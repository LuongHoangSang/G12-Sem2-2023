% --- chapter
\newcommand{\chapter}[2][]{
	\newcommand{\chapname}{#2}
	\begin{flushleft}
		\begin{minipage}[t]{\linewidth}
			\includegraphics[height=1cm]{hdht-logo.png}
			\hspace{0pt}	
			\sffamily\bfseries\large Bài  31. Hiện tượng quang điện trong
			\begin{flushleft}
				\huge\bfseries #1
			\end{flushleft}
		\end{minipage}
	\end{flushleft}
	\vspace{1cm}
	\normalfont\normalsize
}
%-----------------------------------------------------
\chapter[Hiện tượng quang điện trong]{Hiện tượng quang điện trong}


\subsection{Chất quang dẫn và hiện tượng quang điện trong}
\subsubsection{Chất quang dẫn}
Chất quang dẫn là chất dẫn điện kém khi không bị chiếu sáng và trở thành chất dẫn điện tốt khi bị chiếu ánh sáng thích hợp.
%\manatip{Quang (ánh sáng), dẫn (dẫn điện). Quang dẫn (chiếu ánh sáng vào và dẫn điện tốt).}
\luuy{Ánh sáng thích hợp để gây ra hiện tượng quang dẫn có bước sóng nhỏ hơn hoặc bằng giới hạn quang dẫn $\lambda \leq \lambda_0$.}
\subsubsection{Hiện tượng quang điện trong}
Hiện tượng ánh sáng giải phóng các electron liên kết để chúng trở thành các electron dẫn đồng thời tạo ra các lỗ trống cùng tham gia vào quá trình dẫn điện, gọi là hiện tượng quang điện trong.
\luuy{
\begin{itemize}
	\item Hiện tượng quang điện trong không làm bật các electron ra khỏi bề mặt mà chỉ giải phóng các electron ra khỏi liên kết với nguyên tử.
	\item Năng lượng cần thiết để gây ra hiện tượng quang điện trong nhỏ hơn rất nhiều so với hiện tượng quang điện ngoài.
\end{itemize}
}
\subsection{Ứng dụng của hiện tượng quang điện trong}
\subsubsection{Quang điện trở}
	Quang điện trở là điện trở làm bằng chất quang dẫn. Điện trở của quang điện trở có thể thay đổi từ vài triệu $\Omega$ khi không được chiếu sáng đến vài chục $\Omega$ khi được chiếu ánh sáng thích hợp.
\subsubsection{Pin quang điện}
	Pin quang điện là pin chạy bằng năng lượng ánh sáng. Nó biến đổi trực tiếp quang năng thành điện năng. Pin hoạt động dựa vào hiện tượng quang điện trong.