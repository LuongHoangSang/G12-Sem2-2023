
\chapter[Quang phổ vạch của nguyên tử hiđrô]{Quang phổ vạch của nguyên tử hiđrô}

\section{Lý thuyết}

\subsection{Vận dụng công thức tính số vạch quang phổ tối đa mà khối khí hiđrô phát ra được}

Khi khối khí hiđrô có các nguyên tử đang ở trạng thái dừng có mức năng lượng $E_n$ thì số vạch quang phổ tối đa mà khối khí đó phát ra được khi chuyển sang các trạng thái dừng khác có mức năng lượng thấp hơn là
\begin{equation}
	\dfrac{n(n-1)}{2}.
\end{equation}

\begin{center}
	\begin{tabular}{|m{15em}|c|c|c|c|c|c|}
		\hline
		\textbf{Mức năng lượng} 
		&$E_1$
		&$E_2$
		&$E_3$
		&$E_4$
		&$E_5$
		&$E_6$
		\\ \hline
		\textbf{Số vạch quang phổ tối đa}
		& $0$ 
		& $1$ 
		& $3$ 
		& $6$ 
		& $10$ 
		& $15$ 
		\\ \hline
	\end{tabular}
\end{center}

\subsection{Liên hệ bước sóng hoặc tần số của các vạch quang phổ}

Công thức:
\begin{equation}
	\dfrac{1}{\lambda_{31}} = \dfrac{1}{\lambda_{32}}+ \dfrac{1}{\lambda_{21}},
\end{equation}
trong đó:
\begin{itemize}
	\item $\lambda_{31}$ là bước sóng phát ra khi nguyên tử chuyển từ trạng thái $n_3$ (cao) sang trạng thái $n _1$ (thấp);
	\item $\lambda_{32}$ là bước sóng phát ra khi nguyên tử chuyển từ trạng thái $n_3$ (cao) sang trạng thái $n _2$ (trung gian);
	\item $\lambda_{21}$ là bước sóng phát ra khi nguyên tử chuyển từ trạng thái $n_2$ (trung gian) sang trạng thái $n _1$ (thấp).
\end{itemize}
\subsubsection{Mối liên hệ tần số của các vạch quang phổ}
Công thức:
\begin{equation}
	f_{31} = f_{32}+ f_{21}.
\end{equation}
\begin{itemize}
	\item $f_{31}$ là tần số phát ra khi nguyên tử chuyển từ trạng thái $n_3$ (cao) sang trạng thái $n _1$ (thấp);
	\item $f_{32}$ là tần số phát ra khi nguyên tử chuyển từ trạng thái $n_3$ (cao) sang trạng thái $n _2$ (trung gian);
	\item $f_{21}$ là tần số phát ra khi nguyên tử chuyển từ trạng thái $n_2$ (trung gian) sang trạng thái $n _1$ (thấp).
\end{itemize}

\section{Mục tiêu bài học - Ví dụ minh họa}

\begin{dang}{Vận dụng công thức tính số vạch quang phổ tối đa mà khối khí hiđrô phát ra được.}
	
	\ppgiai{
		\begin{description}
			\item[Bước 1:] Xác định trạng thái dừng của nguyên tử hiđrô dựa vào bán kính quỹ đạo ($r_n = n ^2 r_0$) hoặc mức năng lượng tương ứng ($E_n = -\dfrac{13.6}{n^2}\ \text{eV}$);
			\item[Bước 2:] Thay $n$ vào công thức $\dfrac{n(n-1)}{2}$ để tìm số vạch quang phổ tối đa mà khối khí phát ra được.
		\end{description}
	}
	
	\viduii{2}{
		Nguyên tử hiđrô được kích thích để chuyển lên quỹ đạo dừng M. Khi nó chuyển về các trạng thái dừng có mức năng lượng thấp hơn thì sẽ phát ra
		\begin{mcq}(2)
			\item một bức xạ.
			\item hai bức xạ.
			\item ba bức xạ.
			\item bốn bức xạ.
	\end{mcq}}
	{\begin{center}
			\textbf{Hướng dẫn giải}
		\end{center}
		
		Nguyên tử hiđrô được kích thích để chuyển lên quỹ đạo dừng M, nên $n=3$. Áp dụng công thức $\dfrac{n(n-1)}{2}$ để tìm số vạch quang phổ tối đa mà nguyên tử hiđrô phát ra được.
		
		Áp dụng công thức $\dfrac{n(n-1)}{2}$, tính được số vạch quang phổ tối đa mà nguyên tử hiđrô phát ra được là 3 vạch.
		
		\begin{center}
			\textbf{Câu hỏi tương tự}
		\end{center}
		
		Nguyên tử hiđrô được kích thích để chuyển lên quỹ đạo dừng N. Khi nó chuyển về các trạng thái dừng có mức năng lượng thấp hơn thì sẽ phát ra
		\begin{mcq}(2)
			\item sáu bức xạ.
			\item hai bức xạ.
			\item tám bức xạ.
			\item bốn bức xạ.
		\end{mcq}
		\textbf{Đáp án:} A.
	}
	
	\viduii{3}{Khối khí hiđrô có các nguyên tử đang ở trạng thái kích thích thứ nhất thì khối khí nhận thêm năng lượng và chuyển lên trạng thái kích thích mới. Biết rằng ở trạng thái kích thích mới, electron chuyển động trên quỹ đạo có bán kính gấp 9 lần trạng thái kích thích cũ. Số các bức xạ có tần số khác nhau mà khối khí hiđrô có thể phát ra tối đa là
		\begin{mcq}(4)
			\item 15.
			\item 18.
			\item 21.
			\item 24.
	\end{mcq}}
	{\begin{center}
			\textbf{Hướng dẫn giải}
		\end{center}
		
		Ở trạng thái kích thích cũ (trạng thái kích thích thứ nhất), nguyên tử hiđrô có $n_1= 2$, suy ra
		\begin{equation*}
			r_{n_1} = n_1 ^2 r_0 = 4 r_0.
		\end{equation*}
		
		Ở trạng thái kích thích mới, nguyên tử hiđrô có $r_{n_2} = 9 r_{n_1}$, suy ra
		\begin{equation*}
			r_{n _2} = 36 r_0.
		\end{equation*}
		
		Áp dụng công thức $r_{n_2} = n _2 ^2 r_0$, ta có
		\begin{equation*}
			n _2 ^2 r_0 = 36 r_0 \Rightarrow n _2 = 6 .
		\end{equation*}
		
		Số vạch quang phổ tối đa mà nguyên tử hiđrô phát ra được:
		\begin{equation*}
			\dfrac{n_2(n_2-1)}{2}=15.
		\end{equation*}
		
		\begin{center}
			\textbf{Câu hỏi tương tự}
		\end{center}
		
		Khối khí hiđrô có các nguyên tử đang ở trạng thái kích thích thứ nhất thì khối khí nhận thêm năng lượng và chuyển lên trạng thái kích thích mới. Biết rằng ở trạng thái kích thích mới, electron chuyển động trên quỹ đạo có bán kính gấp 4 lần trạng thái kích thích cũ. Số các bức xạ có tần số khác nhau mà khối khí hiđrô có thể phát ra tối đa là
		\begin{mcq}(4)
			\item 6.
			\item 9.
			\item 3.
			\item 8.
		\end{mcq}	
		
		\textbf{Đáp án:} A.
	}
	
\end{dang}




\begin{dang}{Liên hệ bước sóng hoặc tần số của các vạch quang phổ.}
	
	\ppgiai{
		\begin{description}
			\item[Bước 1:] Xác định trạng thái có mức năng lượng cao nhất ($n_3$) và mức năng lượng thấp nhất ($n_1$) trong sự chuyển mức;
			\item[Bước 2:] Xác định trạng thái có mức năng lượng trung gian ($n_2$);
			\item[Bước 3:] Áp dụng công thức $\dfrac{1}{\lambda_{31}} = \dfrac{1}{\lambda_{32}}+ \dfrac{1}{\lambda_{21}}$ hoặc $f_{31} = f_{32}+ f_{21}$ để tính bước sóng hoặc tần số của vạch quang phổ cần tìm.
		\end{description}
	}
	
	\viduii{3}{
		Khi chuyển từ quỹ đạo M về quỹ đạo L, nguyên tử hiđrô phát ra phôtôn có bước sóng $\SI{0.6563}{\micro \meter}$. Khi chuyển từ quỹ đạo N về quỹ đạo L, nguyên tử hiđrô phát ra phôtôn có bước sóng $\SI{0.4861}{\micro \meter}$. Khi chuyển từ quỹ đạo N về quỹ đạo M, nguyên tử hiđrô phát ra phôtôn có bước sóng
		\begin{mcq}(2)
			\item $\SI{1.1424}{\micro \meter}$.
			\item $\SI{1.8744}{\micro \meter}$.
			\item $\SI{0.1702}{\micro \meter}$.
			\item $\SI{0.2793}{\micro \meter}$.
	\end{mcq}}
	{\begin{center}
			\textbf{Hướng dẫn giải}
		\end{center}
		
		Gọi:
		\begin{itemize}
			\item Bước sóng phát ra khi nguyên tử hiđrô chuyển từ quỹ đạo M ($n_2$) về quỹ đạo L ($n_1$) là $\lambda_{21}$.
			\item Bước sóng phát ra khi nguyên tử hiđrô chuyển từ quỹ đạo N ($n_3$) về quỹ đạo L ($n_1$) là $\lambda_{31}$.
			\item Bước sóng phát ra khi nguyên tử hiđrô chuyển từ quỹ đạo N ($n_3$) về quỹ đạo M ($n_2$) là $\lambda_{32}$.
		\end{itemize}
		
		Áp dụng công thức
		\begin{equation*}
			\dfrac{1}{\lambda_{31}} = \dfrac{1}{\lambda_{32}}+ \dfrac{1}{\lambda_{21}},
		\end{equation*}
		suy ra
		\begin{equation*}
			\lambda_{32} = \dfrac{\lambda_{31} \lambda_{21}}{\lambda_{21} - \lambda_{31}} = \SI{1.8744}{\micro \meter} .
		\end{equation*}
		
		\begin{center}
			\textbf{Câu hỏi tương tự}
		\end{center}
		
		Khi chuyển từ quỹ đạo M về quỹ đạo L, nguyên tử hiđrô phát ra phôtôn có tần số $ \SI{4,57 e14}{Hz} $. Khi chuyển từ quỹ đạo N về quỹ đạo L, nguyên tử hiđrô phát ra phôtôn có tần số $\SI{6,17 e14}{\micro \meter}$. Khi chuyển từ quỹ đạo N về quỹ đạo M, nguyên tử hiđrô phát ra phôtôn có tần số
		\begin{mcq}(2)
			\item $ \SI{2,626 e14}{Hz} $.
			\item $ \SI{1,601 e14}{Hz} $.
			\item $ \SI{1,763 e15}{Hz} $.
			\item $ \SI{1,074 e15}{Hz} $.
		\end{mcq}
		
		\textbf{Đáp án:} B.}
	
	\viduii{3}
	{
		Đối với nguyên tử Hiđrô, khi electron chuyển từ quỹ đạo L về quỹ đạo K thì nguyên tử phát ra photon tương ứng với bước sóng $ \SI{121,8}{nm} $. Khi electron chuyển từ quỹ đạo M về quỹ đạo L nguyên tử phát ra photon tương ứng với bước sóng $ \SI{656,3}{nm} $. Khi electron chuyển từ quỹ đạo M về quỹ đạo K, nguyên tử phát ra photon tương ứng với bước sóng
		\begin{mcq}(2)
			\item $ \SI{309,1}{nm} $.
			\item $ \SI{534,5}{nm} $.
			\item $ \SI{95,7}{nm} $.
			\item $ \SI{102,7}{nm} $.
		\end{mcq}
	}
	{
		\begin{center}
			\textbf{Hướng dẫn giải}
		\end{center}
		Ta có: $ \lambda_{LK} = \SI{121,8}{nm} $ và $ \lambda_{ML} = \SI{656,3}{nm} $. \\
		$$
		\dfrac{1}{\lambda_{MK}} = \dfrac{1}{\lambda_{ML}} + \dfrac{1}{\lambda_{LK}} \rightarrow \lambda_{MK} = \SI{102,7}{nm}.
		$$
		\begin{center}
			\textbf{Câu hỏi tương tự}
		\end{center}
		Đối với nguyên tử Hiđrô, khi electron chuyển từ quỹ đạo L về quỹ đạo K thì nguyên tử phát ra photon tương ứng với tần số $ \SI{121,8}{nm} $. Khi electron chuyển từ quỹ đạo M về quỹ đạo L nguyên tử phát ra photon tương ứng với tần số $ \SI{656,3}{nm} $. Khi electron chuyển từ quỹ đạo M về quỹ đạo K, nguyên tử phát ra photon tương ứng với tần số
		\begin{mcq}(2)
			\item $ \SI{9,706 e14}{Hz} $.
			\item $ \SI{5,613 e14}{Hz} $.
			\item $ \SI{3,135 e15}{Hz} $.
			\item $ \SI{2,921 e14}{Hz} $.
		\end{mcq}
		\textbf{Đáp án:} D.
	}
	
\end{dang}
\section{Bài tập tự luyện}
\begin{enumerate}[label=\bfseries Câu \arabic*:]
	
	% Câu No 
	\item \mkstar{1} [3]
		\cauhoi
	{Trong quang phổ vạch của nguyên tử Hidro, ở vùng ánh sáng nhìn thấy có 4 vạch đặc trưng là vạch
		\begin{mcq}(2)
			\item đỏ, lam, chàm, tím. 
			\item đỏ, cam, vàng, tím.
			\item đỏ, cam, chàm, tím. 
			\item đỏ, lục, lam, tím. 
		\end{mcq}
	}
	
	\loigiai
	{		\textbf{Đáp án: A.}
		
		Trong quang phổ vạch của nguyên tử Hidro, ở vùng ánh sáng nhìn thấy có 4 vạch đặc trưng là vạch đỏ, lam, chàm, tím.
	}
	
	% Câu No 
	\item \mkstar{3} [3]
		\cauhoi
	{Theo mẫu Bo về nguyên tử Hidro, nếu lực tượng tác tĩnh điện khi electron và hạt nhân khi chuyển động trên quỹ đạo dừng là $ L $ là $ F $ thì khi electron chuyển động trên quỹ đạo dừng $ n $ lực tĩnh điện là $ \dfrac{F}{16} $. Tỉ số giữa bước sóng dài nhất và bước sóng ngắn nhất mà nguyên tử Hidro có thể phát ra bằng
		\begin{mcq}(4)
			\item $ \dfrac{7}{135} $. 
			\item $ \dfrac{3}{128} $.
			\item $ \dfrac{135}{7} $. 
			\item $ \dfrac{128}{3} $. 
		\end{mcq}
	}
	
	\loigiai
	{		\textbf{Đáp án: A.}
		
		Lực tương tác tĩnh điện trên quỹ đạo $ L $ là
		$$
		F = k \cdot \dfrac{e^{2}}{\left( 2^{2} r_{0} \right)^2} = k \cdot \dfrac{e^{2}}{16 r_{0}^{2}}.
		$$
		Lực tương tác tĩnh điện trên quỹ đạo $ n $ là
		$$
		\dfrac{F}{16} = k \cdot \dfrac{e^{2}}{\left( n^{2} r_{0} \right)^2} \rightarrow n = \num{4}.
		$$
		Ta có:
		$$
		\dfrac{\lambda_{max}}{\lambda_{min}} = \dfrac{\varepsilon_{max}}{\varepsilon_{min}} = \dfrac{E_{4} - E_{1}}{E_{4} - E_{3}} = \dfrac{\dfrac{-13,6}{4^{4}} - \dfrac{-13,6}{1^{2}}}{\dfrac{-13,6}{4^{2}} - \dfrac{-13,6}{3^{2}}} = \dfrac{135}{7}.
		$$
	}
	
	% Câu No 
	\item \mkstar{3} [3]
		\cauhoi
	{Các mức năng lượng của các trạng thái dừng của nguyên tử Hidro được xác định bằng biểu thức $ E_{n} = \xsi{\dfrac{-13,6}{n^{2}}}{eV} $ với $ n = 1,2,3... $ . Mức năng lượng của nguyên tử Hidro ở trạng thái dừng $ M $ là
		\begin{mcq}(4)
			\item $ \SI{-1,51}{eV} $. 
			\item $ \SI{-0,544}{eV} $.
			\item $ \SI{-0,85}{eV} $. 
			\item $ \SI{-3,4}{eV} $. 
		\end{mcq}
	}
	
	\loigiai
	{		\textbf{Đáp án: A.}
		
		Mức năng lượng của nguyên tử Hidro ở trạng thái dừng $ M $ là
		$$
		E_{3} = \dfrac{-13,6}{3^{2}} = \SI{-1,51}{eV}.
		$$
	}
	
	% Câu No 
	\item \mkstar{3} [3]
		\cauhoi
	{Mẫu nguyên tử Bo, bán kính quỹ đạo $ K $ của electron trong nguyên tử Hidro là $ r_{0} $. Khi electron chuyển từ quỹ đạo $ M $ lên quỹ đạo $ N $ thì bán kính quỹ đạo tăng thêm
		\begin{mcq}(4)
			\item $ 7 r_{0} $. 
			\item $ 16 r_{0} $.
			\item $ 27 r_{0} $. 
			\item $ 5 r_{0} $. 
		\end{mcq}
	}
	
	\loigiai
	{		\textbf{Đáp án: A.}
		
		Bán kính quỹ đạo $ r_{M} $ và $ r_{N} $ lần lượt là
		$
		\left\{
		\begin{aligned}
			& r_{M} = 3^{2} r_{0} = 9 r_{0} \\
			& r_{N} = 4^{2} r_{0} = 16 r_{0}.
		\end{aligned}
		\right.
		$
		
		Vậy khi electron chuyển từ quỹ đạo $ M $ lên quỹ đạo $ N $ thì bán kính quỹ đạo tăng thêm $ 7 r_{0} $.
	}
	
	% Câu No 
	\item \mkstar{3} [4]
	\cauhoi
	{Trong nguyên tử Hidro, với $ r_{0} $ là bán kính Bo thì bán kính dừng của electron có thể là
		\begin{mcq}(4)
			\item $ 12 r_{0} $. 
			\item $ 26 r_{0} $.
			\item $ 9 r_{0} $. 
			\item $ 10 r_{0} $. 
		\end{mcq}
	}
	
	\loigiai
	{		\textbf{Đáp án: A.}
		
		Bán kính quỹ đạo dừng phải có dạng $ r_{n} = n^{2} r_{0} $. Nên bán kính quỹ đạo dừng chỉ có thể là $ 9 r_{0} $. 
	}
	
	% Câu No 
	\item \mkstar{3} [4]
	\cauhoi
	{Nguyên tử Hidro chuyển từ trạng thái $ E_{M} = \SI{-1,5}{eV} $ sang trạng thái dừng có năng lượng $ E_{L} = \SI{-3,4}{eV} $ thì nó sẽ
		\begin{mcq}
			\item hấp thụ một photon có năng lượng bằng $ \varepsilon = \SI{1,9e-19}{J} $. 
			\item phát ra một photon có năng lượng bằng $ \varepsilon = \SI{1,9}{eV} $. 
			\item hấp thụ một photon có năng lượng bằng $ \varepsilon = \SI{3,04e-19}{J} $.  
			\item phát ra một photon có năng lượng bằng $ \varepsilon = \SI{3,04}{eV} $.  
		\end{mcq}
	}
	
	\loigiai
	{		\textbf{Đáp án: B.}
		
		Khi chuyển từ mức $ M $ sang mức $ L $ thì năng lượng mà photon phát ra cho bởi:
		$$
		\varepsilon = E_{L} - E_{M} = \SI{1,9}{eV}.
		$$
	}
	
	% Câu No 
	\item \mkstar{3} [4]
	\cauhoi
	{Khi electron ở quỹ đạo dừng thứ $ n $ thì năng lượng của nguyên tử Hidro được xác định bởi công thức $ E_{n} = \xsi{\dfrac{-13,6}{n^{2}}}{eV} $ (với $ n= 1,2,3... $). Khi electron trong nguyên tử Hidro đang chuyển từ quỹ đạo dừng $ n = 3 $ về quỹ đạo dừng $ n = 1 $ thì nguyên tử phát ra bức xạ có bước sóng $ \lambda_{1} $. Khi electron chuyển từ quỹ đạo dừng $ n = 5 $ về quỹ đạo dừng $ n = 2 $ thì nguyên tử phát ra bức xạ có bước sóng $ \lambda_{2} $. Mối liên hệ giữa $ \lambda_{2} $ và $ \lambda_{1} $ là
		\begin{mcq}(4)
			\item $ \lambda_{2} = 5 \lambda_{1} $. 
			\item $  27 \lambda_{2} = 128 \lambda_{1} $.
			\item $ \lambda_{2} = 4 \lambda_{1} $. 
			\item $ 189 \lambda_{2} = 800 \lambda_{1} $. 
		\end{mcq}
	}
	
	\loigiai
	{		\textbf{Đáp án: A.}
		
		Mối liên hệ giữa $ \lambda_{2} $ và $ \lambda_{1} $ là
		$$
		\dfrac{\lambda_{2}}{\lambda_{1}} = \dfrac{E_{3} - E_{1}}{E_{5} - E_{2}} = \dfrac{800}{189}.
		$$
		Vậy $ 189 \lambda_{2} = 800 \lambda_{1} $.
	}
	
	% Câu No 
	\item \mkstar{3} [10]
	\cauhoi
	{Trong quang phổ vạch của Hidro , bước sóng của vạch thứ nhất trong dãy Laiman ứng với sự chuyển của electron từ quỹ đạo $ L $ về quỹ đạo $ K $ là $ \SI{0,1217}{\mu m} $, vạch thứ nhất của dãy Banme ứng với sự chuyển từ $ M $ sang $ L $  là $ \SI{0,6563}{\mu m} $. Bước sóng của vạch quang phổ thứ hai trong dãy Laiman ứng với sự chuyển từ $ M $ sang $ K $ bằng 
		\begin{mcq}(4)
			\item $ \SI{0,1027}{\mu m} $.  
			\item $ \SI{0,5346}{\mu m} $. 
			\item $ \SI{0,7780}{\mu m} $.   
			\item $ \SI{0,3890}{\mu m} $.   
		\end{mcq}
	}
	
	\loigiai
	{		\textbf{Đáp án: A.}
		
		Vạch quang phổ thứ hai trong dãy Laiman cho bởi:
		\begin{align*}
			\dfrac{hc}{\lambda_{31}} &= E_{3} - E_{1} \\ 
			&= \left( E_{3} - E_{2} \right) + \left( E_{2} - E_{1} \right) \\
			&= \dfrac{hc}{\lambda_{32}} + \dfrac{hc}{\lambda_{21}}.
		\end{align*}
		Suy ra,
		\begin{equation*}
			\dfrac{1}{\lambda_{31}} = \dfrac{1}{\lambda_{32}} + \dfrac{1}{\lambda_{21}} \rightarrow \lambda_{32} = \SI{0,1027}{\mu m}.  
		\end{equation*}
	}
	
	% Câu No 
	\item \mkstar{2} [12]
	\cauhoi
	{Trong nguyên tử Hidro, khi electron chuyển từ quỹ đạo $ P $ có năng lượng $ E_{P} $ về quỹ đạo $ L $ có năng lượng $ E_{L} $ thì phát ra photon có năng lượng $ \varepsilon $. Hệ thức nào dưới đây đúng?
		\begin{mcq}(4)
			\item $ \varepsilon = \dfrac{E_{P} - E_{L}}{hc} $.  
			\item $ \varepsilon = \dfrac{E_{P} + E_{L}}{hc} $. 
			\item $ \varepsilon = E_{P} - E_{L}$.   
			\item $ \varepsilon = E_{L} - E_{P}$.   
		\end{mcq}
	}
	
	\loigiai
	{		\textbf{Đáp án: C.}
		
		Theo tiên đề Bo,
		\begin{equation}
			\varepsilon = E_{P} - E_{L}. 
		\end{equation}
	}
	
	% Câu No 
	\item \mkstar{2} [12]
	\cauhoi
	{Biết năng lượng của nguyên tử Hidro ở các trạng thái dừng được tính bằng công thức: $ E_{n} = \dfrac{-13,6}{n^{2}} $, với $ n = 1,2,3... $. Nguyên tử Hidro đang ở trạng thái cơ bản, nếu hấp thụ một photon có năng lượng $ \dfrac{1632}{125} $ (eV) thì nó chuyển lên trạng thái dừng có năng lượng 
		\begin{mcq}(4)
			\item $ \xsi{-\dfrac{17}{5}}{eV} $.  
			\item $ \xsi{-\dfrac{17}{20}}{eV} $. 
			\item $ \xsi{-\dfrac{68}{45}}{eV} $.   
			\item $ \xsi{-\dfrac{68}{125}}{eV} $.   
		\end{mcq}
	}
	
	\loigiai
	{		\textbf{Đáp án: C.}
		
		Khi ở trạng thái cơ bản và hấp thụ một photon có năng lượng $ \dfrac{1632}{125} $ (eV) thì nó chuyển lên trạng thái dừng có năng lượng có độ lớn bằng
		\begin{equation*}
			-13,6 + \dfrac{1632}{125} = \xsi{-\dfrac{68}{45}}{eV}
		\end{equation*}
	}
	
	% Câu No 
	\item \mkstar{3} [13]
	\cauhoi
	{Khi electron trong nguyên tử Hidro chuyển từ quỹ đạo $ M $ về quỹ đạo $ L $, nguyên tử Hidro phát ra photon có bước sóng $ \SI{0,6563}{\mu m} $. Khi chuyển từ quỹ đạo $ N $ về quỹ đạo $ L $ nguyên tử Hidro phát ra photon có bước sóng $ \SI{0,4861}{\mu m} $. Khi chuyển từ quỹ đạo $ N $ về $ M $, nguyên tử Hidro phát ra bước sóng bằng
		\begin{mcq}(4)
			\item $ \SI{1,8744}{\mu m} $.  
			\item $ \SI{1,1424}{\mu m} $. 
			\item $ \SI{0,5335}{\mu m} $.   
			\item $ \SI{0,1702}{\mu m} $.   
		\end{mcq}
	}
	
	\loigiai
	{		\textbf{Đáp án: A.}
		
		Khi chuyển từ quỹ đạo $ N $ về $ M $, nguyên tử Hidro phát ra bước sóng bằng
		\begin{align*}
			E_{NM} &= E_{N} - E_{M} \\
			&= \left( E_{N} - E_{L} \right) - \left( E_{M} - E_{L} \right) \\
			&= E_{NL} - E_{ML} \\
		\end{align*}
		Suy ra,
		\begin{equation*}
			\dfrac{1}{\lambda_{NM}} = \dfrac{1}{\lambda_{NL}} - \dfrac{1}{\lambda_{ML}} \rightarrow \lambda_{NM} = \SI{1,8744}{\mu m}.  
		\end{equation*}
	}
	
	% Câu No 
	\item \mkstar{2} [5]
	\cauhoi
	{Trong nguyên tử Hidro, bán kính Bo là $ r_{0} = \SI{5,3e-11}{m} $. Bán kính quỹ đạo dừng $ M $ là
		\begin{mcq}(4)
			\item $ \SI{132,5e-11}{m} $. 
			\item $ \SI{47,7e-11}{m} $. 
			\item $ \SI{21,2e-11}{m} $.   
			\item $ \SI{84,8e-11}{m} $.   
		\end{mcq}
	}
	
	\loigiai
	{		\textbf{Đáp án: B.}
		
		Bán kính quỹ đạo dừng $ M $ là
		\begin{equation*}
			r_{M} = 3^{2} r_{0} = \SI{47,7e-11}{m}.
		\end{equation*}
	}
	
	% Câu No 
	\item \mkstar{3} [5]
	\cauhoi
	{Trong quang phổ vạch của nguyên tử Hidro, gọi $ d_{1} $ là khoảng cách giữa mức $ L $ và mức $ M $, $ d_{2} $ là khoảng cách giữa mức $ M $ và $ N $. Tỉ số giữa $ d_{2} $ và $ d_{1} $ là 
		\begin{mcq}(4)
			\item $ \num{1,4} $. 
			\item $ \num{1} $. 
			\item $ \num{0,7} $.   
			\item $ \num{2,4} $.   
		\end{mcq}
	}
	
	\loigiai
	{		\textbf{Đáp án: B.}
		
		Ta có:
		$$
		\left\{
		\begin{aligned}
			d_{1} &= r_{M} - r_{L} = 3^{2} r_{0} - 2^{2} r_{0} = 5 r_{0} \\
			d_{2} &= r_{N} - r_{M} = 4^{2} r_{0} - 3^{2} r_{0} = 7 r_{0}.
		\end{aligned}
		\right.
		$$
		Vậy tỉ số giữa $ d_{2} $ và $ d_{1} $ là $ \num{0,7}. $
	}
	
	% Câu No 
	\item \mkstar{2} [5]
	\cauhoi
	{Dãy Ban–me ứng với sự chuyển động electron từ quỹ đạo ở xa hạt nhân về quỹ đạo nào sau đây?
		\begin{mcq}(4)
			\item Quỹ đạo $ K $.
			\item Quỹ đạo $ L $.
			\item Quỹ đạo $ M $. 
			\item Quỹ đạo $ N $. 
		\end{mcq}
	}
	
	\loigiai
	{		\textbf{Đáp án: B.}
		
		Dãy Ban-me ứng với sự chuyển động của elctron từ quỹ đạo ở xa hạt nhân về quỹ  đạo $ L $.
	}	
	
	% Câu No 
	
	
	% Câu No 
	\item \mkstar{2} [7]
	\cauhoi
	{Xét nguyên tử Hidro theo mẫu nguyên tử Bo. Gọi $ r_{0} $ là bán kính Bo. Trong các quỹ đạo dừng của electron có bán kính lần lượt là $ r_{0}, 4 r_{0}, 9 r_{0}, 16 r_{o} $, bán kính dừng khi electron chuyển động trên quỹ đạo $ L $ là
		
		\begin{mcq}(4)
			\item $ 9 r_{0} $.
			\item $ 16 r_{0} $.
			\item $ 4 r_{0} $. 
			\item $ r_{0} $. 
		\end{mcq}
	}
	
	\loigiai
	{		\textbf{Đáp án: C.}
		
		bán kính dừng khi electron chuyển động trên quỹ đạo $ L $ là $ 4 r_{0} $. 
	}	
	
	% Câu No 
	\item \mkstar{3} [7]
		\cauhoi
	{Khi electron trong nguyên tử Hidro chuyển từ quỹ đạo dừng có mức năng lượng $ E_{M} = \SI{-0,85}{eV} $ sang quỹ đạo dừng có năng lượng $ E_{N} = \SI{-13,60}{eV} $ thì nguyên tử phát bức xạ điện từ có bước sóng
		\begin{mcq}(4)
			\item $ \SI{0,4340}{\mu m} $.    
			\item $ \SI{0,0974}{\mu m} $. 
			\item $ \SI{0,4860}{\mu m} $.     
			\item $ \SI{0,6563}{\mu m} $.  
		\end{mcq}
	}
	
	\loigiai
	{		\textbf{Đáp án: B.}
		
		Nguyên từ phát ra bức xạ điện từ có năng lượng là
		$$
		\varepsilon = E_{M} - E_{N} = \SI{12,75}{eV}.
		$$
		Bức xạ điện từ có bước sóng là
		$$
		\lambda = \dfrac{hc}{\varepsilon} = \SI{0,0974}{\mu m}.
		$$
	}	
	
	% Câu No 
	\item \mkstar{2} [7]
		\cauhoi
	{Theo mẫu nguyên tử Bo, nguyên tử Hidro tồn tại ở các trạng thái dừng có năng lượng tương ứng là $ E_{K} = -144E, E_{L} = -36E, E_{M} = -16E, E_{N} = -9E,... $ ($ E $ là hằng số). Khi một nguyên tử Hidro chuyển từ trạng thái dừng có năng lượng $ E_{M} $ về trạng thái dừng có năng lượng $ E_{K} $ thì phát ra một photon có năng lượng
		\begin{mcq}(4)
			\item $ 135E $.    
			\item $ 128E $. 
			\item $ 7E $.   
			\item $ 9E $.  
		\end{mcq}
	}
	
	\loigiai
	{		\textbf{Đáp án: B.}
		
		Photon phát ra có năng lượng là
		$$
		\varepsilon = E_{M} - E_{K} = 128E.
		$$
	}	
	
	% Câu No 
	\item \mkstar{3} [7]
		\cauhoi
	{ Một đám nguyên tử Hidro đang ở trạng thái kích thích mà electron chuyển động trên quỹ đạo dừng $ N $. Khi electron chuyển về các quỹ đạo dừng bên trong thì quang phổ vạch phát xạ của đám nguyên tử đó có bao nhiêu vạch?
		
		\begin{mcq}(4)
			\item $ 3 $.    
			\item $ 1 $. 
			\item $ 6 $.   
			\item $ 4 $.  
		\end{mcq}
	}
	
	\loigiai
	{		\textbf{Đáp án: C.}
		
		Quỹ đạo $ N $ nên ta có $ n = 4 $. Vậy nên số vạch phát ra tối đa là $ \dfrac{n(n-1)}{2} = 6 $.
	}	
		\item \mkstar{1}
		\cauhoi
	{ Bốn vạch $H_\alpha, H_\beta, H_\gamma, H_\delta$ của nguyên tử hydro thuộc dãy nào?
		
		
		\begin{mcq}(4)
			\item  Lyman.  
			\item  Ban-me.	
			\item  Pa-sen.  
			\item  Vừa Ban-me vừa Lyman.
			
		\end{mcq}
	}
	
	\loigiai
	{		\textbf{Đáp án: B.}
		
		
	}	
		\item \mkstar{1}
		\cauhoi
	{ 
		Dãy Lyman trong quang phổ vạch của hydro ứng với sự dịch chuyển của các electron từ các quỹ đạo dừng có năng lượng cao về quỹ đạo
		
		\begin{mcq}(4)
			\item K.   
			\item L. 
			\item M.   
			\item N.
		\end{mcq}
	}
	
	\loigiai
	{		\textbf{Đáp án: A.}
		
		Dãy Lyman trong quang phổ vạch của hydro ứng với sự dịch chuyển của các electron từ các quỹ đạo dừng có năng lượng K. 
	}	
		\item \mkstar{1}
		\cauhoi
	{ 
		Dãy Pa-sen trong quang phổ vạch của hydro ứng với sự dịch chuyển của các electron từ các quỹ đạo dừng có năng lượng cao về quỹ đạo 
		
		\begin{mcq}(4)
			\item K.   
			\item L. 
			\item M.   
			\item N.
		\end{mcq}
	}
	
	\loigiai
	{		\textbf{Đáp án: C.}
		
		
	}	
		\item \mkstar{1}
		\cauhoi
	{ Vạch quang phổ có bước sóng $\lambda = \SI{0,6563}{\mu m}$ là vạch thuộc dãy nào?
		
		
		\begin{mcq}(4)
			\item  Lyman.  
			\item  Banme.
			\item  Banme hoặc Pasen.  
			\item  Pasen.
			
		\end{mcq}
	}
	
	\loigiai
	{		\textbf{Đáp án: B.}
		
		
	}	
		\item \mkstar{1}
		\cauhoi
	{ Dãy Pa-sen nằm trong vùng
		
		
		\begin{mcq}(2)
			\item tử ngoại.   
			\item ánh sáng nhìn thấy.	 
			\item hồng ngoại.  
			\item ánh sáng nhìn thấy và một phần vùng tử ngoại.
			
		\end{mcq}
	}
	
	\loigiai
	{		\textbf{Đáp án: C.}
		
		
	}	
		\item \mkstar{2}
		\cauhoi
	{ 
		Chùm nguyên tử hydro đang ở trạng thái cơ bản, bị kích thích phát sáng thì chúng có thể phát ra tối đa 3 vạch quang phổ. Khi bị kích thích electron trong nguyên tử hydro đã chuyển sang quỹ đạo
		\begin{mcq}(4)
			\item M.   
			\item L. 
			\item O.   
			\item N.
		\end{mcq}
	}
	
	\loigiai
	{		\textbf{Đáp án: A.}
		
		Số bức xạ phát tối đa được xác định theo công thức 
		
		$$N = \dfrac{n(n-1)}{2} = 2\Rightarrow n =3.$$
	}	
	\item \mkstar{2}
		\cauhoi
	{ Nguyên tử hydro bị kích thích chiếu sáng và electron của nguyên tử đã chuyển từ quỹ đạo K lên quỹ đạo M. Sau khi ngừng chiếu sáng, nguyên tử hydro phát xạ thứ cấp, phổ xạ này gồm
		
		
		\begin{mcq}
			\item hai vạch của dãy Lyman.
			
			\item hai vạch của dãy Ban-me. 
			\item một vạch của dãy Lyman và một vạch của dãy Ban-me.
			   
			\item một vạch của dãy Ban-me và hai vạch của dãy Lyman.
			
		\end{mcq}
	}
	
	\loigiai
	{		\textbf{Đáp án: D.}
		
		Khi ngừng chiếu sáng nguyên tử hydro phát xạ gồm một vạch ở dãy Banme và hai vạch dãy Lyman. 
	}	
	\item \mkstar{2}
		\cauhoi
	{ 
		Trong quang phổ của nguyên tử hydro, nếu biết bước sóng dài nhất của vạch quang phổ trong dãy Lyman là $\lambda_1$ và bước sóng của vạch kề với nó trong dãy này là $\lambda_2$ thì bước sóng $\lambda_\alpha$ của vạch quang phổ $H_\alpha$ trong dãy Ban-me là 
		
		\begin{mcq}(4)
			\item  $\lambda_\alpha = \lambda_1 + \lambda_2$.  
			\item  $\lambda_\alpha = \dfrac{\lambda_1 \lambda_2}{\lambda_1 - \lambda_2}.$
			\item  $\lambda_\alpha = \lambda_1 - \lambda_2.$  
			\item $\lambda_\alpha = \dfrac{\lambda_1 \lambda_2}{\lambda_1 + \lambda_2}.$
		\end{mcq}
	}
	
	\loigiai
	{		\textbf{Đáp án: B.}
		
		Ta có:
		
		$$\dfrac{1}{\lambda_\alpha} = \dfrac{1}{\lambda_2} - \dfrac{1}{\lambda_1} \Rightarrow \lambda_\alpha = \dfrac{\lambda_1 \lambda_2}{\lambda_1 - \lambda_2}.$$
	}	
	\item \mkstar{2}
		\cauhoi
	{ Năng lượng ion hóa nguyên tử hydro là $\SI{13,6}{eV}$. Bước sóng ngắn nhất mà nguyên tử có thể phát ra là 
		
		
		\begin{mcq}(4)
			\item $\SI{0,122}{\mu m}$. 
			\item $\SI{0,0913}{\mu m}$. 
			\item $\SI{0,0656}{\mu m}$.   
			\item $\SI{0,5672}{\mu m}$.
		\end{mcq}
	}
	
	\loigiai
	{		\textbf{Đáp án: B.}
		
		Khi nguyên tử hydro hấp thụ một năng lượng $\SI{13,6}{eV}$ thì nó phát ra một photon có bước sóng ngắn nhất thỏa mãn
		
		$$\dfrac{hc}{\lambda_\text{min}} = E_\infty - E_1 \SI{13,6}{eV} \Rightarrow \lambda_\text{min} = \SI{0,0913}{\mu m}.$$ 
	}	

\end{enumerate}

