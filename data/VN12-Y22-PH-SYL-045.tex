
\chapter[Bài tập: Truyền tải điện năng và công suất hao phí trên đường dây tải điện;\\Bài tập: Hiệu suất truyền tải điện]{Bài tập: Truyền tải điện năng và công suất hao phí\\ trên đường dây tải điện;\\Bài tập: Hiệu suất truyền tải điện}
\section{Lý thuyết}
\subsection{Truyền tải điện năng và công suất hao phí trên đường dây tải điện}
\begin{itemize}
	\item Điện năng truyền tải đi xa thường bị tiêu hao đáng kể, chủ yếu do tỏa nhiệt trên đường dây.
	\item Công suất nơi phát điện:
	\begin{equation*}
		P=UI\cos \varphi,
	\end{equation*}
	\item Công suất hao phí do tỏa nhiệt trên đường dây:
	
	\begin{equation*}
		\Delta P = I^2R =R\dfrac{P^2}{(U \cos \varphi)^2},
	\end{equation*}
	
	\item Để giảm hao phí trên đường dây tải điện ta tăng điện áp nơi truyền đi.
	\item Điện năng hao phí trên đường dây tải điện 
	\begin{equation*}
		A=\Delta P \cdot t,
	\end{equation*}
	
	trong đó:
	
	
	+ $U$ là điện áp tại nơi truyền đi (V),
	
	+ $I$ là cường độ dòng điện trên dây dẫn truyền tải (A),
	
	+ $\cos \varphi$ là hệ số công suất,
	
	+ $t$ là thời gian truyền tải điện (s),
	
	+ $R$ là điện trở tổng cộng của đường dây tải điện ($\Omega$).
\end{itemize}
\subsection{Hiệu suất truyền tải điện}
	\begin{equation*}
		H=\dfrac{P'}{P}=\dfrac{P-\Delta P}{P}=1-\dfrac{\Delta P}{P},
	\end{equation*}
	trong đó:
	
	+ $P$ là công suất phát điện từ nhà máy phát điện (W),
	
	+ $P'$ là công suất tiêu thụ tại nơi tiêu thụ (W).

\section{Mục tiêu bài học - Ví dụ minh họa}
\begin{dang}{Ghi nhớ truyền tải điện năng đi xa thì sẽ bị tiêu hao do quá trình tỏa nhiệt trên đường dây tải điện}
	
	\viduii{1}{Khi truyền tải điện năng đi xa bằng đường dây dẫn
		\begin{mcq}
			\item toàn bộ điện năng ở nơi cấp sẽ truyền đến nơi tiêu thụ.
			\item có một phần điện năng hao phí do hiện tượng tỏa nhiệt trên đường dây.
			\item hiệu suất truyền tải là $100\%$.
			\item không có hao phí do tỏa nhiệt trên đường dây.
		\end{mcq}
	}
	{\begin{center}
			\textbf{Hướng dẫn giải}
		\end{center}
		
		Khi truyền tải điện năng đi xa bằng đường dây dẫn sẽ có một phần điện năng hao phí do hiện tượng tỏa nhiệt trên đường dây.
		
		\textbf{Đáp án: B.}
	}
	\viduii{1}{Phương án làm giảm hao phí hữu hiệu nhất là
		\begin{mcq}
			\item tăng tiết diện dây dẫn.
			\item chọn dây dẫn có điện trở suất nhỏ.
			\item tăng hiệu điện thế.
			\item giảm tiết diện dây dẫn.
		\end{mcq}
	}
	{\begin{center}
			\textbf{Hướng dẫn giải}
		\end{center}
		
		Công suất hao phí trên đường dây tải điện là $P'=I^2R=\dfrac{P^2R}{U^2}.$ Phương án làm giảm hao phí hữu hiệu nhất là tăng hiệu điện thế.
		
		\textbf{Đáp án: C.}
	}
	
\end{dang}
\begin{dang}{Ghi nhớ công thức tính công suất hao phí trên đường dây tải điện}
	
	\viduii{2}{Trong truyền tải điện năng đi xa bằng máy biến áp. Biết cường độ dòng điện luôn cùng pha so với điện áp hai đầu nơi truyền đi. Nếu điện áp ở nơi phát tăng 20 lần thì công suất hao phí trên đường dây giảm
		\begin{mcq}(4)
			\item 200 lần.
			\item 40 lần.
			\item 400 lần.
			\item 20 lần.
		\end{mcq}
	}
	{\begin{center}
			\textbf{Hướng dẫn giải}
		\end{center}
		
		Công suất hao phí trong quá trình truyền tải
		\begin{equation*}
			\Delta P = I^2R =R\dfrac{P^2}{(U \cos \varphi)^2}.
		\end{equation*}
		
		Dựa vào công thức ta có $U$ tăng lên 20 lần thì hao phí trên dây giảm 400 lần.
		
		\textbf{Đáp án: C.}
	}
	\viduii{1}{Một trong những biện pháp làm giảm hao phí điện năng trên đường dây tải điện khi truyền tải điện năng đi xa đang được áp dụng rộng rãi là
		\begin{mcq}
			\item tăng điện áp hiệu dụng ở trạm phát điện.
			\item tăng chiều dài đường dây truyền tải điện.
			\item giảm tiết diện dây truyền tải điện.
			\item giảm điện áp hiệu dụng ở trạm phát điện.
		\end{mcq}
	}
	{\begin{center}
			\textbf{Hướng dẫn giải}
		\end{center}
		
		Công suất hao phí trên dây:
		$$\Delta P=R{{I}^{2}}=R.\dfrac{{{P}^{2}}}{{{\left( Ucos\varphi \right)}^{2}}}.$$
		
		Như vậy, với $\cos\varphi$ và công suất nguồn $P$ cố định, ta có hai cách để làm giảm $\Delta P$, đó là:
		
		Cách 1: Làm giảm $R$, với $R=\rho \dfrac{l}{S}$ thì ta phải tăng tiết diện S của đường dây nhưng cách làm này tốn kém vì phải tốn kim loại làm dây và phải tăng sức chịu của cột điện.
		
		Cách 2: Tăng điện áp $U$ ở trạm phát điện. Cách làm này được thực hiện đơn giản bằng máy biến áp, do đó được áp dụng rộng rãi.
		
		\textbf{Đáp án: A.}
	}
	
\end{dang}
\begin{dang}{Liên hệ công thức tính công suất hao phí trên đường dây tải điện để xác định các đại lượng cần tìm}
	
	\viduii{3}{Điện năng được truyền từ một nhà máy phát điện gồm $8$ tổ máy đến nơi tiêu thụ bằng đường dây tải điện một pha. Giờ cao điểm cần cả $8$ tổ máy hoạt động, hiệu suất truyền tải đạt $70\ \%$. Coi điện áp hiệu dụng ở nhà máy không đổi, hệ số công suất của mạch điện bằng $1$, công suất phát điện của các tổ máy khi hoạt động là không đổi và như nhau. Khi công suất tiêu thụ điện ở nơi tiêu thụ giảm còn $83\ \%$ so với giờ cao điểm thì cần bao nhiêu tổ máy hoạt động?
		\begin{mcq}(4)
			\item 4.
			\item 5.
			\item 6.
			\item 7.
		\end{mcq}
	}
	{\begin{center}
			\textbf{Hướng dẫn giải}
		\end{center}
		
		Gọi công suất mỗi tổ máy là ${{P}_{0}}$.
		
		\textbf{Ban đầu:}
		\begin{itemize}
			\item Công suất phát là: $8{{P}_{0}}$.
			\item Công suất tiêu thụ là: ${{P}_{1}}'=0,70{{P}_{1}}$.
			\item Hao phí là:
			$\Delta {{P}_{1}}=0,30{{P}_{1}}=\dfrac{RP_{1}^{2}}{{{U}^{2}}} \dfrac{R}{{{U}^{2}}}=\dfrac{0,30}{{{P}_{1}}}$.
		\end{itemize}
		
		\textbf{Khi công suất ở nơi tiêu thụ giảm còn $83\%$:}
		\begin{itemize}
			\item Công suất tiêu thụ là: ${{P}_{2}}'=0,83{{P}_{1}}'=0,83.0.7{{P}_{1}}=0,581{{P}_{1}}$.	
			\item Hao phí là: $\Delta {{P}_{2}}=\dfrac{RP_{2}^{2}}{{{U}^{2}}}=\dfrac{0.3}{{{P}_{1}}}P_{2}^{2}$.		
			\item Công suất phát là: ${{P}_{2}}={{P}_{2}}'+\Delta {{P}_{2}}=0,581{{P}_{1}}+\dfrac{0.3}{{{P}_{1}}}P_{2}^{2}=0,75{{P}_{1}}=0,75.8{{P}_{0}}=6{{P}_{0}}$.
		\end{itemize}
		
		Vậy cần $6$ tổ máy.
		
		\textbf{Đáp án: C.}
	}
	\viduii{3}{Điện năng từ một trạm phát điện được đưa đến một khu tái định cư bằng đường dây truyền tải một pha. Cho biết, nếu điện áp tại đầu truyền đi tăng từ $U$ lên $2U$ thì số hộ dân được trạm cung cấp đủ điện năng tăng từ 120 lên 144. Cho rằng chỉ tính đến hao phí trên đường dây, công suất tiêu thụ điện của các hộ dân đều như nhau, công suất của trạm phát không đổi và hệ số công suất trong các trường hợp đều bằng nhau. Nếu điện áp truyền đi là $4U$ thì trạm phát này cung cấp đủ điện 
		\begin{mcq}(2)
			\item 150 hộ dân.
			\item 168 hộ dân.
			\item 504 hộ dân. 
			\item 192 hộ dân.
		\end{mcq}
	}
	{\begin{center}
			\textbf{Hướng dẫn giải}
		\end{center}
		
		Công suất hao phí:
		$$ P_{hp}=\frac{P^2}{U^2}R$$ 
		(Với $R$ là điện trở trên đường dây, $P$ là công suất của trạm phát, $U$ là điện áp truyền, $P_0$ là công suất tiêu thụ của mỗi hộ dân, $n$ là số hộ dân).
		
		Ta có:
		$$P\ =\frac{P^2}{U^2}R+120.P_o\ \left(1\right)$$
		
		$$P=\frac{P^2}{4U^2}R+144P_o\ \left(2\right)$$
		
		$$P=\frac{P^2}{16U^2}R+x.P_o\ (3)$$
		
		Từ (1)  và (2): $P_{hp} = 32P_o\ (4)$
		
		Từ (3) và (1), kết hợp với (4) ta có: $$32P_o+120P_o=\frac{1}{16}{.32P}_o+x.P_o$$
		
		Thực hiện tính toán ta suy ra nếu điện áp truyền đi là $4U$ thì trạm phát này cung cấp đủ điện cho 150 hộ dân.
		
		\textbf{Đáp án: A.}
	}
	
	
	
	
\end{dang}
\begin{dang}{Liên hệ giữa công thức tính công suất và công suất tính hao phí để tìm ra công suất thực tế, từ đó lập tỉ số với công suất toàn mạch để tìm ra hiệu suất truyền tải điện}
	
	\viduii{3}{Người ta truyền tải điện xoay chiều một pha từ một trạm phát điện đến nơi tiêu thụ bằng dây dẫn có tổng chiều dài $\SI{20}{km}$. Dây dẫn làm bằng kim loại có điện trở suất $\text{2,5} \cdot 10^{-8}\ \Omega \text{m}$, tiết diện $\SI{0,4}{cm}^2$, hệ số công suất của mạch điện là 1. Điện áp hiệu dụng và công suất truyền đi ở trạm phát điện là $\SI{10}{kV}$ và $\SI{500}{kW}$. Hiệu suất truyền tải điện là
		\begin{mcq}(4)
			\item $\text{93,75}\%$.   
			\item $\text{96,14}\%$.
			\item $\text{97,41}\%$.
			\item $\text{96,88}\%$.
		\end{mcq}
	}
	{\begin{center}
			\textbf{Hướng dẫn giải}
		\end{center}
		
		Điện trở của dây dẫn:
		$$R =\rho \dfrac{l}{S} = \text{12,5}\ \Omega.$$
		
		Công suất hao phí:
		$$ \Delta P  = \dfrac{P^2R}{U^2 \cos^2 \varphi}.$$
		
		Hiệu suất truyền tải điện:
		$$H = \dfrac{P -\delta P}{P} = 1- 	\dfrac{P^2R}{U^2 \cos^2 \varphi} =\text{93,75}\%.$$
		
		\textbf{Đáp án: A.}
	}
	\viduii{3}{Một trạm phát điện xoay chiều có công suất không đổi, truyền điện đi xa với điện áp đưa lên đường dây là $\SI{200}{kV}$ thì tổn hao điện năng là $30\%$. Biết hệ số công suất đường dây bằng 1. Nếu tăng điện áp truyền tải lên $\SI{500}{kV}$ thì tổn hao điện năng là
		\begin{mcq}(4)
			\item $12\%$.
			\item $75\%$.
			\item $24\%$.
			\item $\text{4,8}\%$.
		\end{mcq}
	}
	{\begin{center}
			\textbf{Hướng dẫn giải}
		\end{center}
		
		Hiệu suất hao phí khi điện áp đưa lên đường dây là $\SI{200}{kV}$:
		
		$$h_1 =\dfrac{\Delta P_1}{P} = \dfrac{PR}{U^2_1}.$$
		
		Hiệu suất hao phí khi điện áp đưa lên đường dây là $\SI{500}{kV}$:
		
		$$h_2 =\dfrac{\Delta P_2}{P} = \dfrac{PR}{U^2_2}.$$
		
		Suy ra:
		
		$$h_2 = h_1 \dfrac{U^2_1}{U^2_2} = \text{4,8}\%.$$
		
		\textbf{Đáp án: D.}
	}
	
	\viduii{3}{ Điện năng được truyền từ một trạm phát điện đến nơi tiêu thụ bằng đường dây tải điện một pha. Biết đoạn mạch tại nơi tiêu thụ (cuối đường dây tải điện) tiêu thụ điện với công suất không đổi và có hệ số công suất luôn bằng $0,8$. Để tăng hiệu suất của quá trình truyền tải từ $80\%$ lên $90\%$ thì cần tăng điện áp hiệu dụng ở trạm phát điện lên
		\begin{mcq}(4)
			\item $1,46$ lần.
			\item $1,38$ lần.
			\item $1,33$ lần.
			\item $1,41$ lần.
		\end{mcq}
	}
	{\begin{center}
			\textbf{Hướng dẫn giải}
		\end{center}
		
		Chứng minh công thức:
		
		$$\dfrac{{{U}_{t}}}{\sin \varphi }=\dfrac{{{U}_{p}}}{\sin \left( \pi -{{\varphi }_{t}} \right)}\Rightarrow {{U}_{t}}\sin {{\varphi }_{t}}={{U}_{p}}\sin \varphi$$
		
		$$\Rightarrow {{U}_{t}}I\cos {{\varphi }_{t}}\tan {{\varphi }_{t}}={{U}_{p}}I\cos \varphi \tan \varphi$$
		
		$$\Rightarrow {{P}_{t}}\tan {{\varphi }_{t}}={{P}_{p}}\tan \varphi \Rightarrow \dfrac{{{P}_{t}}}{P}\tan {{\varphi }_{t}}=\tan \varphi\Rightarrow H\tan {{\varphi }_{t}}=\tan \varphi$$
		
		Mà:
		
		$${{\cos }^{2}}\varphi =\dfrac{1}{1+{{\tan }^{2}}\varphi }=\dfrac{1}{1+{{H}^{2}}{{\tan }^{2}}{{\varphi }_{t}}}$$
		
		$$\Rightarrow H=1-\dfrac{R{{P}_{t}}}{H{{U}^{2}}{{\cos }^{2}}\varphi }\Rightarrow 1-H=\dfrac{R{{P}_{t}}}{H{{U}^{2}}{{\cos }^{2}}\varphi }$$
		
		Lập tỉ số ta suy ra:
		
		$$\dfrac{U_{2}^{2}}{U_{1}^{2}}=\dfrac{(1-{{H}_{1}}){{H}_{1}}}{(1-{{H}_{2}}){{H}_{2}}}.\dfrac{{{\cos }^{2}}{{\varphi }_{1}}}{{{\cos }^{2}}{{\varphi }_{2}}}$$
		
		$$\Rightarrow \dfrac{U_{2}^{2}}{U_{1}^{2}}=\dfrac{(1-{{H}_{1}}){{H}_{1}}}{(1-{{H}_{2}}){{H}_{2}}}.\dfrac{1+H_{2}^{2}{{\tan }^{2}}{{\varphi }_{t}}}{1+H_{1}^{2}{{\tan }^{2}}{{\varphi }_{t}}}$$
		
		Áp dụng công thức ta suy ra:
		
		$$\dfrac{{{U}_{2}}}{{{U}_{1}}}\approx 1,38.$$
		
		\textbf{Đáp án: B.}
	}
	\viduii{3}{Điện năng được truyền từ một trạm phát điện có điện áp $\SI{10}{kV}$ đến nơi tiêu thụ bằng đường dây tải điện một pha. Biết công suất truyền đi là $\SI{500}{kW}$, tổng điện trở đường dây tải điện là $\SI{20}{\ohm}$ và hệ số công suất của mạch điện bằng 1. Hiệu suất của quá trình truyền tải này bằng
		\begin{mcq}(4)
			\item $70\%$.
			\item $75\%$.
			\item $80\%$.
			\item $90\%$.
		\end{mcq}
	}
	{\begin{center}
			\textbf{Hướng dẫn giải}
		\end{center}
		Công suất hao phí:
		
		$$\Delta P=I^2 R= \left( \dfrac {P}{U\cos \varphi}\right)^2 R= \dfrac {P^2 R}{U^2 \cos^2 \varphi}=\SI{50000}{W}.$$
		
		Hiệu suất của quá trình truyền tải:
		
		$$H=1-\dfrac {\Delta P}{P} = 1-\dfrac {\SI{50}{kW}}{\SI{500}{kW}}= 0,9.$$
		
		\textbf{Đáp án: D.}
	}
	
\end{dang}