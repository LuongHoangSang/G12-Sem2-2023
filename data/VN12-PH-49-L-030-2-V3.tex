
\chapter[Phản ứng nhiệt hạch (đọc thêm)]{Phản ứng nhiệt hạch (đọc thêm)}
\section{Lý thuyết}
\subsection{Cơ chế của phản ứng nhiệt hạch}
\subsubsection{Khái niệm}
Phản ứng nhiệt hạch là phản ứng trong đó hai hay nhiều hạt nhân nhẹ tổng hợp lại thành một hạt nhân nặng hơn.
\subsubsection{Điều kiện thực hiện}
Điều kiện thực hiện phản ứng nhiệt hạch là
\begin{itemize}
	\item nhiệt độ cao ($50\div100$ triệu độ);
	\item mật độ hạt nhân trong plasma ($n$) phải đủ lớn;
	\item thời gian duy trì trạng thái plasma ($\tau$) ở nhiệt độ cao (100 triệu độ) phải đủ lớn.
\end{itemize}
\subsection{Năng lượng nhiệt hạch}
Năng lượng toả ra bởi các phản ứng nhiệt hạch được gọi là năng lượng
nhiệt hạch.
\subsection{Phản ứng nhiệt hạch trong vũ trụ}
Phản ứng nhiệt hạch trong lòng Mặt Trời và các ngôi sao là nguồn gốc năng lượng của chúng.
\subsection{Phản ứng nhiệt hạch trên Trái Đất}
Trên Trái Đất, con người chỉ mới thực hiện được phản ứng nhiệt hạch dưới dạng không kiểm soát. Đó là sự nổ của bom nhiệt hạch (còn gọi là bom hiđrô hay bom khinh khí).