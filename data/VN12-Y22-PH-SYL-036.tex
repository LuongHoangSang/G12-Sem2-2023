
\chapter[Lý thuyết: Nguyên tắc tạo ra dòng điện xoay chiều;\\Bài tập: Biểu thức dòng điện, điện áp và suất điện động xoay chiều]{Lý thuyết: Nguyên tắc tạo ra dòng điện xoay chiều;\\Bài tập: Biểu thức dòng điện, điện áp và suất điện động xoay chiều}
\section{Lý thuyết}
\subsection{Hiện tượng cảm ứng điện từ và suất điện động cảm ứng}
\subsubsection{Hiện tượng cảm ứng điện từ}
\begin{itemize}
	\item Mỗi khi từ thông qua mạch kín (C) biến thiên thì trong mạch kín (C) xuất hiện một dòng điện gọi là dòng điện cảm ứng. Hiện tượng xuất hiện dòng điện cảm ứng trong (C) gọi là hiện tượng cảm ứng điện từ.
	\item Hiện tượng cảm ứng điện từ chỉ tồn tại trong khoảng thời gian từ thông qua mạch kín biến thiên.
\end{itemize}
\subsubsection{Định luật Lenz về chiều dòng điện cảm ứng}
Dòng điện cảm ứng xuất hiện trong mạch kín có chiều sao cho từ trường cảm ứng có tác dụng chống lại sự biến thiên của từ thông ban đầu qua mạch kín.
\subsubsection{Định luật Faraday về suất điện động cảm ứng}
Độ lớn của suất điện động cảm ứng xuất hiện trong mạch kín tỉ lệ với tốc độ biến thiên từ thông qua mạch kín đó.
\begin{equation*}
	|\calE_\text c| = \left|-\dfrac{\Delta \Phi}{\Delta t}\right|,
\end{equation*}
trong đó:
\begin{itemize}
	\item $\Delta \Phi$ là độ biến thiên từ thông qua một mạch kín (C) đặt trong từ trường;
	\item $\Delta t$ là khoảng thời gian diễn ra sự biến thiên từ thông đó.
\end{itemize}
\subsection{Nguyên tắc tạo ra dòng điện xoay chiều}
\subsubsection{Khái niệm về dòng điện xoay chiều}
Dòng điện xoay chiều hình $\sin$, gọi tắt là dòng điện xoay chiều, là dòng điện có cường độ biến thiên tuần hoàn với thời gian theo quy luật của hàm số $\sin$ hay $\cos$.
\luuy{
	\begin{center}\textbf{Phân biệt dòng điện xoay chiều, dòng điện một chiều, dòng điện không đổi}\end{center}
	\begin{itemize}
		\item Dòng điện xoay chiều là dòng điện có chiều và cường độ biến đổi theo thời gian;
		\item Dòng điện một chiều là dòng điện có chiều không đổi theo thời gian;
		\item Dòng điện không đổi là dòng điện có chiều và cường độ không đổi theo thời gian.
	\end{itemize}
}
\subsubsection{Nguyên tắc tạo ra dòng điện xoay chiều}
Cho một cuộn dây kín có diện tích $S$ quay đều với tốc độ góc $\omega$ quanh một trục cố định $\Delta$ đặt trong một từ trường đều $\vec B$ có phương vuông góc với trục quay. Khi đó trong cuộn dây sẽ xuất hiện dòng điện xoay chiều.

Biểu thức từ thông qua cuộn dây:
\begin{equation*}
	\Phi =\Phi_0 \cos \omega t= NBS \cos \omega t,
\end{equation*}
trong đó,
\begin{itemize}
	\item $\Phi_0$ là từ thông cực đại qua cuộn dây;
	\item $N$ là số vòng dây;
	\item $B$ là độ lớn cảm ứng từ;
	\item $S$ là tiết diện;
	\item $\omega$ là tốc độ góc;
	\item $t$ là thời điểm.
\end{itemize}
Suất điện động cảm ứng xuất hiện trong cuộn dây:
\begin{equation*}
	\calE_\text c=-\dfrac{d \Phi}{d t} = \calE_0 \sin \omega t =NBS\omega \sin \omega t,
\end{equation*}
trong đó $\calE_0$ là suất điện động cực đại xuất hiện trong cuộn dây.

\luuy{\begin{center}\textbf{Quy tắc đổi đơn vị của tần số góc $\omega$}\end{center}
	Các đơn vị thường gặp của $\omega$ là: vòng/ phút, $\SI{}{\radian / \second}$. Quy tắc đổi đơn vị như sau:
	\begin{equation*}
		x\ \text{vòng / phút} \rightarrow \dfrac {x \cdot 2\pi}{60}\ \text{rad/s}.
\end{equation*}}
\subsection{Các đại lượng cơ bản trong biểu thức}
\subsubsection{Các đại lượng cơ bản trong biểu thức}
\begin{tabular}{|>{\centering\arraybackslash}m{6em}|>{\centering\arraybackslash}m{10em}|>{\centering\arraybackslash}m{10em}|>{\centering\arraybackslash}m{10em}|}
	\hline
	\textbf{Đại lượng} & \textbf{Suất điện động} & \textbf{Cường độ dòng điện} & \textbf{Điện áp} \\
	\hline
	Giá trị tức thời& $\calE=\calE_0\sin (\omega t + \varphi_\calE)$ & $i=I_0\cos (\omega t + \varphi_i)$& $	u=U_0\cos (\omega t + \varphi_u)$ \\
	\hline
	Giá trị cực đại & $\calE_0=\omega NBS$ & $I_0=\omega NBS/R $ & $U_0$\\
	\hline 
	Tần số góc & \multicolumn{3}{c|}{$\omega=2\pi f = 2\pi / T$}\\
	\hline
	Pha ban đầu & $\varphi_\calE$ & $\varphi_i$ & $\varphi_u$ \\
	\hline
	Pha & $\omega t + \varphi_\calE$ & $\omega t + \varphi_i$ & $\omega t + \varphi_u$ \\
	\hline
\end{tabular}
\subsubsection{Các giá trị hiệu dụng}
Giá trị hiệu dụng bằng giá trị cực đại chia cho $\sqrt 2$:
\begin{align*}
	I&=\dfrac{I_0}{\sqrt 2};\\
	U&=\dfrac{U_0}{\sqrt 2}.
\end{align*}
\luuy{Các thiết bị đo (vôn kế, ampe kế) cho biết các giá trị hiệu dụng.}

\subsection{Các dạng toán liên quan đến bóng đèn, điện trở}
\subsubsection{Các thông số ghi trên bóng đèn}
Trên các bóng đèn có ghi hai thông số là $U_\text{đm}$ và $\calP_\text{đm}$ , trong đó:
\begin {itemize}
\item $U_\text{đm}$ là điện áp định mức của bóng đèn hay điện áp để đèn sáng bình thường;
\item $\calP_\text{đm}$ là công suất tiêu thụ định mức của bóng đèn.
\end{itemize}
\subsubsection{Công suất định mức, công suất tỏa nhiệt}
Khi đèn hoạt động ở điện áp định mức thì nó tiêu thụ một công suất đúng bằng công suất định mức.
Công suất tỏa nhiệt trên điện trở là
\begin{equation*}
\calP = I^2 R=U \cdot I= \dfrac{U^2}{R}.
\end{equation*}
\subsubsection{Điện năng tiêu thụ, nhiệt lượng tỏa ra}
Khi đèn hoạt động với công suất không đổi $\calP$ trong khoảng thời gian $t$ thì điện năng mà đèn tiêu thụ là
\begin{equation*}
A = UIt = \calP t.
\end{equation*}

Nhiệt lượng tỏa ra trên điện trở là
\begin{equation*}
Q=I^2 Rt.
\end{equation*}

\section{Mục tiêu bài học - Ví dụ minh họa}
\begin{dang}{Ghi nhớ nguyên tắc tạo ra\\ dòng điện xoay chiều}
	
	\viduii{2}{Một khung dây dẫn có diện tích $S=\SI{50}{\centi \meter \squared}$ gồm 150 vòng dây quay đều với vận tốc $n\ \text{vòng/ phút}$ trong một từ trường đều $\vec B$ vuông góc với trục quay $\Delta$ và có độ lớn $B=\SI{0.02}{\tesla}$. Từ thông cực đại gửi qua khung là
		\begin{mcq}(4)
			\item $\SI{0.015}{\weber}$.
			\item $\SI{0.15}{\weber}$.
			\item $\SI{1.5}{\weber}$.
			\item $\SI{15}{\weber}$.
		\end{mcq}
	}
	{\begin{center}
			\textbf{Hướng dẫn giải}
		\end{center}
		
		Đổi $\SI{50}{\centi \meter \squared} = \SI{50e-4}{\meter \squared}$.
		
		Từ thông cực đại gửi qua khung là
		$$\Phi_0=NBS = \SI{0.015}{\weber}.$$
		
		
		
		\textbf{Đáp án: A.}
	}
	\viduii{3}	{Một khung dây dẫn quay đều quanh trục $\Delta$ với tốc độ $150\ \text{vòng/ phút}$ trong từ trường đều có cảm ứng từ $\vec{B}$ vuông góc với trục quay của khung. Từ thông cực đại gửi qua khung dây là $10/ \pi \ \text{Wb}$. Suất điện động hiệu dụng trong khung dây bằng
		\begin{mcq}(4)
			\item $25\ \text V$.
			\item $25\sqrt 2 \text V$.
			\item $50\ \text V$.
			\item $50\sqrt 2 \ \text V$.
		\end{mcq}
	}
	{\begin{center}
			\textbf{Hướng dẫn giải}
		\end{center}
		
		Đổi $\omega = 150\ \text{vòng / phút} \rightarrow\ \omega= \xsi{5\pi}{\radian/ \second}$.
		
		Biểu thức liên hệ giữa từ thông cực đại và suất điện động cực đại:
		$$\calE_0 = \omega \Phi_0.$$
		
		
		Suất điện động hiệu dụng trong khung là
		$$\calE=\dfrac{\calE_0}{\sqrt 2} = \dfrac{\omega \Phi_0}{\sqrt 2} = 25\sqrt 2\ \text V.$$
		
		\textbf{Đáp án: B.}
	}
	
	
\end{dang}
\begin{dang}{Sử dụng được phương trình cường độ dòng điện, điện áp, suất điện động\\ để xác định các giá trị liên quan}
	
	
	\viduii{3}{Điện áp tức thời giữa hai đầu của một đoạn mạch xoay chiều là $u=80\cos 100 \pi t \ \text{(V)}$. Điện áp hiệu dụng giữa hai đầu đoạn mạch đó là bao nhiêu?
		\begin{mcq}(4)
			\item $\SI{80}{\volt}$.
			\item $\SI{40}{\volt}$.
			\item $80\sqrt 2\ \text V$.
			\item $40\sqrt 2\ \text V$.
		\end{mcq}
	}
	{\begin{center}
			\textbf{Hướng dẫn giải}
		\end{center}
		
		Điện áp hiệu dụng giữa hai đầu đoạn mạch là
		$U = \dfrac{U_0}{\sqrt 2}=\dfrac{80}{\sqrt 2}\ \text V = 40\sqrt 2\ \text V.$
		
		\textbf{Đáp án: D.}
	}
	
	\viduii{3}{Một đèn điện có ghi $110\ \text V - 100\ \text W$ mắc nối tiếp với điện trở $R$ vào một mạch điện xoay chiều có $u=220 \sqrt 2 \sin 100 \omega t\ \text{(V)}$. Để đèn sáng bình thường, $R$ phải có giá trị là bao nhiêu?
		\begin{mcq}(4)
			\item $\SI{1210}{\Omega}$.
			\item $\xsi{10/11}{\Omega}$.
			\item $\SI{121}{\Omega}$.
			\item $\SI{110}{\Omega}$.
		\end{mcq}
	}
	{\begin{center}
			\textbf{Hướng dẫn giải}
		\end{center}
		
		Điện áp định mức của đèn là $U=110\ \text V$.
		
		Công suất định mức của đèn là $\calP = 100\ \text W$.
		
		Điện trở của đèn là
		$$R_\text{đ}=\dfrac{U_\text{đm}^2}{\calP_\text{đm}} = 121\ \Omega.$$
		
		
		Cường độ dòng điện hiệu dụng chạy qua mạch:
		$$I=\dfrac{U}{R_\text{đ} + R} = \dfrac{220}{121+R}.$$
		
		
		Do mạch mắc nối tiếp nên cường độ dòng điện hiệu dụng chạy qua đèn cũng bằng $I$, suy ra điện áp giữa hai đầu bóng đèn là
		$$U_\text{đ} = IR_\text{đ} = \dfrac{220\cdot121}{121+R}.$$
		
		
		Để đèn sáng bình thường thì $U_\text{đ}=U_\text{đm}$, suy ra
		$$	\dfrac{220\cdot121}{121+R} = 110 \Rightarrow R = 121\ \Omega.$$
		
		\textbf{Đáp án: C.}
	}
	
\end{dang}
\begin{dang}{ Xây dựng được phương trình cường độ dòng điện, điện áp, suất điện động}
	
	\viduii{3}{Cường độ dòng điện trong đoạn mạch có biểu thức $i=2\cos 100\pi t\ \text{A}$, điện áp giữa hai đầu đoạn mạch có giá trị hiệu dụng là $12\text{V}$ và sớm pha $\dfrac{\pi}{3}$  so với dòng điện. Biểu thức của điện áp giữa hai đầu đoạn mạch là
		\begin{mcq}(2)
			\item $u=12\cos 100\pi t\ \text{V}$.
			\item $u=12\sqrt{2}\cos 100\pi t\ \text{V}$.
			\item $u=12\sqrt{2}\cos \left( 100\pi t-\dfrac{\pi}{3}\right) \ \text{V}$.
			\item $u=12\cos \left( 100\pi t+\dfrac{\pi}{3}\right) \ \text{V}$.
		\end{mcq}
	}
	{\begin{center}
			\textbf{Hướng dẫn giải}
		\end{center}
		
		Điện áp hiệu dụng $U = 12\ \text{V}$ nên điện áp cực đại $U_0=12\sqrt{2}$.
		
		Tần số của điện áp bằng tần số của dòng điện  $\omega=100\pi\ \text{rad/s}$.
		
		Điện áp sớm pha $\dfrac{\pi}{3}$  so với dòng điện, suy ra $\varphi_u=\varphi_i+\dfrac{\pi}{3}=\dfrac{\pi}{3}$.
		
		Vậy $u=12\cos\left(  100\pi t+\dfrac{\pi}{3}\right) \ \text{V}$.
		
		
		\textbf{Đáp án: D.}
	}
	\viduii{3}{ Một mạch điện xoay chiều có điện áp giữa hai đầu mạch là $u=200\cos 100\pi t+\dfrac{\pi}{6}\ \text{V}$. Cường độ hiệu dụng của dòng điện chạy trong mạch là $2\sqrt{2}$. Biết rằng, dòng điện nhanh pha hơn điện áp hai đầu mạch góc $\dfrac{\pi}{3}$, biểu thức của cường độ điện trong mạch là
		
		
		\begin{mcq}(2)
			\item  $i=4\cos \left( 100\pi t+\dfrac{\pi}{3}\right) \ \text{A}$.
			\item  $i=4\cos \left( 100\pi t+\dfrac{\pi}{2}\right) \ \text{A}$.
			\item  $i=2\sqrt{2}\cos \left( 100\pi t-\dfrac{\pi}{6}\right) \ \text{A}$.
			\item  $i=2\sqrt{2}\cos\left(  100\pi t-\dfrac{\pi}{2}\right) \ \text{A}$.
		\end{mcq}
	}
	{\begin{center}
			\textbf{Hướng dẫn giải}
		\end{center}
		
		
		Ta có:  $I=2\sqrt{2}\Rightarrow I_0=4\ \text{A}$.
		
		Dòng điện nhanh pha hơn điện áp hai đầu mạch góc $\dfrac{\pi}{3}\ \text{rad}$ suy ra $\varphi_i=\varphi_u+\dfrac{\pi}{3}=\dfrac{\pi}{2}$.
		
		Biểu thức cường độ dòng điện là  $i=4\cos \left( 100\pi t+\dfrac{\pi}{2}\right) \ \text{A}$.
		
		\textbf{	Đáp án: B.}
		
		
	}
\end{dang}



