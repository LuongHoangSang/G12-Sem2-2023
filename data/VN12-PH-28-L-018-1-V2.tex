% --- chapter
\newcommand{\chapter}[2][]{
	\newcommand{\chapname}{#2}
	\begin{flushleft}
		\begin{minipage}[t]{\linewidth}
			\includegraphics[height=1cm]{hdht-logo.png}
			\hspace{0pt}	
			\sffamily\bfseries\large Bài  21. Điện từ trường
			\begin{flushleft}
				\huge\bfseries #1
			\end{flushleft}
		\end{minipage}
	\end{flushleft}
	\vspace{1cm}
	\normalfont\normalsize
}
%-----------------------------------------------------
\chapter[Điện từ trường]{Điện từ trường}

\subsection{Mối quan hệ giữa điện trường và từ trường}
\textbf{Từ trường biến thiên và điện trường xoáy}
 Điện trường xoáy là điện trường có đường sức là đường cong kín.
 Nếu tại một nơi có từ trường biến thiên theo thời gian thì tại nơi đó xuất hiện điện trường xoáy.

\textbf{Điện trường biến thiên và từ trường}
 Nếu tại một nơi có điện trường biến thiên theo thời gian thì tại nơi đó xuất hiện từ trường.
Đường sức của từ trường luôn khép kín.
\subsection {Điện từ trường }
Điện trường biến thiên và từ trường biến thiên liên quan mật thiết với nhau, là hai thành phần của một trường thống nhất - điện từ trường (trường điện từ).