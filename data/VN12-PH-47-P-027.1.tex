\chapter{Phản ứng hạt nhân. Các định luật bảo toàn}
\section{Dạng 1: Viết phương trình phản ứng hạt nhân}
\begin{enumerate}
	\item {Cho hạt prôtôn bắn vào các hạt nhân $^9_4Be$ đang đứng yên, người ta thấy các hạt tạo thành gồm $^4_2He$ và hạt nhân X. Hạt nhân X có cấu tạo gồm
		\begin{mcq}(2)
			\item 3 prôtôn và 3 nơtrôn.	
			\item 3 prôtôn và 6 nơtrôn.	
			\item 2 prôtôn và 2 nơtrôn.	\item 2 prôtôn và 3 nơtrôn.
		\end{mcq}
	}	
	\item {Cho phản ứng hạt nhân $^{\text{A}}_{\text{Z}} \text{B} \longrightarrow ^{\text{A}}_{\text{Z+1}} \text{B} + \text{X}$, X là
		\begin{mcq} (2)
			\item hạt $\alpha$.	
			\item hạt $\beta^{-}$.	
			\item hạt $\beta^{+}$ .	
			\item hạt phôtôn.
		\end{mcq}
	}
	\item{Urani 238 sau một loạt phóng xạ $\alpha$ và biến thành chì. Phương trình của phản ứng là: $\\^{238}_{92} U \longrightarrow ^{206} _{82} Pb + x ^4_2He + y ^0_{-1} \beta^{-}$ . $y$ có giá trị là
		\begin{mcq}(4)
			\item $y = 4$.	
			\item $y = 5$.	
			\item $y = 6$.	
			\item $y = 8$.
		\end{mcq}
	}
	\item {Trong phản ứng sau đây: $n+ ^{235}_{92} U \longrightarrow ^{95}_{42} Mo + ^{139}_{57} La + 2\text{X} + 7\beta^{-}$. Hạt X là
		\begin{mcq}(4)
			\item Electrôn.	
			\item Prôtôn.	
			\item Hêli.	
			\item Nơtrôn.
		\end{mcq}
	}
\end{enumerate}




\textbf{ĐÁP ÁN}
\begin{longtable}[\textwidth]{|p{0.1\textwidth}|p{0.1\textwidth}|p{0.1\textwidth}|p{0.1\textwidth}|p{0.1\textwidth}|p{0.1\textwidth}|p{0.1\textwidth}|p{0.1\textwidth}|}
	% --- first head
	\hline%\hspace{2 pt}
	\multicolumn{1}{|c|}{\textbf{Câu 1}} & \multicolumn{1}{c|}{\textbf{Câu 2}} & \multicolumn{1}{c|}{\textbf{Câu 3}} &
	\multicolumn{1}{c|}{\textbf{Câu 4}} &
	\multicolumn{1}{c|}{\textbf{Câu 5}} &
	\multicolumn{1}{c|}{\textbf{Câu 6}} &
	\multicolumn{1}{c|}{\textbf{Câu 7}} &
	\multicolumn{1}{c|}{\textbf{Câu 8}} \\
	\hline
	A.&B. &C. &D. & & & &\\
	\hline
	
	
\end{longtable} 
\section{Dạng 2: Năng lượng của phản ứng hạt nhân}
\begin{enumerate}
	\item {Cho phản ứng hạt nhân $^9_4 Be + \alpha \longrightarrow ^{12}_6 C + n$, trong đó khối lượng các hạt tham gia và tạo thành trong phản ứng là $m_{\alpha} = \text{4,0015}\ \text{u}$; $m_{Be} = \text{9,0122}\ \text{u}$; $m_C =\text{12,0000}\ \text{u}$; $m_n = \text{1,0087}\ \text{u}$ và $1\ \text{u} =\text{931,5}\ \text{MeV/c}^2$. Phản ứng hạt nhân này
		\begin{mcq}(2)
			\item thu vào 4,66 MeV.	
			\item tỏa ra 4,66 MeV.	
			\item thu vào 6,46 MeV.	
			\item tỏa ra 6,46 MeV.
		\end{mcq}
	}
	\item{Cho phản ứng hạt nhân $^{27}_{13} Al +\alpha \longrightarrow ^{30}_{15}P +n$, trong đó khối lượng các hạt tham gia và tạo thành trong phản ứng là $m_{\alpha} =\text{4,0016}\ \text{u}$; $m_{Al} =\text{26,9743}\ \text{u}$; $m_P=\text{29,9701}\ \text{u}$; $m_n=\text{1,0087}\ \text{u}$ và $1\ \text{u} =\text{931,5}\ \text{MeV/c}^2$. Phản ứng hạt nhân này
		\begin{mcq}(2)
			\item thu vào 2,7 MeV.	
			\item tỏa ra 2,7 MeV.	
			\item thu vào 4,3 MeV.	
			\item tỏa ra 4,3 MeV.
		\end{mcq}
	}
	\item {[Trích đề thi THPT QG năm 2012] Tổng hợp hạt nhân heli $^4_2 He$ từ phản ứng hạt nhân $^1_1 H + ^7_3Li \longrightarrow ^4_2 He + \text{X}$. Mỗi phản ứng trên tỏa năng lượng 17,3 MeV. Năng lượng tỏa ra khi tổng hợp được 0,5 mol heli là
		\begin{mcq}(4)
			\item 2,6$\cdot 10^{24}$ MeV.	
			\item 2,4$\cdot 10^{24}$ MeV.	
			\item 5,2$\cdot 10^{24}$ MeV.	
			\item 1,3$\cdot 10^{24}$ MeV.
		\end{mcq}
	}
	\item{[Trích đề thi THPT QG năm 2009] Cho phản ứng hạt nhân: $^3_1T+^2_1D \longrightarrow ^4_2He + \text{X}$. Lấy độ hụt khối của hạt nhân $T$, hạt nhân $D$, hạt nhân He lần lượt là 0,009106 u; 0,002491 u; 0,030382 u và $1\ \text{u} =\text{931,5}\ \text{MeV/c}^2$. Năng lượng tỏa ra của phản ứng xấp xỉ bằng
		\begin{mcq}(4)
			\item 21,076 MeV.	
			\item 200,025 MeV.	
			\item 17,498 MeV.	
			\item 15,017 MeV.
		\end{mcq}
	}
	\item {Cho phản ứng hạt nhân $^{235}_{92} U + n \longrightarrow ^{94}_{38} Sr + ^{140}_{54} Xe + 2n$. Biết năng lượng liên kết riêng của các hạt nhân trong phản ứng: $U$ bằng 7,59 MeV; $Sr$ bằng 8,59 MeV và $Xe$ bằng 8,29 MeV. Năng lượng tỏa ra của phản ứng là
		\begin{mcq}(4)
			\item 148,4 MeV.	
			\item 144,8 MeV.	
			\item 418,4 MeV.	
			\item 184,4 MeV.
		\end{mcq}
	}
\end{enumerate}
\textbf{ĐÁP ÁN}
\begin{longtable}[\textwidth]{|p{0.1\textwidth}|p{0.1\textwidth}|p{0.1\textwidth}|p{0.1\textwidth}|p{0.1\textwidth}|p{0.1\textwidth}|p{0.1\textwidth}|p{0.1\textwidth}|}
	% --- first head
	\hline%\hspace{2 pt}
	\multicolumn{1}{|c|}{\textbf{Câu 1}} & \multicolumn{1}{c|}{\textbf{Câu 2}} & \multicolumn{1}{c|}{\textbf{Câu 3}} &
	\multicolumn{1}{c|}{\textbf{Câu 4}} &
	\multicolumn{1}{c|}{\textbf{Câu 5}} &
	\multicolumn{1}{c|}{\textbf{Câu 6}} &
	\multicolumn{1}{c|}{\textbf{Câu 7}} &
	\multicolumn{1}{c|}{\textbf{Câu 8}} \\
	\hline
	B.&A. &A. &C. &D. & & &\\
	\hline
\end{longtable} 

\section{Dạng 3: Bài toán liên quan đến bảo toàn động lượng và bảo toàn năng lượng}
\begin{description}
	\item[Bước 1] Viết phương trình định luật bảo toàn vectơ động lượng:
	\begin{equation*}
		\vec{p_{\text{A}}}+ \vec{p_{\text{B}}} = \vec{p_{\text{X}}} + \vec{p_{\text{Y}}} \Rightarrow m_{\text{A}} \vec{v_{\text{A}}} +m_{\text{B}} \vec{v_{\text{B}}} = m_{\text{X}} \vec{v_{\text{X}}} + m_{\text{Y}} \vec{v_{\text{Y}}}.
	\end{equation*}
	
	Biểu diễn các vec-tơ bằng sơ đồ hình học, từ đó rút ra phương trình độ lớn của các vec-tơ động lượng ta được phương trình (1).
	\item [Bước 2] Viết phương trình định luật bảo toàn năng lượng toàn phần:
	\begin{equation*}
		K_{\text{t}}+ (m_{\text{A}}+ m_{\text{B}})c^2 = K_{\text{s}} + (m_{\text{X}} + m_{\text{Y}})c^2.
	\end{equation*}
	Sử dụng mối liên hệ giữa $P$ và $K$ là $P=\sqrt{2mK}$ hoặc $K=\dfrac{P^2}{2m}$, ta được phương trình (2).
	\item [Bước 3] Giả hệ phương trình (1) và (2).
\end{description}
\begin{enumerate}
	\item {Cho phản ứng hạt nhân sau $^2_1D+^2_1D \longrightarrow ^3_2He + n + \text{3,25}\ \text{MeV}$. Biết độ hụt khối của $^2_1H$ là $\Delta m_D = \text{0,0024}\ \text{u}$; và $1\ \text{u} = 931\ \text{MeV/c}^2$. Năng lượng liên kết của hạt nhân $^3_2He$ là
		\begin{mcq}(4)
			\item 7,7188 MeV.	
			\item 77,188 MeV.	
			\item 771,88 MeV.	
			\item 7,7188 eV.
		\end{mcq}
	}
	\item {Hạt $^{234}_{92}U$ đang đứng yên thì bị vỡ thành hạt $\alpha$ và hạt $^{230}_{90}Th$. Cho $m_{\alpha}=\text{4,0015}\ \text{u}$; $m_{Th}=\text{229,9737}\ \text{u}$ và $1\ \text{u} = \text{931,5}\ \text{MeV/c}^2$. Phản ứng không bức xạ sóng gamma. Động năng của hạt $\alpha$ sinh ra bằng 4,0 MeV. Khối lượng hạt nhân $^{234}_{92} U$ bằng
		\begin{mcq}(4)
			\item 233,9796 u.	
			\item 234,0032 u.	
			\item 233,6796 u.	
			\item 233,7965 u.
		\end{mcq}
	}
	\item {Một hạt $\alpha$ bắn vào hạt nhân $^{27}_{13}Al$ đứng yên tạo ra hạt nơtron và hạt X. Cho $m_{\alpha}= \text{4,0016}\ \text{u}$; $m_n = \text{1,00866}\ \text{u}$; $m_{Al} = \text{26,9744}\ \text{u}$; $m_{\text{X}} =\text{29,9701}\ \text{u}$; $1\ \text{u} = \text{931,5}\ \text{MeV/c}^2$. Các hạt nơtron và X có động năng là 4 MeV và 1,8 MeV. Động năng của hạt $\alpha$ là
		\begin{mcq}(4)
			\item 5,87 MeV.	
			\item 8,37 MeV.	
			\item 7,87 MeV.	
			\item 7,27 MeV.
		\end{mcq}
		
	}
	\item {[Trích đề thi THPT QG năm 2011] Bắn một prôtôn vào hạt nhân $^7_3Li$ đứng yên. Phản ứng tạo ra hai hạt nhân X giống nhau bay ra với cùng tốc độ theo các phương hợp với phương tới của prôtôn các góc bằng nhau là $60^\circ$. Lấy khối lượng của mỗi hạt nhân tính theo đơn vị u bằng số khối của nó. Tỉ số giữa tốc độ của prôtôn và tốc độ của hạt nhân X là
		\begin{mcq}(4)
			\item $4$.	
			\item $\dfrac{1}{4}$.	
			\item $2$.	
			\item $\dfrac{1}{2}$.
		\end{mcq}
	}
	\item {Bắn hạt nhân $\alpha$ có động năng 18 MeV vào hạt nhân $^{14}_7 N$ đứng yên ta có phản ứng $\\ \alpha + ^{14}_7 N \longrightarrow ^1_1p + ^{17}_8 O$. Biết các hạt nhân sinh ra cùng véctơ vận tốc. Cho $m_{\alpha} = \text{4,0015}\ \text{u}$; $m_p=\text{1,0073}\ \text{u}$; $m_{N}=\text{13,9992}\ \text{u}$; $m_{O}=\text{16,9947}\ \text{u}$; và $1\ \text{u} = \text{931,5}\ \text{MeV/c}^2$. Động năng của hạt prôtôn sinh ra có giá trị bằng
		\begin{mcq}(4)
			\item 0,111 MeV.	
			\item 0,555 MeV.	
			\item 0,333 MeV.	
			\item 0,938 MeV.
		\end{mcq} 
	}
	\item {[Trích đề thi THPT QG năm 2013] Dùng một hạt $\alpha$ có động năng 7,7 MeV bắn vào hạt nhân $^{14}_7N$ đang đứng yên gây ra phản ứng $\alpha + ^{14}_7 N \longrightarrow ^1_1p + ^{17}_8 O$. Hạt prôtôn bay ra theo phương vuông góc với phương bay tới của hạt $\alpha$. Cho khối lượng các hạt nhân: $m_{\alpha} = \text{4,0015}\ \text{u}$; $m_p=\text{1,0073}\ \text{u}$; $m_{N14}=\text{13,9992}\ \text{u}$; $m_{O17}=\text{16,9947}\ \text{u}$; và $1\ \text{u} = \text{931,5}\ \text{MeV/c}^2$. Động năng của hạt nhân là
		\begin{mcq}(4)
			\item 6,145 MeV.	
			\item 2,214 MeV.	
			\item 1,345 MeV.	
			\item 2,075 Mev.
		\end{mcq}
	}
\end{enumerate}
\begin{longtable}[\textwidth]{|p{0.1\textwidth}|p{0.1\textwidth}|p{0.1\textwidth}|p{0.1\textwidth}|p{0.1\textwidth}|p{0.1\textwidth}|p{0.1\textwidth}|p{0.1\textwidth}|}
	% --- first head
	\hline%\hspace{2 pt}
	\multicolumn{1}{|c|}{\textbf{Câu 1}} & \multicolumn{1}{c|}{\textbf{Câu 2}} & \multicolumn{1}{c|}{\textbf{Câu 3}} &
	\multicolumn{1}{c|}{\textbf{Câu 4}} &
	\multicolumn{1}{c|}{\textbf{Câu 5}} &
	\multicolumn{1}{c|}{\textbf{Câu 6}} &
	\multicolumn{1}{c|}{\textbf{Câu 7}} &
	\multicolumn{1}{c|}{\textbf{Câu 8}} \\
	\hline	A.&A. &B. &A. &D. &D. & &	\\
	\hline
\end{longtable} 
