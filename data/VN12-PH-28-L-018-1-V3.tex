
\chapter[Điện từ trường]{Điện từ trường}
\section{Lý thuyết}
\subsection{Mối quan hệ giữa điện trường và từ trường}
\textbf{Từ trường biến thiên và điện trường xoáy}
Điện trường xoáy là điện trường có đường sức là đường cong kín.
Nếu tại một nơi có từ trường biến thiên theo thời gian thì tại nơi đó xuất hiện điện trường xoáy.

\textbf{Điện trường biến thiên và từ trường}
Nếu tại một nơi có điện trường biến thiên theo thời gian thì tại nơi đó xuất hiện từ trường.
Đường sức của từ trường luôn khép kín.
\subsection {Điện từ trường }
Điện trường biến thiên và từ trường biến thiên liên quan mật thiết với nhau, là hai thành phần của một trường thống nhất - điện từ trường (trường điện từ).
\section{Bài tập tự luyện}
\begin{enumerate}[label=\bfseries Câu \arabic*:]
	
	%=======================
	\item \mkstar{1} [4]
	\cauhoi
	{Khi nói về điện từ trường, phát biểu nào sau đây là sai?
		\begin{mcq}(1)
			\item Nếu tại một nơi có từ trường biến thiên theo thời gian thì tại đó xuất hiện điện trường xoáy. 
			\item Điện trường và từ trường là hai mặt thể hiện khác nhau của một trường duy nhất gọi là điện từ trường.
			\item Trong quá trình lan truyền điện từ trường, vectơ cường độ điện trường và vectơ cảm úng từ tại mọi điểm luôn vuông góc nhau.
			\item Điện trường không lan truyền được trong điện môi.
		\end{mcq}
	}
	
	\loigiai
	{		\textbf{Đáp án: D.}
		
		Điện trường lan truyền cả trong môi trường điện môi lẫn chân không.
		
	}
	
	%=======================	
	\item \mkstar{1} [10]
	\cauhoi
	{Khi một từ trường biến thiên không đều và không tắt theo thời gian sẽ sinh ra
		\begin{mcq}(2)
			\item điện trường xoáy.      
			\item từ trường đều. 
			\item dòng điện không đổi. 
			\item điện trường đều. 
		\end{mcq}
	}
	
	\loigiai
	{		\textbf{Đáp án: A.}
		
		Từ trường biến thiên sinh ra điện trường xoáy.
		
	}
	
\end{enumerate}

