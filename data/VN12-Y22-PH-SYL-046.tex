
\chapter[Lý thuyết: Máy biến áp;\\Bài tập: Máy biến áp có số vòng dây không đổi;\\Bài tập: Máy biến áp thay đổi số vòng dây]{Lý thuyết: Máy biến áp;\\Bài tập: Máy biến áp có số vòng dây không đổi;\\Bài tập: Máy biến áp thay đổi số vòng dây}
\section{Lý thuyết}
\subsection{Máy biến áp}
\subsubsection{Cấu tạo và nguyên tắc hoạt động}
\begin{itemize}
	\item Định nghĩa: Máy biến áp là những thiết bị biến đổi điện áp xoay chiều mà không làm thay đổi tần số.
	\item Nguyên tắc hoạt động: dựa trên hiện tượng cảm ứng điện từ.
	\item Cấu tạo: cuộn sơ cấp có $N_1$ vòng dây và cuộn thứ cấp có $N_2$ vòng dây được quấn trên cùng một lõi biến áp (khung sắt non pha silic).	 
\end{itemize}
\subsubsection{Sự biến đổi điện áp và cường độ dòng điện}
\begin{itemize}
	\item Nếu điện trở của các cuộn dây có thể bỏ qua thì điện áp hiệu dụng ở hai đầu mỗi cuộn dây tỉ lệ với số vòng dây
	\begin{equation*}
		\dfrac{U_2}{U_1}=\dfrac{N_2}{N_1},
	\end{equation*}
	nếu: $\dfrac{N_2}{N_1}>1:$ máy tăng áp; $\dfrac{N_2}{N_1}<1:$ máy hạ áp.
	\item Nếu điện năng hao phí không đáng kể thì cường độ dòng điện qua mỗi cuộn dây tỉ lệ nghịch với điện áp hiệu dụng ở hai đầu mỗi cuộn dây
	\begin{equation*}
		\dfrac{U_2}{U_1}=\dfrac{I_1}{I_2}.
	\end{equation*}
	\item Cuộn dây nào có dòng điện có cường độ lớn hơn thì tiết diện dây của cuộn đó lớn hơn.
	
\end{itemize}
\subsection{Máy biến áp có số vòng dây không đổi}
\begin{itemize}
	\item Máy biến áp không tải:
	\begin{equation*}
		\dfrac{U_2}{U_1}=\dfrac{N_2}{N_1},
	\end{equation*}
	trong đó:
	
	+ $U_1$ là điện áp hiệu dụng ở hai đầu cuộn sơ cấp (V),
	
	+ $U_2$ là điện áp hiệu dụng ở hai đầu cuộn thứ cấp (V), 
	
	+ $N_1$ là số vòng dây của cuộn sơ cấp,
	
	+ $N_2$ là số vòng dây của cuộn thứ cấp.
	
	\item Máy biến áp có tải:
	
	\begin{equation*}
		\dfrac{U_2}{U_1}=\dfrac{N_2}{N_1}=\dfrac{I_1}{I_2},
	\end{equation*}
	
	trong đó:
	
	+ $I_1$ là cường độ dòng điện hiệu dụng ở cuộn sơ cấp (A), 
	
	+ $I_2$ là cường độ dòng điện hiệu dụng ở cuộn thứ cấp (A).
	
	\item Hiệu suất của máy biến áp:
	\begin{equation*}
		H=\dfrac{U_2 I_2 \cos \varphi_2}{U_1I_1 \cos \varphi_1}, 
	\end{equation*}
	trong đó:
	
	+ $\cos \varphi_1$ là hệ số công suất của mạch sơ cấp,
	
	+ $\cos \varphi_2$ là hệ số công suất của mạch thứ cấp.
	
	\item Máy biến áp lý tưởng $H=1$.
	
\end{itemize}
\subsection{Máy biến áp thay đổi số vòng dây}
\begin{itemize}
	\item Khi máy biến áp có số vòng dây ở cuộn sơ cấp thay đổi:
	\begin{equation*}
		\begin{cases}
			\dfrac{U_1}{U_2}=\dfrac{N_1}{N_2}, \\ 
			\dfrac{U_1}{U'_2}=\dfrac{N_1 \pm n }{N_2}.
		\end{cases}
	\end{equation*}
	
	\item Khi máy biến áp có số vòng dây ở cuộn thứ cấp thay đổi:
	
	\begin{equation*}
		\begin{cases}
			\dfrac{U_2}{U_1}=\dfrac{N_2}{N_1}, \\ 
			\dfrac{U_2}{U'_1}=\dfrac{N_2 \pm n }{N_1}.
		\end{cases}
	\end{equation*}
	với $n$ là số vòng dây. 
\end{itemize}

\section{Mục tiêu bài học - Ví dụ minh họa}
\begin{dang}{Ghi nhớ khái niệm máy tăng áp\\ và máy hạ áp}
	\viduii{1}{Một máy tăng áp có cuộn thứ cấp mắc với điện trở thuần, cuộn sơ cấp mắc với nguồn điện xoay chiều. Tần số dòng điện trong cuộn thứ cấp
		\begin{mcq}
			\item có thể nhỏ hơn hoặc lớn hơn tần số dòng điện trong cuộn sơ cấp. 
			\item bằng tần số dòng điện trong cuộn sơ cấp.
			\item luôn nhỏ hơn tần số dòng điện trong cuộn sơ cấp.
			\item luôn lớn hơn tần số dòng điện trong cuộn sơ cấp.
		\end{mcq}
	}
	{\begin{center}
			\textbf{Hướng dẫn giải}
		\end{center}
		
		Tần số dòng điện trong cuộn thứ cấp bằng tần số dòng điện trong cuộn sơ cấp.
		
		\textbf{Đáp án: B.}
	}
	\viduii{1}{Một máy biến áp có số vòng dây của cuộn sơ cấp lớn hơn số vòng dây của cuộn thứ cấp. Máy biến áp này có tác dụng 
		\begin{mcq}
			\item giảm điện áp mà không thay đổi tần số của dòng điện xoay chiều.
			\item giảm điện áp và giảm tần số của dòng điện xoay chiều. 
			\item tăng điện áp mà không thay đổi tần số của dòng điện xoay chiều. 
			\item tăng điện áp và tăng tần số của dòng điện xoay chiều. 
		\end{mcq}
	}
	{\begin{center}
			\textbf{Hướng dẫn giải}
		\end{center}
		
		Một máy biến áp có số vòng dây của cuộn sơ cấp lớn hơn số vòng dây của cuộn thứ cấp là máy hạ áp. Máy biến áp không thể làm thay đổi tần số dòng điện.
		
		\textbf{Đáp án: A.}
	}
	
\end{dang}
\begin{dang}{Ghi nhớ công thức tính điện áp, cường độ dòng điện, số vòng dây của máy biến áp}
	\viduii{2}{Một máy biến áp lý tưởng có cuộn sơ cấp gồm 1000 vòng, cuộn thứ cấp gồm 50 vòng. Điện áp hiệu dụng giữa hai đầu cuộn sơ cấp 220 V. Bỏ qua hao phí. Điện áp hiệu dụng giữa hai đầu cuộn thứ cấp để hở là
		\begin{mcq}(4)
			\item  440 V.          
			\item  44 V.          
			\item  110 V.          
			\item  11 V.
		\end{mcq}
	}
	{\begin{center}
			\textbf{Hướng dẫn giải}
		\end{center}
		
		\begin{itemize}
			\item Áp dụng công thức mối liên hệ giữa điện áp hiệu dụng và số vòng dây:
			\begin{equation*}
				\dfrac{U_1}{U_2}=\dfrac{N_1}{N_2}
			\end{equation*}
			\item Suy ra điện áp hiệu dụng giữa hai đầu cuộn thứ cấp để hở:
			\begin{equation*}
				U_2=U_1 \dfrac{N_2}{N_1} = 11\ \text{V}.
			\end{equation*}
		\end{itemize}
		
		
		\textbf{Đáp án: D.}
	}
	
	\viduii{2}{Một máy biến áp có số vòng dây trên cuộn sơ cấp và số vòng dây của cuộn thứ cấp là 2000 vòng và 500 vòng. Điện áp hiệu dụng và cường độ hiệu dụng ở mạch thứ cấp là $\SI{50}{V}$ và 6 A. Xác định điện áp hiệu dụng và cường độ hiệu dụng ở mạch sơ cấp
		
		\begin{mcq}(2)
			\item 200 V; 1,5 A.
			\item 100 V; 3 A.
			\item 200 V; 3 A.
			\item 100 V; 1,5 A.
		\end{mcq}
	}
	{\begin{center}
			\textbf{Hướng dẫn giải}
		\end{center}
		
		Áp dụng công thức mối liên hệ giữa hiệu điện thế, cường độ dòng điện và số vòng dây:
			\begin{equation*}
				\dfrac{U_1}{U_2}=\dfrac{N_1}{N_2}=\dfrac{I_2}{I_1}.
			\end{equation*}
			\begin{equation*}
				\Rightarrow U_1=U_2 \dfrac{N_1}{N_2}= 200\ \text{V}.
			\end{equation*}
			\begin{equation*}
				\Rightarrow I_1=I_2 \dfrac{N_2}{N_1}= \text{1,5}\ \text{A}.
			\end{equation*}
		
		\textbf{Đáp án: A.}
	}
	
\end{dang}
\begin{dang}{Sử dụng tỉ số giữa cường độ dòng điện, hiệu điện thế, số vòng dây của cuộn\\ sơ cấp và thứ cấp để xác định các\\ đại lượng cần tìm}
	\viduii{3}{Một máy biến áp lý tưởng có số vòng dây của cuộn sơ cấp nhiều hơn số vòng dây của cuộn thứ cấp là $1200$ vòng, tổng số vòng dây của hai cuộn là $2400$ vòng. Nếu đặt vào hai đầu cuộn sơ cấp một điện áp xoay chiều có giá trị hiệu dụng $120\ \text V$ thì điện áp hiệu dụng ở hai đầu cuộn thứ cấp để hở là
		\begin{mcq}(4)
			\item $\SI{40}{V}$.
			\item $\SI{240}{V}$.
			\item $\SI{60}{V}$.
			\item $\SI{360}{V}$.
		\end{mcq} 
	}
	{	\begin{center}
			\textbf{Hướng dẫn giải}
		\end{center}
		
		Gọi $N_1$, $U_1$ lần lượt là số vòng dây và điện áp hiệu dụng của hai đầu cuộn sơ cấp; $N_2$,$U_2$ lần lượt là số vòng dây và điện áp hiệu dụng của hai đầu cuộn thứ cấp.
		
		Theo bài ra ta có:
		
		$$\left\{\begin{array}{l}N_{1}+N_{2}=2400 \\ N_{1}+N_{2}=1200\end{array}\right.\Rightarrow\left\{\begin{array}{l}N_{1}=1800 \\ N_{2}=600.\end{array}\right.$$
		
		Áp dụng công thức biến áp: $$\frac{U_{1}}{U_{2}}=\frac{N_{1}}{N_{2}} \Rightarrow \frac{120}{U_{2}}=\frac{1800}{600}  \\ \Rightarrow U_{2}=40 \mathrm{V}.$$
		
		\textbf{Đáp án: A.}
	}
	\viduii{4}{Điện năng được truyền từ đường dây điện một pha có điện áp hiệu dụng ổn định $\SI{220}{V}$ vào nhà một hộ dân bằng đường dây tải điện có chất lượng kém. Trong nhà của hộ dân này, dùng một máy biến áp lí tưởng để duy trì điện áp hiệu dụng ở đầu ra luôn là $\SI{220}{V}$ (gọi là máy ổn áp). Máy ổn áp này chỉ hoạt động khi điện áp hiệu dụng ở đầu vào lớn hơn $\SI{110}{V}$. Tính toán cho thấy, nếu công suất sử dụng điện trong nhà là $\SI{1,1}{kW}$ thì tỉ số giữa điện áp hiệu dụng ở đầu ra và điện áp hiệu dụng ở đầu vào (tỉ số tăng áp) của máy ổn áp là $1,1$. Coi điện áp và cường độ dòng điện luôn cùng pha. Nếu công suất sử dụng điện trong nhà là $\SI{2,2}{kW}$ thì tỉ số tăng áp của máy ổn áp bằng
		
		\begin{mcq}(4)
			\item 2,20.
			\item 1,62.
			\item 1,26.
			\item 1,55.
		\end{mcq}
	}
	{	\begin{center}
			\textbf{Hướng dẫn giải}
		\end{center}
		
		Theo đề bài, điện áp đầu ra của ổn áp luôn là $220\textrm{ V}$ nên	
		$$U_{21}=U_{22}=220\textrm{ V}.$$
		
		\textbf{TH1:} Khi công suất tiêu thụ điện của hộ gia đình là $1,1\textrm{ kW}$ thì	 
		$$P_1=U_{21}.I_{21}\Rightarrow I_{21}=5\textrm{ A}.$$
		
		Hệ số tăng áp của MBA là $1,1$ nên
		$$U_{21}/U_{11} =1,1\Rightarrow U_{11}=200\textrm{ V}\Rightarrow I_{11}/I_{21} =1,1 \Rightarrow I_{11}=5,5\textrm{ A}.$$
		
		Độ giảm thế trên đường dây truyền tải:
		$$\Delta U_1=U_0-U_{11}=20\textrm{ V}\Rightarrow R=\Delta U_1/I_{11} =40/11  \SI{}{\ohm}.$$
		
		\textbf{TH2:} Khi công suất tiêu thụ điện của hộ gia đình là $2,2\textrm{ kW}$ thì
		$$P_2=U_{22}\cdot I_{22}\Rightarrow I_{22}=10 \,\text{A}.$$
		
		Hệ số tăng áp của MBA là $k$ nên
		$$U_{22}/U_{12} = k\Rightarrow U_{12}=\frac{220}{k} \textrm{V}$$  $$I_{21}/I_{22} = k \Rightarrow I_{21}=10k.$$
		
		Độ giảm thế trên đường dây truyền tải
		$$\Delta U_2=U_0-U_{12}=I_{21}\cdot R\Rightarrow 220-\frac{220}{k}=10 k \cdot \frac{40}{11}\Rightarrow \left[\begin{array}{l}k=1,26\\ k=4,78.\end{array}\right.$$
		
		MBA chỉ hoạt động khi  $U_1>110 V \Rightarrow k<2 \Rightarrow k=1,26$.
		
		\textbf{Đáp án: C.}
	}
	\viduii{3}{Cho một máy biến áp có hiệu suất $80\%$. Cuộn sơ cấp có 100 vòng, cuộn thứ cấp có 200 vòng. Mạch sơ cấp lý tưởng, đặt vào hai đầu cuộn sơ cấp điện áp xoay chiều có giá trị hiệu dụng 100 V và tần số 50 Hz. Hai đầu cuộn thứ cấp nối với một cuộn dây có điện trở $50\ \Omega$, độ tự cảm $\dfrac{\text{0,5}}{\pi}\ \text{H}$. Cường độ dòng điện hiệu dụng mạch sơ cấp nhận giá trị
		\begin{mcq}(4)
			\item 5 A.
			\item 10 A.
			\item 2 A.
			\item 2,5 A.
		\end{mcq}
	}
	{	\begin{center}
			\textbf{Hướng dẫn giải}
		\end{center}
		
		
		\begin{itemize}
			\item Áp dụng công thức suy ra điện áp của cuộn thứ cấp:
			\begin{equation*}
				\dfrac{U_2}{U_1}=\dfrac{N_2}{N_1} \Rightarrow U_2= \dfrac{N_2}{N_1}U_1 = 200\ \text{V}.
			\end{equation*}
			\item Cường độ dòng điện qua cuộn thứ cấp:
			\begin{equation*}
				I_2 = \dfrac{U_2}{\sqrt{R^2+Z_L^2}}= 2\sqrt 2\ \text{A}.
			\end{equation*}
			\item Hiệu suất của máy biến áp:
			\begin{equation*}
				H=\dfrac{I^2_2R}{U_1I_1}.
			\end{equation*}
			\item Suy ra cường độ dòng điện qua cuộn sơ cấp:
			\begin{equation*}
				I_1= \dfrac{I^2_2R}{HU_1}=5\ \text{A}.
			\end{equation*}
		\end{itemize}
		
		\textbf{Đáp án: A.}
	}
	\viduii{3}{Một máy biến áp lí tưởng, cuộn sơ cấp $N_1$ bằng 1000 vòng được nối vào điện áp hiệu dụng không đổi $U_1=400\ \text{V}$. Thứ cấp gồm 2 cuộn $N_2$ bằng 50 vòng, $N_3$ bằng 100 vòng. Giữa hai đầu $N_2$ đấu với một điện trở $R=40\ \Omega$, giữa 2 đầu $N_3$ đấu với một điện trở $R'=10\ \Omega$. Coi dòng điện và điện áp luôn cùng pha. Cường độ dòng điện hiệu dụng chạy trong cuộn sơ cấp là
		
		\begin{mcq}(4)
			\item 0,150 A.
			\item 0,450 A.
			\item 0,425 A.
			\item 0,015 A.
		\end{mcq}
	}
	{	\begin{center}
			\textbf{Hướng dẫn giải}
		\end{center}
		
		\begin{itemize}
			\item Điện áp qua cuộn thứ cấp có số vòng dây $N_2$:
			\begin{equation*}
				\dfrac{U_1}{U_2}=\dfrac{N_1}{N_2} \Rightarrow U_2 = U_1 \dfrac{N_2}{N_1} =20\ \text{V}.
			\end{equation*}
			\item Cường độ dòng điện qua cuộn $N_2$:
			\begin{equation*}
				I_2=\dfrac{U_2}{R}=\text{0,5}\ \text{A}.
			\end{equation*}
			\item Điện áp qua cuộn thứ cấp có số vòng dây $N_2$:
			\begin{equation*}
				\dfrac{U_1}{U_3}=\dfrac{N_1}{N_3} \Rightarrow U_3 = U_1 \dfrac{N_3}{N_1} =40\ \text{V}.
			\end{equation*}
			\item Cường độ dòng điện qua cuộn $N_2$:
			\begin{equation*}
				I_3=\dfrac{U_3}{R'}=4\ \text{A}.
			\end{equation*}
			\item Máy biến áp lý tưởng có $H=1$ nên công suất hai đầu của cuộn sơ cấp bằng công suất hai đầu của cuộn thứ cấp.
			\item Điện áp qua cuộn thứ cấp có số vòng dây $N_2$:
			\begin{equation*}
				U_1I_1=U_2I_2+U_3I_3.
			\end{equation*}
			\item Thay các giá trị vừa tìm được vào biểu thức trên 
			\begin{equation*}
				400I_1= 20 \cdot \text{0,5}+40 \cdot 4 \Rightarrow I_1 =\text{0,425}\ \text{A}.
			\end{equation*}
		\end{itemize}
		\textbf{Đáp án: C.}
	}
\end{dang}

\begin{dang}{Sử dụng khái niệm máy tăng áp \\ và máy hạ áp, tỉ số giữa cường độ\\ dòng điện, hiệu điện thế, số vòng dây của cuộn sơ cấp và thứ cấp để xác định \\các đại lượng cần tìm}
	\viduii{3}{Một học sinh quấn một máy biến áp với dự định số vòng dây của cuộn sơ cấp gấp hai lần số vòng dây của cuộn thứ cấp. Do sơ suất nên cuộn thứ cấp bị thiếu một số vòng dây. Muốn xác định số vòng dây thiếu để quấn tiếp thêm vào cuộn thứ cấp cho đủ, học sinh này đặt vào hai đầu cuộn sơ cấp một điện áp xoay chiều có giá trị hiệu dụng không đổi, rồi dùng vôn kế xác định tỉ số điện áp ở cuộn thứ cấp để hở và cuộn sơ cấp. Lúc đầu tỉ số điện áp bằng 0,43. Sau khi quấn thêm vào cuộn thứ cấp 24 vòng dây thì tỉ số điện áp bằng 0,45. Bỏ qua mọi hao phí trong máy biến áp. Để được máy biến áp đúng như dự định, học sinh này tiếp tục quấn thêm vào cuộn thứ cấp
		\begin{mcq}(2)
			\item 40 vòng dây.
			\item 84 vòng dây.
			\item 100 vòng dây.
			\item 60 vòng dây.
		\end{mcq}
	}
	{	\begin{center}
			\textbf{Hướng dẫn giải}
		\end{center}
		\begin{itemize}
			\item Tỉ số điện áp ở cuộn thứ cấp để hở và cuộn sơ cấp 
			\begin{equation*}
				\dfrac{U_2}{U_1}=\dfrac{N_2}{N_1}= \text{0,43}. 
			\end{equation*}
			\item Số vòng của cuộn dây thứ cấp
			\begin{equation*}
				N_2 = \text{0,43}N_1.
			\end{equation*}
			\item Sau khi quấn thêm 24 vòng
			\begin{equation*}
				N_2 +24 = \text{0,45}N_1.
			\end{equation*}
			\item Dựa vào 2 biểu thức trên suy ra $N_1 = 1200\ \text{vòng}$ và $N_2 = 516\ \text{vòng}$.
			\item Để quấn máy biến áp theo dự định thì 
			\begin{equation*}
				N_2 + 24 +n=\text{0,5}N_1.
			\end{equation*}
			\item Thay $N_1$ và $N_2$ vào 
			\begin{equation*}
				516 + 24 + n = \text{0,5} \cdot 1200 \Rightarrow n = 60 \text{ vòng dây}.
			\end{equation*}	
		\end{itemize}	
		\textbf{Đáp án: D.}
	}
	\viduii{4}{Đặt vào hai đầu cuộn sơ cấp của một máy biến áp lí tưởng (bỏ qua hao phí) một điện áp xoay chiều có giá trị hiệu dụng không đổi thì điện áp hiệu dụng giữa hai đầu cuộn thứ cấp để hở là $\SI{100}{V}$. Ở cuộn thứ cấp, nếu giảm bớt n vòng dây thì điện áp hiệu dụng giữa hai đầu để hở của nó là $U$, nếu tăng thêm n vòng dây thì điện áp đó là $2U$. Nếu tăng thêm 3n vòng dây ở cuộn thứ cấp thì điện áp hiệu dụng giữa hai đầu để hở của cuộn này bằng 
		\begin{mcq}(4)
			\item 100 V.
			\item 200 V.
			\item 220 V.
			\item 110 V.
		\end{mcq}
	}
	{	\begin{center}
			\textbf{Hướng dẫn giải}
		\end{center}
		\begin{itemize}
			\item Tỉ số giữa điện áp và số vòng dây ở hai đầu cuộn sơ cấp và thứ cấp
			\begin{equation*}
				\dfrac{100}{U_1}=\dfrac{N_2}{N_1}(*). 
			\end{equation*}
			\item Tỉ số giữa điện áp và số vòng dây ở hai đầu cuộn sơ cấp và thứ cấp
			\begin{equation*}
				\dfrac{U}{U_1}=\dfrac{N_2-n}{N_1}(**). 
			\end{equation*}
			\item Tỉ số giữa điện áp và số vòng dây ở hai đầu cuộn sơ cấp và thứ cấp
			\begin{equation*}
				\dfrac{2U}{U_1}=\dfrac{N_2+n}{N_1}(***). 
			\end{equation*}
			\item Lấy (**) cộng (***) và thay (*) vào suy ra 
			\begin{equation*}
				\dfrac{2N_2}{N_1}=\dfrac{3U}{U_1}=\dfrac{200}{U_1} \Rightarrow U = \dfrac{200}{3}\ \text{V}.
			\end{equation*}
			\item Lấy (***) trừ (**) suy ra 
			\begin{equation*}
				\dfrac{U}{U_1}=\dfrac{2n}{N_1} \Rightarrow \dfrac{1}{2}\dfrac{U}{U_1}=\dfrac{n}{N_1}.
			\end{equation*}
			\item Nếu tăng thêm $3n$ vòng dây
			\begin{equation*}
				\dfrac{U_2}{U_1}=\dfrac{N_2+3n}{N_1} = \dfrac{N_2}{N_1} + 3\dfrac{n}{N_1} = \dfrac{100}{U_1} + 3\dfrac{1}{2} \cdot \dfrac{\dfrac{200}{3}}{U_1} = \dfrac{200}{U_1}.
			\end{equation*}
			\item Suy ra $U_2=200\ \text{V}$.
		\end{itemize}
		
		\textbf{Đáp án: B.}
	}
	
	\viduii{3}{ Đặt vào hai đầu cuộn sơ cấp một máy biến áp lí tưởng ( bỏ qua hao phí) một điện áp xoay chiều có giá trị hiệu dụng không đổi thì điện áp hiệu dụng giữa hai đầu cuộn thứ cấp để hở là $\SI{300}{V}$. Nếu giảm bớt một phần ba tổng số vòng dây của cuộn thứ cấp thì điện áp hiệu dụng giữa hai đầu để hở của nó là
		\begin{mcq}(4)
			\item 100 V.
			\item 200 V.
			\item 220 V.
			\item 110 V.
		\end{mcq}
	}
	{	\begin{center}
			\textbf{Hướng dẫn giải}
		\end{center}
		
		Tỉ số giữa điện áp và số vòng dây ở hai đầu cuộn thứ cấp và sơ cấp ban đầu
		
		$$\dfrac{U_2}{U_1} = \dfrac{N_2}{N_1}\ (1).$$
		
		Tỉ số giữa điện áp và số vòng dây ở hai đầu cuộn thứ cấp và sơ cấp lúc sau
		
		$$\dfrac{U'_2}{U_1} = \dfrac{N_2-\dfrac{N_2}{3}}{N_1}= \dfrac{2}{3} \dfrac{N_2}{N_1}\ (2).$$
		
		Lấy (2) chia (1) suy ra
		
		$$\dfrac{U'_2}{2} = \dfrac{2}{3} \Rightarrow U'_2 =\SI{200}{V}.$$
		
		\textbf{Đáp án: B}.
	}
\end{dang}

