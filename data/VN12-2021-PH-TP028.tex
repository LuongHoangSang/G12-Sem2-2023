\whiteBGstarBegin
\setcounter{section}{0}
\section{Lý thuyết: Tia X}
\begin{enumerate}[label=\bfseries Câu \arabic*:]
	
	%=======================================
	\item \mkstar{1} [9]
	
	\cauhoi
	{Tia \textbf{X} không có tính chất nào sau đây?
		\begin{mcq}(2)
			\item Tác dụng lên phim ảnh. 
			\item Đâm xuyên qua tất cả các kim loại. 
			\item Gây ra quang điện cho hầu hết kim loại. 
			\item Làm phát quang các chất. 
		\end{mcq}
	}
	
	\loigiai
	{		\textbf{Đáp án: B.}
		
		Tia X không thể xuyên qua tấm chì vài centimét.
	}

%=======================================
\item \mkstar{1} [4]

\cauhoi
{Phát biểu nào \textbf{đúng} về tia X?
	\begin{mcq}(1)
		\item Tia X có khả năng đâm xuyên kém hơn tia hồng ngoại.. 
		\item Tia X có tần số nhỏ hơn tần số của tia hồng ngoại. 
		\item Tia X có bước sóng lớn hơn bước sóng ánh sáng nhìn thấy. 
		\item Tia X có tác dụng sinh lí, nó hủy diệt tế bào. 
	\end{mcq}
}

\loigiai
{		\textbf{Đáp án: D.}
	
	Tia X có tác dụng sinh lí, nó hủy diệt tế bào. 
}

%=======================================
\item \mkstar{1} [10]

\cauhoi
{Tia Rơnghen có 
	\begin{mcq}(1)
		\item cùng bản chất với sóng âm. 
		\item bước sóng lớn hơn bước sóng của tia hồng ngoại. 
		\item cùng bản chất với sóng vô tuyến. 
		\item điện tích âm. 
	\end{mcq}
}

\loigiai
{		\textbf{Đáp án: C.}
	
	Tia Rơnghen có cùng bản chất với sóng vô tuyến.
}

%=======================================
\item \mkstar{1} [3]

\cauhoi
{Chùm ánh sáng laze được ứng dụng
	\begin{mcq}(2)
		\item trong truyền tin bằng vệ tinh. 
		\item làm nguồn phát siêu âm. 
		\item làm dao mổ trong y học. 
		\item trong đầu đọc USB. 
	\end{mcq}
}

\loigiai
{		\textbf{Đáp án: C.}
	
	Chùm ánh sáng laze được ứng dụng làm dao mổ trong y học. 
}

%=======================================
\item \mkstar{1} [7]

\cauhoi
{Khi nói về tia X, phát biểu nào sau đây là \textbf{đúng}?
	\begin{mcq}(1)
		\item Tia X có tần số nhỏ hơn tia hồng ngoại. 
		\item Tia X có cùng bản chất với tia Catốt.
		\item Tia X có khả năng đâm xuyên mạnh. 
		\item Tia X có bước sóng lớn hơn bước sóng của ánh sáng nhìn thấy.
	\end{mcq}
}

\loigiai
{		\textbf{Đáp án: C.}
	
	Tia X có khả năng đâm xuyên mạnh.
}

%=======================================
\item \mkstar{1} [3]

\cauhoi
{Phát biểu nào sau đây là \textbf{sai}?
	\begin{mcq}(1)
		\item Tia X có bước sóng nhỏ hơn bước sóng của ánh sáng tím. 
		\item Tia X có tác dụng sinh lý. 
		\item Tia X làm ion hóa không khí. 
		\item Tia X có bước sóng lớn hơn bước sóng của tia hồng ngoại. 
	\end{mcq}
}

\loigiai
{		\textbf{Đáp án: D.}
	
	Tia X có bước sóng lớn hơn bước sóng của tia hồng ngoại là sai. 
}

%=======================================
\item \mkstar{1} [2]

\cauhoi
{Tia X được tạo ra bằng cách nào sau đây?
	\begin{mcq}(1)
		\item Cho electron có động năng lớn đập vào một kim loại có nguyên tử lượng lớn. 
		\item Chiếu tia tử ngoại vào kim loại có nguyên tử lượng lớn. 
		\item Chiếu một chùm ánh sáng nhìn thấy vào một kim loại có nguyên tử lượng lớn. 
		\item Chiếu tia hồng ngoại vào một kim loại có nguyên tử lượng lớn. 
	\end{mcq}
}

\loigiai
{		\textbf{Đáp án: A.}
	
	Tia X được tạo ra bằng cách cho electron có động năng lớn đập vào một kim loại có nguyên tử lượng lớn. 
}
	
\end{enumerate}

\loigiai
{
	\begin{center}
		\textbf{BẢNG ĐÁP ÁN}
	\end{center}
	\begin{center}
		\begin{tabular}{|m{2.8em}|m{2.8em}|m{2.8em}|m{2.8em}|m{2.8em}|m{2.8em}|m{2.8em}|m{2.8em}|m{2.8em}|m{2.8em}|}
			\hline
			01.B  & 02.D  & 03.C  & 04.C  & 05.C  & 06.D  & 07.A & & & \\
			\hline
			
		\end{tabular}
	\end{center}
}

\section{Dạng bài: Tia X}
\begin{enumerate}[label=\bfseries Câu \arabic*:]

%=======================================
\item \mkstar{3} [2]

\cauhoi
{Một ống phát tia Rơn-ghen khi hoạt động thì trong một phút có $\num{6e18}$ điện tử đập vào đối catốt. Cường độ dòng điện qua ống Rơn-ghen là
	\begin{mcq}(4)
		\item $\SI{30,0}{mA}$. 
		\item $\SI{16,0}{mA}$. 
		\item $\SI{35,0}{mA}$. 
		\item $\SI{2,20}{mA}$. 
	\end{mcq}
}

\loigiai
{		\textbf{Đáp án: B.}
	
	Cường độ dòng điện qua ống Rơn-ghen cho bởi
	$$
	I = \dfrac{q}{t} = \dfrac{N\cdot e}{t} = \SI{16,0}{mA}.
	$$
}

%=======================================
\item \mkstar{3} [3]

\cauhoi
{Hiệu điện thế giữa hai cực của ống Cu-lít-giơ (ống tia X) là $U_{AK} = \SI{25}{kV}$. Bỏ qua động năng ban đầu của electron khi bức ra khỏi Catốt. Bước sóng ngắn nhất mà ống tia X có thể phát ra bằng
	\begin{mcq}(4)
		\item $\SI{49,69}{pm}$. 
		\item $\SI{49,69}{nm}$. 
		\item $\SI{49,69}{\mu m}$. 
		\item $\SI{49,69}{mm}$. 
	\end{mcq}
}

\loigiai
{		\textbf{Đáp án: A.}
	
	Bước sóng ngắn nhất mà ống tia X có thể phát ra là
	$$
	\lambda_{min} = \dfrac{hc}{\varepsilon_{max}} = \dfrac{hc}{e \cdot U_{AK}} = \SI{49,69}{pm}.
	$$
}

%=======================================
\item \mkstar{3} [10]

\cauhoi
{Một ống Rơn-ghen phát ra tia X có bước sóng ngắn nhất là $\SI{2e-10}{m}$. Khi hiệu điện thế giữa hai cực của ống tăng thêm một lượng $\SI{3,5}{kV}$ thì bước sóng ngắn nhất do ống tia X phát ra khi đó là
	\begin{mcq}(4)
		\item $\SI{1,28e-10}{m}$. 
		\item $\SI{1,83e-10}{m}$. 
		\item $\SI{2,5e-10}{m}$. 
		\item $\SI{3,67e-10}{m}$. 
	\end{mcq}
}

\loigiai
{		\textbf{Đáp án: A.}
	
	Hiệu điện thế giữa hai đầu ống lúc ban đầu là
	$$
	U_{AK} = \dfrac{hc}{e\cdot \lambda_{min}} = \SI{6,211}{kV}.
	$$
	Hiệu điện thế giữa hai đầu ống lúc sau là
	$$
	U_{AK}' = U_{AK} = \SI{3,5}{kV} = \SI{9,711}{kV}.
	$$
	Bước sóng ngắn nhất mà ống tia X phát ra lúc đó là
	$$
	\lambda_{max} = \dfrac{hc}{e \cdot U_{AK}'} = \SI{1,28e-10}{m}.
	$$
}
	
\end{enumerate}

\loigiai
{
\begin{center}
	\textbf{BẢNG ĐÁP ÁN}
\end{center}
\begin{center}
	\begin{tabular}{|m{2.8em}|m{2.8em}|m{2.8em}|m{2.8em}|m{2.8em}|m{2.8em}|m{2.8em}|m{2.8em}|m{2.8em}|m{2.8em}|}
		\hline
		01.B & 02.A  & 03.A  &  &  &  & & & & \\
		\hline
		
	\end{tabular}
\end{center}
}


\section{Lý thuyết: Thang sóng điện từ}
\begin{enumerate}[label=\bfseries Câu \arabic*:]
	
	%=======================================
	\item \mkstar{1} [2]
	
	\cauhoi
	{Với $\varepsilon_{1}, \varepsilon_{2}, \varepsilon_{3}$ lần lượt là photon ứng với các bức xạ màu vàng, bức xạ tử ngoại và bức xạ hồng ngoại thì
		\begin{mcq}(4)
			\item $\varepsilon_{2} > \varepsilon_{3} > \varepsilon_{1}$. 
			\item $\varepsilon_{1} > \varepsilon_{2} > \varepsilon_{3}$. 
			\item $\varepsilon_{3} > \varepsilon_{1} > \varepsilon_{2}$. 
			\item $\varepsilon_{2} > \varepsilon_{1} > \varepsilon_{3}$. 
		\end{mcq}
	}
	
	\loigiai
	{		\textbf{Đáp án: D.}
		
		Năng lượng photon của bức xạ tử ngoại mạnh hơn bức xạ màu vàng và mạnh hơn bức xạ hồng ngoại.
	}
	
	%=======================================
	\item \mkstar{1} [10]
	
	\cauhoi
	{Trong các loại tia Rơnghen, hồng ngoại, tử ngoại, đơn sắc màu lục, tia có tần số nhỏ nhất là 
		\begin{mcq}(2)
			\item tia tử ngoại. 
			\item tia hồng ngoại. 
			\item tia đơn sắc màu lục. 
			\item tia Rơnghen. 
		\end{mcq}
	}
	
	\loigiai
	{		\textbf{Đáp án: B.}
		
		Tia có tần số nhỏ nhất là tia hồng ngoại.
	}

	%=======================================
	\item \mkstar{1} [12]
	
	\cauhoi
	{Tia hồng ngoại, tia tử ngoại, tia X có tần số lần lượt là $f_{1}, f_{2}, f_{3}$. Sắp xếp theo đúng thứ tự giảm dần là
		\begin{mcq}(4)
			\item $f_{2}, f_{1}, f_{3}$. 
			\item $f_{3}, f_{1}, f_{2}$. 
			\item $f_{2}, f_{3}, f_{1}$. 
			\item $f_{3}, f_{2}, f_{1}$. 
		\end{mcq}
	}
	
	\loigiai
	{		\textbf{Đáp án: D.}
		
		Sắp xếp theo thứ tự tần số giảm dần là: tia X. tia tử ngoại, tia hồng ngoại.
	}

	%=======================================
	\item \mkstar{1} [7]
	
	\cauhoi
	{Các bức xạ được sắp xếp theo thứ tự bước sóng tăng dần là
		\begin{mcq}(1)
			\item ánh sáng tím, tia hồng ngoại, tia tử ngoại, tia X. 
			\item tia X, tia tử ngoại, ánh sáng tím, tia hồng ngoại. 
			\item tia hồng ngoại, ánh sáng tím, tia tử ngoại, tia X. 
			\item tia hồng ngoại, ánh sáng đỏ, tia tử ngoại, tia X. 
		\end{mcq}
	}
	
	\loigiai
	{		\textbf{Đáp án: B.}
		
		Sắp xếp theo thứ tự tăng dần bước sóng là tia X, tia tử ngoại, ánh sáng tím, tia hồng ngoại.
	}

	
\end{enumerate}

\loigiai
{
	\begin{center}
		\textbf{BẢNG ĐÁP ÁN}
	\end{center}
	\begin{center}
		\begin{tabular}{|m{2.8em}|m{2.8em}|m{2.8em}|m{2.8em}|m{2.8em}|m{2.8em}|m{2.8em}|m{2.8em}|m{2.8em}|m{2.8em}|}
			\hline
			01.D & 02.B  & 03.D  & 04.B  &  & & & & & \\
			\hline
			
		\end{tabular}
	\end{center}
}

\whiteBGstarEnd