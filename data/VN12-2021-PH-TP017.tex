\begin{enumerate}[label=\bfseries Câu \arabic*:]
	
	\item \mkstar{1}
	
	\cauhoi{Dòng điện xoay chiều hình sin là dòng điện
		\begin{mcq}
			\item có cường độ không đổi theo thời gian.
			\item có cường độ biến đổi điều hoà theo thời gian.
			\item có chiều không đổi theo thời gian.
			\item có chu kỳ thay đổi theo thời gian.
		\end{mcq}
	}
		\loigiai{
\textbf{Đáp án: B.}
			
			Dòng điện xoay chiều là dòng điện có cường độ biến đổi điều hoà theo thời gian.
			
			}	
	
	%%%%%%%%%%%%%%CÂU2%%%%%%%%%%
	\item \mkstar{1}
	
	\cauhoi{Trong các đại lượng đặc trưng cho dòng điện xoay chiều sau đây, đại lượng nào có dùng giá trị hiệu dụng?
		\begin{mcq}(4)
			\item Điện áp. 
			\item Chu kỳ.
			\item Tần số. 
			\item Công suất.
		\end{mcq}
}	
		\loigiai{
			\textbf{Đáp án: A.}
			
			Trong các đại lượng đặc trưng cho dòng điện xoay chiều, đại lượng có dùng giá trị hiệu dụng là điện áp, cường độ dòng điện, suất điện động.
			
			}	
	
	%%%%%%%%%%%%%%CÂU3%%%%%%%%%%
	\item \mkstar{1}
	
	\cauhoi{Trong các đại lượng đặc trưng cho dòng điện xoay chiều sau đây, đại lượng nào \textbf{không} dùng giá trị hiệu dụng?
		\begin{mcq}(2)
			\item Điện áp.   
			\item Cường độ dòng điện.
			\item Suất điện động.   
			\item Công suất.
		\end{mcq}
}	
		\loigiai{
			\textbf{Đáp án: D.}
			
			Công suất không có giá trị hiệu dụng.
			
			}	
	%%%%%%%%%%%%%%CÂU4%%%%%%%%%%
	\item \mkstar{1}
	
	\cauhoi{Phát biểu nào sau đây là\textbf{ không} đúng?
		\begin{mcq}
			\item Điện áp biến đổi điều hoà theo thời gian gọi là điện áp xoay chiều.
			\item Dòng điện có cường độ biến đổi điều hoà theo thời gian gọi là dòng điện xoay chiều.
			\item Suất điện động biến đổi điều hoà theo thời gian gọi là suất điện động xoay chiều.
			\item Cho dòng điện một chiều và dòng điện xoay chiều lần lượt đi qua cùng một điện trở thì chúng toả ra nhiệt lượng như nhau.
		\end{mcq}
}	
		\loigiai{
			\textbf{Đáp án: D.}
			
			Cho dòng điện một chiều và dòng điện xoay chiều lần lượt đi qua cùng một điện trở thì chúng toả ra nhiệt lượng như nhau là không đúng vì chưa đề cập đến độ lớn của cường độ dòng điện. Nếu muốn chúng tỏa ra cùng một nhiệt lượng thì cường độ dòng điện một chiều phải có giá trị bằng giá trị hiệu dụng của dòng điện xoay chiều.
			
			}	
	
	%%%%%%%%%%%%%%CÂU5%%%%%%%%%%
	\item \mkstar{1}
	
	\cauhoi{Các giá trị hiệu dụng của dòng điện xoay chiều
		\begin{mcq}
			\item được xây dựng dựa trên tác dụng nhiệt của dòng điện.
			\item chỉ được đo bằng ampe kế nhiệt.
			\item bằng giá trị trung bình chia cho $\sqrt{2}$.
			\item bằng giá trị cực đại chia cho 2.
		\end{mcq}
}		
		\loigiai{
		\textbf{Đáp án: A.}
			
			Các giá trị hiệu dụng của dòng điện xoay chiều được xây dựng dựa trên tác dụng nhiệt của dòng điện.
			
			}	
	
	%%%%%%%%%%%%%%CÂU6%%%%%%%%%%
	\item \mkstar{1}
	
	
	\cauhoi{Cường độ dòng điện trong mạch không phân nhánh có dạng $i=2\sqrt{2}\cos(100\pi t)\,\text{A}$. Cường độ dòng điện hiệu dụng trong mạch là
		\begin{mcq}(4)
			\item $I=\SI{4}{A}$.
			\item $I=\SI{2,83}{A}$.
			\item $I=\SI{2}{A}$.
			\item $I=\SI{1,41}{A}$.
		\end{mcq}
}	
		\loigiai{
		\textbf{Đáp án: C.}
			
			Cường độ dòng điện hiệu dụng $$I=\dfrac{I_0}{\sqrt{2}}=\SI{2}{A}.$$
			
			}
	
	
	
	%%%%%%%%%%%%%%CÂU7%%%%%%%%%%
	\item \mkstar{1}
	
	
	\cauhoi{Điện áp tức thời giữa hai đầu đoạn mạch có dạng $i=141\cos(100\pi t)\,\text{V}$. Điện áp hiệu dụng giữa hai đầu đoạn mạch là
		\begin{mcq}(4)
			\item $\SI{141}{V}$.
			\item $\SI{50}{V}$.
			\item $\SI{100}{V}$.
			\item $\SI{200}{V}$.
		\end{mcq}
}		
		\loigiai{
				\textbf{Đáp án: C.}
			
			Điện áp hiệu dụng $$U=\dfrac{U_0}{\sqrt{2}}=\SI{100}{V}.$$
			
		}	
	\item \mkstar{2}

\cauhoi{Phát biểu nào sau đây là \textbf{không} đúng?
	\begin{mcq}
		\item Dòng điện xoay chiều không phải là dòng điện chạy trong các đồ chơi dùng pin.
		\item Dòng điện dân dụng ở Việt Nam là dòng điện xoay chiều.
		\item Dòng điện xoay chiều là dòng điện hình sin.
		\item Dòng điện hình sin là dòng điện xoay chiều.
	\end{mcq}
}	
\loigiai{
	\textbf{Đáp án: C.}
	
	Dòng điện hình sin là dòng điện xoay chiều.
	
	Nhưng dòng điện xoay chiều có thể không phải là dòng điện hình sin mà còn là các hình dạng khác.
	
}		
	\item \mkstar{2}

\cauhoi{Đơn vị nào sau đây dùng để tính lượng điện năng tiêu thụ hàng tháng của một hộ gia đình?
	\begin{mcq}(2)
		\item Am-pe.
		\item Oát.
		\item Ki-lô Oát trên giờ.
		\item Ki-lô Oát giờ.
	\end{mcq}
}	
\loigiai{
	\textbf{Đáp án: D.}
	
	Đơn vị để tính điện năng tiêu thụ là kWh. Với $1\ \text{kWh} = \SI{3e6}{\joule}$.
	
}			
	
	%%%%%%%%%%%%%%CÂU8%%%%%%%%%%
	\item \mkstar{2}
	
	\cauhoi{ Một điện trở thuần $R=\SI{100}{\Omega}$ khi dùng dòng điện có tần số $\SI{50}{Hz}$. Nếu dùng dòng điện có tần số $\SI{100}{Hz}$ thì điện trở
		\begin{mcq}(4)
			\item giảm 2 lần.
			\item tăng 2 lần.
			\item không đổi.
			\item giảm 4 lần.
		\end{mcq}
}	
		\loigiai{
		\textbf{Đáp án: C.}
			
			Giá trị của điện trở không phụ thuộc vào tần số của mạch.
			
			}	
	
	%%%%%%%%%%%%%%CÂU9%%%%%%%%%%
		\item \mkstar{2}
	
	\cauhoi{Dòng điện có biểu thức $i=2\cos100\pi t\, \text{A}$, trong một chu kì, dòng điện đổi chiều bao nhiêu lần?
		\begin{mcq}(4)
			\item 100 lần.
			\item 50 lần.
			\item 2 lần.
			\item 1 lần.
		\end{mcq}
	}	
	\loigiai{
		\textbf{Đáp án: C.}
		
		Trong 1 chu kì, dòng điện đổi chiều 2 lần.
		
	}
		\item \mkstar{2}

\cauhoi{Dòng điện có biểu thức $i=2\cos100\pi t\, \text{A}$, trong $1/50\ \text s$, dòng điện đổi chiều bao nhiêu lần?
	\begin{mcq}(4)
		\item 100 lần.
		\item 50 lần.
		\item 2 lần.
		\item 1 lần.
	\end{mcq}
}	
\loigiai{
	\textbf{Đáp án: C.}
	
 $1/50\ \text s$ tương ứng với 1 chu kì. Vậy dòng điện đổi chiều 2 lần.
	
}		
		\item \mkstar{2}
	
	\cauhoi{Dòng điện có biểu thức $i=2\cos100\pi t\, \text{A}$, trong một giây, dòng điện đổi chiều bao nhiêu lần?
		\begin{mcq}(4)
			\item 100 lần.
			\item 50 lần.
			\item 110 lần.
			\item 99 lần.
		\end{mcq}
	}	
	\loigiai{
		\textbf{Đáp án: A.}
		
		1 giây tương ứng với 50 chu kì.
		
		Vậy số lần đổi chiều của dòng điện trong một giây là $n=\text{100 lần}$.
		
	}
		\item \mkstar{2}

\cauhoi{Một khung dây dẫn có diện tích $S=\SI{50}{\centi \meter \squared}$ gồm 150 vòng dây quay đều với vận tốc $n\ \text{vòng/ phút}$ trong một từ trường đều $\vec B$ vuông góc với trục quay $\Delta$ và có độ lớn $B=\SI{0.02}{\tesla}$. Từ thông cực đại gửi qua khung là
	\begin{mcq}(4)
		\item $\SI{0.015}{\weber}$.
		\item $\SI{0.15}{\weber}$.
		\item $\SI{1.5}{\weber}$.
		\item $\SI{15}{\weber}$.
	\end{mcq}
}	
\loigiai{
	\textbf{Đáp án: A.}
	
	Đổi $\SI{50}{\centi \meter \squared} = \SI{50e-4}{\meter \squared}$.
	
	Từ thông cực đại gửi qua khung là
	$$\Phi_0=NBS = \SI{0.015}{\weber}.$$
	
}		
		\item \mkstar{2}

\cauhoi{Một khung dây dẫn có diện tích $S=\SI{50}{\centi \meter \squared}$ gồm 150 vòng dây quay đều với vận tốc $n\ \text{vòng/ phút}$ trong một từ trường đều $\vec B$ song song với trục quay $\Delta$ và có độ lớn $B=\SI{0.02}{\tesla}$. Từ thông cực đại gửi qua khung là
	\begin{mcq}(4)
		\item $\SI{0.015}{\weber}$.
		\item $\SI{0}{\weber}$.
		\item $\SI{0.05}{\weber}$.
		\item $\SI{0.15}{\weber}$.
	\end{mcq}
}	
\loigiai{
	\textbf{Đáp án: B.}
	
	Vì $\vec B$ song song với trục quay $\Delta$ nên $\cos \alpha=0$.
	
	Từ thông cực đại gửi qua khung là
	$$\Phi_0=NBS\cos\alpha = \SI{0}{\weber}.$$
	
}	
		\item \mkstar{2}

\cauhoi{Một khung dây hình chữ nhật quay đều với tốc độ góc $3000\ \text{vòng/ phút}$ quanh trục $\Delta$ đặt trong từ trường đều có cảm ứng từ vuông góc với trục quay. Suất điện động trong khung biến thiên điều hòa với chu kì
	\begin{mcq}(4)
		\item $\SI{3.14}{\second}$.
		\item $\SI{0.314}{\second}$.
		\item $\SI{0.02}{\second}$.
		\item $\SI{0.2}{\second}$.
	\end{mcq}
}	
\loigiai{
	\textbf{Đáp án: C.}
	
	Đổi $\omega = 30000\ \text{vòng / phút} \rightarrow\ \omega= \xsi{100\pi}{\radian/ \second}$.
	
	Suất điện động trong khung biến thiên với chu kì là
	$$T = \dfrac{2\pi}{\omega} = \SI{0.02}{\second}.$$
	
}	
\item \mkstar{3}

\cauhoi{Một khung dây dẫn quay đều quanh trục $\Delta$ với tốc độ $150\ \text{vòng/ phút}$ trong từ trường đều có cảm ứng từ $\vec{B}$ vuông góc với trục quay của khung. Từ thông cực đại gửi qua khung dây là $10/ \pi \ \text{Wb}$. Suất điện động hiệu dụng trong khung dây bằng
	\begin{mcq}(4)
		\item $25\ \text V$.
		\item $25\sqrt 2 \text V$.
		\item $50\ \text V$.
		\item $50\sqrt 2 \ \text V$.
	\end{mcq}
}	
\loigiai{\textbf{Đáp án: B.}
	
	Đổi $\omega = 150\ \text{vòng / phút} \rightarrow\ \omega= \xsi{5\pi}{\radian/ \second}$.
	
	Biểu thức liên hệ giữa từ thông cực đại và suất điện động cực đại:
	$$\calE_0 = \omega \Phi_0.$$
	
	
	Suất điện động hiệu dụng trong khung là
	$$\calE=\dfrac{\calE_0}{\sqrt 2} = \dfrac{\omega \Phi_0}{\sqrt 2} = 25\sqrt 2\ \text V.$$
	
	}
	%%%%%%%%%%%%%%CÂU10%%%%%%%%%%
	\item \mkstar{3}
	
	\cauhoi{Một mạch điện xoay chiều có phương trình dòng điện trong mạch là $i=5\cos\left(100\pi t-\dfrac{\pi}{2}\right)\, \text{A}$. Xác định điện lượng chuyển qua mạch trong 1/6 chu kì đầu tiên.
		\begin{mcq}(4)
			\item $\dfrac{1}{40\pi}\,\text{C}$.
			\item $\dfrac{1}{40}\,\text{C}$.
			\item $\dfrac{1}{20\pi}\,\text{C}$.
			\item $\dfrac{1}{20}\,\text{C}$.
		\end{mcq}
}	
		\loigiai{
			\textbf{Đáp án: A.}
			
			Điện lượng chuyển qua mạch trong 1/6 chu kì đầu tiên là
			$$q=\int_{0}^{\frac{T}{6}}{idt}= \int_{0}^{\frac{T}{6}}{5\cos\left(100\pi t-\dfrac{\pi}{2}\right)dt}=\dfrac{1}{40\pi}\,\text{C}.$$
			
			}
		
\item \mkstar{3}

\cauhoi{Một bóng đèn có ghi 110 V - 200 W mắc nối tiếp với điện trở $R$ vào một mạch xoay chiều có $u = 220\sqrt 2 \cos 100 \pi t$ (V). Để đèn sáng bình thường, $R$ phải có giá trị là bao nhiêu?
	\begin{mcq}(4)
		\item $R=\text{60,5}\, \Omega.$
		\item $R=\text{40,5}\, \Omega.$
		\item $R=\text{35,5}\, \Omega.$
		\item $R=\text{30,5}\, \Omega.$
	\end{mcq}
}	
	\loigiai{
	\textbf{Đáp án: A.}
		
		Điện áp hiệu dụng giữa hai đầu đoạn mạch đó là: 
		$$U=\dfrac{U_0}{\sqrt{2}}=\dfrac{220 \sqrt 2\,\si{\volt}}{\sqrt 2}=220 \, \si{\volt}.$$
		
		Bóng đèn và điện trở R mắc nối tiếp nên: $$U = U_\textrm{Đ} + U_R \Rightarrow U_R = \SI{110}{\volt}.$$
		
		Cường độ dòng  điện qua điện trở: $$I=I_R=I_\textrm{Đ}=\dfrac{\calP_\textrm{Đ}}{U_\textrm{Đ}}=\dfrac{20}{11}\, \text{A}.$$
		
		Để đèn sáng bình thường: $$R=\dfrac{U_R}{I_R}=\text{60,5}\, \Omega.$$
		
	}
	\item \mkstar{3}
	
	\cauhoi{Một đèn điện có ghi $110\ \text V - 100\ \text W$ mắc nối tiếp với điện trở $R$ vào một mạch điện xoay chiều có $u=220 \sqrt 2 \sin 100 \omega t\ \text{(V)}$. Để đèn sáng bình thường, $R$ phải có giá trị là bao nhiêu?
		\begin{mcq}(4)
			\item $\SI{1210}{\Omega}$.
			\item $\xsi{10/11}{\Omega}$.
			\item $\SI{121}{\Omega}$.
			\item $\SI{110}{\Omega}$.
		\end{mcq}
	}	
	\loigiai{
		\textbf{Đáp án: C.}
		
		Điện áp định mức của đèn là $U=110\ \text V$.
		
		Công suất định mức của đèn là $\calP = 100\ \text W$.
		
		Điện trở của đèn là
		$$R_\text{đ}=\dfrac{U_\text{đm}^2}{\calP_\text{đm}} = 121\ \Omega.$$
		
		
		Cường độ dòng điện hiệu dụng chạy qua mạch:
		$$I=\dfrac{U}{R_\text{đ} + R} = \dfrac{220}{121+R}.$$
		
		
		Do mạch mắc nối tiếp nên cường độ dòng điện hiệu dụng chạy qua đèn cũng bằng $I$, suy ra điện áp giữa hai đầu bóng đèn là
		$$U_\text{đ} = IR_\text{đ} = \dfrac{220\cdot121}{121+R}.$$
		
		
		Để đèn sáng bình thường thì $U_\text{đ}=U_\text{đm}$, suy ra
		$$	\dfrac{220\cdot121}{121+R} = 110 \Rightarrow R = 121\ \Omega.$$
		
	}
\end{enumerate}
\loigiai{\textbf{Đáp án}
	\begin{center}
		\begin{tabular}{|m{2.8em}|m{2.8em}|m{2.8em}|m{2.8em}|m{2.8em}|m{2.8em}|m{2.8em}|m{2.8em}|m{2.8em}|m{2.8em}|}
			\hline
			1. B & 2. A & 3. D & 4. D & 5. A & 6. C  & 7. C  & 8. C & 9. D & 10. C\\
			\hline
			11. C & 12. C & 13. A & 14. A & 15. B & 16. C  & 17. B  & 18. A & 19. A & 20. C\\
			\hline
		\end{tabular}
\end{center}}