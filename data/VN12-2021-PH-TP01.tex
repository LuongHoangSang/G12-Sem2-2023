\setcounter{section}{0}

\begin{enumerate}[label=\bfseries Câu \arabic*:]
	
	%1
	\item \mkstar{1}
	
	\cauhoi{Một đơn vị khối lượng nguyên tử (1 u) bằng
		\begin{mcq}
			\item $1/12$ khối lượng của hạt nhân $^6_3\ce{C}$.	
			\item khối lượng của một proton.	
			\item $\SI{931.5}{MeV \cdot c^2}$.	
			\item Tất cả đều sai.
		\end{mcq}}
	
		\loigiai{\textbf{Đáp án: D.}
			
			Một đơn vị khối lượng nguyên tử (1 u) bằng $1/12$ khối lượng của hạt nhân $^{12}_6\ce{C}$.
			
				}
	%2
	\item \mkstar{1}
	
	\cauhoi{\textbf{[TSĐH 2007]} Hạt nhân Triti ($^3_1\text T$) có
		\begin{mcq}(2)
			\item 3 nuclôn, trong đó có 1 prôtôn.	
			\item 3 nơtrôn và 1 prôtôn.	
			\item 3 nuclôn, trong dó có 1 nơtrôn.	
			\item 3 prôtôn và 1 nơtrôn.
		\end{mcq}}
	
		\loigiai{\textbf{Đáp án: A.}
			
			Hạt nhân Triti ($^3_1\text T$) có $A=3$ nuclôn, $Z=1$ proton, $N=3-1=2$ nơtron.
			
				}
	
	%3
	\item \mkstar{1}
	
	\cauhoi{Đồng vị là
		\begin{mcq}
			\item các nguyên tử mà hạt nhân của chúng có số khối bằng nhau.	
			\item các nguyên tử mà hạt nhân của chúng có số proton bằng nhau, số nơtron khác nhau.	
			\item các nguyên tử mà hạt nhân của chúng có số nơtron bằng nhau, số proton khác nhau.
			\item các nguyên tử mà hạt nhân của chúng có khối lượng bằng nhau. 
		\end{mcq}}
	
		\loigiai{\textbf{Đáp án: B.}	}
	%4
	\item \mkstar{1}
	
	\cauhoi{\textbf{[TSĐH 2007]} Phát biểu nào là \textbf{sai}?
		\begin{mcq}
			\item  Các đồng vị phóng xạ đều không bền.	
			\item Các nguyên tử mà hạt nhân có cùng số prôtôn nhưng có số nơtron khác nhau gọi là đồng vị.	
			\item Các đồng vị của cùng một nguyên tố có số nơtrôn khác nhau nên tính chất hóa học khác nhau.
			\item Các đồng vị của cùng một nguyên tố có cùng vị trí trong bảng hệ thống tuần hoàn. 
		\end{mcq}}
	
		\loigiai{\textbf{Đáp án: C.}
			
			Các đồng vị của cùng một nguyên tố có cùng số $p$, cùng số $e$ nên có cùng tính chất hoá học.
			
				}
	%5
	\item \mkstar{1}
	
	\cauhoi{Tương tác giữa các nuclôn tạo thành hạt nhân là tương tác
		\begin{mcq}(4)
			\item mạnh.	
			\item yếu.	
			\item điện từ.
			\item hấp dẫn. 
		\end{mcq}}
	
		\loigiai{\textbf{Đáp án: A.}	}
	%6
	\item \mkstar{2}
	
	\cauhoi{So với hạt nhân $\ce{Si}^{29}_{14}$, hạt nhân $\ce{Ca}^{40}_{20}$ có nhiều hơn
		\begin{mcq}(2)
			\item 11 nơtron và 6 proton.	
			\item 5 nơtron và 6 proton.	
			\item 6 nơtron và 5 proton.	
			\item 5 nơtron và 12 proton.
		\end{mcq}}
	
		\loigiai{\textbf{Đáp án: B.}
			
			Hạt nhân $\ce{Si}^{29}_{14}$ có 15 nơtron và 14 proton, hạt nhân $\ce{Ca}^{40}_{20}$ có 20 nơtron và 20 proton.
			
				}
	%7
	\item \mkstar{2}
	
	\cauhoi{Hạt nhân nguyên tố chì có 82 proton, 125 nơtron. Hạt nhân nguyên tử này kí hiệu là
		\begin{mcq}(4)
			\item $^{125}_{82}\ce{Pb}$.	
			\item $^{82}_{125}\ce{Pb}$.	
			\item $^{82}_{207}\ce{Pb}$.	
			\item $^{207}_{82}\ce{Pb}$.
		\end{mcq}}
	
		\loigiai{\textbf{Đáp án: D.}
			
			Số khối $A=Z+N=207$.
			
				}	
	%8
	\item \mkstar{2}
	
	\cauhoi{Cho biết khối lượng hạt nhân $^{234}_{92} \text U$ là 233,9904 u. Biết khối lượng của hạt prôtôn và nơtrôn lần lượt là $m_p= \text{1,007276}\ \text{u}$ và $m_n= \text{l,008665}\ \text{u}$. Độ hụt khối của hạt nhân $^{234}_{92}\text U$ bằng
		\begin{mcq}(4)
			\item 1,909 u.	
			\item 3,460 u.	
			\item 0,000.	
			\item 2,056 u.
		\end{mcq}}
		
		\loigiai{\textbf{Đáp án: A.}
			
			Công thức tính độ hụt khối:
			\begin{equation*}
				\Delta m = 92 \cdot m_p + 142 \cdot m_n - m_\text U = \SI{1.909}{u}.
			\end{equation*}	
				}
	%9
	\item \mkstar{3}
	
	\cauhoi{Biết $N_\text A = \SI{6.02e23}{mol^{-1}}$. Trong $\SI{59.50}{\gram}$ $^{238}_{92}\ce{U}$ có số nơtron xấp xỉ là
		\begin{mcq} (4)
			\item $\text{2,38}\cdot 10^{25}$.	
			\item $\text{2,20}\cdot 10^{25}$.	
			\item $\text{1,19}\cdot 10^{25}$.	
			\item $\text{9,21}\cdot 10^{24}$.
		\end{mcq}}
	
		\loigiai{\textbf{Đáp án: B.}
			
			Số nơtron có trong 1 hạt Urani $^{238}_{92} \text U$:
			\begin{equation*}
				N=A-Z=238-92=146
			\end{equation*}
			
			Số hạt $^{238}_{92} \text U$ có trong $\SI{59.50}{\gram}$:
			\begin{equation*}N_\text U = \dfrac{m}{M}N_\text A = \SI{1.505e23}{}.
			\end{equation*}
			Số nơtron có trong $\SI{59.50}{\gram}$: $N\cdot N_\text U \approx \SI{2.2e25}{}$.
			
				}
	
	\item \mkstar{3}
	
	\cauhoi{\textbf{[TSĐH 2007]} Biết số Avôgađrô  là $N_{\text{A}} = \text{6,02} \cdot 10^{23}\ \text{g}/\text{mol}$ và khối lượng mol của Urani $^{238}_{92}\text U$ bằng $238\ \text{g/mol}$. Số nơtrôn có trong $119\ \text g$ Urani $^{238}_{92} \text U$ xấp xỉ bằng
		\begin{mcq} (4)
			\item $\text{8,8}\cdot 10^{25}$.	
			\item $\text{1,2}\cdot 10^{25}$.	
			\item $\text{2,2}\cdot 10^{25}$.	
			\item $\text{4,4}\cdot 10^{25}$.
		\end{mcq}}
	
		\loigiai{\textbf{Đáp án: D.}
			
			Số nơtron có trong 1 hạt Urani $^{238}_{92} \text U$:
			\begin{equation*}
				N=A-Z=238-92=146
			\end{equation*}
			
			Số hạt $^{238}_{92} \text U$ có trong $119\ \text g$:
			\begin{equation*}N_\text U = \dfrac{m}{M}N_\text A = \SI{3.01e23}{}.
			\end{equation*}
			Số nơtron có trong $119\ \text g$: $N\cdot N_\text U \approx \SI{4.4e25}{}$.
			
				}
	
	
\end{enumerate}

\loigiai{\textbf{Đáp án}
	\begin{center}
		\begin{tabular}{|m{2.8em}|m{2.8em}|m{2.8em}|m{2.8em}|m{2.8em}|m{2.8em}|m{2.8em}|m{2.8em}|m{2.8em}|m{2.8em}|}
			\hline
			1. D & 2. A & 3. B & 4. C & 5. A & 6. B & 7. D & 8. A & 9. B & 10. D \\
			\hline
			
			\hline
		\end{tabular}
\end{center}}


