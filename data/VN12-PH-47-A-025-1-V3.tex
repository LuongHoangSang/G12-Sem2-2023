\chapter[Viết phương trình phản ứng hạt nhân]{Viết phương trình phản ứng hạt nhân}
\section{Lý thuyết}

\subsection{Định nghĩa và đặc tính của phản ứng hạt nhân}
Phản ứng hạt nhân là mọi quá trình dẫn đến sự biến đổi hạt nhân
\begin{equation}
	A+B\rightarrow C+D,
\end{equation}
trong đó:
\begin{itemize}
	\item A, B là các hạt nhân tương tác;
	\item C, D là các hạt nhân sản phẩm.
\end{itemize}

\subsection{Định luật bảo toàn điện tích và định luật bảo toàn số khối trong phản ứng hạt nhân}
Xét phản ứng hạt nhân
\begin{equation}
	^{A_1}_{Z_1}A + ^{A_2}_{Z_2}B \rightarrow ^{A_3}_{Z_3}C + ^{A_4}_{Z_4}D
\end{equation}
\subsubsection{Định luật bảo toàn điện tích}
Trong phản ứng hạt nhân, tổng số đại số điện tích các hạt tương tác bằng tổng đại điện tích số các hạt sản phẩm
\begin{equation}
	Z_1+Z_2=Z_3+Z_4.
\end{equation}
\subsubsection{Định luật bảo toàn số khối}
Trong phản ứng hạt nhân, tổng số nuclon các hạt tương tác bằng tổng số nuclon các hạt sản phẩm
\begin{equation}
	A_1+A_2=A_3+A_4.
\end{equation}

\section{Mục tiêu bài học - Ví dụ minh họa}

\begin{dang}{Viết phương trình phản ứng hạt nhân.}
	\ppgiai{
		\begin{description}
			\item[Bước 1:] Viết phương trình phản ứng hạt nhân, xác định số nuclon, proton, nơtron của các hạt đã biết.
			\item[Bước 2:] Áp dụng định luật bảo toàn điện tích và định luật bảo toàn số khối để xác định cấu tạo của các hạt nhân còn thiếu trong phương trình.
		\end{description}
	}
	
	\viduii{2}{
		Cho phản ứng hạt nhân $^1_0 n + ^{235}_{\ 92} \text{U} \rightarrow ^{94}_{38} \text {Sr} + X + 2 ^1_0n$. Hạt nhân X có cấu tạo gồm
		\begin{mcq}(2)
			\item 54 proton và 86 nơtron.
			\item 54 proton và 140 nơtron.
			\item 86 proton và 140 nơtron.
			\item 86 proton và 54 nơtron.
	\end{mcq}}
	{\begin{center}
			\textbf{Hướng dẫn giải}
		\end{center}
		Áp dụng định luật bảo toàn điện tích và bảo toàn số khối
		\begin{equation*}
			\left\{
			\begin{matrix}
				0+92=38+Z_X+2\cdot0\\
				1+235=94+A_X+2\cdot1
			\end{matrix}
			\right.
			\Rightarrow
			\left\{
			\begin{matrix}
				Z_X=54\\
				A_X=140.
			\end{matrix}
			\right.
		\end{equation*}
		
		Vậy hạt nhân $X$ có cấu tạo gồm 54 proton và 86 nơtron.
		
		\begin{center}
			\textbf{Câu hỏi tương tự}
		\end{center}
		
		Cho phản ứng hạt nhân $ ^{238}_{922} \text{U} \longrightarrow ^{95}_{38} \text{Sr} + \text{X} + 3n $. Hạt nhân X có cấu tạo gồm
		\begin{mcq}(2)
			\item 54 proton và 86 nơtron.
			\item 54 proton và 140 nơtron.
			\item 86 proton và 140 nơtron.
			\item 86 proton và 54 nơtron.
		\end{mcq}
		
		\textbf{Đáp án:} A.
	}
	\viduii{2}
	{Cho hạt prôtôn bắn vào các hạt nhân $^9_4\text{Be}$ đang đứng yên, người ta thấy các hạt tạo thành gồm $^4_2\text{He}$ và hạt nhân X. Hạt nhân X có cấu tạo gồm
		\begin{mcq}(2)
			\item 3 proton và 3 nơtron.
			\item 3 proton và 6 nơtron.
			\item 3 proton và 6 nơtron.
			\item 2 proton và 3 nơtron.
		\end{mcq}
	}	{\begin{center}
			\textbf{Hướng dẫn giải}
		\end{center}
		Phương trình phản ứng hạt nhân
		\begin{equation*}
			^1_1 p+^9_4\text{Be}\rightarrow ^4_2\text{He}+^A_Z X
		\end{equation*}
		Áp dụng định luật bảo toàn điện tích và bảo toàn số khối
		\begin{equation*}
			\left\{
			\begin{matrix}
				1+4=2+Z_X\\
				1+9=4+A_X
			\end{matrix}
			\right.
			\Rightarrow
			\left\{
			\begin{matrix}
				Z_X=3\\
				A_X=6.
			\end{matrix}
			\right.
		\end{equation*}
		
		Vậy hạt nhân $X$ có cấu tạo gồm 3 proton và 3 nơtron.
		
		\begin{center}
			\textbf{Câu hỏi tương tự}
		\end{center}
		Cho hạt $ \alpha $ bắn vào một lá nhôm mỏng, người ta thấy các hạt tạo thành gồm proton và hạt nhân X. Hạt nhân X có cấu tạo gồm
		\begin{mcq}(2)
			\item 5 proton và 7 nơtron.
			\item 5 proton và 8 nơtron.
			\item 8 proton và 9 nơtron.
			\item 8 proton và 10 nơtron.
		\end{mcq}
		
		
		\textbf{Đáp án:} D.
	}
\end{dang}

\begin{dang}{Xác định tên hạt nhân còn thiếu.}
	\ppgiai{
		\begin{description}
			\item[Bước 1:] Viết phương trình phản ứng hạt nhân, xác định số nuclon, proton, nơtron của các hạt đã biết.
			\item[Bước 2:] Áp dụng định luật bảo toàn điện tích và định luật bảo toàn số khối để xác định cấu tạo của các hạt nhân còn thiếu trong phương trình.
			\item[Bước 3:] Dựa vào số hiệu nguyên tử xác định tên hạt nhân còn thiếu.
		\end{description}
	}
	
	\viduii{2}
	{
		Cho phản ứng hạt nhân $^{235}_{92} \text{U} + ^{1}_{0} \text{n} \longrightarrow ^{95}_{42} \text{Mo} + \text{X} + 2 \; ^{1}_{0} \text{n} + 4 \; ^{0}_{-1} \text{e}$, trong đó X là hạt nhân
		\begin{mcq}(4)
			\item $^{138}_{53} \text{I}$.
			\item $^{138}_{52} \text{Te}$.
			\item $^{139}_{54} \text{Xe}$.
			\item $^{139}_{57} \text{La}$.
		\end{mcq}
	}
	{
		\begin{center}
			\textbf{Hướng dẫn giải}
		\end{center}
		
		Số khối trước phản ứng: $A_\text{t} = 236$.
		
		Số khối sau phản ứng phải bằng số khối trước phản ứng:
		$$A_\text{s} = A_\text{t} \Rightarrow 95+A_\text X+2\cdot 2 + 4 \cdot 0 = 236 \Rightarrow A_\text X = 139$$
		
		Số điện tích trước phản ứng: $Z_\text{t} = 92$.
		
		Số điện tích sau phản ứng phải bằng số điện tích trước phản ứng:
		$$Z_\text{s} = Z_\text{t} \Rightarrow 42 + Z_\text{X} + 2 \cdot 0 + 4 \cdot(-1) = 92 \Rightarrow Z_\text{X} = 54$$
		
		Vậy X là hạt nhân $ ^{139}_{54} \text{Xe}$.
		
		\begin{center}
			\textbf{Câu hỏi tương tự}
		\end{center}
		
		Trong phản ứng hạt nhân $^{19}_{9} \text{F} + \text{p} \longrightarrow ^{16}_{8} \text{O} + \text{X}$, hạt X là
		\begin{mcq}(4)
			\item nơtron. 
			\item proton. 
			\item hạt $^{4}_{2} \text{He}$. 
			\item electron.
		\end{mcq}
		\textbf{Đáp án:} C.
	}
	
	\viduii{2}
	{
		Trong phản ứng hạt nhân $^{35}_{17} \text{Cl} + \text{X} \longrightarrow ^{32}_{16} \text{S} + ^{4}_{2} \text{He}$, hạt X là
		\begin{mcq}(4)
			\item $^{2}_{1} \text{H}$.
			\item $^{3}_{1} \text{H}$. 
			\item $^{1}_{1} \text{H}$. 
			\item $^{1}_{0} \text{n}$.
		\end{mcq}
	}
	{
		\begin{center}
			\textbf{Hướng dẫn giải}
		\end{center}
		
		Số khối sau phản ứng: $A_\text{s} = 36$.
		
		Số khối sau phản ứng phải bằng số khối trước phản ứng:
		$$A_\text{t} = A_\text{s} \Rightarrow 35+A_\text X = 36 \Rightarrow A_\text X = 1$$
		
		Số điện tích sau phản ứng: $Z_\text{s} = 18$.
		
		Số điện tích sau phản ứng phải bằng số điện tích trước phản ứng:
		$$Z_\text{t} = Z_\text{s} \Rightarrow 17 + Z_\text{X} = 18 \Rightarrow Z_\text{X} = 1$$
		
		Vậy X là hạt nhân $^{1}_{1} \text{H}$.
		
		\begin{center}
			\textbf{Câu hỏi tương tự}
		\end{center}
		
		Trong phản ứng hạt nhân $^{9}_{4} \text{Be} + \alpha \longrightarrow \text{X} + \text{n}$, hạt X là
		\begin{mcq}(4)
			\item $^{13}_{7} \text{N}$.
			\item $^{12}_{6} \text{C}$.
			\item $^{12}_{5} \text{B}$. 
			\item $^{16}_{8} \text{O}$.
		\end{mcq}
		\textbf{Đáp án:} B.
	}
	
\end{dang}
\section{Bài tập tự luyện}


\begin{enumerate}[label=\bfseries Câu \arabic*:]
	\item \mkstar{1} [2]
	\cauhoi
	{Chọn câu \textbf{sai} khi nói về phản ứng hạt nhân tỏa năng lượng.
		\begin{mcq}
			\item Năng lượng tỏa ra dưới dạng động năng của các hạt tạo thành.
			\item Các hạt tạo thành sau phản ứng bền vững hơn các hạt trước phản ứng.
			\item Tổng độ hụt khối của các hạt trước phản ứng lớn hơn tổng độ hụt khối của các hạt sau phản ứng.
			\item Tổng khối lượng các hạt trước phản ứng lớn hơn tổng khối lượng các hạt sau phản ứng.
		\end{mcq}
	}
	
	\loigiai
	{		\textbf{Đáp án: C.}
		
		Trong phản ứng hạt nhân tỏa năng lượng, tổng khối lượng các hạt trước phản ứng lớn hơn tổng khối lượng các hạt sau phản ứng, các hạt tạo thành sau phản ứng bền vững hơn các hạt trước phản ứng, năng lượng tỏa ra dưới dạng động năng của các hạt tạo thành.
		
		Trong phản ứng hạt nhân tỏa năng lượng, tổng độ hụt khối của các hạt trước phản ứng nhỏ hơn tổng độ hụt khối của các hạt sau phản ứng.
		
	}
	\item \mkstar{2} [4]
	\cauhoi
	{Cho phản ứng hạt nhân sau: $\ce{^1_1 H} + \ce{^9_4 Be} \longrightarrow \ce{^4_2 He} + \ce{^7_3 Li} + \SI{2.1}{MeV}$. Năng lượng tỏa ra từ phản ứng trên khi tổng hợp được $\SI{0.5}{mol}$ He là
		\begin{mcq}(4)
			\item $\SI{12.642e23}{MeV}$.
			\item $\SI{6.321e21}{MeV}$.
			\item $\SI{12.642e21}{MeV}$.
			\item $\SI{6.321e23}{MeV}$.
		\end{mcq}
	}
	
	\loigiai
	{		\textbf{Đáp án: D.}
		
		Số hạt nhân He có trong $\SI{0.5}{mol}$ là
		$$N=0,5 \cdot N_\text{A} = \SI{3.011e23}{}$$
		
		Mỗi phản ứng tỏa ra năng lượng là $\SI{2.1}{MeV}$, vậy năng lượng tỏa ra từ $\SI{3.011e23}{}$ phản ứng là
		$$\SI{3.011e23}{} \cdot \SI{2.1}{MeV} = \SI{6.321e23}{MeV}$$
		
	}
	\item \mkstar{2} [1]
	\cauhoi
	{Cho phản ứng hạt nhân $\ce{^235_92 U} + \ce{^1_0 n} \longrightarrow \ce{^95_42 Mo} + \ce{X} + 2\ce{^1_0 n} + 4\ce{^0_{-1} e}$, trong đó X là hạt nhân
		\begin{mcq}(4)
			\item $\ce{^138_53 I}$.
			\item $\ce{^138_52 Te}$.
			\item $\ce{^139_54 Xe}$.
			\item $\ce{^139_57 La}$.
		\end{mcq}
	}
	
	\loigiai
	{		\textbf{Đáp án: C.}
		
		Số khối trước phản ứng: $A_\text{t} = 236$.
		
		Số khối sau phản ứng phải bằng số khối trước phản ứng:
		$$A_\text{s} = A_\text{t} \Rightarrow 95+A_\text X+2\cdot 2 + 4 \cdot 0 = 236 \Rightarrow A_\text X = 139$$
		
		Số điện tích trước phản ứng: $Z_\text{t} = 92$.
		
		Số điện tích sau phản ứng phải bằng số điện tích trước phản ứng:
		$$Z_\text{s} = Z_\text{t} \Rightarrow 42 + Z_\text{X} + 2 \cdot 0 + 4 \cdot(-1) = 92 \Rightarrow Z_\text{X} = 54$$
		
		Vậy X là hạt nhân $\ce{^139_54 Xe}$.
		
	}
	
	\item \mkstar{2} [4]
	\cauhoi
	{Trong phản ứng hạt nhân $\ce{^19_9 F} + \ce{p} \longrightarrow \ce{^16_8 O} + \ce{X}$, hạt X là
		\begin{mcq}(4)
			\item nơtron. 
			\item proton. 
			\item hạt $\ce{^4_2 He}$. 
			\item electron.
		\end{mcq}
	}
	
	\loigiai
	{		\textbf{Đáp án: C.}
		
		Số khối trước phản ứng: $A_\text{t} = 20$.
		
		Số khối sau phản ứng phải bằng số khối trước phản ứng:
		$$A_\text{s} = A_\text{t} \Rightarrow 16+A_\text X = 20 \Rightarrow A_\text X = 4$$
		
		Số điện tích trước phản ứng: $Z_\text{t} = 10$.
		
		Số điện tích sau phản ứng phải bằng số điện tích trước phản ứng:
		$$Z_\text{s} = Z_\text{t} \Rightarrow 8 + Z_\text{X} = 10 \Rightarrow Z_\text{X} = 2$$
		
		Vậy X là hạt nhân $\ce{^4_2 He}$.
		
	}
	
	
	
	\item \mkstar{2} [12]
	\cauhoi
	{Trong phản ứng hạt nhân $\ce{^35_17 Cl} + \ce{X} \longrightarrow \ce{^32_16 S} + \ce{^4_2 He}$, hạt X là
		\begin{mcq}(4)
			\item $\ce{^2_1 H}$.
			\item $\ce{^3_1 H}$. 
			\item $\ce{^1_1 H}$. 
			\item $\ce{^1_0 n}$.
		\end{mcq}
	}
	
	\loigiai
	{		\textbf{Đáp án: C.}
		
		Số khối sau phản ứng: $A_\text{s} = 36$.
		
		Số khối sau phản ứng phải bằng số khối trước phản ứng:
		$$A_\text{t} = A_\text{s} \Rightarrow 35+A_\text X = 36 \Rightarrow A_\text X = 1$$
		
		Số điện tích sau phản ứng: $Z_\text{s} = 18$.
		
		Số điện tích sau phản ứng phải bằng số điện tích trước phản ứng:
		$$Z_\text{t} = Z_\text{s} \Rightarrow 17 + Z_\text{X} = 18 \Rightarrow Z_\text{X} = 1$$
		
		Vậy X là hạt nhân $\ce{^1_1 H}$.
		
	}
	\item \mkstar{2} [13]
	\cauhoi
	{Trong phản ứng hạt nhân $\ce{^9_4 Be} + \ce{\alpha} \longrightarrow \ce{X} + \ce{n}$, hạt X là
		\begin{mcq}(4)
			\item $\ce{^13_7 N}$.
			\item $\ce{^12_6 C}$.
			\item $\ce{^12_5 B}$. 
			\item $\ce{^16_8 O}$.
		\end{mcq}
	}
	
	\loigiai
	{		\textbf{Đáp án: B.}
		
		Số khối trước phản ứng: $A_\text{t} = 13$.
		
		Số khối sau phản ứng phải bằng số khối trước phản ứng:
		$$A_\text{s} = A_\text{t} \Rightarrow 1+A_\text X = 13 \Rightarrow A_\text X = 12$$
		
		Số điện tích trước phản ứng: $Z_\text{t} = 6$.
		
		Số điện tích sau phản ứng phải bằng số điện tích trước phản ứng:
		$$Z_\text{s} = Z_\text{t} \Rightarrow 0 + Z_\text{X} = 6 \Rightarrow Z_\text{X} = 6$$
		
		Vậy X là hạt nhân $\ce{^12_6 C}$.
		
	}
	\item \mkstar{2} [5]
	\cauhoi
	{Trong phản ứng hạt nhân $\ce{^3_1 T} + \ce{X} \longrightarrow \ce{\alpha} + \ce{n}$, hạt X là hạt nhân nào sau đây?
		\begin{mcq}(4)
			\item $\ce{^4_2 He}$. 
			\item $\ce{^2_1 D}$. 
			\item $\ce{^3_1 T}$. . 
			\item $\ce{^1_1 H}$. 
		\end{mcq}
	}
	
	\loigiai
	{		\textbf{Đáp án: B.}
		
		Số khối sau phản ứng: $A_\text{s} = 5$.
		
		Số khối sau phản ứng phải bằng số khối trước phản ứng:
		$$A_\text{t} = A_\text{s} \Rightarrow 3+A_\text X = 5 \Rightarrow A_\text X = 2$$
		
		Số điện tích sau phản ứng: $Z_\text{s} = 2$.
		
		Số điện tích sau phản ứng phải bằng số điện tích trước phản ứng:
		$$Z_\text{t} = Z_\text{s} \Rightarrow 1 + Z_\text{X} = 2 \Rightarrow Z_\text{X} = 1$$
		
		Vậy X là hạt nhân $\ce{^2_1 D}$.
		
	}
	\item \mkstar{2} [5]
	\cauhoi
	{Trong phản ứng hạt nhân $\ce{^37_17 Cl} + \ce{X} \longrightarrow \ce{^37_18 Ar} + \ce{n}$, hạt X là hạt nhân nào sau đây?
		\begin{mcq}(4)
			\item $\ce{^3_1 T}$. 
			\item $\ce{^4_2 He}$. 
			\item $\ce{^2_1 D}$.
			\item $\ce{^1_1 H}$. 
		\end{mcq}
	}
	
	\loigiai
	{		\textbf{Đáp án: D.}
		
		Số khối sau phản ứng: $A_\text{s} = 38$.
		
		Số khối sau phản ứng phải bằng số khối trước phản ứng:
		$$A_\text{t} = A_\text{s} \Rightarrow 37+A_\text X = 38 \Rightarrow A_\text X = 1$$
		
		Số điện tích sau phản ứng: $Z_\text{s} = 18$.
		
		Số điện tích sau phản ứng phải bằng số điện tích trước phản ứng:
		$$Z_\text{t} = Z_\text{s} \Rightarrow 17 + Z_\text{X} = 18 \Rightarrow Z_\text{X} = 1$$
		
		Vậy X là hạt nhân $\ce{^1_1 H}$.
		
	}
	\item \mkstar{2} [7]
	\cauhoi
	{Khi bắn phá hạt nhân $\ce{^14_7 N}$ bằng hạt $\ce{\alpha}$, người ta thu được một hạt proton và một hạt nhân $\ce{X}$. Hạt nhân X là
		\begin{mcq}(4)
			\item $\ce{^17_8 O}$.
			\item $\ce{^16_8 O}$.
			\item $\ce{^12_6 O}$.
			\item $\ce{^14_6 C}$.
		\end{mcq}
	}
	
	\loigiai
	{		\textbf{Đáp án: A.}
		
		Phương trình phản ứng:
		$$\ce{^14_7 N} + \ce{\alpha} \longrightarrow \ce{p} + \ce{X}$$
		
		Số khối trước phản ứng: $A_\text{t} = 18$.
		
		Số khối sau phản ứng phải bằng số khối trước phản ứng:
		$$A_\text{s} = A_\text{t} \Rightarrow 1+A_\text X = 18 \Rightarrow A_\text X = 17$$
		
		Số điện tích trước phản ứng: $Z_\text{t} = 9$.
		
		Số điện tích sau phản ứng phải bằng số điện tích trước phản ứng:
		$$Z_\text{s} = Z_\text{t} \Rightarrow 1 + Z_\text{X} = 9 \Rightarrow Z_\text{X} = 8$$
		
		Vậy X là hạt nhân $\ce{^17_8 O}$.
		
	}
	\item \mkstar{2} [9]
	\cauhoi
	{Trong phản ứng hạt nhân $\ce{\alpha} + \ce{^27_13 Al} \longrightarrow \ce{^30_15 p} + \ce{X}$, hạt X là hạt nhân nào sau đây?
		\begin{mcq}(4)
			\item proton.
			\item electron. 
			\item nơtron. 
			\item pôzitron.
		\end{mcq}
	}
	
	\loigiai
	{		\textbf{Đáp án: C.}
		
		Số khối trước phản ứng: $A_\text{t} = 31$.
		
		Số khối sau phản ứng phải bằng số khối trước phản ứng:
		$$A_\text{s} = A_\text{t} \Rightarrow 30+A_\text X = 31 \Rightarrow A_\text X = 1$$
		
		Số điện tích trước phản ứng: $Z_\text{t} = 15$.
		
		Số điện tích sau phản ứng phải bằng số điện tích trước phản ứng:
		$$Z_\text{s} = Z_\text{t} \Rightarrow 15 + Z_\text{X} = 15 \Rightarrow Z_\text{X} = 0$$
		
		Vậy X là hạt nhân $\ce{^1_0 n}$.
		
	}
	\item \mkstar{2} [4]
	\cauhoi
	{Cho phản ứng hạt nhân $\ce{^4_2 He} + \ce{^27_13 Al} \longrightarrow \ce{^30_15 P} + \ce{n}$, khối lượng của các hạt nhân là $m_{\ce{He}} = \SI{4.0015}{u}$, $m_{\ce{Al}} = \SI{26.97435}{u}$, $m_{\ce{P}} = \SI{29.97005}{u}$, $m_{\ce{n}} = \SI{1.00867}{u}$, $\SI{1}{u} = \SI{931.5}{MeV/c^2}$. Năng lượng mà phản ứng này tỏa ra hoặc thu vào là bao nhiêu?
		\begin{mcq}(2)
			\item Tỏa ra $\SI{5.022648}{MeV}$.
			\item Tỏa ra $\SI{5.022648e-13}{J}$.
			\item Thu vào $\SI{2.673405}{MeV}$.
			\item Thu vào $\SI{2.673405e-13}{J}$.
		\end{mcq}
	}
	
	\loigiai
	{		\textbf{Đáp án: C.}
		
		Năng lượng tỏa ra hoặc thu vào:
		$$Q=(m_{\text{t}} - m_\text{s}) c^2 = \SI{-2.673405}{MeV}$$
		
		Vì $Q<0$ nên phản ứng thu năng lượng.
		
	}
	
	\item \mkstar{2} [2]
	\cauhoi
	{Cho phản ứng hạt nhân $\ce{^55_25 Mn} + \ce{p} \longrightarrow \ce{^55_26 Fe} + \ce{n}$, khối lượng của các hạt nhân là $m_{\ce{Mn}} = \SI{54.9381}{u}$, $m_{\ce{Fe}} = \SI{54.9380}{u}$, $m_{\ce{p}} = \SI{1.0073}{u}$, $m_{\ce{n}} = \SI{1.0087}{u}$, $\SI{1}{u} = \SI{931.5}{MeV/c^2}$. Phản ứng trên
		\begin{mcq}(2)
			\item tỏa năng lượng $\SI{12.1095}{MeV}$.
			\item tỏa năng lượng $\SI{1.21095}{MeV}$.
			\item thu năng lượng $\SI{12.1095}{MeV}$.
			\item thu năng lượng $\SI{1.21095}{MeV}$.
		\end{mcq}
	}
	
	\loigiai
	{		\textbf{Đáp án: D.}
		
		Năng lượng tỏa ra hoặc thu vào:
		$$Q=(m_{\text{t}} - m_\text{s}) c^2 = \SI{-1.21095}{MeV}$$
		
		Vì $Q<0$ nên phản ứng thu năng lượng.
		
	}
	\item \mkstar{2} [2]
	\cauhoi
	{Giả sử trong một phản ứng hạt nhân, tổng khối lương hai hạt trước phản ứng lớn hơn tổng khối lượng hai hạt sau phản ứng là $\SI{0.02}{u}$. Cho $\SI{1}{u} = \SI{931.5}{MeV/c^2}$. Phản ứng hạt nhân này
		\begin{mcq}(2)
			\item tỏa năng lượng $\SI{1.863}{MeV}$.
			\item thu năng lượng $\SI{1.863}{MeV}$.
			\item tỏa năng lượng $\SI{18.63}{MeV}$.
			\item thu năng lượng $\SI{18.63}{MeV}$.
		\end{mcq}
	}
	
	\loigiai
	{		\textbf{Đáp án: C.}
		
		Vì $m_{\text{t}}>m_{\text{s}}$ nên phản ứng hạt nhân tỏa năng lượng:
		
		Năng lượng tỏa ra:
		$$Q=(m_{\text{t}} - m_\text{s}) c^2 = \SI{18.63}{MeV}$$
		
	}
	
	\item \mkstar{2} [3]
	\cauhoi
	{Cho phản ứng hạt nhân $\ce{^23_11 Na} + \ce{^1_1 H} \longrightarrow \ce{^4_2 He} + \ce{^20_10 Ne}$. Lấy khối lượng của các hạt nhân $\ce{^23_11 Na}$, $\ce{^20_10 Ne}$, $\ce{^4_2 He}$, $\ce{^1_1 H}$ lần lượt là $\SI{22.9837}{u}$, $\SI{19.9869}{u}$, $\SI{4.0015}{u}$, $\SI{1.0073}{u}$ và $\SI{1}{u} = \SI{931.5}{MeV/c^2}$. Năng lượng của phản ứng này
		\begin{mcq}(2)
			\item thu vào $\SI{2.4219}{MeV}$.
			\item thu vào $\SI{3.4524}{MeV}$.
			\item tỏa ra $\SI{3.4524}{MeV}$.
			\item tỏa ra $\SI{2.4219}{MeV}$.
		\end{mcq}
	}
	
	\loigiai
	{		\textbf{Đáp án: D.}
		
		Năng lượng tỏa ra hoặc thu vào:
		$$Q=(m_{\text{t}} - m_\text{s}) c^2 = \SI{2.4219}{MeV}$$
		
		Vì $Q>0$ nên phản ứng tỏa năng lượng.
	}
	\item \mkstar{2} [9]
	\cauhoi
	{Xét một phản ứng hạt nhân $\ce{^2_1 H} + \ce{^2_1 H} \longrightarrow \ce{^3_2 He} + \ce{^1_0 n}$. Biết $m_{\ce{H}} = \SI{2.0135}{u}$, $m_{\ce{He}} = \SI{3.0149}{u}$, $m_{\ce{n}} = \SI{1.0087}{u}$, lấy $\SI{1}{u} = \SI{931.5}{MeV/c^2}$. Phản ứng trên
		\begin{mcq}(2)
			\item tỏa năng lượng $\SI{3.1671}{MeV}$.
			\item thu năng lượng $\SI{3.1671}{MeV}$.
			\item tỏa năng lượng $\SI{6.0371}{MeV}$.
			\item thu năng lượng $\SI{6.0371}{MeV}$.
		\end{mcq}
	}
	
	\loigiai
	{		\textbf{Đáp án: A.}
		
		Năng lượng tỏa ra hoặc thu vào:
		$$Q=(m_{\text{t}} - m_\text{s}) c^2 = \SI{3.1671}{MeV}$$
		
		Vì $Q>0$ nên phản ứng tỏa năng lượng.
	}

\item \mkstar{1} [2]
\cauhoi
{Chọn câu \textbf{sai} khi nói về phản ứng hạt nhân tỏa năng lượng.
	\begin{mcq}
		\item Năng lượng tỏa ra dưới dạng động năng của các hạt tạo thành.
		\item Các hạt tạo thành sau phản ứng bền vững hơn các hạt trước phản ứng.
		\item Tổng độ hụt khối của các hạt trước phản ứng lớn hơn tổng độ hụt khối của các hạt sau phản ứng.
		\item Tổng khối lượng các hạt trước phản ứng lớn hơn tổng khối lượng các hạt sau phản ứng.
	\end{mcq}
}

\loigiai
{		\textbf{Đáp án: C.}
	
	Trong phản ứng hạt nhân tỏa năng lượng, tổng khối lượng các hạt trước phản ứng lớn hơn tổng khối lượng các hạt sau phản ứng, các hạt tạo thành sau phản ứng bền vững hơn các hạt trước phản ứng, năng lượng tỏa ra dưới dạng động năng của các hạt tạo thành.
	
	Trong phản ứng hạt nhân tỏa năng lượng, tổng độ hụt khối của các hạt trước phản ứng nhỏ hơn tổng độ hụt khối của các hạt sau phản ứng.
	
}
\item \mkstar{2} [4]
\cauhoi
{Cho phản ứng hạt nhân sau: $\ce{^1_1 H} + \ce{^9_4 Be} \longrightarrow \ce{^4_2 He} + \ce{^7_3 Li} + \SI{2.1}{MeV}$. Năng lượng tỏa ra từ phản ứng trên khi tổng hợp được $\SI{0.5}{mol}$ He là
	\begin{mcq}(4)
		\item $\SI{12.642e23}{MeV}$.
		\item $\SI{6.321e21}{MeV}$.
		\item $\SI{12.642e21}{MeV}$.
		\item $\SI{6.321e23}{MeV}$.
	\end{mcq}
}

\loigiai
{		\textbf{Đáp án: D.}
	
	Số hạt nhân He có trong $\SI{0.5}{mol}$ là
	$$N=0,5 \cdot N_\text{A} = \SI{3.011e23}{}$$
	
	Mỗi phản ứng tỏa ra năng lượng là $\SI{2.1}{MeV}$, vậy năng lượng tỏa ra từ $\SI{3.011e23}{}$ phản ứng là
	$$\SI{3.011e23}{} \cdot \SI{2.1}{MeV} = \SI{6.321e23}{MeV}$$
	
}
\item \mkstar{2} [1]
\cauhoi
{Cho phản ứng hạt nhân $\ce{^235_92 U} + \ce{^1_0 n} \longrightarrow \ce{^95_42 Mo} + \ce{X} + 2\ce{^1_0 n} + 4\ce{^0_{-1} e}$, trong đó X là hạt nhân
	\begin{mcq}(4)
		\item $\ce{^138_53 I}$.
		\item $\ce{^138_52 Te}$.
		\item $\ce{^139_54 Xe}$.
		\item $\ce{^139_57 La}$.
	\end{mcq}
}

\loigiai
{		\textbf{Đáp án: C.}
	
	Số khối trước phản ứng: $A_\text{t} = 236$.
	
	Số khối sau phản ứng phải bằng số khối trước phản ứng:
	$$A_\text{s} = A_\text{t} \Rightarrow 95+A_\text X+2\cdot 2 + 4 \cdot 0 = 236 \Rightarrow A_\text X = 139$$
	
	Số điện tích trước phản ứng: $Z_\text{t} = 92$.
	
	Số điện tích sau phản ứng phải bằng số điện tích trước phản ứng:
	$$Z_\text{s} = Z_\text{t} \Rightarrow 42 + Z_\text{X} + 2 \cdot 0 + 4 \cdot(-1) = 92 \Rightarrow Z_\text{X} = 54$$
	
	Vậy X là hạt nhân $\ce{^139_54 Xe}$.
	
}

\item \mkstar{2} [4]
\cauhoi
{Trong phản ứng hạt nhân $\ce{^19_9 F} + \ce{p} \longrightarrow \ce{^16_8 O} + \ce{X}$, hạt X là
	\begin{mcq}(4)
		\item nơtron. 
		\item proton. 
		\item hạt $\ce{^4_2 He}$. 
		\item electron.
	\end{mcq}
}

\loigiai
{		\textbf{Đáp án: C.}
	
	Số khối trước phản ứng: $A_\text{t} = 20$.
	
	Số khối sau phản ứng phải bằng số khối trước phản ứng:
	$$A_\text{s} = A_\text{t} \Rightarrow 16+A_\text X = 20 \Rightarrow A_\text X = 4$$
	
	Số điện tích trước phản ứng: $Z_\text{t} = 10$.
	
	Số điện tích sau phản ứng phải bằng số điện tích trước phản ứng:
	$$Z_\text{s} = Z_\text{t} \Rightarrow 8 + Z_\text{X} = 10 \Rightarrow Z_\text{X} = 2$$
	
	Vậy X là hạt nhân $\ce{^4_2 He}$.
	
}



\item \mkstar{2} [12]
\cauhoi
{Trong phản ứng hạt nhân $\ce{^35_17 Cl} + \ce{X} \longrightarrow \ce{^32_16 S} + \ce{^4_2 He}$, hạt X là
	\begin{mcq}(4)
		\item $\ce{^2_1 H}$.
		\item $\ce{^3_1 H}$. 
		\item $\ce{^1_1 H}$. 
		\item $\ce{^1_0 n}$.
	\end{mcq}
}

\loigiai
{		\textbf{Đáp án: C.}
	
	Số khối sau phản ứng: $A_\text{s} = 36$.
	
	Số khối sau phản ứng phải bằng số khối trước phản ứng:
	$$A_\text{t} = A_\text{s} \Rightarrow 35+A_\text X = 36 \Rightarrow A_\text X = 1$$
	
	Số điện tích sau phản ứng: $Z_\text{s} = 18$.
	
	Số điện tích sau phản ứng phải bằng số điện tích trước phản ứng:
	$$Z_\text{t} = Z_\text{s} \Rightarrow 17 + Z_\text{X} = 18 \Rightarrow Z_\text{X} = 1$$
	
	Vậy X là hạt nhân $\ce{^1_1 H}$.
	
}
\item \mkstar{2} [13]
\cauhoi
{Trong phản ứng hạt nhân $\ce{^9_4 Be} + \ce{\alpha} \longrightarrow \ce{X} + \ce{n}$, hạt X là
	\begin{mcq}(4)
		\item $\ce{^13_7 N}$.
		\item $\ce{^12_6 C}$.
		\item $\ce{^12_5 B}$. 
		\item $\ce{^16_8 O}$.
	\end{mcq}
}

\loigiai
{		\textbf{Đáp án: B.}
	
	Số khối trước phản ứng: $A_\text{t} = 13$.
	
	Số khối sau phản ứng phải bằng số khối trước phản ứng:
	$$A_\text{s} = A_\text{t} \Rightarrow 1+A_\text X = 13 \Rightarrow A_\text X = 12$$
	
	Số điện tích trước phản ứng: $Z_\text{t} = 6$.
	
	Số điện tích sau phản ứng phải bằng số điện tích trước phản ứng:
	$$Z_\text{s} = Z_\text{t} \Rightarrow 0 + Z_\text{X} = 6 \Rightarrow Z_\text{X} = 6$$
	
	Vậy X là hạt nhân $\ce{^12_6 C}$.
	
}
\item \mkstar{2} [5]
\cauhoi
{Trong phản ứng hạt nhân $\ce{^3_1 T} + \ce{X} \longrightarrow \ce{\alpha} + \ce{n}$, hạt X là hạt nhân nào sau đây?
	\begin{mcq}(4)
		\item $\ce{^4_2 He}$. 
		\item $\ce{^2_1 D}$. 
		\item $\ce{^3_1 T}$. . 
		\item $\ce{^1_1 H}$. 
	\end{mcq}
}

\loigiai
{		\textbf{Đáp án: B.}
	
	Số khối sau phản ứng: $A_\text{s} = 5$.
	
	Số khối sau phản ứng phải bằng số khối trước phản ứng:
	$$A_\text{t} = A_\text{s} \Rightarrow 3+A_\text X = 5 \Rightarrow A_\text X = 2$$
	
	Số điện tích sau phản ứng: $Z_\text{s} = 2$.
	
	Số điện tích sau phản ứng phải bằng số điện tích trước phản ứng:
	$$Z_\text{t} = Z_\text{s} \Rightarrow 1 + Z_\text{X} = 2 \Rightarrow Z_\text{X} = 1$$
	
	Vậy X là hạt nhân $\ce{^2_1 D}$.
	
}
\item \mkstar{2} [5]
\cauhoi
{Trong phản ứng hạt nhân $\ce{^37_17 Cl} + \ce{X} \longrightarrow \ce{^37_18 Ar} + \ce{n}$, hạt X là hạt nhân nào sau đây?
	\begin{mcq}(4)
		\item $\ce{^3_1 T}$. 
		\item $\ce{^4_2 He}$. 
		\item $\ce{^2_1 D}$.
		\item $\ce{^1_1 H}$. 
	\end{mcq}
}

\loigiai
{		\textbf{Đáp án: D.}
	
	Số khối sau phản ứng: $A_\text{s} = 38$.
	
	Số khối sau phản ứng phải bằng số khối trước phản ứng:
	$$A_\text{t} = A_\text{s} \Rightarrow 37+A_\text X = 38 \Rightarrow A_\text X = 1$$
	
	Số điện tích sau phản ứng: $Z_\text{s} = 18$.
	
	Số điện tích sau phản ứng phải bằng số điện tích trước phản ứng:
	$$Z_\text{t} = Z_\text{s} \Rightarrow 17 + Z_\text{X} = 18 \Rightarrow Z_\text{X} = 1$$
	
	Vậy X là hạt nhân $\ce{^1_1 H}$.
	
}
\item \mkstar{2} [7]
\cauhoi
{Khi bắn phá hạt nhân $\ce{^14_7 N}$ bằng hạt $\ce{\alpha}$, người ta thu được một hạt proton và một hạt nhân $\ce{X}$. Hạt nhân X là
	\begin{mcq}(4)
		\item $\ce{^17_8 O}$.
		\item $\ce{^16_8 O}$.
		\item $\ce{^12_6 O}$.
		\item $\ce{^14_6 C}$.
	\end{mcq}
}

\loigiai
{		\textbf{Đáp án: A.}
	
	Phương trình phản ứng:
	$$\ce{^14_7 N} + \ce{\alpha} \longrightarrow \ce{p} + \ce{X}$$
	
	Số khối trước phản ứng: $A_\text{t} = 18$.
	
	Số khối sau phản ứng phải bằng số khối trước phản ứng:
	$$A_\text{s} = A_\text{t} \Rightarrow 1+A_\text X = 18 \Rightarrow A_\text X = 17$$
	
	Số điện tích trước phản ứng: $Z_\text{t} = 9$.
	
	Số điện tích sau phản ứng phải bằng số điện tích trước phản ứng:
	$$Z_\text{s} = Z_\text{t} \Rightarrow 1 + Z_\text{X} = 9 \Rightarrow Z_\text{X} = 8$$
	
	Vậy X là hạt nhân $\ce{^17_8 O}$.
	
}
\item \mkstar{2} [9]
\cauhoi
{Trong phản ứng hạt nhân $\ce{\alpha} + \ce{^27_13 Al} \longrightarrow \ce{^30_15 p} + \ce{X}$, hạt X là hạt nhân nào sau đây?
	\begin{mcq}(4)
		\item proton.
		\item electron. 
		\item nơtron. 
		\item pôzitron.
	\end{mcq}
}

\loigiai
{		\textbf{Đáp án: C.}
	
	Số khối trước phản ứng: $A_\text{t} = 31$.
	
	Số khối sau phản ứng phải bằng số khối trước phản ứng:
	$$A_\text{s} = A_\text{t} \Rightarrow 30+A_\text X = 31 \Rightarrow A_\text X = 1$$
	
	Số điện tích trước phản ứng: $Z_\text{t} = 15$.
	
	Số điện tích sau phản ứng phải bằng số điện tích trước phản ứng:
	$$Z_\text{s} = Z_\text{t} \Rightarrow 15 + Z_\text{X} = 15 \Rightarrow Z_\text{X} = 0$$
	
	Vậy X là hạt nhân $\ce{^1_0 n}$.
	
}

\end{enumerate}


