\chapter{Phóng xạ và các dạng bài tập}
\section{Dạng 1: Bài toán áp dụng định luật phóng xạ}
Xét một mẫu phóng xạ: $\text{X} \longrightarrow \text{Y} + \text{tia phóng xạ}$.

Gọi $N_0$, $m_0$  lần lượt là số hạt nhân và khối lượng của mẫu ban đầu.

\textbf{- Số hạt nhân và khối lượng phóng xạ còn lại:}
\begin{equation}
	\begin{cases}
		N = N_0 \cdot 2^{-\dfrac{t}{T}} = N_0 \cdot e^{-\lambda t}.\\
		m=m_0 \cdot 2^{-\dfrac{t}{T}} = m_0\cdot e^{-\lambda t}. 
	\end{cases}
\end{equation}

Trong đó: $N$, $m$ lần lượt là số hạt nhân, khối lượng của mẫu phóng xạ còn lại sau thời gian $t$.

\textbf{- Số hạt nhân và khối lượng phóng xạ đã bị phân rã:}
\begin{equation}
	\begin{cases}
		\Delta N = N_0 - N = N_0 \left (1-2^{-\dfrac{t}{T}}\right) = N_0 (1-e^{-\lambda t}).\\
		\Delta m=m_0-m=m_0 \left(1-2^{-\dfrac{t}{T}}\right)= m_0(1-e^{-\lambda t}).
	\end{cases}
\end{equation}

Trong đó: $\Delta N$, $\Delta m$  lần lượt là số hạt nhân, khối lượng của mẫu đã bị phân rã.

\textbf{- Phần trăm số hạt, khối lượng phóng xạ còn lại:}  
\begin{equation}
	\dfrac{N}{N_0} =\dfrac{m}{m_0} = \dfrac{H}{H_0} = 2^{-\dfrac{t}{T}} = e^{-\lambda t}.
\end{equation}

\textbf{- Phần trăm số hạt, khối lượng phóng xạ bị phân rã: }
\begin{equation}
	\dfrac{\Delta N}{N_0} =\dfrac{\Delta m}{m_0} = \dfrac{H}{H_0} = 1- 2^{-\dfrac{t}{T}} = 1- e^{-\lambda t}.
\end{equation}

\luuy{Mối liên hệ về số hạt và khối lượng: 
	\begin{equation}
		N=m \cdot N_{\text{A}} = \dfrac{m}{A}N_{\text {A}}.
	\end{equation} 
	Trong đó: $n$ là số mol, $N_{\text{A}}= \text{6,02}\cdot\  \text{mol}^{-1}$  là số Avôgađrô.
}
\begin{enumerate}
	\item {Ban đầu có 100 g lượng chất phóng xạ $^{60}_{27}Co$ với chu kì bán rã $T=\text{5,33}$ năm. Sau 25 năm, khối lượng và số hạt Coban còn lại bao nhiêu?
		\begin{mcq}(2)
			\item $m=\text{3,873}\ \text{g}$; $N=\text{0,389} \cdot 10^{23}$ hạt.	
			\item $m=\text{2,873}\ \text{g}$; $N=\text{0,286} \cdot 10^{23}$ hạt.	 
			\item $m=\text{4,873}\ \text{g}$; $N=\text{0,490} \cdot 10^{23}$ hạt.		
			\item  $m=\text{3,365}\ \text{g}$; $N=\text{0,338} \cdot 10^{23}$ hạt.	
		\end{mcq}
	}
	\item{Cho 2 gam $^{60}_{27}Co$ tinh khiết có phóng xạ $\beta^{-}$ với chu kỳ bán rã là 5,33 năm. Sau 15 năm, khối lượng $^{60}_{27}Co$ còn lại là
		\begin{mcq}(4)
			\item  0,284 g.	
			\item  0,842 g.	
			\item  0,482 g.	
			\item  0,248 g.
		\end{mcq}
	}
	\item{Gọi $\Delta t$ là khoảng thời gian để số hạt nhân của một chất phóng xạ giảm 4 lần. Sau $2\Delta t$  thì số hạt nhân còn lại bằng bao nhiêu phần trăm ban đầu?
		\begin{mcq}(4)
			\item  25,25$\%$.	
			\item  93,75$\%$.	
			\item  13,5$\%$.	
			\item  6,25$\%$.
		\end{mcq}
	}
	\item{Một chất phóng xạ $^{210}_{84} Po$ chu kỳ bán rã là 138 ngày, ban đầu mẫu chất phóng xạ nguyên chất. Sau thời gian $t$ ngày thì số prôtôn có trong mẫu phóng xạ còn lại là $N_1$. Tiếp sau đó $\Delta t$ ngày thì số nơtrôn có trong mẫu phóng xạ còn lại là $N_2$, biết $N_1=\text{1,158}N_2$. Giá trị của $\Delta t$ gần đúng bằng
		\begin{mcq}(4)
			\item  140 ngày	
			\item  130 ngày 	
			\item  120 ngày	
			\item  110 ngày
		\end{mcq}
	}
	\item {Chu kỳ bán rã của hai chất phóng xạ A, B là 20 phút và 40 phút. Ban đầu hai chất phóng xạ có số hạt nhân bằng nhau. Sau 80 phút thì tỉ số các hạt A và B bị phân rã là
		\begin{mcq}(4)
			\item $\dfrac{4}{5}$.	
			\item $\dfrac{5}{4}$.	
			\item $4$.	
			\item $\dfrac{1}{4}$.
		\end{mcq}
	}
	\item{[Trích đề thi THPT QG năm 2018] Pôlôni $^{210}_{84} Po$ là chất phóng xạ $\alpha$. Ban đầu có một mẫu $^{210}_{84}Po$  nguyên chất. Khối lượng $^{210}_{84}Po$ trong mẫu ở các thời điểm $t=t_0$, $t=t_0 +2\Delta t$ và $t=t_0 +3\Delta t (\Delta t>0)$ có giá trị lần lượt là $m_0$, 8g và 1g. Giá trị của $m_0$ là
		\begin{mcq}(4)
			\item 256 g.	
			\item 128 g.	
			\item 64 g.	
			\item 512 g.
		\end{mcq}
	}
	\item{Chu kì bán rã của chất phóng xạ $^{90}_{38}Sr$ là 20 năm. Sau 80 năm có bao nhiêu phần trăm chất phóng xạ đó phân rã thành chất khác?
		\begin{mcq}(4)
			\item  6,25$\%$.	
			\item  12,5$\%$.	
			\item  87,5$\%$.	
			\item  93,75$\%$.
		\end{mcq}
	}
\end{enumerate}

\textbf{ĐÁP ÁN}
\begin{longtable}[\textwidth]{|p{0.1\textwidth}|p{0.1\textwidth}|p{0.1\textwidth}|p{0.1\textwidth}|p{0.1\textwidth}|p{0.1\textwidth}|p{0.1\textwidth}|p{0.1\textwidth}|}
	% --- first head
	\hline%\hspace{2 pt}
	\multicolumn{1}{|c|}{\textbf{Câu 1}} & \multicolumn{1}{c|}{\textbf{Câu 2}} & \multicolumn{1}{c|}{\textbf{Câu 3}} &
	\multicolumn{1}{c|}{\textbf{Câu 4}} &
	\multicolumn{1}{c|}{\textbf{Câu 5}} &
	\multicolumn{1}{c|}{\textbf{Câu 6}} &
	\multicolumn{1}{c|}{\textbf{Câu 7}} &
	\multicolumn{1}{c|}{\textbf{Câu 8}} \\
	\hline
	A.&A. &D. &D. &B. &D. &D. &	\\
	\hline
	
\end{longtable} 

\section{Dạng 2: Bài toán số hạt nhân và khối lượng hạt nhân con tạo thành}	
\begin{enumerate}
	\item {Một hạt $^{226}Ra$ phân rã chuyển thành hạt nhân $^{222} Rn$. Xem khối lượng bằng số khối. Nếu có 226g $^{226}Ra$ thì sau 2 chu kì bán rã khối lượng $^{222} Rn$ tạo thành là:
		\begin{mcq}(4)
			\item 55,5 g.	
			\item 56,5 g.	
			\item 169,5 g.	
			\item 166,5 g.
		\end{mcq}
	}
	\item {Hạt nhân X phóng xạ biến đổi thành hạt nhân bên Y. Ban đầu ($t = 0$) có một mẫu chất X nguyên chất. Tại thời điểm $t_1$ và $t_2$ tỉ số giữa số hạt nhân Y và số hạt nhân X ở trong mẫu tương ứng là 2 và 3. Tại thời điểm $t_3 =2t_1 +3t_2$, tỉ số đó là
		\begin{mcq}(4)
			\item 17.	
			\item 575. 	
			\item 107.	
			\item 72.
		\end{mcq}
	}
	\item{Chất polonium $^{210}_{84} Po$ phóng xạ anpha ($\alpha$) và chuyển thành chì $^{206}_{82} Pb$ với chu kỳ bán rã là 138,4 ngày. Biết tại điều kiện tiêu chuẩn, mỗi mol khí chiếm một thể tích là $\text{22,4}\ l$. Nếu ban đầu có 5 g chất $^{210}_{84} Po$ tinh khiết thì thể tích khí $He$ ở điều kiện tiêu chuẩn sinh ra sau một năm là
		\begin{mcq} (4)
			\item $\text{0,484}\ l$.
			\item $\text{8,44}\ l$.
			\item $\text{0,884}\ l$.
			\item $\text{0,448}\ l$.
		\end{mcq}
	}
	\item{Đồng vị $^{210}_{84} Po$ phóng xạ $\alpha$ tạo thành chì $^{206}_{86}Pb$. Ban đầu trong một mẫu chất $Po$ có khối lượng 1 mg. Tại thời điểm $t_1$ tỉ lệ giữa số hạt $Pb$ và số hạt $Po$ trong mẫu là 7 : 1. Tại thời điểm $t_2=t_1 + 414$ ngày thì tỉ lệ đó là 63:1. Chu kỳ phóng xạ của $Po$ là
		\begin{mcq}(4)
			\item 138,0 ngày.	
			\item 138,4 ngày.	
			\item 137,8 ngày.	
			\item 138,5 ngày.
		\end{mcq}
	}
	\item{[Trích đề thi THPT QG năm 2009] Lấy chu kì bán rã của pôlôni $^{210}_{84}Po$ là 138 ngày và $N_{\text{A}} = \text{6,02} \cdot 10^{23}\ \text{mol}^{-1}$. Độ phóng xạ của 42 mg Pôlôni là 
		\begin{mcq}(4)
			\item $7 \cdot 10^{12}\ \text{Bq}$.
			\item $7 \cdot 10^{10}\ \text{Bq}$.
			\item $7 \cdot 10^{14}\ \text{Bq}$.
			\item $7 \cdot 10^{9}\ \text{Bq}$.
		\end{mcq}
	}
\end{enumerate}
\textbf{ĐÁP ÁN}
\begin{longtable}[\textwidth]{|p{0.1\textwidth}|p{0.1\textwidth}|p{0.1\textwidth}|p{0.1\textwidth}|p{0.1\textwidth}|p{0.1\textwidth}|p{0.1\textwidth}|p{0.1\textwidth}|}
	% --- first head
	\hline%\hspace{2 pt}
	\multicolumn{1}{|c|}{\textbf{Câu 1}} & \multicolumn{1}{c|}{\textbf{Câu 2}} & \multicolumn{1}{c|}{\textbf{Câu 3}} &
	\multicolumn{1}{c|}{\textbf{Câu 4}} &
	\multicolumn{1}{c|}{\textbf{Câu 5}} &
	\multicolumn{1}{c|}{\textbf{Câu 6}} &
	\multicolumn{1}{c|}{\textbf{Câu 7}} &
	\multicolumn{1}{c|}{\textbf{Câu 8}} \\
	\hline
	D.&B. &D. &D. &A. & & &	\\
	\hline
	
\end{longtable} 