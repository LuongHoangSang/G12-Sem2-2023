
\chapter[Năng lượng liên kết, \\năng lượng liên kết riêng của hạt nhân]{Năng lượng liên kết,\\ năng lượng liên kết riêng của hạt nhân}
\section{Lý thuyết}

\subsection{Độ hụt khối}
Khối lượng của một hạt nhân luôn nhỏ hơn tổng khối lượng của các nuclon tạo thành hạt nhân đó. Độ chênh lệch giữa hai khối lượng đó được gọi là độ hụt khối của hạt nhân
\begin{equation}
	\Delta m = m_0 - m_X = Z m_p+ (A-Z) m_n - m_X,
\end{equation}
trong đó:
\begin{itemize}
	\item $m_0=Z m_p+ (A-Z) m_n$ là tổng khối lượng các nuclon lúc đầu chưa liên kết;
	\item $m_X$ khối lượng của hạt nhân $^A_Z X$;
	\item $m_p$ và $m_n$ lần lượt là khối lượng của các proton và nơtron.
\end{itemize}
\subsection{Năng lượng liên kết}
Năng lượng liên kết của hạt nhân được tính bằng tích của độ hụt khối của hạt nhân với bình phương vận tốc ánh sáng
\begin{equation}
	W_\text{lk}=\Delta m c^2 = \left[ Z m_p+ (A-Z) m_n - m_X \right] c^2,
\end{equation}
trong đó:
\begin{itemize}
	\item $m_0=Z m_p+ (A-Z) m_n$ là tổng khối lượng các nuclon lúc đầu chưa liên kết;
	\item $m_X$ là khối lượng của hạt nhân $^A_Z X$;
	\item $m_p$ và $m_n$ lần lượt là khối lượng của các proton và nơtron;
	\item $c$ là tốc độ ánh sáng trong chân không.
\end{itemize}
\subsection{Năng lượng liên kết riêng}
Mức độ bền vững của một hạt nhân không những phụ thuộc vào năng lượng liên kết mà còn phụ thuộc vào số nuclon của hạt nhân đó. Vì vậy người ta định nghĩa năng lượng liên kết riêng là thương số giữa năng lượng liên kết và số nuclon
\begin{equation}
	W_\text{lkr}=\dfrac{W_\text{lk}}{A},
\end{equation}
trong đó:
\begin{itemize}
	\item $W_\text{lkr}$ là năng lượng liên kết riêng;
	\item $W_\text{lk}$ là năng lượng liên kết;
	\item A là số khối.
\end{itemize}

\section{Mục tiêu bài học - Ví dụ minh họa}

\begin{dang}{Áp dụng trực tiếp công thức.}
	
	\ppgiai{
		Áp dụng công thức tính độ hụt khối:
		$$
		Z m_p+ (A-Z) m_n - m_X.
		$$
		Áp dụng công thức tính năng lượng liên kết:
		$$
		W_\text{lk}=\Delta m c^2 = \left[ Z m_p+ (A-Z) m_n - m_X \right] c^2.
		$$
		Áp dụng công thức tính năng lượng liên kết riêng:
		$$
		W_\text{lkr}=\dfrac{W_\text{lk}}{A}
		$$
	}
	
	\viduii{2}
	{
		Xét đồng vị Côban $ ^{60}_{27} \text{Co} $ hạt nhân có khối lượng $ m_{\text{Co}} = \SI{59,934}{u}$. Biết khối lượng của các hạt $ m_{\text{P}} = \SI{1,007276}{u} $; $ m_{\text{n}} = \SI{1,008565}{u} $. Độ hụt khối của hạt nhân đó là
		\begin{mcq}(4)
			\item $ \SI{0,401}{u} $.
			\item $ \SI{0,302}{u} $.
			\item $ \SI{0,548}{u} $.
			\item $ \SI{0,544}{u} $.
		\end{mcq} 
	}
	{
		\begin{center}
			\textbf{Hướng dẫn giải}
		\end{center}
		Ta có:
		$$
		\Delta m = 27 m_{\text{p}} + (60 - 27) m_{\text{n}} - m_{\text{Co}} = \SI{0,548}{u}.
		$$
		
		\begin{center}
			\textbf{Câu hỏi tương tự}
		\end{center}
		Khối lượng của nguyên tử nhôm $ ^{27}_{13} \text{Al} $ là $ \SI{26,9803}{u} $. Khối lượng của nguyên tử $ ^{1}_{1} \text{H} $ là $ \SI{1,007825}{u} $, khối lượng của proton là $ \SI{1,00728}{u} $ và khối lượng của neutron là $ \SI{1,00866}{u} $. Độ hụt khối của hạt nhân nhôm là
		\begin{mcq}(2)
			\item $ \SI{0,242665}{u} $.
			\item $ \SI{0,23558}{u} $.
			\item $ \SI{0,23548}{u} $.
			\item $ \SI{0,23544}{u} $.
		\end{mcq}
		\textbf{Đáp án:} A.
	}
	
	\viduii{2}
	{	[THPT QG 2017 - Mã đề 202]  Hạt nhân $^{235}_{\ 92}\text{U}$ có năng lượng liên kết $\SI{1784}{\mega\electronvolt}$. Năng lượng liên kết riêng của hạt nhân là
		\begin{mcq}(2)
			\item $\SI{5,46}{\mega\electronvolt}$.
			\item $\SI{12,48}{\mega\electronvolt}$.
			\item $\SI{19,39}{\mega\electronvolt}$.
			\item $\SI{7,59}{\mega\electronvolt}$.
	\end{mcq}}
	{
		\begin{center}
			\textbf{Hướng dẫn giải}
		\end{center}
		
		Năng lượng liên kết riêng của hạt nhân là
		\begin{equation*}
			W_\text{lkr}=\dfrac{W_\text{lk}}{A}=\dfrac{\SI{1784}{\mega\electronvolt}}{235}=\SI{7,59}{\mega\electronvolt}.
		\end{equation*}
		
		\begin{center}
			\textbf{Câu hỏi tương tự}
		\end{center}
		Cho $ m_{C} = \SI{12,00000}{u} $ ; $ m_{p} = \SI{1,00728}{u} $ ; $ m_{n} = \SI{1,00867}{u} $ ; $ \SI{1}{u} = \SI{1,66058 e-27}{kg} $ ; $ \SI{1}{eV} = \SI{1,6 e-19}{J} $ ; $ c = \SI{3 e8}{m/s} $. Năng lượng tối thiểu để tách hạt nhân $ ^{12} \text{C} $ thành các nuclôn riêng biệt bằng
		\begin{mcq}(4)
			\item $ \SI{72,7}{MeV} $.
			\item $ \SI{89,4}{MeV} $.
			\item $ \SI{44,7}{MeV} $.
			\item $ \SI{8,94}{MeV} $.
		\end{mcq}
		
		\textbf{Đáp án:} B.
	}
	
\end{dang}

\begin{dang}{So sánh và sắp xếp mức độ bền vững\\ của hạt nhân.}
	
	\ppgiai{
		\begin{description}
			\item[Bước 1:] Tính năng lượng liên kết riêng của từng hạt nhân:
			$$
			W_\text{lkr}=\dfrac{W_\text{lk}}{A}.
			$$
			\item[Bước 2:] Sắp xếp thứ tự các hạt nhân theo yêu cầu bài toán.
		\end{description}
	}
	
	\luuy{Hạt nhân càng bên vững thì có năng lượng liên kết riêng càng lớn và ngược lại.}
	
	
	\viduii{2}
	{[Đề thi đại học khối A, 2010] Cho ba hạt nhân $X$, $Y$ và $Z$ có số nuclôn tương ứng là $A_X$, $A_Y$, $A_Z$ với $A_X = 2A_Y = 0,5A_Z$. Biết năng lượng liên kết của từng hạt nhân tương ứng là $\Delta E_X$, $\Delta E_Y$, $\Delta E_Z$ với $\Delta E_Z < \Delta E_Y < \Delta E_X$. Sắp xếp các hạt nhân này theo thứ tự tính bền vững giảm dần là
		\begin{mcq}(4)
			\item $X$, $Y$, $Z$.
			\item $Z$, $X$, $Y$. 
			\item $Y$, $Z$, $X$.
			\item $Y$, $X$, $Z$.
		\end{mcq}
	}{\begin{center}
			\textbf{Hướng dẫn giải}
		\end{center}
		
		Năng lượng liên kết riêng của $X$, $Y$, $Z$ lần lượt là $W_\textrm{lkr X}=\dfrac{E_X}{A_X}$, $W_\textrm{lkr Y}=\dfrac{E_Y}{A_Y}$ và $W_\textrm{lkr Z}=\dfrac{E_Z}{A_Z}$.
		
		Vì $\Delta E_Z < \Delta E_Y < \Delta E_X$ và $A_X = 2A_Y = 0,5A_Z$ ($A_Z>A_X>A_Y$) nên sắp xếp các hạt nhân này theo thứ tự tính bền vững giảm dần là $Y$, $X$, $Z$.
		
		\begin{center}
			\textbf{Câu hỏi tương tự}
		\end{center}
		Các hạt nhân đơteri $ ^{2}_{1} \text{H} $ ; triti $ ^{3}_{1} \text{H} $ ; heli $ ^{4}_{2} \text{He} $ có năng lượng liên kết lần lượt là $ \SI{2,22}{MeV} $ ; $ \SI{8,49}{MeV} $ ; $ \SI{28,16}{MeV} $. Các hạt nhân trên được sắp xếp theo thứ tự giảm dần về độ bền vững hạt nhân là
		\begin{mcq}(2)
			\item $ ^{2}_{1} \text{H} $ ; $ ^{4}_{2} \text{He} $ ; $ ^{3}_{1} \text{H} $.
			\item $ ^{2}_{1} \text{H} $ ; $ ^{3}_{1} \text{H} $ ; $ ^{4}_{2} \text{He} $.
			\item $ ^{4}_{2} \text{He} $ ; $ ^{3}_{1} \text{H} $ ; $ ^{2}_{1} \text{H} $.
			\item $ ^{3}_{1} \text{H} $ ; $ ^{4}_{2} \text{He} $ ; $ ^{2}_{1} \text{H} $.
		\end{mcq}
		
		\textbf{Đáp án:} C.
	}
	
	\viduii{3}
	{[Đề thi đại học khối A, 2010] Cho khối lượng của proton, nơtron, $^{40}_{18}\text{Ar}$, $^{6}_{3}\text{Li}$ lần lượt là 1,0073 u; 1,0087u; 39,9525u; 6,0145u và $\SI{1}{u}=\SI{931,5}{\mega\electronvolt/c^2}$. So với năng lượng liên kết riêng của hạt nhân $^{6}_{3}\text{Li}$ thì năng lượng liên kết riêng của hạt nhân $^{40}_{18}\text{Ar}$
		\begin{mcq}(2)
			\item nhỏ hơn một lượng $\SI{3,42}{\mega\electronvolt}$.
			\item lớn hơn một lượng $\SI{5,20}{\mega\electronvolt}$.
			\item lớn hơn một lượng $\SI{3,42}{\mega\electronvolt}$.
			\item nhỏ hơn một lượng $\SI{5,20}{\mega\electronvolt}$.
	\end{mcq}}
	{
		\begin{center}
			\textbf{Hướng dẫn giải}
		\end{center}
		
		Năng lượng liên kết riêng của hạt nhân $^{40}_{18}\text{Ar}$ là
		\begin{equation*}
			W_\text{lkr}=\dfrac{W_\text{lk}}{A}= \dfrac{18 m_p+ 22 m_n - m_\text{Ar}}{40}\approx \SI{8,62}{\mega\electronvolt}.
		\end{equation*}
		
		Năng lượng liên kết riêng của hạt nhân $^{6}_{3}\text{Li}$ là
		\begin{equation*}
			W_\text{lkr}=\dfrac{W_\text{lk}}{A}= \dfrac{3 m_p+ 3 m_n - m_\text{Li}}{6}\approx \SI{5,20}{\mega\electronvolt}.
		\end{equation*}
		
		Vậy so với năng lượng liên kết riêng của hạt nhân $^{6}_{3}\text{Li}$ thì năng lượng liên kết riêng của hạt nhân $^{40}_{18}\text{Ar}$ $\SI{8,62}{\mega\electronvolt} - \SI{5,20}{\mega\electronvolt} = \SI{3,42}{\mega\electronvolt}.$
		
		\begin{center}
			\textbf{Câu hỏi tương tự}
		\end{center}
		
		Trong các hạt nhân: $ ^{4}_{2} \text{He} $ , $ ^{7}_{3} \text{Li} $ , $ ^{56}_{26} \text{Fe} $ và $ ^{235}_{92} \text{U} $ hạt nhân bền vững nhất là
		\begin{mcq}(4)
			\item $ ^{235}_{92} \text{U} $.
			\item $ ^{56}_{26} \text{Fe} $.
			\item $ ^{7}_{3} \text{Li} $.
			\item $ ^{4}_{2} \text{He} $.
		\end{mcq}
		
		\textbf{Đáp án:} B.
	}
	
\end{dang}
\section{Bài tập tự luyện}
\begin{enumerate}[label=\bfseries Câu \arabic*:]
	\item \mkstar{1} [1]
	\cauhoi
	{Đại lượng đặc trưng cho mức độ bền vững của hạt nhân là
		\begin{mcq}(2)
			\item độ hụt khối.
			\item năng lượng liên kết.
			\item khối lượng hạt nhân.
			\item năng lượng liên kết riêng.
		\end{mcq}
	}
	
	\loigiai
	{		\textbf{Đáp án: D.}
		
		Đại lượng đặc trưng cho mức độ bền vững của hạt nhân là năng lượng liên kết riêng. Năng lượng liên kết riêng càng lớn thì hạt nhân càng bền vững và ngược lại.
		
	}
	\item \mkstar{1} [2]
	\cauhoi
	{Tìm phát biểu \textbf{sai} khi nói về độ hụt khối.
		\begin{mcq}
			\item Độ hụt khối của một hạt nhân thường khác không nhưng trong trường hợp đặc biệt thì cũng có thể bằng không. 
			\item Độ chênh lệch giữa khối lượng $m$ của hạt nhân và tổng khối lượng $m_0$ của các nuclon cấu tạo nên hạt nhân gọi là độ hụt khối.
			\item Khối lượng của một hạt nhân luôn nhỏ hơn tổng khối lượng của các nuclon cấu tạo thành hạt nhân đó. 
			\item Khối lượng của một hạt nhân luôn lớn hơn tổng khối lượng của các nuclon cấu tạo thành hạt nhân đó.
		\end{mcq}
	}
	
	\loigiai
	{		\textbf{Đáp án: D.}
		
		"Khối lượng của một hạt nhân luôn lớn hơn tổng khối lượng của các nuclon cấu tạo thành hạt nhân đó." là phát biểu sai.
		
		"Khối lượng của một hạt nhân luôn nhỏ hơn tổng khối lượng của các nuclon cấu tạo thành hạt nhân đó." là phát biểu đúng.
		
	}
	\item \mkstar{1} [3]
	\cauhoi
	{Hạt nhân có năng lượng liên kết càng lớn thì
		\begin{mcq}(2)
			\item càng kém bền vững.
			\item độ hụt khối của hạt nhân càng nhỏ.
			\item độ hụt khối của hạt nhân càng lớn.
			\item càng bền vững.
		\end{mcq}
	}
	
	\loigiai
	{		\textbf{Đáp án: C.}
		
		Hạt nhân có năng lượng liên kết càng lớn thì độ hụt khối của hạt nhân càng lớn:
		$$E_\text{lk} = \Delta m c^2$$
		
	}
	\item \mkstar{1} [4]
	\cauhoi
	{Đại lượng nào sau đây đặc trưng cho mức độ bền vững của hạt nhân?
		\begin{mcq}(2)
			\item Năng lượng nghỉ.
			\item Độ hụt khối.
			\item Năng lượng liên kết.
			\item Năng lượng liên kết riêng.
		\end{mcq}
	}
	
	\loigiai
	{		\textbf{Đáp án: D.}
		
		Đại lượng đặc trưng cho mức độ bền vững của hạt nhân là năng lượng liên kết riêng. Năng lượng liên kết riêng càng lớn thì hạt nhân càng bền vững và ngược lại.
		
		
	}
	\item \mkstar{1} [5]
	\cauhoi
	{Để so sánh độ bền vững của các hạt nhân người ta thường dùng đại lượng:
		\begin{mcq}
			\item năng lượng liên kết tính trên một nuclon.
			\item năng lượng liên kết tính cho một hạt nhân.
			\item năng lượng liên kết giữa hai nuclon.
			\item năng lượng liên kết giữa hạt nhân và lớp vỏ nguyên tử.
		\end{mcq}
	}
	
	\loigiai
	{		\textbf{Đáp án: A.}
		
		Để so sánh độ bền vững của các hạt nhân người ta thường dùng năng lượng liên kết riêng: năng lượng liên kết tính trên một nuclon.
		
	}
	\item \mkstar{1} [5]
	\cauhoi
	{Năng lượng liên kết riêng là năng lượng liên kết ...
		\begin{mcq}(2)
			\item tính riêng cho hạt nhân ấy.
			\item tính cho một nuclon.
			\item của một cặp proton - proton.
			\item của một cặp proton - nơtron.
		\end{mcq}
	}
	
	\loigiai
	{		\textbf{Đáp án: B.}
		
		Năng lượng liên kết riêng là năng lượng liên kết tính cho một nuclon.
		
	}
	\item \mkstar{1} [7]
	\cauhoi
	{Năng lượng liên kết riêng của một hạt nhân
		\begin{mcq}(2)
			\item có thể âm hoặc dương.
			\item càng lớn, thì càng kém bền vững.
			\item càng nhỏ, thì càng bền vững.
			\item càng lớn, thì càng bền vững.
		\end{mcq}
	}
	
	\loigiai
	{		\textbf{Đáp án: D.}
		
		Năng lượng liên kết riêng của một hạt nhân càng lớn thì hạt nhân đó càng bền vững.
		
	}
	\item \mkstar{1} [9]
	\cauhoi
	{Hạt nhân càng bền vững khi có
		\begin{mcq}(2)
			\item năng lượng liên kết càng lớn.
			\item số nuclon càng nhỏ.
			\item số nuclon càng lớn.
			\item năng lượng liên kết riêng càng lớn.
		\end{mcq}
	}
	
	\loigiai
	{		\textbf{Đáp án: D.}
		
		Càng nhân càng bền vững khi có năng lượng liên kết riêng càng lớn.
		
	}
	\item \mkstar{2} [1]
	\cauhoi
	{Độ hụt khối khi hình thành hạt nhân $\ce{^12_6 C}$ bằng $\SI{89.424}{MeV/c^2}$. Năng lượng liên kết riêng của hạt nhân này bằng
		\begin{mcq}(4)
			\item $\SI{7.452}{MeV}$.
			\item $\SI{6.387}{MeV}$.
			\item $\SI{6.837}{MeV}$.
			\item $\SI{7.542}{MeV}$.
		\end{mcq}
	}
	
	\loigiai
	{		\textbf{Đáp án: A.}
		
		Năng lượng liên kết:
		$$E_\text{lk} = \Delta m c^2 = \SI{89.424}{MeV/c^2} \cdot c^2 = \SI{89.424}{MeV}$$
		
		Năng lượng liên kết riêng:
		$$E_\text{lkr} = \dfrac{E_\text{lk}}{A} = \SI{7.452}{MeV}$$
		
	}
	
	\item \mkstar{2} [1]
	\cauhoi
	{Hạt nhân đơteri $\ce{^2_1 D}$ có năng lượng liên kết là $\SI{2.23}{MeV}$. Biết khối lượng của proton là $\SI{1.0073}{u}$ và khối lượng của nơtron là $\SI{1.0087}{u}$. Hạt nhân $\ce{^2_1 D}$ có khối lượng là
		\begin{mcq}(4)
			\item $\SI{2.1036}{u}$. 
			\item $\SI{2.0136}{u}$. 
			\item $\SI{2.1360}{u}$. 
			\item $\SI{2.0361}{u}$. 
		\end{mcq}
	}
	
	\loigiai
	{		\textbf{Đáp án: B.}
		
		Độ hụt khối của hạt nhân:
		$$E_\text{lk} = \Delta m c^2 \Rightarrow \Delta m = \dfrac{E_\text{lk}}{c^2} = \dfrac{2,23}{931,5} = \SI{2.4e-3}{u}$$
		
		Khối lượng hạt nhân $\ce{^2_1 D}$:
		$$\Delta m = Zm_p + (A-Z)m_n - m_X \Rightarrow m_X = \SI{2.0136}{u}$$
	}
	
	
	
	\item \mkstar{2} [2]
	\cauhoi
	{Hạt nhân đơ-tơ-ri $\ce{^2_1 D}$ có khối lượng $\SI{2.0136}{u}$. Biết khối lượng của proton là $\SI{1.0073}{u}$ và khối lượng của nơtron là $\SI{1.0087}{u}$. Lấy $\SI{1}{u}=\SI{931.5}{MeV/c^2}$. Năng lượng liên kết riêng của hạt nhân $\ce{^2_1 D}$ là
		\begin{mcq}(2)
			\item $\SI{1.1178}{MeV/nuclon}$. 
			\item $\SI{2.2356}{MeV/nuclon}$. 
			\item $\SI{0.6753}{MeV/nuclon}$.
			\item $\SI{2.0218}{MeV/nuclon}$.
		\end{mcq}
	}
	
	\loigiai
	{		\textbf{Đáp án: A.}
		
		Độ hụt khối:
		$$\Delta m = Zm_p + (A-Z)m_n - m_X = \SI{2.4e-3}{u}$$
		
		Năng lượng liên kết:
		$$E_\text{lk} = \Delta m c^2 = \SI{2.2356}{MeV}$$
		
		Năng lượng liên kết riêng:
		$$E_\text{lkr} = \dfrac{E_\text{lk}}{A} =\SI{1.1178}{MeV/nuclon} $$
		
	}
	
	\item \mkstar{2} [3]
	\cauhoi
	{Hạt nhân đơ-tơ-ri $\ce{^2_1 D}$ có khối lượng $\SI{2.0136}{u}$. Biết khối lượng của proton là $\SI{1.0073}{u}$ và khối lượng của nơtron là $\SI{1.0087}{u}$. Lấy $\SI{1}{u}=\SI{931.5}{MeV/c^2}$. Năng lượng liên kết của hạt nhân $\ce{^2_1 D}$ là
		\begin{mcq}(4)
			\item $\SI{3.06}{MeV}$. 
			\item $\SI{1.12}{MeV}$. 
			\item $\SI{4.48}{MeV}$.
			\item $\SI{2.24}{MeV}$.
		\end{mcq}
	}
	
	\loigiai
	{		\textbf{Đáp án: D.}
		
		Độ hụt khối:
		$$\Delta m = Zm_p + (A-Z)m_n - m_X = \SI{2.4e-3}{u}$$
		
		Năng lượng liên kết:
		$$E_\text{lk} = \Delta m c^2 = \SI{2.24}{MeV}$$
		
	}
	\item \mkstar{2} [2]
	\cauhoi
	{Hạt nhân $\ce{^37_17 Cl}$ có khối lượng nghỉ bằng $\SI{36.956563}{u}$. Biết khối lượng của nơtron là $\SI{1.008670}{u}$, khối lượng của proton là $\SI{1.007276}{}$ và $\SI{1}{u} = \SI{931.5}{MeV/c^2}$. Năng lượng liên kết riêng của hạt nhân $\ce{^37_17 Cl}$ bằng
		\begin{mcq}(4)
			\item $\SI{6.325}{MeV}$.
			\item $\SI{7.368}{MeV}$.
			\item $\SI{8.468}{MeV}$.
			\item $\SI{8.573}{MeV}$.
		\end{mcq}
	}
	
	\loigiai
	{		\textbf{Đáp án: D.}
		
		Độ hụt khối:
		$$\Delta m = Zm_p + (A-Z)m_n - m_X = \SI{0.340529}{u}$$
		
		Năng lượng liên kết:
		$$E_\text{lk} = \Delta m c^2 = \SI{317.2028}{MeV}$$
		
		Năng lượng liên kết riêng:
		$$E_\text{lkr} = \dfrac{E_\text{lk}}{A} =\SI{8.573}{MeV/nuclon} $$
		
	}
	
	\item \mkstar{2} [4]
	\cauhoi
	{Biết khối lượng của hạt nhân $\ce{^235_92 U}$ là $\SI{234.99}{u}$, của proton là $\SI{1.0073}{u}$ và của nơtron là $\SI{1.0087}{u}$. Năng lượng liên kết riêng của hạt nhân $\ce{^235_92 U}$ là
		\begin{mcq}(2)
			\item $\SI{7.95}{MeV/nuclon}$.
			\item $\SI{6.73}{MeV/nuclon}$.
			\item $\SI{8.71}{MeV/nuclon}$.
			\item $\SI{7.63}{MeV/nuclon}$.
		\end{mcq}
	}
	
	\loigiai
	{		\textbf{Đáp án: D.}
		
		Độ hụt khối:
		$$\Delta m = Zm_p + (A-Z)m_n - m_X = \SI{1.9257}{u}$$
		
		Năng lượng liên kết:
		$$E_\text{lk} = \Delta m c^2 = \SI{1793.79}{MeV}$$
		
		Năng lượng liên kết riêng:
		$$E_\text{lkr} = \dfrac{E_\text{lk}}{A} =\SI{7.63}{MeV/nuclon} $$
		
	}
	
	\item \mkstar{2} [12]
	\cauhoi
	{Biết năng lượng liên kết của hạt nhân $\ce{^7_3 Li}$ là $\SI{62.40}{MeV}$. Năng lượng liên kết riêng của nó xấp xỉ bằng
		\begin{mcq}(4)
			\item $\SI{20.80}{MeV/nuclon}$.
			\item $\SI{4.455}{MeV/nuclon}$.
			\item $\SI{10.40}{MeV/nuclon}$.
			\item $\SI{8.910}{MeV/nuclon}$.
		\end{mcq}
	}
	
	\loigiai
	{		\textbf{Đáp án: D.}
		
		Năng lượng liên kết riêng:
		$$E_\text{lkr} = \dfrac{E_\text{lk}}{A} \approx \SI{8.910}{MeV/nuclon} $$
		
	}
	
	
	
	\item \mkstar{2} [5]
	\cauhoi
	{Hạt nhân Côban $\ce{^60_27 Co}$ có khối lượng $m_{\ce{Co}} = \SI{55.940}{u}$. Biết khối lượng của proton là $m_p=\SI{1.0073}{u}$ và khối lượng của nơtron là $m_n=\SI{1.0087}{u}$. Năng lượng liên kết riêng của hạt nhân $\ce{^60_27 Co}$ là
		\begin{mcq}(2)
			\item $\SI{48.9}{MeV/nuclon}$.
			\item $\SI{54.4}{MeV/nuclon}$.
			\item $\SI{70.4}{MeV/nuclon}$.
			\item $\SI{70.5}{MeV/nuclon}$.
		\end{mcq}
	}
	
	\loigiai
	{		\textbf{Đáp án: D.}
		
		Độ hụt khối:
		$$\Delta m = Zm_p + (A-Z)m_n - m_X = \SI{4.5442}{u}$$
		
		Năng lượng liên kết:
		$$E_\text{lk} = \Delta m c^2 = \SI{4232.9223}{MeV}$$
		
		Năng lượng liên kết riêng:
		$$E_\text{lkr} = \dfrac{E_\text{lk}}{A} =\SI{70.5}{MeV/nuclon} $$
		
	}
	
	\item \mkstar{2} [7]
	\cauhoi
	{Một hạt nhân $\ce{^4_2 He}$ có độ hụt khối là $\SI{0.0305}{u}$. Lấy $\SI{1}{u} = \SI{931.5}{MeV/c^2}$. Năng lượng liên kết riêng của hạt nhân này là
		\begin{mcq}(2)
			\item $\SI{28.41075}{MeV/nuclon}$.
			\item $\SI{7.10269}{eV/nuclon}$.
			\item $\SI{7.10269}{MeV/nuclon}$.
			\item $\SI{7.01269}{MeV/nuclon}$.
		\end{mcq}
	}
	
	\loigiai
	{		\textbf{Đáp án: C.}
		
		
		Năng lượng liên kết:
		$$E_\text{lk} = \Delta m c^2 = \SI{28.41075}{MeV}$$
		
		Năng lượng liên kết riêng:
		$$E_\text{lkr} = \dfrac{E_\text{lk}}{A} =\SI{7.10269}{MeV/nuclon} $$
	}
	
	\item \mkstar{2} [9]
	\cauhoi
	{Hạt nhân Rađi $\ce{^226_88 Ra}$ có khối lượng bằng $\SI{225.9771}{u}$. Biết khối lượng của nơtron là $\SI{1.00867}{u}$, khối lượng của proton là $\SI{1.00728}{u}$. Năng lượng liên kết riêng của hạt nhân $\ce{^226_88 Ra}$ bằng
		\begin{mcq}(2)
			\item $\SI{1732.59}{MeV/nuclon}$.
			\item $\SI{1667.85}{MeV/nuclon}$.
			\item $\SI{7.67}{MeV/nuclon}$.
			\item $\SI{7.38}{MeV/nuclon}$.
		\end{mcq}
	}
	
	\loigiai
	{		\textbf{Đáp án: C.}
		
		Độ hụt khối:
		$$\Delta m = Zm_p + (A-Z)m_n - m_X = \SI{1.86}{u}$$
		
		Năng lượng liên kết:
		$$E_\text{lk} = \Delta m c^2 = \SI{1732.59}{MeV}$$
		
		Năng lượng liên kết riêng:
		$$E_\text{lkr} = \dfrac{E_\text{lk}}{A} =\SI{7.67}{MeV/nuclon} $$
		
	}
	\item \mkstar{3} [9]
	\cauhoi
	{Hạt nhân $\ce{^4_2 He}$ có năng lượng liên kết riêng là $\SI{7.1}{MeV/nuclon}$. Cho $\SI{1}{u} = \SI{931.5}{MeV/c^2}$. Độ hụt khối của hạt nhân $\ce{^4_2 He}$ là
		\begin{mcq}(4)
			\item $\SI{0.0076}{u}$.
			\item $\SI{0.0305}{u}$.
			\item $\SI{0.751}{u}$.
			\item $\SI{1.917}{u}$.
		\end{mcq}
	}
	
	\loigiai
	{		\textbf{Đáp án: B.}
		
		Năng lượng liên kết:
		$$E_\text{lk} = E_\text{lkr} A = \SI{28.4}{MeV}$$
		
		Độ hụt khối của hạt nhân:
		$$E_\text{lk} = \Delta m c^2 \Rightarrow \Delta m = \dfrac{E_\text{lk}}{c^2} = \dfrac{28,4}{931,5} = \SI{0.0305}{u}$$
		
	}
	\item \mkstar{3} [13]
	\cauhoi
	{Biết khối lượng của nơtron là $\SI{1.00867}{u}$, khối lượng của proton là $\SI{1.00728}{u}$, khối lượng của hạt nhân $\ce{^10_5 B}$ là $\SI{10.0102}{u}$. Năng lượng liên kết riêng của hạt nhân này bằng
		\begin{mcq}(4)
			\item $\SI{5.885}{MeV}$.
			\item $\SI{6.479}{MeV}$.
			\item $\SI{12.948}{MeV}$.
			\item $\SI{64.79}{MeV}$.
		\end{mcq}
	}
	
	\loigiai
	{		\textbf{Đáp án: B.}
		
		Độ hụt khối:
		$$\Delta m = Zm_p + (A-Z)m_n - m_X = \SI{0.0696}{u}$$
		
		Năng lượng liên kết:
		$$E_\text{lk} = \Delta m c^2 = \SI{64.79}{MeV}$$
		
		Năng lượng liên kết riêng:
		$$E_\text{lkr} = \dfrac{E_\text{lk}}{A} =\SI{6.479}{MeV/nuclon} $$
		
		Vậy $\SI{6.474}{MeV}$ là đáp án gần đúng nhất.
		
	}
\end{enumerate}

