% --- chapter
\newcommand{\chapter}[2][]{
	\newcommand{\chapname}{#2}
	\begin{flushleft}
		\begin{minipage}[t]{\linewidth}
			\includegraphics[height=1cm]{hdht-logo.png}
			\hspace{0pt}	
			\sffamily\bfseries\large Bài  36. Năng lượng liên kết của hạt nhân. Phản ứng hạt nhân
			\begin{flushleft}
				\huge\bfseries #1
			\end{flushleft}
		\end{minipage}
	\end{flushleft}
	\vspace{1cm}
	\normalfont\normalsize
}
%-----------------------------------------------------
\chapter[Viết phương trình phản ứng hạt nhân]{Viết phương trình phản ứng hạt nhân}
\section{Lý thuyết}

\subsection{Định nghĩa và đặc tính của phản ứng hạt nhân}
Phản ứng hạt nhân là mọi quá trình dẫn đến sự biến đổi hạt nhân
\begin{equation}
A+B\rightarrow C+D,
\end{equation}
trong đó:
\begin{itemize}
	\item A, B là các hạt nhân tương tác;
	\item C, D là các hạt nhân sản phẩm.
\end{itemize}

\subsection{Định luật bảo toàn điện tích và định luật bảo toàn số khối trong phản ứng hạt nhân}
Xét phản ứng hạt nhân
\begin{equation}
^{A_1}_{Z_1}A + ^{A_2}_{Z_2}B \rightarrow ^{A_3}_{Z_3}C + ^{A_4}_{Z_4}D
\end{equation}
\subsubsection{Định luật bảo toàn điện tích}
Trong phản ứng hạt nhân, tổng số đại số điện tích các hạt tương tác bằng tổng đại điện tích số các hạt sản phẩm
\begin{equation}
Z_1+Z_2=Z_3+Z_4.
\end{equation}
\subsubsection{Định luật bảo toàn số khối}
Trong phản ứng hạt nhân, tổng số nuclon các hạt tương tác bằng tổng số nuclon các hạt sản phẩm
\begin{equation}
A_1+A_2=A_3+A_4.
\end{equation}

\section{Mục tiêu bài học - Ví dụ minh họa}

\begin{dang}{Viết phương trình phản ứng hạt nhân.}
\ppgiai{
	\begin{description}
		\item[Bước 1:] Viết phương trình phản ứng hạt nhân, xác định số nuclon, proton, nơtron của các hạt đã biết.
		\item[Bước 2:] Áp dụng định luật bảo toàn điện tích và định luật bảo toàn số khối để xác định cấu tạo của các hạt nhân còn thiếu trong phương trình.
	\end{description}
}

\viduii{2}{
Cho phản ứng hạt nhân $^1_0 n + ^{235}_{\ 92} \text{U} \rightarrow ^{94}_{38} \text {Sr} + X + 2 ^1_0n$. Hạt nhân X có cấu tạo gồm
\begin{mcq}(2)
	\item 54 proton và 86 nơtron.
	\item 54 proton và 140 nơtron.
	\item 86 proton và 140 nơtron.
	\item 86 proton và 54 nơtron.
\end{mcq}}
{\begin{center}
	\textbf{Hướng dẫn giải}
\end{center}
	Áp dụng định luật bảo toàn điện tích và bảo toàn số khối
	\begin{equation*}
	\left\{
	\begin{matrix}
	0+92=38+Z_X+2\cdot0\\
	1+235=94+A_X+2\cdot1
	\end{matrix}
	\right.
	\Rightarrow
	\left\{
	\begin{matrix}
	Z_X=54\\
	A_X=140.
	\end{matrix}
	\right.
	\end{equation*}
	
	Vậy hạt nhân $X$ có cấu tạo gồm 54 proton và 86 nơtron.
	
\begin{center}
	\textbf{Câu hỏi tương tự}
\end{center}

Cho phản ứng hạt nhân $ ^{238}_{922} \text{U} \longrightarrow ^{95}_{38} \text{Sr} + \text{X} + 3n $. Hạt nhân X có cấu tạo gồm
\begin{mcq}(2)
	\item 54 proton và 86 nơtron.
	\item 54 proton và 140 nơtron.
	\item 86 proton và 140 nơtron.
	\item 86 proton và 54 nơtron.
\end{mcq}

\textbf{Đáp án:} A.
}
\viduii{2}
{Cho hạt prôtôn bắn vào các hạt nhân $^9_4\text{Be}$ đang đứng yên, người ta thấy các hạt tạo thành gồm $^4_2\text{He}$ và hạt nhân X. Hạt nhân X có cấu tạo gồm
	\begin{mcq}(2)
		\item 3 proton và 3 nơtron.
		\item 3 proton và 6 nơtron.
		\item 3 proton và 6 nơtron.
		\item 2 proton và 3 nơtron.
	\end{mcq}
}	{\begin{center}
		\textbf{Hướng dẫn giải}
	\end{center}
	Phương trình phản ứng hạt nhân
	\begin{equation*}
	^1_1 p+^9_4\text{Be}\rightarrow ^4_2\text{He}+^A_Z X
	\end{equation*}
	Áp dụng định luật bảo toàn điện tích và bảo toàn số khối
	\begin{equation*}
	\left\{
	\begin{matrix}
	1+4=2+Z_X\\
	1+9=4+A_X
	\end{matrix}
	\right.
	\Rightarrow
	\left\{
	\begin{matrix}
	Z_X=3\\
	A_X=6.
	\end{matrix}
	\right.
	\end{equation*}
	
	Vậy hạt nhân $X$ có cấu tạo gồm 3 proton và 3 nơtron.
	
\begin{center}
	\textbf{Câu hỏi tương tự}
\end{center}
Cho hạt $ \alpha $ bắn vào một lá nhôm mỏng, người ta thấy các hạt tạo thành gồm proton và hạt nhân X. Hạt nhân X có cấu tạo gồm
	\begin{mcq}(2)
		\item 5 proton và 7 nơtron.
		\item 5 proton và 8 nơtron.
		\item 8 proton và 9 nơtron.
		\item 8 proton và 10 nơtron.
	\end{mcq}

	
	\textbf{Đáp án:} D.
}
\end{dang}

\begin{dang}{Xác định tên hạt nhân còn thiếu.}
\ppgiai{
	\begin{description}
		\item[Bước 1:] Viết phương trình phản ứng hạt nhân, xác định số nuclon, proton, nơtron của các hạt đã biết.
		\item[Bước 2:] Áp dụng định luật bảo toàn điện tích và định luật bảo toàn số khối để xác định cấu tạo của các hạt nhân còn thiếu trong phương trình.
		\item[Bước 3:] Dựa vào số hiệu nguyên tử xác định tên hạt nhân còn thiếu.
	\end{description}
}

\viduii{2}
{
Cho phản ứng hạt nhân $^{235}_{92} \text{U} + ^{1}_{0} \text{n} \longrightarrow ^{95}_{42} \text{Mo} + \text{X} + 2 \; ^{1}_{0} \text{n} + 4 \; ^{0}_{-1} \text{e}$, trong đó X là hạt nhân
		\begin{mcq}(4)
			\item $^{138}_{53} \text{I}$.
			\item $^{138}_{52} \text{Te}$.
			\item $^{139}_{54} \text{Xe}$.
			\item $^{139}_{57} \text{La}$.
		\end{mcq}
}
{
\begin{center}
	\textbf{Hướng dẫn giải}
\end{center}

Số khối trước phản ứng: $A_\text{t} = 236$.
		
		Số khối sau phản ứng phải bằng số khối trước phản ứng:
		$$A_\text{s} = A_\text{t} \Rightarrow 95+A_\text X+2\cdot 2 + 4 \cdot 0 = 236 \Rightarrow A_\text X = 139$$
		
		Số điện tích trước phản ứng: $Z_\text{t} = 92$.
		
		Số điện tích sau phản ứng phải bằng số điện tích trước phản ứng:
		$$Z_\text{s} = Z_\text{t} \Rightarrow 42 + Z_\text{X} + 2 \cdot 0 + 4 \cdot(-1) = 92 \Rightarrow Z_\text{X} = 54$$
		
		Vậy X là hạt nhân $ ^{139}_{54} \text{Xe}$.
		
\begin{center}
	\textbf{Câu hỏi tương tự}
\end{center}

Trong phản ứng hạt nhân $^{19}_{9} \text{F} + \text{p} \longrightarrow ^{16}_{8} \text{O} + \text{X}$, hạt X là
		\begin{mcq}(4)
			\item nơtron. 
			\item proton. 
			\item hạt $^{4}_{2} \text{He}$. 
			\item electron.
		\end{mcq}
\textbf{Đáp án:} C.
}

\viduii{2}
{
Trong phản ứng hạt nhân $^{35}_{17} \text{Cl} + \text{X} \longrightarrow ^{32}_{16} \text{S} + ^{4}_{2} \text{He}$, hạt X là
		\begin{mcq}(4)
			\item $^{2}_{1} \text{H}$.
			\item $^{3}_{1} \text{H}$. 
			\item $^{1}_{1} \text{H}$. 
			\item $^{1}_{0} \text{n}$.
		\end{mcq}
}
{
\begin{center}
	\textbf{Hướng dẫn giải}
\end{center}

Số khối sau phản ứng: $A_\text{s} = 36$.
		
		Số khối sau phản ứng phải bằng số khối trước phản ứng:
		$$A_\text{t} = A_\text{s} \Rightarrow 35+A_\text X = 36 \Rightarrow A_\text X = 1$$
		
		Số điện tích sau phản ứng: $Z_\text{s} = 18$.
		
		Số điện tích sau phản ứng phải bằng số điện tích trước phản ứng:
		$$Z_\text{t} = Z_\text{s} \Rightarrow 17 + Z_\text{X} = 18 \Rightarrow Z_\text{X} = 1$$
		
		Vậy X là hạt nhân $^{1}_{1} \text{H}$.

\begin{center}
	\textbf{Câu hỏi tương tự}
\end{center}

Trong phản ứng hạt nhân $^{9}_{4} \text{Be} + \alpha \longrightarrow \text{X} + \text{n}$, hạt X là
		\begin{mcq}(4)
			\item $^{13}_{7} \text{N}$.
			\item $^{12}_{6} \text{C}$.
			\item $^{12}_{5} \text{B}$. 
			\item $^{16}_{8} \text{O}$.
		\end{mcq}
\textbf{Đáp án:} B.
}

\end{dang}


