
\chapter[Tia hồng ngoại]{Tia hồng ngoại}
\section{Lý thuyết}
\subsection {Bản chất}
\begin{itemize}
	\item Tia hồng ngoại là những bức xạ không nhìn thấy được, có bước sóng lớn hơn bước sóng của ánh sáng đỏ ($> \text{0,76}\ \mu \text{m}$). 
	\item Thu được cùng với các tia sáng thông thường.
	\item Có cùng bản chất với ánh sáng.
\end{itemize}

\subsection{Tính chất chung}
\begin{itemize}
	\item Tuân theo các định luật: truyền thẳng, phản xạ, khúc xạ.
	\item Gây được hiện tượng nhiễu xạ, giao thoa như ánh sáng thông thường. 
\end{itemize}

\subsection{Nguồn phát}
\begin{itemize}
	\item Mọi vật có nhiệt độ cao hơn 0 K ($-273^\circ \text{C}$) đều phát ra tia hồng ngoại.
	\item Vật có nhiệt độ cao hơn môi trường xung quanh thì phát bức xạ hồng ngoại ra môi trường.
	\item Nguồn phát tia hồng ngoại thông dụng: bóng đèn dây tóc, bếp ga, bếp than, điôt hồng ngoại,...
\end{itemize}

\subsection{Tính chất và công dụng}

\subsubsection{Tính chất}

\begin{itemize}
	\item  Tác dụng nhiệt là tính chất nổi bật nhất. Vật hấp thụ tia hồng ngoại sẽ nóng lên.
	\item  Có khả năng gây ra một số phản ứng hóa học, có thể tác dụng lên phim ảnh.
	\item  Có thể biến điệu như sóng điện từ cao tần.
	\item  Gây ra hiện tượng quang điện trong với một số chất bán dẫn.
\end{itemize}

\subsubsection{Công dụng}

\begin{itemize}
	\item Sấy khô, sưởi ấm, đun nấu.
	\item Chụp ảnh ban đêm, chụp ảnh của nhiều thiên thể, chụp ảnh trái đất từ vệ tinh.
	\item Sử dụng trong các bộ điều khiển từ xa (điều khiển ti vi, điều hòa, ...).
	\item Quân sự: ống nhòm hồng ngoại dùng để quan sát và lái xe ban đêm, camera hồng ngoại chụp ảnh và quay phim ban đêm, tên lửa tự động tìm mục tiêu dựa vào tia hồng ngoại do mục tiêu phát ra.
\end{itemize}
\section{Bài tập tự luyện}
\begin{enumerate}[label=\bfseries Câu \arabic*:]
	
	%=======================================
	\item \mkstar{1} [2]
	\cauhoi
	{Khi nói về tia hồng ngoại, phát biểu nào dưới đây là \textbf{sai}?
		\begin{mcq}(1)
			\item Tia hồng ngoại có khả năng làm phát quang một số chất. 
			\item Tia hồng ngoại có khả năng gây ra một số phản ứng hóa học. 
			\item Tác dụng nổi bật nhất của tia hồng ngoại là tác dụng nhiệt. 
			\item Tia hồng ngoại cũng có thể biến điệu được như sóng điện từ cao tần. 
		\end{mcq}
	}
	
	\loigiai
	{		\textbf{Đáp án: A.}
		
		Tia hồng ngoại không có khả năng gây phát quang.
	}
	
	%=======================================
	\item \mkstar{1} [6]
	\cauhoi
	{Để đo thân nhiệt của một người mà không cần tiếp xúc trực tiếp, ta dùng máy đo thân nhiệt điện tử. Máy này tiếp nhận năng lượng bức xạ phát ra từ người cần đo. Nhiệt độ của người càng cao thì máy tiếp nhận năng lượng càng lớn. Bức xạ chủ yếu mà máy nhận được từ người thuộc miền 
		\begin{mcq}(4)
			\item tia Y. 
			\item hồng ngoại. 
			\item tia X. 
			\item tử ngoại. 
		\end{mcq}
	}
	
	\loigiai
	{		\textbf{Đáp án: B.}
		
		Nhiệt là tính chất đặc trưng của bức xạ hồng ngoại nên bức xạ chủ yếu mà máy thu được từ người thuộc miền hồng ngoại.
	}
	
	%=======================================
	\item \mkstar{1} [7]
	\cauhoi
	{Tính chất nổi bật của tia hồng ngoại là
		\begin{mcq}(2)
			\item tác dụng nhiệt.
			\item làm ion hóa không khí. 
			\item tác dụng lên kính ảnh. 
			\item khả năng đâm xuyên. 
		\end{mcq}
	}
	
	\loigiai
	{		\textbf{Đáp án: A.}
		
		Tính chất nổi bật của tia hồng ngoại là tác dụng nhiệt. 
	}
		\item \mkstar{1} 
	\cauhoi
	{Bức xạ hồng ngoại là bức xạ 
		
		\begin{mcq}(1)
			\item màu hồng.
			
			\item màu đỏ sẫm.
			
			\item mắt không nhìn thấy ở ngoài miền đỏ.
			
			\item có bước sóng nhỏ hơn so với ánh sáng thường.
			
		\end{mcq}
	}
	
	\loigiai
	{		\textbf{Đáp án: C.}
		
		 Bức xạ hồng ngoại là bức xạ mắt không nhìn thấy và ở ngoài miền đỏ. 
	
	}
	\item \mkstar{1} 
	\cauhoi
	{Bức xạ (hay tia) hồng ngoại là bức xạ
		
		\begin{mcq}(1)
			\item đơn sắc, có màu hồng.
			
			\item  đơn sắc, không màu ở ngoài đầu đỏ của quang phổ.
			
			\item có bước sóng nhỏ dưới $\SI{0,4}{m}$.
			
			\item có bước sóng từ $\SI{0,75}{m}$ tới cỡ milimét.
			
		\end{mcq}
	}
	
	\loigiai
	{		\textbf{Đáp án: D.}
		
		
	}
	\item \mkstar{1} 
	\cauhoi
	{Tia hồng ngoại là những bức xạ có
		
		\begin{mcq}(1)
			\item bản chất là sóng điện từ.
			
			\item  khả năng ion hoá mạnh không khí.
			
			\item khả năng đâm xuyên mạnh, có thể xuyên qua lớp chì dày cỡ cm.
			
			\item bước sóng nhỏ hơn bước sóng của ánh sáng đỏ.
		\end{mcq}
	}
	
	\loigiai
	{		\textbf{Đáp án: A.}
		
		Tia hồng ngoại có bản chất là sóng điện từ có bước sóng lớn hơn bước sóng ánh sáng đỏ, khả năng ion hoá không khí và không có khả năng đâm xuyên qua lớp chì. 
	}
	\item \mkstar{1} 
	\cauhoi
	{Phát biểu nào sau đây là không đúng?
		
		\begin{mcq}(1)
			\item Tia hồng ngoại do các vật bị nung nóng phát ra.
			
			\item Tia hồng ngoại là sóng điện từ có bước sóng lớn hơn $\SI{0,76}{m}$.
			\item Tia hồng ngoại có tác dụng lên mọi kính ảnh.
			
			\item Tia hồng ngoại có tác dụng nhiệt rất mạnh.
			
		\end{mcq}
	}
	
	\loigiai
	{		\textbf{Đáp án: C.}
		
		Tia hồng ngoại có tác dụng lên kính ảnh hồng ngoại. 
	}
	\item \mkstar{1} 
	\cauhoi
	{Có thể nhận biết tia hồng ngoại bằng
		\begin{mcq}(2)
			\item màn huỳnh quang. 	
			\item quang phổ kế.
			
			\item mắt người.
			\item pin nhiệt điện.
			
		\end{mcq}
	}
	
	\loigiai
	{		\textbf{Đáp án: D.}
		
		Tính chất nổi bật của tia hồng ngoại là tác dụng nhiệt $\rightarrow$ nhận biết tia hồng ngoại bằng pin nhiệt điện.
	}
	\item \mkstar{2} 
	\cauhoi
	{Trong các nhà hàng, khách sạn, rạp chiếu phim, v.v. có lắp máy sấy tay cảm ứng trong nhà vệ sinh. Khi người sử dụng đưa tay vào vùng cảm ứng, thiết bị sẽ tự động sấy để làm khô tay và ngắt khi người sử dụng đưa tay ra. Máy sấy tay này hoạt động dựa trên
		
		\begin{mcq}(1)
			\item cảm ứng tia tử ngoại phát ra từ bàn tay.
			\item cảm ứng độ ẩm của bàn tay.
			\item cảm ứng tia hồng ngoại phát ra từ bàn tay.
			\item cảm ứng tia X phát ra từ bàn tay.
			
		\end{mcq}
	}
	
	\loigiai
	{		\textbf{Đáp án: C.}
		
		Máy sấy tay này hoạt động dựa trên cảm ứng tia hồng ngoại phát ra từ bàn tay. 
	}
\end{enumerate}

