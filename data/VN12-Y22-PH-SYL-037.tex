
\chapter[Lý thuyết và bài tập: Các mạch điện xoay chiều]{Lý thuyết và bài tập: Các mạch điện xoay chiều}
\subsection{Mạch điện xoay chiều chỉ có một điện trở}
\subsubsection{Định luật Ôm}
Cường độ dòng điện hiệu dụng trong mạch điện xoay chiều chỉ có một điện trở có giá trị bằng thương số của điện áp hiệu dụng giữa hai đầu mạch và giá trị của điện trở:
\begin{equation*}
	I = \dfrac{U}{R}.
\end{equation*}
\subsubsection{Đặc điểm về pha của dòng điện và điện áp}
Trong mạch điện xoay chiều chỉ chứa điện trở, cường độ dòng điện qua điện trở cùng pha với điện áp hai đầu đoạn mạch:
\begin{equation*}
	\varphi_u - \varphi_i = 0.
\end{equation*}
\subsubsection{Hệ thức độc lập thời gian}
\begin{equation*}
	\dfrac{u}{U_0}-\dfrac{i}{I_0}=0.
\end{equation*}
\luuy{Giá trị tức thời của dòng điện ($i$) và điện áp ($u$) thay đổi theo thời gian.}
\subsection{Mạch điện xoay chiều chỉ có một tụ điện}
\subsubsection{Định luật Ôm}
Cường độ dòng điện hiệu dụng trong mạch điện xoay chiều chỉ có một tụ điện có giá trị bằng thương số của điện áp hiệu dụng giữa hai đầu mạch và dung kháng:
\begin{equation*}
	I = \dfrac{U}{Z_C},
\end{equation*}
trong đó $Z_C = \dfrac{1}{\omega C}$ là dung kháng của tụ điện.
\subsubsection{Đặc điểm về pha của dòng điện và điện áp}
Trong mạch điện xoay chiều chỉ chứa tụ điện, cường độ dòng điện qua tụ điện sớm pha $\dfrac{\pi}{2}$ so với điện áp hai đầu đoạn mạch:
\begin{equation*}
	\varphi_u - \varphi_i = -\dfrac{\pi}{2}.
\end{equation*}
\subsubsection{Hệ thức độc lập thời gian}
\begin{equation*}
	\dfrac{u^2}{U_0^2}+\dfrac{i^2}{I_0^2}=1.
\end{equation*}
\subsection{Mạch điện xoay chiều chỉ có một cuộn cảm thuần}
\subsubsection{Định luật Ôm}
Cường độ dòng điện hiệu dụng trong mạch điện xoay chiều chỉ có một cuộn cảm thuần có giá trị bằng thương số của điện áp hiệu dụng giữa hai đầu mạch và cảm kháng:
\begin{equation*}
	I = \dfrac{U}{Z_L},
\end{equation*}
trong đó $Z_L =\omega L$ là cảm kháng của cuộn cảm thuần.
\subsubsection{Đặc điểm về pha của dòng điện và điện áp}
Trong mạch điện xoay chiều chỉ chứa cuộn cảm thuần, cường độ dòng điện qua cuộn cảm trễ pha $\dfrac{\pi}{2}$ so với điện áp hai đầu cuộn cảm:
\begin{equation*}
	\varphi_u - \varphi_i = \dfrac{\pi}{2}.
\end{equation*}
\subsubsection{Hệ thức độc lập thời gian}
\begin{equation*}
	\dfrac{u^2}{U_0^2}+\dfrac{i^2}{I_0^2}=1.
\end{equation*}

\section{Mục tiêu bài học - Ví dụ minh họa}
\begin{dang}{Sử dụng được công thức xác định giá trị hiệu dụng, giá trị tức thời, phương trình, độ lệch pha, điện trở trong đoạn mạch\\ chỉ có điện trở}
	
	
	\viduii{2}{Một mạch điện xoay chiều chỉ có điện trở, mối quan hệ về pha của $u$ và $i$ trong mạch là
		\begin{mcq}(2)
			\item $u$ và $i$ cùng pha với nhau.
			\item $u$ sớm pha hơn $i$ góc $\pi/2$.
			\item $u$ và $i$ ngược pha với nhau.
			\item $i$ sớm pha hơn $u$ góc $\pi/2$.
		\end{mcq}
	}
	{\begin{center}
			\textbf{Hướng dẫn giải}
		\end{center}
		
		Một mạch điện xoay chiều chỉ có điện trở, mối quan hệ về pha của $u$ và $i$ trong mạch là $u$ cùng pha $i$.
		
		\textbf{Đáp án: A.}
	}
	
	\viduii{3}{Một điện trở $R=100\ \text{A}$ mắc vào một mạch điện xoay chiều có $u=220 \sqrt 2 \cos 100 \omega t\ \text{(V)}$. Cường độ dòng điện hiệu dụng chạy qua $R$ có giá trị là bao nhiêu?
		\begin{mcq}(4)
			\item $\text{1,1}\ \text{A}$.
			\item $\text{2,2}\sqrt{2}\ \text{A}$.
			\item $\text{2,2}\ \text{A}$.
			\item $\text{1,1}\sqrt{2}\ \text{A}$.
		\end{mcq}
	}
	{\begin{center}
			\textbf{Hướng dẫn giải}
		\end{center}
		
		
		Cường độ dòng điện hiệu dụng chạy qua mạch:
		$$I=\dfrac{U}{R} = \text{2,2}\ \text{A}$$
		
		
		
		\textbf{Đáp án: C.}
	}
	
\end{dang}
\begin{dang}{  Sử dụng được công thức xác định giá trị hiệu dụng, giá trị tức thời, phương trình, độ lệch pha, dung kháng trong đoạn mạch chỉ có tụ điện}
	
	\viduii{3}{Mạch điện $X$ chỉ có tụ điện $C$, biết $C=\dfrac{10^{-4}}{\pi}\,\text{F}$, mắc mạch điện trên vào mạng điện có phương trình $u=100\sqrt{2}\cos\left(100\pi t+\dfrac{\pi}{6}\right)\,\text{V}$. Xác định phương trình dòng điện trong mạch.
		\begin{mcq}(2)
			\item $i=\sqrt{2}\cos\left(100\pi t+\dfrac{2\pi}{3}\right)\,\text{A}$.
			\item $i=\sqrt{2}\cos\left(100\pi t+\dfrac{\pi}{6}\right)\,\text{A}$.
			\item $i=\cos\left(100\pi t+\dfrac{2\pi}{3}\right)\,\text{A}$.
			\item $i=\cos\left(100\pi t+\dfrac{\pi}{6}\right)\,\text{A}$.
		\end{mcq}
	}
	{\begin{center}
			\textbf{Hướng dẫn giải}
		\end{center}
		
		Phương trình dòng điện có dạng $i=I_0\cos\left(100\pi t+\dfrac{\pi}{6}+\dfrac{\pi}{2}\right)\,\text{A}$.
		
		Trong đó
		$I_0=\sqrt{2}\,\text{A}$.
		
		Vậy $i=\sqrt{2}\cos\left(100\pi t+\dfrac{2\pi}{3}\right)\,\text{A}$.
		
		
		\textbf{Đáp án: A.}
	}
	\viduii{3}{Đặt điện áp $u=U_0 \cos \left( 100 \pi t + \dfrac{\pi}{4} \right)\ \text {(V)}$ vào hai đầu đoạn mạch chỉ có tụ điện thì cường độ dòng điện trong mạch là $i=I_0 \cos \left(100 \pi t + \varphi \right)$. Giá trị của $\varphi$ bằng
		\begin{mcq}(4)
			\item $\dfrac{3\pi}{4}\ \text {rad}$.
			\item $\dfrac{\pi}{2}\ \text {rad}$.
			\item $-\dfrac{3\pi}{4}\ \text {rad}$.
			\item $-\dfrac{\pi}{2}\ \text {rad}$.
		\end{mcq}
	}
	{\begin{center}
			\textbf{Hướng dẫn giải}
		\end{center}
		
		
		Trong đoạn mạch xoay chiều chỉ có tụ điện thì mối liên hệ về pha của dòng điện và điện áp là
		\begin{equation*}
			\varphi_u - \varphi_i = -\dfrac{\pi}{2},
		\end{equation*}
		suy ra
		\begin{equation*}
			\varphi_i = \varphi_u + \dfrac{\pi}{2}= \dfrac{3\pi}{4}\ \text {rad}.
		\end{equation*}
		
		\textbf{	Đáp án: A.}
		
		
	}
\end{dang}
\begin{dang}{ Sử dụng được công thức xác định giá trị hiệu dụng, giá trị tức thời, phương trình, độ lệch pha, cảm kháng trong đoạn mạch chỉ có cuộn cảm thuần}
	
	\viduii{3}{Đặt điện áp $u=300\sqrt{2}\cos100 \pi t \ \text {(V)}$ vào hai đầu đoạn mạch chỉ có cuộn cảm thuần với độ tự cảm $L=\dfrac{0,2}{\pi}\ \text H$. Xác biểu thức cường độ dòng điện trong mạch.
		\begin{mcq}(2)
			\item $i=15 \cos 100\pi t\ \text {(A)}$.
			\item $i=15 \cos \left(100 \pi t - \dfrac{\pi}{2}\right) \ \text {(A)}$.
			\item $i=15\sqrt 2 \cos 100 \pi t\ \text {(A)}$.
			\item $i=15\sqrt 2 \cos \left(100\pi t - \dfrac{\pi}{2}\right)\ \text {(A)}$.
		\end{mcq}
	}
	{\begin{center}
			\textbf{Hướng dẫn giải}
		\end{center}
		
		Cảm kháng của cuộn cảm thuần:
		\begin{equation*}
			Z_L = \omega L  = \SI{20}{\Omega}.
		\end{equation*}
		
		Cường độ dòng điện hiệu dụng chạy qua đoạn mạch:
		\begin{equation*}
			I=\dfrac{U}{Z_L}=\SI{15}{\ampere}.
		\end{equation*}
		
		Trong đoạn mạch xoay chiều chỉ có cuộn cảm thuần thì mối liên hệ về pha của dòng điện và điện áp là
		\begin{equation*}
			\varphi_u - \varphi_i = \dfrac{\pi}{2},
		\end{equation*}
		suy ra
		\begin{equation*}
			\varphi_i = \varphi_u-\dfrac{\pi}{2} = -\dfrac{\pi}{2}\ \text {rad}.
		\end{equation*}	
		
		Biểu thức cường độ dòng điện trong mạch là
		\begin{equation*}
			i=I\sqrt 2 \cos (\omega t + \varphi_i) = 15\sqrt 2 \cos \left(100 \pi t -\dfrac{\pi}{2} \right) \ \text{(A)}.
		\end{equation*}
		
		
		\textbf{Đáp án: D.}
	}
	\viduii{3}{Một mạch điện chỉ có cuộn cảm có hệ số tự cảm $L=\dfrac{1}{\pi}\,\text{H}$ mắc vào mạng điện và có phương trình dòng điện $i=2\cos\left(100\pi t+\dfrac{\pi}{6}\right)\,\text{A}$. Hãy viết phương trình hiệu điện thế giữa hai đầu mạch điện.
		\begin{mcq}
			\item $u_L=200\cos\left(100\pi t+\dfrac{2\pi}{3}\right)\,\text{V}$.
			\item $u_L=200\cos\left(100\pi t+\dfrac{\pi}{6}\right)\,\text{V}$.
			\item $u_L=200\sqrt{2}\cos\left(100\pi t+\dfrac{2\pi}{3}\right)\,\text{V}$.
			\item $u_L=200\sqrt{2}\cos\left(100\pi t+\dfrac{\pi}{6}\right)\,\text{V}$.
		\end{mcq}
	}
	{\begin{center}
			\textbf{Hướng dẫn giải}
		\end{center}
		
		
		$u_L$ có dạng $u_L=U_{0L}\cos\left(100\pi t+\dfrac{\pi}{6}+\dfrac{\pi}{2}\right)\,\text{V}$.
		
		Trong đó
		$U_{0L}=I_0Z_L=\SI{200}{V}$.
		
		Vậy $u_L=200\cos\left(100\pi t+\dfrac{2\pi}{3}\right)\,\text{V}$.
		
		\textbf{	Đáp án: A.}
		
		
	}
\end{dang}


