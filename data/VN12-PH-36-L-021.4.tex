% --- chapter
\newcommand{\chapter}[2][]{
	\newcommand{\chapname}{#2}
	\begin{flushleft}
		\begin{minipage}[t]{\linewidth}
			\includegraphics[height=1cm]{hdht-logo.png}
			\hspace{0pt}	
			\sffamily\bfseries\large Bài  27. Tia hồng ngoại và tia tử ngoại
			\begin{flushleft}
				\huge\bfseries #1
			\end{flushleft}
		\end{minipage}
	\end{flushleft}
	\vspace{1cm}
	\normalfont\normalsize
}
%-----------------------------------------------------
\chapter[Tia tử ngoại]{Tia tử ngoại}


\subsection {Bản chất}

\begin{itemize}
	\item Tia tử ngoại là những bức xạ không nhìn thấy được, có bước sóng nhỏ hơn bước sóng của ánh sáng tím ($< \text{0,38}\ \mu \text{m}$). 
	\item Thu được cùng với các tia sáng thông thường.
	\item Có bản chất với ánh sáng.
\end{itemize}

\subsection{Tính chất chung}
\begin{itemize}
	\item Tuân theo các định luật: truyền thẳng, phản xạ, khúc xạ.
	\item Gây được hiện tượng nhiễu xạ, giao thoa như ánh sáng thông thường. 
\end{itemize}

\subsection{Nguồn phát}

\begin{itemize}
	\item Tất cả những vật có nhiệt độ trên $2000^\circ \text{C}$ thì phát được tia tử ngoại.
	\item Nhiệt độ của vật càng cao thì phổ tử ngoại của vật trải càng dài hơn về phía sóng ngắn
	\item Nguồn phát tia tử ngoại: hồ quang điện, mặt trời, đèn hơi thủy ngân,...
\end{itemize}

\subsection{Tính chất và công dụng}

\subsubsection{Tính chất}

\begin{itemize}
	\item  Tác dụng hóa học: gây phản ứng trên phim ảnh, kích thích nhiều phản ứng hóa học.
	\item  Tác dụng sinh học: hủy diệt tế bào, diệt khuẩn nấm mốc, là tiền tố tổng hợp vitamin D.
	\item Kích thích sự phát quang của nhiều chất.
	\item Làm ion hóa không khí và nhiều chất khí khác. \item Gây ra hiện tượng quang điện.
	\item Bị nước và thủy tinh hấp thụ rất mạnh nhưng lại có thể truyền qua được thạch anh. Ngoài ra tầng ozon hấp thụ hết các tia có bước sóng dưới $300\ \text{nm}$ và là tấm áo giáp bảo vệ sinh vật trên Trái Đất.
\end{itemize}

\subsubsection{Công dụng}

\begin{itemize}
	\item Y học: dùng để tiệt trùng dụng cụ phẫu thuật, chữa bệnh còi xương.
	\item Công nghiệp thực phẩm: tiệt trùng thực phẩm.
	\item Công nghiệp cơ khí: tìm về nứt (khuyết tật) trên bề mặt sản phẩm.
\end{itemize}