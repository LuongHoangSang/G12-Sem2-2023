% --- chapter
\newcommand{\chapter}[2][]{
	\newcommand{\chapname}{#2}
	\begin{flushleft}
		\begin{minipage}[t]{\linewidth}
			\includegraphics[height=1cm]{hdht-logo.png}
			\hspace{0pt}	
			\sffamily\bfseries\large Bài  30. Hiện tượng quang điện. Thuyết lượng tử ánh sáng
			\begin{flushleft}
				\huge\bfseries #1
			\end{flushleft}
		\end{minipage}
	\end{flushleft}
	\vspace{1cm}
	\normalfont\normalsize
}
%-----------------------------------------------------
\chapter[Thuyết lượng tử ánh sáng và lưỡng tính sóng hạt của ánh sáng]{ Thuyết lượng tử ánh sáng \\và lưỡng tính sóng hạt của ánh sáng}

\subsection{Thuyết lượng tử ánh sáng}
Nội dung thuyết lượng tử ánh sáng (thuyết phôtôn):
\begin{itemize}
	\item Ánh sáng được tạo thành bởi các hạt gọi là phôtôn;
	\item Với mỗi ánh sáng đơn sắc có tần số $f$, các phôtôn đều giống nhau, mỗi phôtôn mang năng lượng bằng $hf$;
	\item Trong chân không, phôtôn bay với tốc độ $c=\SI{3e8}{\meter / \second}$ dọc theo các tia sáng;
	\item Mỗi lần một nguyên tử hay phân tử phát xạ hoặc hấp thụ ánh sáng thì chúng phát ra hay hấp thụ một phôtôn.
	\luuy{Phôtôn chỉ tồn tại trong trạng thái chuyển động. Không có phôtôn đứng yên.}
\end{itemize}
\subsection{Lưỡng tính sóng - hạt của ánh sáng}
Ánh sáng vừa có tính chất sóng, vừa có tính chất hạt: ánh sáng có lưỡng tính sóng - hạt. Ánh sáng có bước sóng càng lớn thì càng thể hiện rõ tính chất sóng (giao thoa, nhiễu xạ, \ldots). Ngược lại ánh sáng có bước sóng càng nhỏ thì càng thể hiện rõ tính chất hạt (quang điện, đâm xuyên, \ldots).
\luuy{ Dù tính chất nào của ánh sáng thể hiện ra thì ánh sáng vẫn có bản chất điện từ.}
