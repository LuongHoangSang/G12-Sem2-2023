
\chapter[Lý thuyết: Con lắc đơn;\\
Bài tập: Phương trình dao động điều hòa của con lắc đơn]{Lý thuyết: Con lắc đơn;\\	Bài tập: Phương trình dao động điều hòa của con lắc đơn}
\section{Lý thuyết}
\subsection{Cấu tạo của con lắc đơn}
\begin{itemize}
	\item Con lắc đơn gồm vật nặng $m$ gắn vào sợi dây có chiều dài $l$.
	\item Điều kiện để con lắc đơn dao động điều hòa: bỏ qua ma sát, lực cản, dây không giãn và rất nhẹ, vật coi là chất điểm và $\alpha_0 <<1\ \text{rad}$ hay $s_0 << l$.
\end{itemize}
\luuy{Các phép xấp xỉ gần đúng áp dụng được cho các bài toán con lắc đơn
	Các giá trị lượng giác của một góc nhỏ có thể áp dụng các phép xấp xỉ gần đúng sau:
	\begin{eqnarray*}
		\sin \alpha &\approx& \alpha;\\
		\cos \alpha &\approx& 1 - \dfrac {\alpha ^2}{2};\\
		\tan \alpha &\approx& \alpha.
	\end{eqnarray*}
}
\subsection{Li độ, tần số và chu kỳ của con lắc đơn}

\subsubsection{Li độ góc và li độ cong}
\begin{tabular}{|m{7em}|m{14em}|m{14em}|}
	\hline
	\thead{} & \thead{Li độ góc} & \thead{Li độ cong} \\
	\hline
	\textbf{Kí hiệu} & \nfhead{$\alpha$} & \nfhead{$s$} \\
	\hline
	\textbf{Biên độ} & \nfhead{$\alpha _0$} & \nfhead{$s_0$} \\
	\hline
	\textbf{Phương trình} &  \nfhead{$\alpha = \alpha_0 \cos (\omega t + \varphi)$}  & \nfhead{$s=s_0 \cos (\omega t + \varphi)$\\ với $s_0 = \alpha _0 l$}\\
	\hline
\end{tabular}
\subsubsection{Tần số góc, chu kỳ, tần số}
\begin{itemize}
	\item Tần số góc:
	\begin{equation*}
		\omega =\sqrt{\dfrac{g}{l}}.
	\end{equation*}
	\item Chu kỳ:
	\begin{equation*} T=\dfrac{2\pi}{\omega}=2\pi\sqrt{\dfrac{l}{g}}.
	\end{equation*}
	\item Tần số:
	\begin{equation*} f=\dfrac{1}{T}=\dfrac{\omega}{2\pi}=\dfrac{1}{2\pi}\sqrt{\dfrac{g}{l}}.
	\end{equation*}
\end{itemize}
\subsection{Hệ thức độc lập với thời gian}
\begin{eqnarray*}
	s^2_0 &=&s^2+\left(\dfrac{v}{\omega}\right)^2; \\ \alpha^2_0 &=& \alpha^2 +\dfrac{v^2}{gl}.
\end{eqnarray*}
\subsection{Phương trình dao động điều hòa của con lắc đơn}
\subsubsection{Phương trình dao động điều hòa theo li độ dài}
\begin{equation*}
	s=s_0 \cos(\omega t + \varphi)
\end{equation*}
\subsubsection{Phương trình dao động điều hòa theo li độ góc}
\begin{equation*}
	\alpha =\alpha_0 \cos (\omega t + \varphi)\ \text{(rad)}
\end{equation*}

\luuy{Mối liên hệ giữa li độ dài và li độ góc: $s=\alpha l,\ s_0=\alpha_0 l.$}

\subsection{Phương trình vận tốc, gia tốc của con lắc đơn dao động điều hòa}
\subsubsection{Phương trình vận tốc}
\begin{equation*}
	v=s'=-\omega s_0 \sin(\omega t +\varphi) = -\omega l \alpha_0 \sin(\omega t + \varphi).
\end{equation*}
\subsubsection{Phương trình gia tốc} 
\begin{equation*}
	a=v' = -\omega^2 s_0 \cos (\omega t + \varphi) = -\omega^2 l\alpha_0 \cos (\omega t +\varphi ).
\end{equation*}
\subsection{Hệ thức độc lập}
\begin{equation*}
	s^2_0 =s^2+\left(\dfrac{v}{\omega}\right)^2;
\end{equation*}
\begin{equation*}
	\alpha^2_0 = \alpha^2 +\dfrac{v^2}{gl};
\end{equation*}
\begin{equation*}
	a= - \omega^2 s=-\omega^2 \alpha l.
\end{equation*}
\section{Mục tiêu bài học - Ví dụ minh họa}
\begin{dang}{Ghi nhớ được các công thức tính li độ,\\ tần số và chu kì của con lắc đơn}
	\viduii{1}
	{
		Chu kì của con lắc đơn dao động nhỏ ($\sin \alpha \approx \alpha\ (\SI{}{\radian})$) là
		\begin{mcq}(2)
			\item $T=\dfrac {1}{2\pi} \sqrt {\dfrac{l}{g}}$.
			\item $T=\dfrac {1}{2\pi} \sqrt {\dfrac{g}{l}}$.
			\item $T=\sqrt {2\pi \dfrac {l}{g}}$.
			\item $T=2\pi \sqrt {\dfrac{l}{g}}$.
		\end{mcq}
	}
	{\begin{center}
			\textbf{Hướng dẫn giải}
		\end{center}
		
		Chu kì dao động của con lắc đơn dao động nhỏ ($\sin \alpha \approx \alpha\ (\SI{}{\radian})$) là $T=2\pi \sqrt {\dfrac{l}{g}}$.
		
		\textbf{Đáp án: D.}
	}
	\viduii{2}
	{
		Tìm độ dài của con lắc đơn có chu kì $\SI{1}{\second}$ ở nơi có gia tốc trọng trường $g=\SI{9.81}{\meter / \second ^2}$.
	}
	{
		\begin{center}
			\textbf{Hướng dẫn giải}
		\end{center}
		
		Độ dài của con lắc đơn: $$T=2\pi \sqrt {\dfrac{l}{g}} \Rightarrow l=\left(\dfrac {T}{2\pi}\right)^2 \cdot g = \left(\dfrac {\SI{1}{\second}}{\xsi{2\pi}{\radian}}\right)^2 \cdot \SI{9.81}{\meter / \second^2} = \SI{0.25}{\meter}.$$
	}
\end{dang}
\begin{dang}{Sử dụng được các công thức tính li độ,\\ tần số và chu kì của con lắc đơn}
	\viduii{3}{Một con lắc đơn gồm sợi dây có chiều dài $20\ \text{cm}$ treo tại một điểm cố định. Kéo con lắc khỏi phương thẳng đứng một góc bằng 0,1 rad về phía bên phải, rồi truyền cho con lắc một tốc độ bằng $14\sqrt 3\ \text{cm/s}$ theo phương vuông góc với dây. Coi con lắc dao động điều hòa. Cho gia tốc trọng trường $9,8\ m/s^2$. Tính biên độ dài của con lắc.
	}
	{\begin{center}
			\textbf{Hướng dẫn giải}
		\end{center}
		
		\begin{itemize}
			\item Li độ của con lắc:
			\begin{equation*}
				s = l \alpha = 0,02\ \text{m}. 
			\end{equation*}
			\item Tốc độ góc của con lắc:
			\begin{equation*}
				\omega = \sqrt{\dfrac{g}{l}} = 7\ \text{rad/s}.
			\end{equation*}
			\item Áp dụng hệ thức độc lập:
			\begin{equation*}
				s^2_0 =s^2+\left(\dfrac{v}{\omega}\right)^2 \Rightarrow s_0 = 0,04\ \text{m}.
			\end{equation*}
		\end{itemize}
	}
	\viduii{3}{Một con lắc đơn dài $l=\SI{2.00}{\meter}$ dao động điều hòa tại một nơi có gia tốc rơi tự do $g=\SI{9.80}{\meter / \second ^2}$. Hỏi con lắc thực hiện được bao nhiêu dao động toàn phần trong $\SI{5.00}{}$ phút?
	}
	{\begin{center}
			\textbf{Hướng dẫn giải}
		\end{center}
		
		Tần số dao động của con lắc đơn:
		$$f=\dfrac {1}{2\pi} \sqrt {\dfrac{g}{l}} = \dfrac {1}{2\pi} \sqrt {\dfrac{\SI{9.80}{\meter / \second ^2}}{\SI{2.00}{\meter}}} \approx \SI{0.35}{\hertz}.$$
		
		Số dao động toàn phần mà con lắc thực hiện được trong $\SI{5.00}{}$ phút:
		$$N=ft =\SI{0.35}{\hertz} \cdot \SI{5.00}{}\cdot \SI{60}{\second} = 105.$$
	}
	
\end{dang}
\begin{dang}{Giải thích được các đại lượng\\ có trong phương trình dao động điều hòa,\\ phương trình vận tốc, gia tốc\\ của con lắc đơn}
	\viduii{2}
	{
		Con lắc đơn dao động điều hòa với phương trình $s=\cos(2t+0,69)\ \text{cm}$, $t$ tính theo đơn vị $\text s$. Khi $t=0,135\ \text{s}$ thì pha dao động là
		\begin{mcq}(4)
			\item $0,57\ \text{rad}$.
			\item $0,75\ \text{rad}$.
			\item $0,96\ \text{rad}$.
			\item $0,69\ \text{rad}$.
		\end{mcq}
	}
	{\begin{center}
			\textbf{Hướng dẫn giải}
		\end{center}
		
		Pha của dao động là $(\omega t + \varphi)$. Thay số, ta được $0,96\ \text{rad}$.
		
		\textbf{Đáp án: C.}
	}
	\viduii{2}
	{
		Một con lắc đơn gồm vật nặng có khối lượng $m$ dao động điều hòa với biên độ góc $\alpha_0$. Biểu thức tính vận tốc của vật nặng đó ở li độ $\alpha$ là
		\begin{mcq}(2)
			\item $v=\pm\sqrt{gl(\alpha^2-\alpha_0^2)}$.
			\item $v=\pm\sqrt{2gl(\alpha^2-\alpha_0^2)}$.
			\item $v=\pm\sqrt{2gl(\alpha_0^2-\alpha^2)}$.
			\item $v=\pm\sqrt{gl(\alpha_0^2-\alpha^2)}$.
		\end{mcq}
	}
	{
		\begin{center}
			\textbf{Hướng dẫn giải}
		\end{center}
		
		Từ hệ thức độc lập:
		$$\alpha^2_0 = \alpha^2 +\dfrac{v^2}{gl}$$
		Suy ra $v=\pm\sqrt{gl(\alpha_0^2-\alpha^2)}$.
		
		\textbf{Đáp án: D.}
	}
\end{dang}
\begin{dang}{Xây dựng được phương trình\\ dao động điều hòa, phương trình vận tốc, gia tốc của con lắc đơn}
	\viduii{3}{Một con lắc đơn có chiều dài $l = 16\ \text{cm}$. Kéo con lắc lệch khỏi vị trí cân bằng một góc $9^\circ$ rồi thả nhẹ. Bỏ qua mọi ma sát, lấy $g = 10\ \text{m/s}^2$, $\pi^2= 10$. Chọn gốc thời gian lúc thả vật, chiều dương cùng chiều với chiều chuyển động ban đầu của vật. Viết phương trình dao động theo li độ góc của con lắc đơn.
	}
	{\begin{center}
			\textbf{Hướng dẫn giải}
		\end{center}
		
		\begin{itemize}
			\item Tần số góc của dao động:
			\begin{equation*}
				\omega = \dfrac{g}{l} = \text{2,5}\pi\ \text{rad/s}.
			\end{equation*}
			\item Đổi $\alpha_0 = 9^\circ =\text{0,157}\ \text{rad}$. 
			\item Pha ban đầu:
			\begin{equation*}
				\cos \varphi =\dfrac{\alpha}{\alpha_0} = \dfrac{-\alpha_0}{\alpha_0}=-1 \Rightarrow \varphi = \pi\ \text{rad}.
			\end{equation*}
			\item Phương trình dao động theo li độ góc: 
			\begin{equation*}
				\alpha = \text{0,157} \cos\ (\text{2,5}\pi + \pi)\ \text{rad}.
			\end{equation*}
		\end{itemize}
	}
	\viduii{3}{Một con lắc đơn dao động điều hòa với chu kì $T = 2 \ \text{s}$. Lấy $g = 10\ \text{m/s}^2$, $\pi^2 = 10$. Viết phương trình dao động của con lắc theo li độ dài. Biết rằng tại thời điểm ban đầu vật có li độ góc $\alpha = \text{0,05}\ \text{rad}$ và vận tốc $v = - \text{15,7}\ \text{cm/s}$.
	}
	{\begin{center}
			\textbf{Hướng dẫn giải}
		\end{center}
		
		\begin{itemize}
			\item Tốc độ góc của dao động:
			\begin{equation*}
				\omega =\dfrac{2\pi}{T} = \pi\ \text{rad/s}.
			\end{equation*}
			\item Chiều dài dây treo con lắc đơn:
			\begin{equation*}
				l= \dfrac{g}{\omega^2} = 1\ \text{m} = 100\ \text{cm}.
			\end{equation*}
			\item Biên độ cực đại:
			\begin{equation*}
				s_0 =\sqrt {(\alpha l)^2 + \dfrac{v^2}{\omega^2}} = 5\sqrt 2\ \text{cm}.
			\end{equation*}
			\item Pha ban đầu:
			\begin{equation*}
				\cos \varphi  = \dfrac{\alpha l}{s_0} =\dfrac{1}{\sqrt 2} = \cos \left(\pm \dfrac{\pi}{4}\right);\ \text{vì}\ v<0\ \text{nên}\ \varphi = \dfrac{\pi}{4}\ \text{rad}.
			\end{equation*}
			\item Phương trình dao động theo li độ dài:
			\begin{equation*}
				s=5\sqrt 2 \cos \left(\pi t + \dfrac{\pi}{4}\right)\ \text{cm}.
			\end{equation*}
		\end{itemize}
	}
	
\end{dang}
\begin{dang}{Sử dụng được phương trình\\ dao động điều hòa, phương trình vận tốc, gia tốc để xác định xác định các giá trị tức thời, hiệu dụng, cực đại}
	\viduii{3}{Một con lắc đơn dao động với phương trình $s=10\sin(2t)\ \text{cm}$. Tính vận tốc của con lắc ở thời điểm $t=\pi /6\ \text s$.
		\begin{mcq}(2)
			\item $v=10\sqrt{3}\ \text{cm/s}$.
			\item $v=10\sqrt{2}\ \text{cm/s}$.
			\item $v=10\ \text{cm/s}$.
			\item $v=5\ \text{cm/s}$.
		\end{mcq}
	}
	{\begin{center}
			\textbf{Hướng dẫn giải}
		\end{center}
		
		\begin{itemize}
			\item Phương trình vận tốc:
			\begin{equation*}
				v=s' = 20 \cos (2t)
			\end{equation*}
			\item Vận tốc ở thời điểm $t=\pi /6\ \text s$: 
			\begin{equation*}
				v= 20 \cos (2\cdot \dfrac{\pi}{6}) = 10\ \text{cm/s}.
			\end{equation*}
		\end{itemize}
		
		\textbf{Đáp án: C.}
	}
	\viduii{3}{Một con lắc đơn dao động với phương trình $s=10\sin(2t)\ \text{cm}$. Tính gia tốc của con lắc ở thời điểm $t=\pi /6\ \text s$.
	}
	{\begin{center}
			\textbf{Hướng dẫn giải}
		\end{center}
		
		\begin{itemize}
			\item Li độ ở thời điểm $t=\pi /6\ \text s$:
			\begin{equation*}
				s = 10 \sin (2 \cdot \dfrac{\pi}{6}) = 5\sqrt{3}\ \text{cm}.
			\end{equation*}
			\item Gia tốc ở thời điểm $t=\pi /6\ \text s$:
			\begin{equation*}
				a=-\omega^2 s = -20\sqrt{3}\ \text{cm/s}^2.
			\end{equation*}
		\end{itemize}
	}
\end{dang}