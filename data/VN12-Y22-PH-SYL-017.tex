
\chapter[Dạng bài: Năng lượng của con lắc lò xo dao động điều hòa]{Dạng bài: Năng lượng của con lắc lò xo dao động điều hòa}
\section{Lý thuyết}
\subsection{Động năng}
Động năng của con lắc lò xo là động năng của vật $m$:
\begin{equation*}
	W_{\text {đ}} = \dfrac {1}{2}mv^2,
\end{equation*}
trong đó:
\begin{itemize}
	\item  $m$ là khối lượng của vật,
	\item  $v$ là vận tốc của vật,
	\item $W_{\text {đ}}$ là động năng của vật.
\end{itemize}
\subsection{Thế năng}
Chọn gốc thế năng tại vị trí cân bằng:
\begin{equation*}
	W_{\text {t}} = \dfrac {1}{2}kx^2,
\end{equation*}
trong đó:
\begin{itemize}
	\item  $k$ là độ cứng của lò xo,
	\item  $x$ là li độ của vật,
	\item $W_\text{t}$ là thế năng của vật.
\end{itemize}
\subsection{Cơ năng}
Cơ năng của con lắc lò xo là tổng động năng của vật và thế năng đàn hồi:
\begin{equation*}
	W =W_{\text {đ}}+W_{\text {t}}= \dfrac {1}{2}mv^2 + \dfrac {1}{2}kx^2 = \dfrac {1}{2}kA^2 = \dfrac {1}{2} m \omega ^2 A^2.
\end{equation*}

Cơ năng của con lắc được \textbf{bảo toàn} nếu bỏ qua \textbf{ma sát}:
\begin{equation*}
	W =W_{\text {đ}_1}+W_{\text {t}_1}= W_{\text {đ}_2}+W_{\text {t}_2}=\dfrac {1}{2}kA^2 = \dfrac {1}{2} m \omega ^2 A^2= \text{hằng số}.
\end{equation*}


\luuy{
	Trong quá trình dao động của con lắc lò xo:
	\begin{itemize}
		\item Khi động năng tăng thì thế năng giảm, khi động năng cực đại thì thế năng bằng 0 và ngược lại.
		\item Khi vật từ biên về vị trí cân bằng, vật chuyển động nhanh dần, khi đó động năng tăng dần, thế năng giảm dần. Khi vật chuyển động từ vị trí cân bằng ra biên, vật chuyển động chậm dần nên động năng giảm dần, thế năng tăng dần.
		\item 	Tại vị trí biên, động năng của vật bằng 0, thế năng cực đại. Tại vị trí cân bằng động năng của vật cực đại và thế năng bằng 0.
		\item Động năng, thế năng biến thiên tuần hoàn với tần số $f' = 2 f$, chu kỳ $T'=\dfrac{T}{2}$, tần số góc $\omega '=2\omega$. 
	\end{itemize}	
}
\section{Mục tiêu bài học - Ví dụ minh họa}
\begin{dang}{ Xác định động năng, thế năng đàn hồi, thế năng trọng trường, cơ năng\\ của con lắc lò xo.
	}
	
	
	
	\viduii{2}{Một con lắc lò xo có độ cứng $k=100\ \text{N/m}$. Vật nặng dao động với biên độ $A=20\ \text{cm}$, khi vật đi qua li độ $x=12\ \text{cm}$ thì động năng của vật bằng
		\begin{mcq}(4)
			\item $\text{1,44}\ \text{J}$.
			\item $\text{1,28}\ \text{J}$.
			\item $\text{2,56}\ \text{J}$.
			\item $\text{0,72}\ \text{J}$.
		\end{mcq}
	}
	{\begin{center}
			\textbf{Hướng dẫn giải}
		\end{center}
		
		Động năng của vật là $W_{\text {đ}}=W-W_{\text {t}}=\dfrac{1}{2}kA^2-\dfrac{1}{2}kx^2=\text{1,28}\ \text{J}$.
		
		\textbf{Đáp án: B.}
	}
	\viduii{3}{Cho một con lắc lò xo dao động điều hoà với phương trình $x = 5\cos \left(20t+\dfrac{\pi}{6}\right)\ \text{cm}$. Biết vật nặng có khối lượng $m = \SI{200}{g}$. Cơ năng của con lắc trong quá trình dao động bằng
		\begin{mcq}(4)
			\item 0,7 J.
			\item 0,5 J.
			\item 0,9 J.
			\item 0,1 J.
		\end{mcq}
	}
	{\begin{center}
			\textbf{Hướng dẫn giải}
		\end{center}
		
		Áp dụng công thức
		
		$W=\dfrac{1}{2}m\omega^2A^2= \dfrac{1}{2} 0,2 \cdot 20^2 \cdot 0,05^2 =\SI{0,1}{J}$.
		
		\textbf{Đáp án: D.}
	}
\end{dang}
\begin{dang}{Sử dụng định luật bảo toàn năng lượng kết hợp với dữ kiện đã cho về mối liên hệ giữa động năng, cơ năng, thế năng để xác định động năng, cơ năng, thế năng đàn hồi,\\ thế năng trọng trường của con lắc lò xo.}
	\ppgiai{
		\begin{itemize}
			\item Động năng của con lắc có giá trị gấp n lần thế năng: 
			
			$W_{\text {đ}}=nW_{\text {t}}\Rightarrow W =W_{\text {đ}}+W_{\text {t}}=\left(\text{n}+1 \right)W_{\text {t}}$.
			
			$\Rightarrow W_{\text {t}}=\dfrac{W}{\text{n}+1}\Leftrightarrow \dfrac{1}{2}kx^2= \dfrac{\dfrac{1}{2}kA^2}{\text{n}+1}$
			\begin{equation*}
				\Leftrightarrow x=\pm \dfrac{A}{\sqrt{n+1}}.
			\end{equation*}
			Tương tự, ta cũng chứng minh được: 
			\begin{equation*} v=\pm \sqrt{\dfrac{n}{n+1}}v_\text{max}.\end{equation*}
			
			\item Thế năng của con lắc có giá trị gấp n lần động năng: 
			
			$W_{\text {t}}=nW_{\text {đ}}\Rightarrow W =\dfrac{W_{\text {t}}}{n}+W_{\text {t}}=\dfrac{\text{n}+1}{n} W_{\text {t}}$.
			
			$\Rightarrow W_{\text {t}}=\dfrac{\text{n}}{\text{n}+1}W\Leftrightarrow \dfrac{1}{2}kx^2=\dfrac{\text{n}}{\text{n}+1}\cdot \dfrac{1}{2}kA^2 $
			\begin{equation*}
				\Leftrightarrow x=\pm \sqrt{\dfrac{n}{n+1}}A.
			\end{equation*}
			Tương tự, ta cũng chứng minh được: 
			\begin{equation*} v=\pm \dfrac{v_\text{max}}{\sqrt{n+1}}.\end{equation*}
			\luuy{Khoảng thời gian giữa hai lần tiên tiếp động năng bằng thế năng là $\dfrac{T}{4}$, với $T$ là chu kỳ dao động của con lắc lò xo.}
		\end{itemize}
		
	}	
	\viduii{2}{Một con lắc lò xo dao động điều hoà với biên độ $A$ trên mặt phẳng nằm ngang. Khi động năng của vật gấp 2 lần thế năng thì vận tốc của vật là $10\ \text{cm/s}$. Vận tốc cực đại của vật trong quá trình dao động là
		
		\begin{mcq}(4)
			\item $10\ \text{cm/s}$.
			\item $5\sqrt{6}\ \text{cm/s}$.
			\item $20\ \text{cm/s}$.
			\item $20\sqrt{3}\ \text{cm/s}$.
		\end{mcq}
	}
	{\begin{center}
			\textbf{Hướng dẫn giải}
		\end{center}
		
		Động năng của vật gấp 2 lần thế năng nên $W_{\text {đ}}=2W_{\text {t}}$ với $n=2$.
		
		Khi đó, vận tốc của vật là $v=\pm \sqrt{\dfrac{n}{n+1}}v_\text{max}=\pm \sqrt{\dfrac{2}{2+1}}v_\text{max}$
		
		$\Rightarrow v_\text{max}= \dfrac{\sqrt{6}}{2}v=5\sqrt{6}\ \text{cm/s}$.
		
		\textbf{Đáp án: B.}
	}
	\viduii{3}{Một con lắc lò xo dao động điều hoà theo phương nằm ngang. Biết rằng khi tốc độ của vật là $\text{48}\pi\ \text{cm/s}$ thì động năng bằng n lần thế năng, còn khi vật có li độ $x=4\ \text{cm}$ thì thế năng bằng n lần động năng. Chu kỳ dao động của con lắc là
		\begin{mcq}(4)
			\item $\text{0,52}\ \text{s}$.
			\item $\text{0,6}\ \text{s}$.
			\item $\text{0,167}\ \text{s}$. 
			\item $\text{0,256}\ \text{s}$.
		\end{mcq}
	}
	{\begin{center}
			\textbf{Hướng dẫn giải}
		\end{center}
		
		Khi $W_{\text {đ}}=nW_{\text {t}}$ thì  $v=\pm \sqrt{\dfrac{n}{n+1}}v_\text{max}$.
		
		Do đó $\left( \dfrac{v}{v_\text{max}} \right)^2=\dfrac{n}{n+1}\Rightarrow \left( \dfrac{\text{48}\pi\ \text{cm/s}}{\omega A} \right)^2=\dfrac{n}{n+1}$.
		
		Khi $W_{\text {t}}=nW_{\text {đ}}$ thì $x=\pm \sqrt{\dfrac{n}{n+1}}A\Rightarrow \left( \dfrac{\text{4}\ \text{cm}}{A} \right)^2=\dfrac{n}{n+1} $
		
		
		Do đó $\left( \dfrac{\text{48}\pi\ \text{cm/s}}{\omega A} \right)^2=\left( \dfrac{\text{4}\ \text{cm}}{A} \right)^2\Rightarrow \omega= 12\pi \ \text{rad/s}\Rightarrow T=\text{0,167}\ \text{s}$.
		
		\textbf{Đáp án: C.}
	}
\end{dang}
