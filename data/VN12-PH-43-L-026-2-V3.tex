
\chapter[Ứng dụng của LASER (đọc thêm)]{Ứng dụng của LASER (đọc thêm)}

\section{Mục tiêu bài học - Ví dụ minh họa}

\begin{dang}{Công suất chùm LASER.}
	
	\ppgiai{
		Công suất của chùm tia LASER:
		\begin{equation}
			\calP = n_p \varepsilon,
		\end{equation}
		trong đó:
		\begin{itemize}
			\item $n_p$ là số phôtôn do LASER phát ra trong $\SI{1}{\second}$;
			\item $\varepsilon$ là năng lượng của phôtôn trong chùm tia LASER.
		\end{itemize}
		Vì LASER có tính đơn sắc cao (các phôtôn trong chùm tia LASER có cùng tần số) nên năng lượng của phôtôn trong chùm tia LASER xác định:
		\begin{equation}
			\varepsilon = hf = \dfrac{hc}{\lambda}.
		\end{equation}
	}
	
	\viduii{2}{ Cho nguồn LASER phát ra chùm bức xạ có bước sóng $\SI{0.45}{\micro \meter}$ với công suất $\SI{1.2}{\watt}$. Trong mỗi giây, số phôtôn do chùm sáng phát ra là
		\begin{mcq}(2)
			\item $\SI{4.42e12}{}$.
			\item $\SI{4.42e18}{}$.
			\item $\SI{2.72e12}{}$.
			\item $\SI{2.72e18}{}$.
	\end{mcq}}
	{\begin{center}
			\textbf{Hướng dẫn giải}
		\end{center}
		
		Áp dụng công thức
		\begin{equation*}
			\calP = n_p \varepsilon,
		\end{equation*}
		suy ra
		\begin{equation*}
			n_p= \dfrac{\calP}{\varepsilon} = \dfrac{\calP \lambda}{hc} \approx \SI{2.72e18}{}.
		\end{equation*}
		
		\begin{center}
			\textbf{Câu hỏi tương tự}
		\end{center}
		
		Cho nguồn LASER phát ra chùm bức xạ có bước sóng $\SI{0.5}{\micro \meter}$ với công suất $\SI{1.5}{\watt}$. Trong mỗi giây, số phôtôn do chùm sáng phát ra là
		\begin{mcq}(2)
			\item $\SI{3,78 e12}{}$.
			\item $\SI{4,43 e18}{}$.
			\item $\SI{4,43 e12}{}$.
			\item $\SI{3,78 e18}{}$.
		\end{mcq}
		
		\textbf{Đáp án:} D.
	}
	\viduii{3}{ Nguồn sáng thứ nhất có công suất $\calP_1$ phát ra chùm tia LASER có bước sóng $\lambda_1 = \SI{450}{\nano \meter}$. Nguồn sáng thứ hai có công suất $\calP_2$ phát ra chùm tia LASER có bước sóng $\lambda_2 = \SI{0.6}{\micro \meter}$. Trong cùng một khoảng thời gian, tỉ số giữa số phôtôn mà nguồn thứ nhất phát ra so với số phôtôn mà nguồn thứ hai phát ra là $3:1$. Tỉ số $\calP_1:\calP_2$ là 
		\begin{mcq}(2)
			\item $4:1$.
			\item $9:4$.
			\item $4:3$.
			\item $3:1$.
	\end{mcq}}
	{\begin{center}
			\textbf{Hướng dẫn giải}
		\end{center}
		
		Ta có hệ phương trình:
		\begin{equation*}\begin{cases}
				\calP_1 = n_1 \varepsilon_1 \\
				\calP_2 = n_2 \varepsilon_2
			\end{cases},\end{equation*}
		suy ra
		\begin{equation*}
			\dfrac{\calP_1}{\calP_2} = \dfrac{n_1 \varepsilon_1}{n_2 \varepsilon_2} = \dfrac{n_1 \lambda_2}{n_2 \lambda_1} = \dfrac{3}{1} \cdot \dfrac{\SI{0.6}{\micro \meter}}{\SI{0.45}{\micro \meter}} = 4.
		\end{equation*}
		
		Vậy $\calP_1:\calP_2 = 4:1$.
		
		\begin{center}
			\textbf{Câu hỏi tương tự}
		\end{center}
		
		Nguồn sáng thứ nhất có công suất $\calP_1$ phát ra chùm tia LASER có bước sóng $\lambda_1 = \SI{560}{\nano \meter}$. Nguồn sáng thứ hai có công suất $\calP_2$ phát ra chùm tia LASER có bước sóng $\lambda_2 = \SI{0.65}{\micro \meter}$. Trong cùng một khoảng thời gian, tỉ số giữa số phôtôn mà nguồn thứ nhất phát ra so với số phôtôn mà nguồn thứ hai phát ra là $3:2$. Tỉ số $\calP_1:\calP_2$ là 
		\begin{mcq}(2)
			\item $43:16$.
			\item $92:41$.
			\item $65:56$.
			\item $33:13$.
		\end{mcq}
		
		\textbf{Đáp án:} C.}
	
\end{dang}

\begin{dang}{Một số ứng dụng của LASER.}
	
	\ppgiai{
		Nhiệt lượng cần thiết để đưa một khối vật chất có khối lượng $ m $ và nhiệt dung riêng $ c $ từ nhiệt độ $ t_{1} $ đến nhiệt độ $ t_{2} $ là
		\begin{equation}
			Q = mc\left( t_{2} - t_{1} \right) .
		\end{equation}
		Nhiệt lượng cần thiết để đưa một khối vật chất $ m $ và có nhiệt nóng chảy $ \lambda $ đang ở nhiệt độ nóng chảy chuyển từ thể rắn sang thể lỏng là
		\begin{equation}
			Q = m \lambda .
		\end{equation}
		Nhiệt lượng cần thiết để một khối chất lỏng có khối lượng $ m $ và nhiệt hóa hơi $ L $ đang ở nhiệt dộ hóa hơi chuyển từ thể lỏng sang thể hơi là
		\begin{equation}
			Q = m L .
		\end{equation}
	}
	
	\viduii{3}{
		Trong y học, người ta dùng một laze phát ra chùm sáng có bước sóng $ \lambda $ để "đốt" các mô mềm. Biết rằng để đốt được một phần mô mềm có thể tích $ \SI{6}{mm^{3}} $ thì phần mô này cần hấp thụ hoàn toàn năng lượng của $ \num{45 e18} $ photon của chùm laze trên. Coi năng lượng trung bình để đốt hoàn toàn $ \SI{1}{mm^{3}} $ mô là $ \SI{2,53}{J} $. Lấy $ h = \SI{6,625 e-34}{J.s} $ và $ c = \SI{3 e8}{m/s} $. Giá trị của $ \lambda $ là
		\begin{mcq}(4)
			\item $ \SI{589}{nm} $.
			\item $ \SI{683}{nm} $.
			\item $ \SI{485}{nm} $.
			\item $ \SI{489}{nm} $.
		\end{mcq}
	}
	{
		\begin{center}
			\textbf{Hướng dẫn giải}
		\end{center}
		
		Ta có:
		$$
		Q = N \dfrac{hc}{\lambda} \rightarrow \lambda = \SI{0,589 e-6}{m}.
		$$
		\begin{center}
			\textbf{Câu hỏi tương tự}
		\end{center}
		
		Dùng laze $ \text{CO}_{2} $ có công suất $ P = \SI{10}{W} $ để làm dao mổ. Khi tia laze được chiếu vào vị trí cần mổ sẽ làm cho nước ở phần mô chỗ đó bốc hơi và mô bị cắt. Biết chùm laze có bán kính $ r = \SI{0,1}{mm} $ và di chuyển với vận tốc $ v = \SI{0,5}{cm/s} $ trên bề mặt của mô mềm. Biết thể tích nước bốc hơi trong $ \SI{1}{s} $ là $ \SI{3,5}{mm^{3}} $. Chiều sâu cực đại của vết cắt là
		\begin{mcq}(4)
			\item $ \SI{1}{mm} $.
			\item $ \SI{2}{mm} $.
			\item $ \SI{3,5}{mm} $.
			\item $ \SI{4}{mm} $.	
		\end{mcq}
		\textbf{Đáp án:} C.
	}
	
	\viduii{3}
	{
		Dùng chùm tia laze có công suất $ P = \SI{10}{W} $ để nấu chảy khối thép có khối lượng $ \SI{1}{kg} $. Nhiệt độ ban đầu của khối thép $ t_{0} = 30\circ $, nhiệt dung riêng của thép $ c = \SI{488}{J/kg \cdot \text{độ}} $, nhiệt nóng chảy của thép $ L = \SI{270}{kJ/kg} $, điểm nóng chảy của thép $ T_{c} = 1535\circ $. Coi rằng không bị mất nhiệt lượng ra môi trường. Thời gian làm nóng chảy hoàn toàn khối thép là
		\begin{mcq}(4)
			\item $ \SI{26}{h} $.
			\item $ \SI{0,94}{h} $.
			\item $ \SI{100}{h} $.
			\item $ \SI{94}{h} $.
		\end{mcq}
	}
	{
		\begin{center}
			\textbf{Hướng dẫn giải}
		\end{center}
		Nhiệt lượng cần thiết để đưa khối thép lên điểm nóng chảy:
		$$
		Q_{1} = mc\left( T_{c} - t_{0} \right) = \SI{67420}{J}.
		$$
		Nhiệt lượng cần thiết để chuyển khối thép từ thể rắn sang thể lỏng ở điểm nóng chảy:
		$$
		Q_{2} = mL = \SI{270000}{J}.
		$$
		Tổng nhiệt lượng để nấu chảy hoàn toàn khối thép:
		$$
		Q = Q_{1} + Q_{2} = \SI{944240}{J}.
		$$
		Thời gian cần để nấu chảy khối thép:
		$$
		t = \dfrac{Q}{P} = \SI{26}{h}.
		$$
		
		\begin{center}
			\textbf{Câu hỏi tương tự}
		\end{center}
		Người ta dùng một laze hoạt động dưới chế độ liên tục để khoan một tấm thép. Công suất của chùm laze là $ P = \SI{10}{W} $. Đường kính của một chùm sáng là $ d = \SI{1}{mm} $. Bề dày của tấm thép là $ e = \SI{2}{mm} $. Nhiệt độ ban đầu là $ t_{0} = 30\circ $. Khối lượng riêng của thép là $ \rho = \SI{7800}{kg/m^{3}} $. Nhiệt dung riêng của thép $ c = \SI{448}{J/kg \cdot \text{độ}} $. Nhiệt nóng chảy riêng của thép là $ \lambda = \SI{270}{kJ/kg} $. Điểm nóng chảy của thép là $ T_{c} = 1535\circ $. Bỏ qua mọi hao phí thì thời gian khoan thép là
		\begin{mcq}(4)
			\item $ \SI{2,16}{S} $.
			\item $ \SI{1,16}{S} $.
			\item $ \SI{1,18}{S} $.
			\item $ \SI{1,26}{S} $.
		\end{mcq}
		\textbf{Đáp án:} B.
	}
	
\end{dang}
\section{Bài tập tự luyện}
\begin{enumerate}[label=\bfseries Câu \arabic*:]
	\item \mkstar{2}
	\cauhoi
	{ Một laze He – Ne phát ánh sáng có bước sóng $\SI{632,8}{nm}$ và có công suất đầu ra là $\SI{2,3}{mW}$. Số photon phát ra trong mỗi phút là
		
		\begin{mcq}(4)
			\item $22 \cdot 10^{15}$.
			\item $24 \cdot 10^{15}$.
			\item $44 \cdot 10^{16}$.
			\item $44 \cdot 10^{15}$.
		\end{mcq}
	}
	
	\loigiai
	{		\textbf{Đáp án: C.}
		
		 Gọi $n$ là số photon phát ra trong một giây, thì công suất phát của laze được tính: 
		 
		 $$P = n\varepsilon \Rightarrow n = \dfrac{P}{\varepsilon}.$$
		 
		 Số photon phát ra trong một phút
		 
		 $$N = 60n = 60\dfrac{P}{\varepsilon} = 60 \dfrac{P\lambda}{hc} \approx 44 \cdot 10^{16}.$$
		
	}
	
	\item \mkstar{2} 
		\cauhoi
	{ Một laze rubi phát ra ánh sáng có bước sóng $694,4 nm$. Nếu xung laze được phát ra trong $\to$ s và năng lượng giải phóng bởi mỗi xung là $Q = \SI{0,15}{J}$ thì số photon trong mỗi xung là
		
		\begin{mcq}(4)
			\item $22 \cdot 10^{16}$.
			\item $24 \cdot 10^{17}$.
			\item $\text{5,24} \cdot 10^{17}$.
			\item $\text{5,44} \cdot 10^{15}$.
		\end{mcq}
	}
	
	\loigiai
	{		\textbf{Đáp án: C.}
		
		Gọi $n$ là số photon phát ra trong mỗi xung, thì năng lượng được giải phóng ra trong mỗi xung là
		
		$$Q = N\varepsilon \Rightarrow N  = \dfrac{Q}{\varepsilon} = \dfrac{Q\lambda}{hc} \approx \text{5,24} \cdot 10^{17}.$$
	}
	\item \mkstar{2} 
	\cauhoi
	{ Người ta dùng một loại laze CO$_2$ có công suất $P=  \SI{10}{W}$ để làm dao mổ. Tia laze chiếu vào chỗ mô sẽ làm nước ở phần mô chỗ đó bốc hơi và mô bị cắt. Nhiệt dung riêng của nước $C = \text{4,18}\ \text{kJ}/(\text{kg}\cdot \text K)$ nhiệt hoá hơi của nước $L = \SI{2260}{kJ/kg}$, khối lượng riêng của nước $D = \SI{1}{g/cm}^3$, nhiệt độ cơ thể là $37^\circ\ \text C$. Thể tích nước mà tia laze làm bốc hơi trong $\SI{1}{s}$ là
		
		\begin{mcq}(4)
			\item $\SI{2,892}{mm}^3$.
			\item $\SI{3,963}{mm}^3$.
			\item $\SI{4,011}{mm}^3$.
			\item $\SI{2,553}{mm}^3$
		\end{mcq}
	}
	
	\loigiai
	{		\textbf{Đáp án: B.}
		
		Năng lượng mà tia laze này cung cấp trong $\SI{1}{s}$ là 
		
		$$A = P \cdot t = \SI{10}{J}.$$
		
		Năng lượng cần thiết 
		
		$$Q = A \Rightarrow V = \dfrac{A}{D(C\Delta t + L)} \approx \SI{3,963}{mm}^3.$$
		
	}
\end{enumerate}