\let\lesson\undefined
\newcommand{\lesson}{\phantomlesson{Bài 20: Mạch dao động}}
\chapter[ Bài toán tương tự dao động cơ]{ Bài toán tương tự dao động cơ}
\section{Lý thuyết}
- Điện tích $q$ của một bản tụ điện và cường độ dòng điện $i$ trong mạch dao động biến thiên điều hòa theo thời gian, $i$ sớm pha $\dfrac{\pi}{2} $ so với $q$.
\begin{equation}
	q=q_0\cos(\omega t +\varphi)\ \text{C}, 
\end{equation}
\begin{equation}
	i=\frac{dq}{dt}=I_0 \cos(\omega +\varphi+ \frac{\pi}{2})\ \text{A},
\end{equation}
trong đó: $I_0=q_0\omega, \omega =\dfrac{1}{\sqrt {LC}}$.

- Biểu thức hiệu điện thế hai đầu tụ điện

$$ u = \dfrac{q}{C} = U_0 \cos (\omega t + \varphi_u)\ \text{V}.$$

với $U_0 =\dfrac{Q_0}{C}.$

- Công thức độc lập:

$$ \left(\dfrac{q}{Q_0}\right)^2 +  \left(\dfrac{i}{I_0}\right)^2 =1; \left(\dfrac{u}{U_0}\right)^2 +  \left(\dfrac{i}{I_0}\right)^2 =1.$$


\section{Mục tiêu bài học - Ví dụ minh họa}
\begin{dang}{Xác định tần số, chu kỳ của\\ mạch dao động.}
	\viduii{2}{
		Cường độ dòng điện tức thời trong mạch dao động $LC$ có dạng $i = \text{0,5} \cos \left( 3000t \right)$. Tần số dao động riêng của mạch là  
		\begin{mcq}(2)
			\item $\SI{3000}{rad/s}$. 
			\item $\SI{477,46}{Hz}$. 
			\item $\xsi{6000\pi}{Hz}$. 
			\item $\SI{3000}{Hz}$. 
		\end{mcq}
	}
	{	\begin{center}
			\textbf{Hướng dẫn giải}
		\end{center}
		
		Tần số dao động riêng trong mạch cho bởi biểu thức:
		
		$$f=\dfrac{\omega}{2 \pi}= \SI{477,76}{Hz}.$$
		
		\textbf{Đáp án: B.}
		
		\begin{center}
			\textbf{Câu hỏi tương tự}
		\end{center}
		
		Một mạch dao động $LC$ lí tưởng có $L = \SI{20}{mH}$ và $C = \SI{200}{pF}$. Chu kì riêng của dao động điện từ trong mạch xấp xỉ bằng
		\begin{mcq}(2)
			\item $\xsi{1,3\cdot10^{-5}}{s}$. 
			\item $\xsi{1,9\cdot10^{-4}}{s}$. 
			\item $\xsi{12,5\cdot10^{-3}}{s}$. 
			\item $\xsi{3,9\cdot10^{-4}}{s}$. 
		\end{mcq}
		
		\textbf{Đáp án: A.} 
	}
	\viduii{3}{
		
		Một mạch dao động điện từ lí tưởng đang có dao động điện từ tự do. Biết điện tích cực đại trên một bản tụ điện là $\xsi{4\sqrt{2}}{\mu C}$ và cường độ dòng điện cực đại trong mạch là $\xsi{0,5\sqrt{2}}{A}$. Thời gian ngắn nhất để điện tích trên một bản tụ giảm từ giá trị cực đại đến nửa giá trị cực đại là
		\begin{mcq}(4)
			\item $\xsi{\dfrac{4\pi}{3}}{\mu s}$. 
			\item $\xsi{\dfrac{16\pi}{3}}{\mu s}$. 
			\item $\xsi{\dfrac{2\pi}{3}}{\mu s}$. 
			\item $\xsi{\dfrac{8\pi}{3}}{\mu s}$. 
		\end{mcq}
	}
	{	\begin{center}
			\textbf{Hướng dẫn giải}
		\end{center}
		
		Tần số riêng của mạch dao động là 
		
		$$\omega = \dfrac{I_0}{q_0} = \SI{125 e3}{rad/s}.$$ 
		
		Từ đường tròn pha, ta thấy để điện tích trên một bản tụ giảm từ giá trị cực đại đến nửa giá trị cực đại thì vector quay quay được một góc nhỏ nhất là $\Delta \varphi = \pi /3.$ 
		
		Vậy thời gian ngắn nhất để điện tích trên một bản tụ giảm từ giá trị cực đại đến nửa giá trị cực đại là
		
		$$\Delta t = \dfrac{\Delta \varphi}{\omega}  = \xsi{\dfrac{8\pi}{3}}{\mu s}.$$
		
		\textbf{Đáp án: D.}
		\begin{center}
			\textbf{Câu hỏi tương tự}
		\end{center}
		
		Trong mạch dao động $LC$ lí tưởng đang có dao động điện từ tự do. Thời gian ngắn nhất để năng lượng điện trường giảm từ giá trị cực đại xuống còn một nửa giá trị cực đại là $\xsi{1,5\cdot10^{-4}}{s}$. Thời gian ngắn nhất để điện tích trên tụ giảm từ giá trị cực đại xuống còn một nửa giá trị đó là
		\begin{mcq}(2)
			\item $\xsi{2\cdot10^{-4}}{s}$. 
			\item $\xsi{6\cdot10^{-4}}{s}$. 
			\item $\xsi{12\cdot10^{-4}}{s}$. 
			\item $\xsi{3\cdot10^{-4}}{s}$. 
		\end{mcq}
		
		\textbf{Đáp án: A.} 
	}
	\viduii{2}{
		
		Một mạch dao động $LC$ lí tưởng với $L = \SI{2,4}{mH}$ và $C = \SI{1,5}{mF}$. Gọi $I_0$ là cường độ dòng điện cực đại trong mạch. Khoảng thời gian ngắn nhất giữa hai lần liên tiếp mà $i =\dfrac{I_0}{3}$ là
		\begin{mcq}(4)
			\item $\SI{4,76}{ms}$. 
			\item $\SI{4,67}{ms}$. 
			\item $\SI{0,29}{ms}$. 
			\item $\SI{4,54}{ms}$. 
		\end{mcq}
	}
	{	\begin{center}
			\textbf{Hướng dẫn giải}
		\end{center}
		
		Từ đường tròn pha, ta xác định được khoảng thời gian ngắn nhất giữa hai lần liên tiếp $i =\dfrac{I_0}{3}$ là
		
		$$\Delta t=\dfrac{2}{\omega} \arccos \left(\dfrac{i}{I_{0}}\right)=\dfrac{2}{\omega} \arccos \dfrac{1}{3}.$$
		
		Lại có,
		
		$$\omega=\dfrac{1}{\sqrt{L C}} \approx \SI{527}{rad/s}.$$
		
		Khi đó,
		
		$$\Delta t=\dfrac{2}{527} \arccos \dfrac{1}{3}= \xsi{4,67\cdot10^{-3}}{s}.$$
		
		\textbf{Đáp án: B.}
		
		
		\begin{center}
			\textbf{Câu hỏi tương tự}
		\end{center}
		
		Một mạch dao động điện từ lí tưởng đang có dao động điện từ tự do. Tại thời điểm $t = 0$, tụ điện bắt đầu phóng điện. Sau khoảng thời gian ngắn nhất $\Delta t = 10^{-6}\ \text{s}$ thì điện tích trên một bản tụ điện bằng một nửa giá trị cực đại. Tính chu kì dao động riêng của mạch.
		
		\textbf{Đáp án:} $T = 6 \cdot 10^{-6}\ \text s.$
	}
	
	\viduii{2}{
		
		Một mạch dao động lí tưởng gồm cuộn cảm thuần có độ tự cảm $\SI{4}{\mu H}$ và một tụ điện có điện dung biến đổi từ $\SI{10}{pF}$ đến $\SI{360}{pF}$. Lấy $\pi^2 = 10$. Chu kì dao động riêng của mạch có giá trị
		\begin{mcq}(2)
			\item $\xsi{4\cdot10^{-8}}{s}$ đến $\xsi{3,2\cdot10^{-7}}{s}$. 
			\item $\xsi{2\cdot10^{-8}}{s}$ đến $\xsi{3,6\cdot10^{-7}}{s}$. 
			\item $\xsi{4\cdot10^{-8}}{s}$ đến $\xsi{2,4\cdot10^{-7}}{s}$. 
			\item $\xsi{2\cdot10^{-8}}{s}$ đến $\xsi{3\cdot10^{-7}}{s}$. 
		\end{mcq}
	}
	{	\begin{center}
			\textbf{Hướng dẫn giải}
		\end{center}
		
		Chu kì dao động riêng của mạch:
		
		$$T=2 \pi \sqrt{L C}.$$
		
		Khi $C=\SI{10}{pF}$, ta có:
		
		$$T=2 \pi \sqrt{4\cdot10^{-6} \cdot 10\cdot10^{-12}}= \xsi{4\cdot10^{-8}}{s}.$$
		
		Khi $C=\SI{360}{pF}$, ta có:
		
		$$T=2 \pi \sqrt{4\cdot10^{-6} \cdot 360\cdot10^{-12}}= \SI{2,4 e-7}{s}.$$
		
		Vậy chu kì dao động riêng của mạch biến thiên từ $\SI{4 e-8}{s}$ đến $\SI{2,4 e-7}{s}$.
		
		\textbf{Đáp án: C.}
		
		
		\begin{center}
			\textbf{Câu hỏi tương tự}
		\end{center}
		
		Một mạch dao động ở máy vào của một máy thu thanh gồm cuộn thuần cảm có độ tự cảm $\SI{3}{\mu H}$ và tụ điện có điện dung biến thiên trong khoảng $\SI{10}{pF}$ và $\SI{500}{pF}$. Biết rằng, muốn thu tần số riêng của mạch dao động phải bằng tần số của mạch cần thu (để có cộng hưởng). Trong không khí, tốc độ truyền sóng điện từ là $\xsi{3\cdot10^{8}}{m/s}$, máy thu này có thể thu được sóng điện từ trong khoảng
		\begin{mcq}(2)
			\item từ $\SI{100}{m}$ đến $\SI{730}{m}$. 
			\item từ $\SI{10,32}{m}$ đến $\SI{73}{m}$. 
			\item từ $\SI{1,24}{m}$ đến $\SI{73}{m}$. 
			\item từ $\SI{10}{m}$ đến $\SI{73}{m}$. 
		\end{mcq}
		
		\textbf{Đáp án: B.}
		
	}
	
\end{dang}

\begin{dang}{Xác định cường độ dòng điện, hiệu điện thế, điện tích trong mạch dao động.}
	\viduii{3}{
		Một mạch dao động $LC$ có $C = \SI{2}{nF}$ đang thực hiện dao động điện từ tự do. Tại thời điểm $t_1$, cường độ dòng điện trong mạch có độ lớn $\SI{8}{mA}$, tại thời điểm $t_2 = t_1 +\dfrac{T}{4}$, hiệu điện thế giữa hai bản tụ có độ lớn $\SI{6}{V}$. Giá trị của $L$ là 
		\begin{mcq}(4)
			\item $\SI{2,250}{H}$. 
			\item $\SI{1,125}{H}$. 
			\item $\SI{2,250}{mH}$. 
			\item $\SI{1,125}{mH}$. 
		\end{mcq}
	}
	{	\begin{center}
			\textbf{Hướng dẫn giải}
		\end{center}
		
		Ta có thời điểm $t_2$ trễ hơn $\dfrac{T}{4}$ so với thời điểm $t_1$ nên đây là hai thời điểm vuông pha. 
		
		Suy ra $i_2$ vuông pha với $i_1$. 
		
		Mà $u_2$ thì vuông pha với $i_2$. 
		
		Nên $i_1$ và $u_2$ hoặc là cùng pha, hoặc là ngược pha với nhau. 
		
		Nếu chỉ xét độ lớn, ta có:
		
		$$\dfrac{|i_1|}{I_0} = \dfrac{|u_2|}{U_0}.$$ 
		
		Lại có $U_0 = \dfrac{Q_0}{C} = \dfrac{I_0}{\omega C}$. Thay vào biểu thức trên, ta được:
		
		$$\dfrac{|i_1|}{I_0} = \dfrac{|u_2|}{\dfrac{I_0}{\omega C}}.$$
		
		Suy ra 
		
		$$|i_1| = |u_2|\omega C \Rightarrow |i_1| = |u_2|\dfrac{1}{\sqrt{LC}} C.$$
		
		Từ đó suy ra $L = \SI{1,125}{mH}.$
		
		\textbf{Đáp án: D.}
		
		\begin{center}
			\textbf{Câu hỏi tương tự}
		\end{center}
		
		Cường độ dòng điện tức thời trong mạch dao động $LC$ lí tưởng là $i = \text{0,08} \sin \left( 2000t \right)$ (A). Cuộn dây có độ tự cảm là $L = \SI{50}{mH}$. Hiệu điện thế giữa hai bản tụ tại thời điểm cường độ dòng điện tức thời trong mạch bằng cường độ dòng điện hiệu dụng là
		\begin{mcq}(4)
			\item $\SI{32}{V}$. 
			\item $\xsi{4\sqrt{2}}{V}$. 
			\item $\SI{8}{V}$. 
			\item $\xsi{2\sqrt{2}}{V}$. 
		\end{mcq}
		
		\textbf{Đáp án: B.} 
	}
	\viduii{3}{
		Hai mạch dao động điện từ lí tưởng đang có dao động điện từ tự do. Điện tích của tụ điện trong mạch thứ nhất và mạch thứ hai lần lượt là $q_1$ và $q_2$ với $4{q_1}^{2}+{q_2}^{2} = \text{1,3}\cdot10^{-17}$, $q$ tính bằng C. Ở thời điểm $t$ điện tích của tụ điện và cường độ dòng điện trong mạch dao động thứ nhất lần lượt là $\xsi{10^{-9}}{C}$ và $\SI{6}{mA}$, cường độ dòng điện trong mạch thứ hai có độ lớn bằng
		\begin{mcq}(4)
			\item $\SI{4}{mA}$. 
			\item $\SI{10}{mA}$. 
			\item $\SI{8}{mA}$. 
			\item $\SI{6}{mA}$. 
		\end{mcq}
	}
	{	\begin{center}
			\textbf{Hướng dẫn giải}
		\end{center}
		
		Thay $q_1 = \xsi{10^{-9}}{C}$ vào $4{q_1}^{2}+{q_2}^{2} = \text{1,3}\cdot10^{-17}$ ta tìm được $q_2 = \SI{3}{nC}$. 
		
		Lấy đạo hàm hai vế phương trình $4{q_1}^{2}+{q_2}^{2} = \text{1,3}\cdot10^{-17}$ ta được:
		
		$$8{q_1}{i_1} + 2{q_2}{i_2} = 0$$
		
		Thay $q_1, i_1, q_2$ vào phương trình trên ta tìm được $i_2 = \SI{-8}{mA}.$
		
		\textbf{Đáp án: C.}
		
		
		\begin{center}
			\textbf{Câu hỏi tương tự}
		\end{center}
		
		Có hai mạch dao động điện từ lý tưởng đang có dao động điện từ tự do. Ở thời điểm $t$, gọi $q_1$ và $q_2$ lần lượt là điện tích của tụ điện trong mạch dao động thứ nhất và thứ hai. Biết $36q_1^2 + 36q_2^2 = 24^2$. Ở thời điểm $t = t_1$, trong mạch dao động thứ nhất: điện tích của tụ điện $q_1 = \SI{2,4}{nC}$; cường độ dòng điện qua cuộn cảm $i_1 = \SI{3,2}{mA}$. Khi đó, cường độ dòng điện qua cuộn cảm trong mạch dao động thứ hai là
		\begin{mcq}(2)
			\item $i_2 = \SI{5,4}{mA}.$
			
			\item $i_2 = \SI{3,2}{mA}.$
			
			\item $i_2 = \SI{6,4}{mA}.$
			
			\item $i_2 = \SI{4,5}{mA}.$
		\end{mcq}
		\textbf{Đáp án: B} 
	}
	\viduii{2}{
		
		Một mạch dao động $LC$ có tụ điện $\SI{25}{pF}$ và cuộn cảm $10^{-4}\ \text H$. Biết ở thời điểm ban đầu của dao động, cường độ dòng điện có giá trị cực đại và bằng $\SI{40}{mA}$. Tìm biểu thức của cường độ dòng điện, của điện tích trên bản cực của tụ điện và biểu thức của hiệu điện thế giữa hai bản cực của tụ điện.
		\begin{mcq}(2)
			\item $i = 4\cdot 10^{-2} \cos(2 \cdot 10^7 t)\ \text A$. 
			\item $i = 4\cdot 10^{-2} \cos(2 \cdot 10^{-7} t)\ \text A$. 
			\item $i = 4\cdot 10^{-2} \cos \left(2 \cdot 10^7 t + \dfrac{\pi}{2}\right)\ \text A$. 
			\item $i = 4\cdot 10^{-2} \cos \left(2 \cdot 10^7 t - \dfrac{\pi}{2}\right)\ \text A$. 
		\end{mcq}
	}
	{	\begin{center}
			\textbf{Hướng dẫn giải}
		\end{center}
		
		Tần số góc 
		
		$$\omega  = \dfrac{1}{\sqrt{LC}} = 2 \cdot 10^7\ \text{rad/s}.$$
		
		Biểu thức cường độ dòng điện 
		
		$$i = I_0 \cos (\omega t + \varphi).$$
		
		Vì lúc $t = 0$ thì $i = I_0 = \SI{40}{mA} = 4 \cdot 10^{-2}\ \text A$ nên $\varphi = 0$, do đó: 
		
		$$i = 4\cdot 10^{-2} \cos (2 \cdot 10^7 t)\ \text A.$$
		
		\textbf{Đáp án: A.}
		
		
		\begin{center}
			\textbf{Câu hỏi tương tự}
		\end{center}
		
		Một mạch dao động $LC$ gồm tụ điện có điện dung $C = \SI{40}{pF}$ và cuộn cảm có độ tự cảm $L = \SI{10}{\mu H}$. Ở thời điểm ban đầu, cường độ dòng điện có giá trị cực đại và bằng $\SI{0,05}{A}$. Biểu thức hiệu điện thế ở hai cực của tụ điện?
		
		\begin{mcq}(2)
			\item $u = 50 \cos \left(5 \cdot 10^7 t - \dfrac{\pi}{2}\right)\ \text V$. 
			\item $u = 50 \cos \left(5 \cdot 10^7 t + \dfrac{\pi}{2}\right)\ \text V$. 
			\item $u = 25 \cos \left(5 \cdot 10^7 t - \dfrac{\pi}{2}\right)\ \text V$. 
			\item $u = 25 \cos \left(5 \cdot 10^7 t + \dfrac{\pi}{2}\right)\ \text V$. 
		\end{mcq}
		
		\textbf{Đáp án: C.} 
	}
	\viduii{2}{
		
		Một mạch dao động lí tưởng gồm tụ điện có điện dung $C = \SI{4}{\mu F}$ và cuộn cảm thuần có độ tự $L = \SI{1}{H}$, đang thực hiện dao động điện từ tự do với hiệu điện thế cực đại giữa hai bản tụ điện là $\SI{6}{V}$ thì dòng điện qua cuộn cảm có giá trị cực đại là
		\begin{mcq}(4)
			\item $\xsi{24 \sqrt{2}}{mA}$. 
			\item $\SI{12}{mA}$. 
			\item $\xsi{12 \sqrt{2}}{mA}$. 
			\item $\SI{24}{mA}$. 
		\end{mcq}
	}
	{	\begin{center}
			\textbf{Hướng dẫn giải}
		\end{center}
		
		Trong một mạch dao động, mối quan hệ giữa dòng điện cực đại $I_{0}$ và hiệu điện thế cực đại $U_{0}$ giữa hai bàn tụ là
		
		$$I_{0}=\sqrt{\dfrac{C}{L}} U_{0}=  \SI{12}{mA}.$$
		
		\textbf{Đáp án: B.}
		
		
		\begin{center}
			\textbf{Câu hỏi tương tự}
		\end{center}
		
		Trong mạch dao động $LC$ có dao động điện từ tự do (dao động riêng) với tần số góc $\xsi{10^{4}}{rad/s}$. Điện tích cực đại trên tụ điện là $\xsi{10^{-9}}{C}$. Cường độ dòng điện trong mạch cực đại bằng
		\begin{mcq}(4)
			\item $\xsi{2\cdot10^{-5}}{A}$. 
			\item $\xsi{10^{-5}}{A}$. 
			\item $\xsi{10^{-4}}{A}$. 
			\item $\xsi{2\cdot10^{-4}}{A}$. 
		\end{mcq}
		
		\textbf{Đáp án: B.} 
	}
	\viduii{2}{
		
		Mạch dao động gồm tụ điện có điện dung $\SI{4500}{pF}$ và cuộn dây thuần cảm có độ tự cảm $\SI{5}{\mu H}$. Hiệu điện thế cực đại ở hai đầu tụ điện là $\SI{2}{V}$. Cường độ dòng điện trong mạch cực đại bằng
		\begin{mcq}(4)
			\item $\SI{0,03}{A}$. 
			\item $\SI{0,06}{A}$. 
			\item $\xsi{6\cdot10^{-4}}{A}$. 
			\item $\xsi{3\cdot10^{-4}}{A}$. 
		\end{mcq}
	}
	{	\begin{center}
			\textbf{Hướng dẫn giải}
		\end{center}
		
		Cường độ dòng điện cực đại trong mạch 
		
		$$I_{0}=\sqrt{\dfrac{C}{L}} U_{0}=\SI{0,06}{A}.$$
		
		\textbf{Đáp án: B.}
		
		
		\begin{center}
			\textbf{Câu hỏi tương tự}
		\end{center}
		
		Mạch dao động điện từ điều hoà $LC$ gồm tụ điện $C = \SI{30}{nF}$ và cuộn cảm $L = \SI{25}{mH}$. Nạp điện cho tụ điện đến hiệu điện thế $\SI{4,8}{V}$ rồi cho tụ phóng điện qua cuộn cảm, cường độ dòng điện hiệu dụng trong mạch là bao nhiêu?
		
		\textbf{Đáp án:} $I = \SI{3,72}{mA}.$
	}
	
	
\end{dang}
\section{Bài tập tự luyện}
\begin{enumerate}[label=\bfseries Câu \arabic*:]
	

		\item \mkstar{1} 
	
	{ Một mạch dao động $LC$ gồm một cuộn cảm có độ tự cảm $L=\dfrac{1}{\pi}\ \text{H}$ và một tụ điện có điện dung $C$. Tần số dao động riêng của mạch là $\SI{1}{MHz}$. Giá trị của $C$ bằng
		
		\begin{mcq}(4)
			\item $C = \dfrac{1}{4\pi}\ \text{pF}.$
			\item $C = \dfrac{1}{4\pi}\ \text{F}.$
			\item $C = \dfrac{1}{4\pi}\ \text{mF}.$ 
			\item $C = \dfrac{1}{4\pi}\ \mu\text{F}.$
		\end{mcq}
	}
	\hideall
	{		\textbf{Đáp án: A.}
		
		$$C= \dfrac{1}{4\pi^2f^2 L}  = \dfrac{1}{4\pi}\ \text{pF}.$$
		
	}
		\item \mkstar{1} 
	
	{ Gọi $A$ và $v_\text{M}$ lần lượt là biên độ và vận tốc cực đại của một vật trong dao động điều hoà; $Q_0$ và $I_0$ lần lượt là điện tích cực đại trên một bản tụ điện và cường độ dòng điện cực đại trong mạch dao động $LC$ đang hoạt động. Biểu thức $\dfrac{v_\text{M}}{A}$ có cùng đơn vị với biểu thức:
		
		\begin{mcq}(4)
			\item $\dfrac{I_0}{Q_0}$.
			\item $Q_0I_0^2$.
			\item $\dfrac{Q_0}{I_0}$. 
			\item $I_0Q_0^2$
		\end{mcq}
	}
	\hideall
	{		\textbf{Đáp án: A.}
		
		
		
	}
		\item \mkstar{2} 
	
	{ Trong mạch dao động $LC$ có dao động điện từ tự do (dao động riêng) với tần số góc $\xsi{10^4}{rad/s}$. Điện tích cực đại trên tụ điện là  $\xsi{10^{-9}}{C}$. Khi cường độ dòng điện trong mạch bằng $\xsi{6\cdot 10^{-6}}{A}$ thì điện tích trên tụ điện là
		
		\begin{mcq}(4)
			\item $6\cdot \xsi{10^{-10}}{C}$.
			\item $8\cdot \xsi{10^{-10}}{C}$.
			\item $2\cdot \xsi{10^{-10}}{C}$. 
			\item $4\cdot \xsi{10^{-10}}{C}$.
		\end{mcq}
	}
	\hideall
	{		\textbf{Đáp án: B.}
		
		Do $i$ và $q$ vuông pha với nhau nên theo hệ thức độc lập ta có:
		
		$$\left(\dfrac{i}{I_0}\right)^2 +\left(\dfrac{q}{Q_0}\right)^2 =1 \Rightarrow q = \xsi{8 \cdot 10^{-8}}{C} $$
		
		
		
	}
		\item \mkstar{2} 
	
	{Một mạch dao động $LC$ gồm một cuộn cảm $L=\SI{500}{\mu H}$ và một tụ điện có điện dung $C = \SI{5}{\mu F}$. Lấy $\pi^2 = 10$. Giả sử tại thời điểm ban đầu điện tích của tụ điện đạt giá trị cực đại $Q_0 = \xsi{6 \cdot 10^{-4}}{C}$. Biểu thức của cường độ đòng điện qua mạch là
		
		
		\begin{mcq}(2)
			\item $i = 6 \cos \left(2\cdot 10^4 t + \dfrac{\pi}{2}\right)\ \text A.$
			\item $i = 12 \cos \left(2\cdot 10^4 t - \dfrac{\pi}{2}\right)\ \text A.$
			\item $i = 6 \cos \left(2\cdot 10^4 t - \dfrac{\pi}{2}\right)\ \text A.$ 
			\item $i = 12 \cos \left(2\cdot 10^4 t + \dfrac{\pi}{2}\right)\ \text A.$
		\end{mcq}
	}
	\hideall
	{		\textbf{Đáp án: B.}
		
		Tần số góc của mạch dao động 
		
		$$\omega = \dfrac{1}{LC} = 2\cdot 10^4\ \text{rad/s}.$$
		
		Dòng điện cực đại chạy trong mạch
		
		$$I_0 = \omega Q_0 = \dfrac{Q_0}{\sqrt{LC}} = \SI{12}{A}.$$
		
		
		
		
		
	}
	\item \mkstar{1}
	
	{Cường độ dòng điện tức thời trong mạch dao động LC có dạng $i = \text{0,5} \cos \left( 3000t \right)$. Tần số dao động riêng của mạch là  
		\begin{mcq}(4)
			\item $\SI{3000}{rad/s}$. 
			\item $\SI{477,46}{Hz}$. 
			\item $\xsi{6000\pi}{Hz}$. 
			\item $\SI{3000}{Hz}$. 
		\end{mcq}
	}
	
	\hideall
	{		\textbf{Đáp án: B.}
		
		Tần số dao động riêng trong mạch cho bởi biểu thức:
		$$
		f=\dfrac{\omega}{2 \pi}=\dfrac{3000}{2 \pi}= \SI{477,76}{Hz}.
		$$
		
	}
	
	%--------------------------------------------------------------------------------------------------------------
	%--------------------------------------------------------------------------------------------------------------
	\item \mkstar{1}
	
	{Một mạch dao động LC lí tưởng có $L = \SI{20}{mH}$ và $C = \SI{200}{pF}$. Chu kì riêng của dao động điện từ trong mạch xấp xỉ bằng
		\begin{mcq}(4)
			\item $\xsi{1,3\cdot10^{-5}}{s}$. 
			\item $\xsi{1,9\cdot10^{-4}}{s}$. 
			\item $\xsi{12,5\cdot10^{-3}}{s}$. 
			\item $\xsi{3,9\cdot10^{-4}}{s}$. 
		\end{mcq}
	}
	
	\hideall
	{		\textbf{Đáp án: A.}
		
		Chu kì dao động riêng của mạch cho bởi 
		$$ T = 2\pi \sqrt{LC} = 2\pi \sqrt{20\cdot10^{-3} \cdot 200\cdot10^{-12}} = \xsi{1,3\cdot10^{-5}}{s} $$.
		
	}
	
	%--------------------------------------------------------------------------------------------------------------
	%--------------------------------------------------------------------------------------------------------------
	\item \mkstar{3}
	
	{Một mạch dao động điện từ lí tưởng đang có dao động điện từ tự do. Biết điện tích cực đại trên một bản tụ điện là $\xsi{4\sqrt{2}}{\mu C}$ và cường độ dòng điện cực đại trong mạch là $\xsi{0,5\sqrt{2}}{A}$. Thời gian ngắn nhất để điện tích trên một bản tụ giảm từ giá trị cực đại đến nửa giá trị cực đại là
		\begin{mcq}(4)
			\item $\xsi{\dfrac{4\pi}{3}}{\mu s}$. 
			\item $\xsi{\dfrac{16\pi}{3}}{\mu s}$. 
			\item $\xsi{\dfrac{2\pi}{3}}{\mu s}$. 
			\item $\xsi{\dfrac{8\pi}{3}}{\mu s}$. 
		\end{mcq}
	}
	
	\hideall
	{		\textbf{Đáp án: D.}
		
		Tần số riêng của mạch dao động là 
		$$
		\omega = \dfrac{I_0}{q_0} = \dfrac{\text{0,5} \sqrt{2}}{4 \sqrt{2} \cdot10^{-6}} = \SI{125 e3}{rad/s}.
		$$ \\
		Từ đường tròn pha, ta thấy để điện tích trên một bản tụ giảm từ giá trị cực đại đến nửa giá trị cực đại thì vector quay quay được một góc nhỏ nhất là $\Delta \varphi = \pi /3.$ \\
		Vậy thời gian ngắn nhất để điện tích trên một bản tụ giảm từ giá trị cực đại đến nửa giá trị cực đại là
		$$
		\Delta t = \dfrac{\Delta \varphi}{\omega} = \dfrac{\pi / 3}{125\cdot10^{3}} = \xsi{\dfrac{8\pi}{3}}{\mu s}.
		$$
	}
	
	
	
	%--------------------------------------------------------------------------------------------------------------
	%--------------------------------------------------------------------------------------------------------------
	\item \mkstar{3}
	
	{Một mạch dao động LC có $C = \SI{2}{nF}$ đang thực hiện dao động điện từ tự do. Tại thời điểm $t_1$, cường độ dòng điện trong mạch có độ lớn $\SI{8}{mA}$, tại thời điểm $t_2 = t_1 +T/4$, hiệu điện thế giữa hai bản tụ có độ lớn $\SI{6}{V}$. Giá trị của L là 
		\begin{mcq}(4)
			\item $\SI{2,250}{H}$. 
			\item $\SI{1,125}{H}$. 
			\item $\SI{2,250}{mH}$. 
			\item $\SI{1,125}{mH}$. 
		\end{mcq}
	}
	
	\hideall
	{		\textbf{Đáp án: D.}
		
		Ta có thời điểm $t_2$ trễ hơn $T/4$ so với thời điểm $t_1$ nên đây là hai thời điểm vuông pha. \\
		Suy ra $i_2$ vuông pha với $i_1$. \\
		Mà $u_2$ thì vuông pha với $i_2$. \\
		Nên $i_1$ và $u_2$ hoặc là cùng pha, hoặc là ngược pha với nhau. \\
		Nếu chỉ xét độ lớn, ta có:
		$$
		\dfrac{|i_1|}{I_0} = \dfrac{|u_2|}{U_0}.
		$$ \\
		Lại có $U_0 = \dfrac{Q_0}{C} = \dfrac{I_0}{\omega C}$. Thay vào biểu thức trên, ta được:
		$$
		\dfrac{|i_1|}{I_0} = \dfrac{|u_2|}{\dfrac{I_0}{\omega C}}.
		$$
		Suy ra 
		
		$$|i_1| = |u_2|\omega C \Rightarrow |i_1| = |u_2|\dfrac{1}{\sqrt{LC}} C \Rightarrow 8\cdot10^{-3} = 6 \cdot \dfrac{1}{\sqrt{L \cdot 2\cdot10^{-9}}} \cdot 2\cdot10^{-9}$$
		
		Từ đó suy ra $L = \SI{1,125}{mH}.$
		
	}
	
	%--------------------------------------------------------------------------------------------------------------
	%--------------------------------------------------------------------------------------------------------------
	
	\item \mkstar{3}
	
	{Trong mạch dao động LC lí tưởng đang có dao động điện từ tự do. Thời gian ngắn nhất để năng lượng điện trường giảm từ giá trị cực đại xuống còn một nửa giá trị cực đại là $\xsi{1,5\cdot10^{-4}}{s}$. Thời gian ngắn nhất để điện tích trên tụ giảm từ giá trị cực đại xuống còn một nửa giá trị đó là
		\begin{mcq}(4)
			\item $\xsi{2\cdot10^{-4}}{s}$. 
			\item $\xsi{6\cdot10^{-4}}{s}$. 
			\item $\xsi{12\cdot10^{-4}}{s}$. 
			\item $\xsi{3\cdot10^{-4}}{s}$. 
		\end{mcq}
	}
	
	\hideall
	{		\textbf{Đáp án: A.}
		
		Thời gian ngắn nhất để năng lượng điện trường giảm từ giá trị cực đại xuống còn một nửa giá trị cực đại là $T/4$. \\
		Ta có $T/8 = \SI{1,5 e-4}{s} \Rightarrow T = \SI{1,2 e-3}{s}.$ \\ 
		Từ đường tròn pha, ta xác định được thời gian ngắn nhất để điện tích trên tụ giảm từ giá trị cực đại xuống còn một nửa giá trị đó là
		$$
		\Delta t = \dfrac{T}{6} = \xsi{2\cdot10^{-4}}{s}.
		$$	 
	}
	
	%--------------------------------------------------------------------------------------------------------------
	%--------------------------------------------------------------------------------------------------------------	
	\item \mkstar{3}
	
	{Hai mạch dao động điện từ lí tưởng đang có dao động điện từ tự do. Điện tích của tụ điện trong mạch thứ nhất và mạch thứ hai lần lượt là $q_1$ và $q_2$ với $4{q_1}^{2}+{q_2}^{2} = \text{1,3}\cdot10^{-17}$, q tính bằng C. Ở thời điểm $t$ điện tích của tụ điện và cường độ dòng điện trong mạch dao động thứ nhất lần lượt là $\xsi{10^{-9}}{C}$ và $\SI{6}{mA}$, cường độ dòng điện trong mạch thứ hai có độ lớn bằng
		\begin{mcq}(4)
			\item $\SI{4}{mA}$. 
			\item $\SI{10}{mA}$. 
			\item $\SI{8}{mA}$. 
			\item $\SI{6}{mA}$. 
		\end{mcq}
	}
	
	\hideall
	{		\textbf{Đáp án: C.}
		
		Thay $q_1 = \xsi{10^{-9}}{C}$ vào $4{q_1}^{2}+{q_2}^{2} = \text{1,3}\cdot10^{-17}$ ta tìm được $q_2 = \SI{3}{nC}$. \\
		Lấy đạo hàm hai vế phương trình $4{q_1}^{2}+{q_2}^{2} = \text{1,3}\cdot10^{-17}$ ta được:
		$$
		8{q_1}{i_1} + 2{q_2}{i_2} = 0
		$$
		Thay $q_1, i_1, q_2$ vào phương trình trên ta tìm được $i_2 = \SI{-8}{mA}.$
		
	}
	
	
	
	
	%--------------------------------------------------------------------------------------------------------------
	%--------------------------------------------------------------------------------------------------------------
	\item \mkstar{3}
	
	{Một mạch dao động ở máy vào của một máy thu thanh gồm cuộn thuần cảm có độ tự cảm $\SI{3}{\mu H}$ và tụ điện có điện dung biến thiên trong khoảng $\SI{10}{pF}$ và $\SI{500}{pF}$. Biết rằng, muốn thu tần số riêng của mạch dao động phải bằng tần số của mạch cần thu (để có cộng hưởng). Trong không khí, tốc độ truyền sóng điện từ là $\xsi{3\cdot10^{8}}{m/s}$, máy thu này có thể thu được sóng điện từ trong khoảng
		\begin{mcq}(2)
			\item từ $\SI{100}{m}$ đến $\SI{730}{m}$. 
			\item từ $\SI{10,32}{m}$ đến $\SI{73}{m}$. 
			\item từ $\SI{1,24}{m}$ đến $\SI{73}{m}$. 
			\item từ $\SI{10}{m}$ đến $\SI{730}{m}$. 
		\end{mcq}
	}
	
	\hideall
	{		\textbf{Đáp án: B.}
		
		Sóng điện từ thu được từ mạch dao động cho bởi biểu thức:
		$$
		\lambda = 2\pi c \sqrt{LC}.
		$$ 
		Sóng điện từ có bước sóng dài nhất thu được khi $C = C_{max} =\SI{500}{pF}$. Khi đó:
		$$
		\lambda_{max} = 2\pi c \sqrt{LC_{max}} = 2\pi \cdot 3\cdot10^{8} \cdot \sqrt{3\cdot10^{-6} \cdot 500\cdot10^{-12}} = \SI{73}{m}.
		$$
		Sóng điện từ có bước sóng ngắn nhất thu được khi $C = C_{min} =\SI{10}{pF}$. Khi đó:
		$$
		\lambda_{min} = 2\pi c \sqrt{LC_{min}} = 2\pi \cdot 3\cdot10^{8} \cdot \sqrt{3\cdot10^{-6} \cdot 10\cdot10^{-12}} = \SI{10,32}{m}.
		$$
		
	}
	
	%--------------------------------------------------------------------------------------------------------------
	%--------------------------------------------------------------------------------------------------------------	
	\item \mkstar{3}
	
	{Một mạch dao động LC lí tưởng với $L = \SI{2,4}{mH}$ và $C = \SI{1,5}{mF}$. Gọi $I_0$ là cường độ dòng điện cực đại trong mạch. Khoảng thời gian ngắn nhất giữa hai lần liên tiếp mà $i = I_o /3$ là
		\begin{mcq}(4)
			\item $\SI{4,76}{ms}$. 
			\item $\SI{4,67}{ms}$. 
			\item $\SI{0,29}{ms}$. 
			\item $\SI{4,54}{ms}$. 
		\end{mcq}
	}
	
	\hideall
	{		\textbf{Đáp án: B.}
		
		Từ đường tròn pha, ta xác định được khoảng thời gian ngắn nhất giữa hai lần liên tiếp $i=I_{0} / 3$ là
		$$
		\Delta t=\dfrac{2}{\omega} \arccos \left(\dfrac{i}{I_{0}}\right)=\dfrac{2}{\omega} \arccos \dfrac{1}{3}.
		$$
		Lại có,
		$$
		\omega=\dfrac{1}{\sqrt{L C}}=\dfrac{1}{\sqrt{\text{2,4}\cdot10^{-3} \cdot \text{1,5} \cdot 10^{-3}}} \approx \SI{527}{rad/s}.
		$$
		Khi đó,
		$$
		\Delta t=\dfrac{2}{527} \arccos \dfrac{1}{3}= \xsi{4,67\cdot10^{-3}}{s}.
		$$
		
		
	}
	
	%--------------------------------------------------------------------------------------------------------------%--------------------------------------------------------------------------------------------------------------	
	\item \mkstar{3}
	
	{Một mạch dao động lí tưởng gồm cuộn cảm thuần có độ tự cảm $\SI{4}{\mu H}$ và một tụ điện có điện dung biến đổi từ $\SI{10}{pF}$ đến $\SI{360}{pF}$. Lấy $\pi^2 = 10$. Chu kì dao động riêng của mạch có giá trị
		\begin{mcq}(2)
			\item $\xsi{4\cdot10^{-8}}{s}$ đến $\xsi{3,2\cdot10^{-7}}{s}$. 
			\item $\xsi{2\cdot10^{-8}}{s}$ đến $\xsi{3,6\cdot10^{-7}}{s}$. 
			\item $\xsi{4\cdot10^{-8}}{s}$ đến $\xsi{2,4\cdot10^{-7}}{s}$. 
			\item $\xsi{2\cdot10^{-8}}{s}$ đến $\xsi{3\cdot10^{-7}}{s}$. 
		\end{mcq}
	}
	
	\hideall
	{		\textbf{Đáp án: C.}
		
		Chu kì dao động riêng của mạch cho bời:
		$$
		T=2 \pi \sqrt{L C}
		$$
		Khi $\SI{10}{pF}$, ta có:
		$$
		T=2 \pi \sqrt{4\cdot10^{-6} \cdot 10\cdot10^{-12}}= \xsi{4\cdot10^{-8}}{s}.
		$$
		Khi $\SI{360}{pF}$, ta có:
		$$
		T=2 \pi \sqrt{4\cdot10^{-6} \cdot 360\cdot10^{-12}}= \SI{2,4 e-7}{s}.
		$$
		Vậy chu kì dao động riêng của mạch biến thiên từ $\SI{4 e-8}{s}$ đến $\SI{2,4 e-7}{s}$.
		
	}
	
	%--------------------------------------------------------------------------------------------------------------%--------------------------------------------------------------------------------------------------------------	
	\item \mkstar{3}
	
	{Một mạch dao động lí tưởng gồm tụ điện có điện dung $C = \SI{2}{\mu F}$ và cuộn cảm thuần có độ tự cảm L, đang thực hiện dao động điện từ tự do tại thời điểm hiệu điện thế giữa hai bản tụ điện là $\SI{5}{V}$ thì điện tích trên một bản tụ điện bằng
		\begin{mcq}(4)
			\item $\SI{5}{\mu C}$. 
			\item $\SI{2}{\mu C}$. 
			\item $\SI{4}{\mu C}$. 
			\item $\SI{10}{\mu C}$. 
		\end{mcq}
	}
	
	\hideall
	{		\textbf{Đáp án: D.}
		
		Trong một mạch dao động đang hoạt động, mối quan hệ giữa điện thế $u$ và điện tích $q$ ở một thời điểm bất kì cho bởi
		$$
		q=C u=2\cdot10^{-6} \cdot 5=\SI{10}{\mu C}.
		$$
		
	}
	
	%--------------------------------------------------------------------------------------------------------------%--------------------------------------------------------------------------------------------------------------	
	\item \mkstar{3}
	
	{Một mạch dao động lí tưởng gồm tụ điện có điện dung $C = \SI{4}{\mu F}$ và cuộn cảm thuần có độ tự $L = \SI{1}{H}$, đang thực hiện dao động điện từ tự do với hiệu điện thế cực đại giữa hai bản tụ điện là $\SI{6}{V}$ thì dòng điện qua cuộn cảm có giá trị cực đại là
		\begin{mcq}(4)
			\item $\xsi{24 \sqrt{2}}{mA}$. 
			\item $\SI{12}{mA}$. 
			\item $\xsi{12 \sqrt{2}}{mA}$. 
			\item $\SI{24}{mA}$. 
		\end{mcq}
	}
	
	\hideall
	{		\textbf{Đáp án: B.}
		
		Trong một mạch dao động, mối quan hệ giữa dòng điện cực đại $I_{0}$ và hiệu điện thế cực đại $U_{0}$ giữa hai bàn tụ là
		$$
		I_{0}=\sqrt{\dfrac{C}{L}} U_{0}=\sqrt{\dfrac{4\cdot10^{-6}}{1}} \cdot 6=  \SI{12}{mA}.
		$$
		
		
	}
	
	
	%--------------------------------------------------------------------------------------------------------------%--------------------------------------------------------------------------------------------------------------	
	\item \mkstar{3}
	
	{Trong mạch dao động LC lí tưởng đang có dao động điện từ tự do, với hiệu điện thế cực đại giữa hai bản tụ là $U_0$ và cường độ dòng điện cực đại trong mạch là $I_0$. Tại thời điểm $t$, hiệu điện thế giữa hai bản tụ điện là $u$ và cường độ dòng điện trong mạch là $i$. Hệ thức liên hệ giữa $u$ và $i$ là
		\begin{mcq}(2)
			\item $i^{2} = LC ({U_0}^{2} - u^{2})$. 
			\item $i^{2} = \dfrac{L ({U_0}^{2} - u^{2})}{C}$. 
			\item $i^{2} = \sqrt{LC} ({U_0}^{2} - u^{2})$. 
			\item $i^{2} = \dfrac{C({U_0}^2 - u^{2})}{L}$. 
		\end{mcq}
	}
	
	\hideall
	{		\textbf{Đáp án: D.}
		
		Điện áp hai đầu tụ điện $u$ và cường độ dòng điện $i$ ở một thời điểm $t$ là những đại lượng vuông pha nhau. Vậy nên ta có hệ thức độc lập:
		$$
		\dfrac{i^{2}}{I_{0}^{2}}+\dfrac{u^{2}}{U_{0}^{2}}=1 \rightarrow i^{2}=\dfrac{I_{0}^{2}}{U_{o}^{2}}\left(U_{0}^{2}-u^{2}\right)=\dfrac{C}{L}\left(U_{0}^{2}-u^{2}\right).
		$$
	}
	
	%--------------------------------------------------------------------------------------------------------------%--------------------------------------------------------------------------------------------------------------	
	\item \mkstar{3}
	
	{Một mạch dao động điện từ lí tưởng đang có dao động điện từ tự do. Tại thời điểm $t = 0$, điện tích trên một bản tụ điện có giá trị cực đại. Sau khoảng thời gian ngắn nhất $\Delta t$ thì điện tích trên bản tụ này bằng một nửa giá trị cực đại. Chu kì dao động riêng của mạch dao động này là
		\begin{mcq}(4)
			\item $4 \Delta t$. 
			\item $6 \Delta t$. 
			\item $3 \Delta t$. 
			\item $12 \Delta t$. 
		\end{mcq}
	}
	
	\hideall
	{		\textbf{Đáp án: B.}
		
		Từ đường tròn pha, ta có khoảng thời gian ngắn nhất để điện tích trên bản tụ điện giảm từ giá trị cực đại xuống còn một nửa giá trị cực đại là
		$$
		\Delta t=\dfrac{T}{6} \rightarrow T=6 \Delta t.
		$$
		
	}
	
	%--------------------------------------------------------------------------------------------------------------%--------------------------------------------------------------------------------------------------------------	
	\item \mkstar{3}
	
	{Trong mạch dao động LC có dao động điện từ tự do (dao động riêng) với tần số góc $\xsi{10^{4}}{rad/s}$. Điện tích cực đại trên tụ điện là $\xsi{10^{-9}}{C}$. Cường độ dòng điện trong mạch cực đại bằng
		\begin{mcq}(4)
			\item $\xsi{2\cdot10^{-5}}{A}$. 
			\item $\xsi{10^{-5}}{A}$. 
			\item $\xsi{10^{-4}}{A}$. 
			\item $\xsi{2\cdot10^{-4}}{A}$. 
		\end{mcq}
	}
	
	\hideall
	{		\textbf{Đáp án: B.}
		
		Cường độ dòng điện cực đại trong mạch cho bởi
		$$
		I_{0}=\omega Q_{0}=10^{4} \cdot 10^{-9}= \xsi{10^{-5}}{A}.
		$$
		
	}
	
	%--------------------------------------------------------------------------------------------------------------%--------------------------------------------------------------------------------------------------------------	
	\item \mkstar{3}
	
	{Mạch dao động gồm tụ điện có điện dung $\SI{4500}{pF}$ và cuộn dây thuần cảm có độ tự cảm $\SI{5}{\mu H}$. Hiệu điện thế cực đại ở hai đầu tụ điện là $\SI{2}{V}$. Cường độ dòng điện trong mạch cực đại bằng
		\begin{mcq}(4)
			\item $\SI{0,03}{A}$. 
			\item $\SI{0,06}{A}$. 
			\item $\xsi{6\cdot10^{-4}}{A}$. 
			\item $\xsi{3\cdot10^{-4}}{A}$. 
		\end{mcq}
	}
	
	\hideall
	{		\textbf{Đáp án: B.}
		
		Cường độ dòng điện cực đại trong mạch cho bởi biểu thức;
		$$
		I_{0}=\sqrt{\dfrac{C}{L}} U_{0}=\sqrt{\dfrac{4500\cdot10^{-12}}{5\cdot10^{-6}}} \cdot 2=\SI{0,06}{A}
		$$
		
	}
	
	%--------------------------------------------------------------------------------------------------------------%--------------------------------------------------------------------------------------------------------------	
	\item \mkstar{3} 
	
	{Cường độ dòng điện tức thời trong mạch dao động LC lí tưởng là $i = \text{0,08} \sin \left( 2000t \right)$ (A). Cuộn dây có độ tự cảm là $L = \SI{50}{mH}$. Hiệu điện thế giữa hai bản tụ tại thời điểm cường độ dòng điện tức thời trong mạch bằng cường độ dòng điện hiệu dụng là
		\begin{mcq}(4)
			\item $\SI{32}{V}$. 
			\item $\xsi{4\sqrt{2}}{V}$. 
			\item $\SI{8}{V}$. 
			\item $\xsi{2\sqrt{2}}{V}$. 
		\end{mcq}
	}
	
	\hideall
	{		\textbf{Đáp án: B.}
		
		Điện dung của tụ điện cho bởi biều thức
		$$
		C=\dfrac{1}{\omega^{2} L}=\dfrac{1}{2000^{2} \cdot 50\cdot10^{-3}}= \SI{5}{\mu C}.
		$$
		Điện áp cực đại giữa hai đầu bản tụ cho bởi:
		$$
		U_{0}=\sqrt{\dfrac{L}{C}} I_{0}=\sqrt{\dfrac{50\cdot10^{-3}}{5\cdot10^{-6}}} \cdot \text{0,08}= \SI{8}{V}.
		$$
		Thay $i=\dfrac{I_{0} \sqrt{2}}{2}$ vào hệ thức độc lập giữa $i$ và $u$ ta được:
		$$
		\left(\dfrac{i}{I_{0}}\right)^{2}+\left(\dfrac{u}{U_{0}}\right)^{2}=1 \rightarrow\left(\dfrac{\sqrt{2}}{2}\right)^{2}+\left(\dfrac{u}{8}\right)^{2}=1 \rightarrow|u|= \xsi{4\sqrt{2}}{V}.
		$$
		
	}
	
\end{enumerate}

