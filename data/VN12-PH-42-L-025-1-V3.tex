
\chapter[Mẫu nguyên tử Bohr: 2 tiên đề Bohr]{Mẫu nguyên tử Bohr: 2 tiên đề Bohr}

\section{Lý thuyết}

\subsection{Các tiên đề của Bohr về cấu tạo nguyên tử}
\subsubsection{Tiên đề về các trạng thái dừng}
\begin{itemize}
	\item Nguyên tử chỉ tồn tại trong một số trạng thái có năng lượng xác định, gọi là các trạng thái dừng, khi ở trạng thái dừng thì nguyên tử không bức xạ.
	\item Trong các trạng thái dừng của nguyên tử, electron chỉ chuyển động quanh hạt nhân trên những quỹ đạo có bán kính hoàn toàn xác định gọi là các quỹ đạo dừng.
\end{itemize}

\subsubsection{Tiên đề về sự bức xạ và hấp thụ năng lượng của nguyên tử}
\begin{itemize}
	\item Khi nguyên tử chuyển từ trạng thái dừng có năng lượng ($E_ n$) sang trạng thái dừng có năng lượng thấp hơn ($E_m$) thì nó phát ra một phôtôn có năng lượng đúng bằng hiệu $E_n-E_m$:
	\begin{equation}
		\varepsilon = h f_{nm}=E_n-E_m.
	\end{equation}
	\item Ngược lại, nếu nguyên tử đang ở trong trạng thái dừng có năng lượng $E_m$ mà hấp thụ được một phôtôn có năng lượng đúng bằng hiệu $E_n-E_m$ thì nó chuyển lên trạng thái dừng có năng lượng cao $E_n$.
\end{itemize}
\luuy{\begin{itemize}
		\item Bình thường, nguyên tử ở trong trạng thái dừng có năng lượng thấp nhất và electron chuyển động trên quỹ đạo gần hạt nhân nhất. Đó là trạng thái cơ bản. \item Khi hấp thụ năng lượng thì nguyên tử chuyển lên các trạng thái dừng có năng lượng cao hơn và electron chuyển động trên những quỹ đạo xa hạt nhân hơn. Đó là các trạng thái kích thích.
\end{itemize}}
\subsection{Ứng dụng mẫu nguyên tử Bohr cho nguyên tử hiđrô}
\subsubsection{Bán kính quỹ đạo dừng}
Đối với nguyên tử hiđrô, bán kính các quỹ đạo dừng tăng tỉ lệ với bình phương của các số nguyên liên tiếp:
\begin{equation}
	r_n = n^2 r_0,
\end{equation}
trong đó:
\begin{itemize}
	\item $r_0 = \SI{5.3e-11}{\meter}$ là bán kính Bohr (nguyên tử ở trạng thái cơ bản);
	\item $n$ là các số nguyên được xác định qua bảng sau:
	\begin{center}
		\begin{tabular}{|m{8em}|m{5em}|m{5em}|m{5em}|m{5em}|m{5em}|m{5em}|}
			\hline
			\textbf{Bán kính} 
			&\multicolumn{1}{c|}{$r_0$}
			&\multicolumn{1}{c|}{$4r_0$}
			&\multicolumn{1}{c|}{$9r_0$}
			&\multicolumn{1}{c|}{$16r_0$}
			&\multicolumn{1}{c|}{$25r_0$}
			&\multicolumn{1}{c|}{$36r_0$}
			\\ \hline
			\textbf{Tên quỹ đạo}
			& \multicolumn{1}{c|}{\begin{tabular}[c]{@{}c@{}} $\text K$\\ $(n=1)$\end{tabular}} 
			& \multicolumn{1}{c|}{\begin{tabular}[c]{@{}c@{}} $\text L$\\ $(n=2)$\end{tabular}} 
			& \multicolumn{1}{c|}{\begin{tabular}[c]{@{}c@{}} $\text M$\\ $(n=3)$\end{tabular}} 
			& \multicolumn{1}{c|}{\begin{tabular}[c]{@{}c@{}} $\text N$\\ $(n=4)$\end{tabular}} 
			& \multicolumn{1}{c|}{\begin{tabular}[c]{@{}c@{}} $\text O$\\ $(n=5)$\end{tabular}} 
			& \multicolumn{1}{c|}{\begin{tabular}[c]{@{}c@{}} $\text P$\\ $(n=6)$\end{tabular}} 
			\\ \hline
		\end{tabular}
	\end{center}
\end{itemize}
\subsubsection{Năng lượng của nguyên tử ở các trạng thái dừng}
Năng lượng của nguyên tử hiđrô ở các trạng thái dừng được xác định bởi công thức:
\begin{equation}
	E_n= \dfrac {-13.6}{n ^2} \SI{}{\electronvolt},
\end{equation}
trong đó $\SI{13.6}{\electronvolt}$ là năng lượng ion hóa của nguyên tử hiđrô.

\section{Mục tiêu bài học - Ví dụ minh họa}
\begin{dang}{Vận dụng kết hợp công thức tính lực hướng tâm và công thức tính lực điện.}
	\viduii{3}{Cho bán kính Bohr là $r_0=\SI{5.3e-11}{\meter}$, hằng số điện $k=\SI{9e9}{N \meter ^2 / \coulomb ^2}$, điện tích nguyên tố $e=\SI{1.6e-19}{\coulomb}$, khối lượng electron $m_e=\SI{9.1e-31}{\kilogram}$. Trong nguyên tử hiđrô, nếu electron chuyển động tròn đều quanh hạt nhân thì ở quỹ đạo L, tốc độ góc của electron là
		\begin{mcq}(2)
			\item $\SI{0.5e16}{\radian / \second}$.
			\item $\SI{2.4e16}{\radian / \second}$.
			\item $\SI{1.5e16}{\radian / \second}$.
			\item $\SI{4.6e16}{\radian / \second}$.
	\end{mcq}}
	{\begin{center}
			\textbf{Hướng dẫn giải}
		\end{center}
		
		Nếu coi electron chuyển động tròn đều quanh hạt nhân thì lực điện giữa electron và hạt nhân đóng vai trò lực hướng tâm:
		\begin{equation*}
			k \dfrac {|q_1 q_2|}{r^2} = m_e \omega ^2 r = m_e \dfrac{v^2}{r},
		\end{equation*}
		trong đó $r$ được xác định bởi công thức $r_n = n^2 r_0$.
		
		Bán kính nguyên tử ở quỹ đạo L ($n = 2$):
		\begin{equation*}
			r_n =n^2 r_0 = \SI{2.12e-10}{\meter}.
		\end{equation*}
		
		Tốc độ góc của electron:
		\begin{equation*}
			k \dfrac {|q_1q_2|}{r^2} = m_e \omega ^2 r \Rightarrow \omega = \sqrt {k\dfrac{|q_1q_2|}{m_er^3}} \approx \SI{0.5e16}{\radian / \second}.
		\end{equation*}
		
		\begin{center}
			\textbf{Câu hỏi tương tự}
		\end{center}
		
		Cho bán kính Bohr là $r_0=\SI{5.3e-11}{\meter}$, hằng số điện $k=\SI{9e9}{N \meter ^2 / \coulomb ^2}$, điện tích nguyên tố $e=\SI{1.6e-19}{\coulomb}$, khối lượng electron $m_e=\SI{9.1e-31}{\kilogram}$. Trong nguyên tử hiđrô, nếu electron chuyển động tròn đều quanh hạt nhân thì ở quỹ đạo L, tốc độ dài của electron là
		\begin{mcq}(2)
			\item $ \SI{1,06 e6}{m/s} $.
			\item $ \SI{5,09 e6}{m/s} $.
			\item $ \SI{3,18 e6}{m/s} $.
			\item $ \SI{9,75 e6}{m/s} $.
		\end{mcq}	
		
		\textbf{Đáp án:} A.}
	
	\viduii{3}{ Theo các tiên đề Bohr, giả sử chuyển động của electron quanh hạt nhân là chuyển động tròn đều. Tỉ số giữa tốc độ của electron trên quỹ đạo K với tốc độ của electron trên quỹ đạo N bằng
		\begin{mcq}(4)
			\item 4.
			\item 3.
			\item 6.
			\item 9.
	\end{mcq}}
	{	\begin{center}
			\textbf{Hướng dẫn giải}
		\end{center}
		
		Từ phương trình:
		\begin{equation*}
			k \dfrac {|q_1q_2|}{r^2} = m_e \dfrac{v^2}{r},
		\end{equation*}
		ta thấy
		\begin{equation*}
			v^2 = k\dfrac{m|q_1q_2|}{r}
		\end{equation*}
		hay
		\begin{equation*}
			v^2 \sim \dfrac{1}{r}.
		\end{equation*}
		
		Mà $r_n =n ^2 r_0$, suy ra
		\begin{equation*}
			v \sim \dfrac{1}{n}.
		\end{equation*}
		
		Tỉ số giữa tốc độ của electron trên quỹ đạo K ($n_\text K = 1$) với tốc độ của electron trên quỹ đạo N ($n_\text N= 4$):
		\begin{equation*}
			\dfrac{v_\text K}{v_\text N} = \dfrac{n_\text N}{n_\text K}=\dfrac{4}{1}=4.
		\end{equation*}
		
		\begin{center}
			\textbf{Câu hỏi tương tự}
		\end{center}
		
		Theo các tiên đề Bohr, giả sử chuyển động của electron quanh hạt nhân là chuyển động tròn đều. Tỉ số giữa tốc độ của electron trên quỹ đạo K với tốc độ của electron trên quỹ đạo L bằng
		\begin{mcq}(4)
			\item 4.
			\item 3.
			\item 2.
			\item 9.
		\end{mcq}
		
		\textbf{Đáp án:} C.}
\end{dang}
\begin{dang}{Vận dụng công thức tính năng lượng bức xạ hoặc hấp thụ của nguyên tử.}
	\vidu{3}{ Electron trong nguyên tử hiđrô chuyển từ quỹ đạo có năng lượng $E_\text M = \SI{-1.5}{\electronvolt}$ xuống quỹ đạo có năng lượng $E_\text L = \SI{-3.4}{\electronvolt}$. Tìm bước sóng của vạch quang phổ phát ra.
		\begin{mcq}(2)
			\item $\lambda = \SI{0.654}{\micro \meter}$.
			\item $\lambda= \SI{0.643}{\micro \meter}$.
			\item $\lambda = \SI{0.564}{\micro \meter}$.
			\item $\lambda = \SI{0.458}{\micro \meter}$.
	\end{mcq}}
	{\begin{center}
			\textbf{Hướng dẫn giải}
		\end{center}
		
		Áp dụng công thức tính năng lượng của phôtôn phát ra khi nguyên tử chuyển từ trạng thái dừng có mức năng lượng $E_\text M$ (cao) sang trạng thái dừng có mức năng lượng $E_\text L$ (thấp): \begin{equation*}
			\varepsilon = \dfrac {hc}{\lambda} = E_\text M - E_ \text L. 
		\end{equation*}
		
		Đổi $E_\text M = \SI{-1.5}{\electronvolt} \rightarrow \SI{-2.4e-19}{\joule}$ và $E_\text L = \SI{-3.4}{\electronvolt} \rightarrow \SI{-5.44e-19}{\joule}$.
		
		Áp dụng công thức:
		\begin{align*}
			\dfrac{hc}{\lambda}&=E_\text M - E_ \text L \\
			\Rightarrow \lambda &= \dfrac {hc}{E_\text M - E_ \text L} \approx \SI{0.654}{\micro \meter}.
		\end{align*}
		
		\begin{center}
			\textbf{Câu hỏi tương tự}
		\end{center}
		
		Electron trong nguyên tử hiđrô chuyển từ quỹ đạo có năng lượng $E_\text M = \SI{-1.5}{\electronvolt}$ xuống quỹ đạo có năng lượng $E_\text L = \SI{-3.4}{\electronvolt}$. Vạch quang phổ phát ra bước sóng $ \SI{0,654}{\mu m} $. Giá trị của $ E_\text L $ là
		\begin{mcq}(4)
			\item $ \SI{-3,4}{eV} $.
			\item $ \SI{-2,4}{eV} $.
			\item $ \SI{-4,2}{eV} $.
			\item $ \SI{-3,0}{eV} $.
		\end{mcq}
		
		\textbf{Đáp án:} A.
	}
\end{dang}

\section{Bài tập tự luyện}
\begin{enumerate}[label=\bfseries Câu \arabic*:]
	
	% Câu No 
	\item \mkstar{1} [5]
	\cauhoi
	{Ở trạng thái dừng, mỗi electron chuyển động trên hạt nhân với quỹ đạo dừng có bán kính
		\begin{mcq}(4)
			\item giảm dần. 
			\item giảm rồi tăng.
			\item tăng dần. 
			\item xác định. 
		\end{mcq}
	}
	
	\loigiai
	{		\textbf{Đáp án: D.}
		
		Ở trạng thái dừng, mỗi electron chuyển động trên hạt nhân với quỹ đạo dừng có bán kính xác định.		
	}
	
	% Câu No 
	\item \mkstar{2} [1] 
		\cauhoi
	{Xét nguyên tử Hidro theo mẫu nguyên tử Bo. Gọi $ r_{0} $ là bán kính Bo thì quỹ đạo dừng N có bán kính bằng
		\begin{mcq}(4)
			\item $ 4 r_{0} $. 
			\item $ 16 r_{0} $.
			\item $ 2 r_{0} $. 
			\item $ 8 r_{0} $. 
		\end{mcq}
	}
	
	\loigiai
	{		\textbf{Đáp án: B.}	
		
		Quỹ đạo dừng N có $ n = 3 $. Nên bán kính quỹ đạo dừng N là $ n^{2} r_{0} = 16 r_{0} $.
	}
	
	% Câu No 
	\item \mkstar{2} [1]
		\cauhoi
	{Theo lý thuyết Bo, khi chuyển từ trạng thái $ \SI{-3,4}{eV} $ lên trạng thái dừng có năng lượng $ \SI{-0,85}{eV} $ nguyên tử Hidro phải hấp thụ photon có năng lượng
		\begin{mcq}(4)
			\item $ \SI{0,85}{eV} $. 
			\item $ \SI{4,25}{eV} $.
			\item $ \SI{3,4}{eV} $. 
			\item $ \SI{2,55}{eV} $. 
		\end{mcq}
	}
	
	\loigiai
	{		\textbf{Đáp án: D.}
		
		Nguyên tử Hidro phải hấp thụ photon có năng lượng là
		$$
		\varepsilon = \SI{-0,85}{eV} - \left( \SI{-3,4}{eV} \right) = \SI{2,55}{eV}.
		$$	
	}
	
	% Câu No 
	\item \mkstar{3} [1]
		\cauhoi
	{Theo lí thuyết Bo, năng lượng trong nguyên tử Hidro được xác định bằng công thức $ E_{n} = \xsi{\dfrac{-13,6}{n^{2}}}{eV}$ với $ n = 1, 2, 3, ...\infty $ ứng với các quỹ đạo K, L, M, ... Gọi $ r_{0} $ là bán kính Bo. Khi nguyên tử ở trạng thái dừng có năng lượng $ \SI{-0,544}{eV} $, electron trong nguyên tử đang chuyển động với quỹ đạo có bán kính
		\begin{mcq}(4)
			\item $ 25 r_{0} $. 
			\item $ 5 r_{0} $.
			\item $ 4 r_{0} $. 
			\item $ 16 r_{0} $. 
		\end{mcq}
	}
	
	\loigiai
	{		\textbf{Đáp án: A.}
		
		Ta có:
		$$
		E_{n} = \dfrac{-13,6}{n^{2}} \ rightarrow \num{-0,544} = \dfrac{-13,6}{n^{2}} \rightarrow n = 5.
		$$		
		Khi đó, bán kính của quỹ đạo electron là $ n^{2} r_{0} = 25 r_{0}$.
	}
	
	% Câu No 
	\item \mkstar{3} [2]
		\cauhoi
	{Xét nguyên tử Hidro theo mẫu nguyên tử Bo. Electron trong nguyên tử chuyển từ quỹ đạo dừng n lên quỹ đạo dừng m thì bán kính quỹ đạo tăng 9 lần. Gọi $ r_{0} $ là bán kính quỹ đạo Bo thì bán kính quỹ đạo dừng m \textbf{không} thể nhận giá trị nào sau đây?
		\begin{mcq}(4)
			\item $ 100 r_{0} $. 
			\item $ 144 r_{0} $.
			\item $ 36 r_{0} $. 
			\item $ 18 r_{0} $. 
		\end{mcq}
	}
	
	\loigiai
	{		\textbf{Đáp án: D.}
		
		Vì khi chuyển từ quỹ đạo dừng n lên quỹ đạo dừng m thì bán kính tăng 9 lần nên
		$$
		r_{m} = 9 \cdot r_{n} \rightarrow m^{2} \cdot r_{0} = 9 n^{2} \cdot r_{0} \ rightarrow n^{2} = \dfrac{m^{2}}{9}.
		$$		
		Ta thấy khi $ r_{m} = 100 r_{0} $ thì $ r_{n} = \SI{1111,1}{r_{0}}$.
		Vậy $ r_{m} $ không thể nhận giá trị $ 100 r_{0} $
	}
	
	% Câu No 
	\item \mkstar{3} [2]
		\cauhoi
	{Trong nguyên tử Hidro, bán kính Bo là $ r_{0} = \SI{5,3e-11}{m} $. Ở một trạng thái kích thích của nguyên tử Hidro, electron chuyển động trên quỹ đạo dừng có bán kính $ r = \SI{2,12e-10}{m} $. Quỹ đạo đó có tên quỹ đạo dừng
		\begin{mcq}(4)
			\item L. 
			\item M.
			\item O. 
			\item N. 
		\end{mcq}
	}
	
	
	\loigiai
	{		\textbf{Đáp án: A.}
		
		Ta có:
		$$
		r = n^{2} \cdot r_{0} \rightarrow n = \num{2}.
		$$
		Vậy electron này dang ở trên quỹ đạo L.		
	}
	
	% Câu No 
	\item \mkstar{3} [2]
	\cauhoi
	{Theo mẫu Bo về nguyên tử Hidro, lực tương tác tĩnh điện giữa electron và hạt nhân khi electron chuyển động trên quỹ đạo dừng K là F. Khi electron chuyển động từ quỹ đạo dừng N về quỹ đạo dừng L thì lực tương tác tĩnh điện giữa electron và hạt nhân tăng thêm
		\begin{mcq}(4)
			\item $ \dfrac{15}{16} F $. 
			\item $ \dfrac{15}{256} F $.
			\item $ 12 F $. 
			\item $ 240 F $. 
		\end{mcq}
	}
	
	\loigiai
	{		\textbf{Đáp án: B.}
		
		Lực tương tác trên quỹ đạo dừng K là
		$$
		F = k \cdot \dfrac{e^{2}}{r_{0}^2}
		$$ 
		Lực tương tác trên quỹ đạo dừng N là
		$$
		F_{N} = k \cdot \dfrac{e^{2}}{16^{2} \cdot r_{0}^2} = \dfrac{F}{256}
		$$   
		Lực tương tác trên quỹ đạo dừng L là
		$$
		F_{L} = k \cdot \dfrac{e^{2}}{4^{2} \cdot r_{0}^2} = \dfrac{F}{16}
		$$           		
		Vậy khi chuyển từ quỹ đạo dừng N về quỹ đạo dừng L thì lực tương tác tĩnh điện giữa electron và hạt nhân tăng thêm $ \dfrac{F}{16} - \dfrac{F}{256} = \dfrac{15}{256} F$.
	}
	
	% Câu No 
	\item \mkstar{1} [4]
	\cauhoi
	{Theo mẫu nguyên tử Bo, trạng thái dừng của nguyên tử
		\begin{mcq}(1)
			\item có thể là trạng thái cơ bản hoặc trạng thái kích thích. 
			\item chỉ là trạng thái kích thích.
			\item là trạng thái mà các electron nguyên tử ngừng chuyển động. 
			\item chỉ là trạng thái cơ bản. 
		\end{mcq}
	}
	
	\loigiai
	{		\textbf{Đáp án: A.}
		
		Theo mẫu nguyên tử Bo, trạng thái dừng của nguyên tử có thể là trạng thái cơ bản hoặc trạng thái kích thích. 
	}
	
	% Câu No 
	\item \mkstar{3} [4]
		\cauhoi
	{Nguyên tử H chuyển từ trạng thái kích thích về trạng thái nguyên tử có mức năng lượng thấp hơn phát ra bức xạ có bước sóng $ \lambda = \SI{486}{nm} $. Độ giảm năng lượng của nguyên tử khi phát ra bức xạ là
		\begin{mcq}(4)
			\item $ \SI{4,09e-15}{J} $. 
			\item $ \SI{4,86e-19}{J} $.
			\item $ \SI{4,09e-19}{J} $. 
			\item $ \SI{3,08e-20}{J} $. 
		\end{mcq}
	}
	
	\loigiai
	{		\textbf{Đáp án: C.}
		
		Độ giảm năng lượng của nguyên tử khi phát ra bức xạ đúng bằng năng lượng của photon phát ra cho bởi:
		$$
		\varepsilon = \dfrac{hc}{\lambda} = \SI{4,09e-19}{J}.
		$$
	}
	
	% Câu No 
	\item \mkstar{1} [10]
	\cauhoi
	{Trạng thái dừng của nguyên tử là
		\begin{mcq}(1)
			\item  trạng thái đứng yên của nguyên tử. 
			\item trạng thái chuyển động đều của nguyên tử.
			\item trạng thái trong đó mọi êlectron của nguyên tử đều không chuyển động đối với hạt nhân.
			\item một trong số các trạng thái có năng lượng xác định, mà nguyên tử có thể tồn tại.
		\end{mcq}
	}
	
	\loigiai
	{		\textbf{Đáp án: D.}
		
		Trạng thái dừng của nguyên tử là một trong số các trạng thái có năng lượng xác định, mà nguyên tử có thể tồn tại.		
	}
	
	% Câu No 
	\item \mkstar{1} [10]
		\cauhoi
	{Phát biểu nào sau đây là đúng khi nói về mẫu nguyên tử Borh?
		\begin{mcq}(1)
			\item Nguyên tử bức xạ khi chuyển từ trạng thái cơ bản lên trạng thái kích thích. 
			\item Trong các trạng thái dừng, động năng của êlectron trong nguyên tử bằng không.
			\item Khi ở trạng thái cơ bản, nguyên tử có năng lượng cao nhất.
			\item Trạng thái kích thích có năng lượng càng cao thì bán kính quỹ đạo của êlectron càng lớn.
		\end{mcq}
	}
	
	\loigiai
	{		\textbf{Đáp án: D.}
		
		Trạng thái kích thích có năng lượng càng cao thì bán kính quỹ đạo của êlectron càng lớn.		
	}
	
	% Câu No 
	\item \mkstar{3} [13]
		\cauhoi
	{Tốc độ của electron trong nguyên tử Hidro khi nó tồn tại ở trạng thái dừng M là
		\begin{mcq}(4)
			\item $ \SI{0,73e6}{m/s} $. 
			\item $ \SI{1,09e6}{m/s} $.
			\item $ \SI{0,55e6}{m/s} $. 
			\item $ \SI{0,22e6}{m/s} $. 
		\end{mcq}
	}
	
	\loigiai
	{		\textbf{Đáp án: A.}
		
		Áp dụng định luật II Newton trên phương hướng tâm ta có:
		$$
		F = m a_{ht} \rightarrow k \cdot \dfrac{e^{2}}{r_{M}^2} = m \cdot \dfrac{v^{2}}{r_{M}^2} \rightarrow k \cdot \dfrac{e^{2}}{\left( 3^{2} r_{0} \right)^{2}} = m \cdot \dfrac{v^{2}}{ 3^{2} r_{0} } \rightarrow v_{0} = \SI{0,73e6}{m/s}.
		$$		
	}
		\item \mkstar{1} 
		\cauhoi
	{Phát biểu nào sau đây là sai, khi nói về mẫu nguyên tử Bohr?
		
		\begin{mcq}
			\item  Trong trạng thái dừng, nguyên tử không bức xạ.
			\item Trong trạng thái dừng, nguyên tử có bức xạ.
			
			\item  Khi nguyên tử chuyển từ trạng thái dừng có năng lượng $E_\text n$ sang trạng thái dừng có năng lượng $E_\text m$ ($E_\text m < E_\text n$) thì nguyên tử phát ra một phôtôn có năng lượng đúng bằng. ($E_\text n - E_\text m$).
			
			\item  Nguyên tử chỉ tồn tại ở một số trạng thái có năng lượng xác định, gọi là các trạng thái dừng. 
		\end{mcq}
	}
	
	\loigiai
	{		\textbf{Đáp án: B.}
		
		Trong trạng thái dừng, nguyên tử không có bức xạ. 
	}
		\item \mkstar{1}
		\cauhoi
	{Để nguyên tử hydro hấp thụ một photon, thì photon phải có năng lượng bằng năng lượng
		\begin{mcq}
			\item của trạng thái dừng có năng lượng thấp nhất. 
			 
			\item của một trong các trạng thái dừng. 
			
			\item của trạng thái dừng có năng lượng cao nhất. 
			
			\item của hiệu năng lượng ở hai trạng thái dừng bất kì.
			
		\end{mcq}
	}
	
	\loigiai
	{		\textbf{Đáp án: D.}
		
	
	}
		\item \mkstar{1}
		\cauhoi
	{Bán kính quỹ đạo dừng thứ n của electron trong nguyên tử hydro 
		
		\begin{mcq}(2)
			\item   tỉ lệ thuận với $n$.	
			\item 	tỉ lệ nghịch với $n$.	
			\item   tỉ lệ thuận với $n^2$.
			\item   tỉ lệ nghịch với $n^2$.
			
		\end{mcq}
	}
	
	\loigiai
	{		\textbf{Đáp án: C.}
		
		Bán kính quỹ đạo dừng thứ n của electron trong nguyên tử hydro tỉ lệ thuận với $n^2$.
	}
\end{enumerate}

