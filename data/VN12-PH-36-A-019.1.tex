% --- chapter
\newcommand{\chapter}[2][]{
	\newcommand{\chapname}{#2}
	\begin{flushleft}
		\begin{minipage}[t]{\linewidth}
			\includegraphics[height=1cm]{hdht-logo.png}
			\hspace{0pt}	
			\sffamily\bfseries\large Bài  28. Tia X
			\begin{flushleft}
				\huge\bfseries #1
			\end{flushleft}
		\end{minipage}
	\end{flushleft}
	\vspace{1cm}
	\normalfont\normalsize
}
%-----------------------------------------------------
\chapter[Tính chất hạt của tia X]{Tính chất hạt của tia X \footnotetext{Dạng bài tập này sử dụng các kiến thức của chương \textit{Lượng tử ánh sáng}}}

\section{Lý thuyết}

\subsection{Tìm bước sóng nhỏ nhất do tia X phát ra (hay tần số lớn nhất)}

\begin{itemize}
\item Hiện tượng: khi các electron được tăng tốc trong điện trường thì năng lượng của chúng gồm động năng ban đầu cực đại và năng lượng điện trường cung cấp.
\item Khi đập vào đối cathode thì năng lượng của electron được chuyển hóa thành gồm nhiệt lượng (làm nóng đối cathode) và năng lượng  phát tia X.
\item Năng lượng của một photon trong chùm tia X phát ra:
\begin{equation}
	\varepsilon_X =\dfrac{hc}{\lambda}.
\end{equation}

\textbf{Năng lượng dòng electron bằng tổng năng lượng tia X và nhiệt năng} 
\begin{equation}
	\varepsilon = \varepsilon_{\text{X}}+ Q \geq \varepsilon_{\text{X}} \quad\Leftrightarrow\quad \dfrac{hc}{\lambda_\text{X}} \leq \varepsilon  \quad\Rightarrow\quad \lambda_{\text{X}} \geq \dfrac{hc}{\varepsilon}
\end{equation}
trong đó $\varepsilon = |e|U_{\text{AK}}$ là năng lượng của electron, $U_{\text{AK}}$ là điện áp đặt vào anode và cathode của ống phóng tia X.
\end{itemize}

\subsection{Tìm vận tốc cực đại của electron khi đập vào Cathode}

\begin{itemize}
	\item Định lý động năng
	\begin{equation}
		W_{\text{đ}}- W_{\text{đ}_0} =eU_{\text{AK}}.
	\end{equation}
	\item Công thức động năng
	\begin{equation}
		W_{\text{đ}}=\dfrac{1}{2}mv^2.
	\end{equation}
\end{itemize}


\subsection{Tính nhiệt lượng làm nóng đối Cathode}

\begin{itemize}
	\item Cường độ dòng điện trong ống Ronghen 
	\begin{equation}
		I = \dfrac{N}{t}|e|.
	\end{equation}
trong đó:

+ $I$ là cường độ dòng điện trong ống Ronghen (A),

+ $e$ là điện tích của electron (C),

+ $N$ là số electron đập vào đối cathode trong khoảng thời gian $t$,

+ $t$ thời gian electron đập vào đối cathode (s).
	\item Công thức tính nhiệt lượng: 
	\begin{equation}
		Q=mc(t_2-t_1).
	\end{equation}
trong đó:

+ $Q$ là nhiệt lượng (J),

+ $m$ là khối lượng của vật (kg),

+ $t_2-t_1$ là độ tăng (giảm) nhiệt độ của vật ($^\circ$ C hoặc K),

+ $c$ là nhiệt dung riêng của chất làm vật (J/kg.K).
	\item Nhiệt lượng làm nóng đối cathode bằng tổng động năng của các electron đến đập vào đối cathode:
	
	\begin{equation}
		Q = W = N\cdot W_{\text{đ}} = N\cdot e.
	\end{equation}
Với $N$ tổng số  electron đến đối cathode.

\end{itemize}

\section{Mục tiêu bài học - Ví dụ minh họa}

\begin{dang}{Tìm bước sóng nhỏ nhất do tia X phát ra (hay tần số lớn nhất).}

\ppgiai{
\begin{description}
	\item[Bước 1] Xác định năng lượng của dòng electron
	\begin{equation*}
		\varepsilon = |e|U_{\text{AK}}.
	\end{equation*}
	\item[Bước 2] Để xác định bước sóng nhỏ nhất do tia X phát ra (hay tần số lớn nhất).
	\begin{equation*}
		\lambda_{min}= \dfrac{hc}{	|e|U_{\text{AK}}}\quad \Rightarrow\quad f_{\text{max}}=\dfrac{c}{\lambda_{min}}.
	\end{equation*}
\end{description}
}

\viduii{2}
{
Hiệu điện thế giữa anode và cathode của ống R\"ontgen là 18,75 kV. Cho $e =\text{1,6} \cdot 10^{-19}\ \text{C}$, $h = \text{6,625}\cdot 10^{-34}\ \text{Js}$, $c = 3\cdot 10^8\ \text{m/s}$. Bỏ qua động năng ban đầu của electron. Bước sóng nhỏ nhất mà tia R\"ontgen phát ra là bao nhiêu?}
{\begin{center}
\textbf{Hướng dẫn giải}
\end{center}
\begin{itemize}
	\item Vận dụng công thức 
	\begin{equation*}
	\lambda_{min}= \dfrac{hc}{|e|U_{\text{AK}}}.
	\end{equation*}
	\item Thay số vào biểu thức trên tìm được $\lambda_{min}=\text{0,6625} \cdot 10^{-10}\ \text{m}$.
\end{itemize}
\begin{center}
\textbf{Câu hỏi tương tự}
\end{center}

Hiệu điện thế giữa anode và cathode của ống R\"ontgen là $ U_{AK} $. Cho $e =\text{1,6} \cdot 10^{-19}\ \text{C}$, $h = \text{6,625}\cdot 10^{-34}\ \text{Js}$, $c = 3\cdot 10^8\ \text{m/s}$. Bước sóng nhỏ nhất mà tia R\"ontgen phát ra là $ \SI{0,6625 e-10}{m} $. Hiệu điện thế $ U_{AK} $ là
\begin{mcq}(4)
	\item $ \SI{17,85}{kV} $.
	\item $ \SI{18,75}{kV} $.
	\item $ \SI{15,78}{kV} $.
	\item $ \SI{18,85}{kV} $.
\end{mcq}

\textbf{Đáp án:} B.

}
\viduii{2}
{
Trong một ống R\"ontgen, biết hiệu điện thế giữa anode va cathode là $U = 2\cdot 10^6\ \text{V}$.  Hãy tìm bước sóng nhỏ nhất của tia R\"ontgen do ống phát ra?
}
{\begin{center}
	\textbf{Hướng dẫn giải}
\end{center}
\begin{itemize}
	\item Vận dụng công thức 
	\begin{equation*}
		\lambda_{min}= \dfrac{hc}{	|e|U_{\text{AK}}}.
	\end{equation*}
	\item Thay số vào biểu thức trên tìm được $\lambda_{min}=\text{0,62} \cdot 10^{-12}\ \text{m}$.
\end{itemize}

\begin{center}
\textbf{Câu hỏi tương tự}
\end{center}

Trong một ống R\"ontgen, biết hiệu điện thế giữa anode va cathode là $U = 2\cdot 10^6\ \text{V}$. Động năng cực đại của electron đập vào đối cathode là 
\begin{mcq}(2)
	\item $ \SI{2,3 e-12}{J} $.
	\item $ \SI{2,3 e-13}{J} $.
	\item $ \SI{3,2 e-12}{J} $.
	\item $ \SI{3,2 e-13}{J} $.
\end{mcq}

\textbf{Đáp án:} D.

}

\end{dang}

\begin{dang}{Tìm vận tốc cực đại của electron khi\\ đập vào Cathode.}

\ppgiai{
\begin{description}
	\item[Bước 1] Áp dụng công thức năng lượng do điện trường cung cấp
	\begin{equation}
		W_{\text{đ}}=A=|e| U_{\text{AK}}.
	\end{equation} 
	\item [Bước 2] Với $W_{\text{đ}}=|e|U_{\text{AK}}=\dfrac{1}{2}mv^2$ từ đó suy ra được vận tốc cực đại của electron khi đập vào cathode.
\end{description}
}

\viduii{2}
{
Hiệu điện thế giữa anode và cathode của ống ống phóng tia X là  20 kV. Cho $e=\text{1,6}\cdot 10^{-19}\ \text{C}$, $h=\text{6,625} \cdot 10^{-34}\ \text{Js}$, $c=3\cdot 10^8\ \text{m/s}$. Bỏ qua động năng ban đầu của electron. Tính vận tốc của electron khi đập vào cathode
}
{\begin{center}
	\textbf{Hướng dẫn giải}
\end{center}
\begin{itemize}
	\item Vận tốc của electron khi đập vào cathode
	\begin{equation*} W_{\text{đ}}=|e|U_{\text{AK}}=\dfrac{1}{2}mv^2 \Rightarrow v =\sqrt {\dfrac{2|e|U_{\text{AK}}}{m}}.
	\end{equation*}
	\item Thay các đại lượng $U_{\text{AK}}$, $e$ và khối lượng electron vào suy ra
	\begin{equation*}
	v =\sqrt {\dfrac{2|e|U_{\text{AK}}}{m}} = \text{8,4} \cdot 10^7 \ \text{m/s}.
	\end{equation*}
\end{itemize}

\begin{center}
\textbf{Câu hỏi tương tự}
\end{center}

Hiệu điện thế giữa anode và cathode của ống ống phóng tia X là  40 kV. Cho $e=\text{1,6}\cdot 10^{-19}\ \text{C}$, $h=\text{6,625} \cdot 10^{-34}\ \text{Js}$, $c=3\cdot 10^8\ \text{m/s}$. Bỏ qua động năng ban đầu của electron. Tính vận tốc của electron khi đập vào cathode
\begin{mcq}(4)
	\item $ \SI{6,4 e-15}{J} $.
	\item $ \SI{4,6 e-15}{J} $.
	\item $ \SI{6,4 e-13}{J} $.
	\item $ \SI{4,6 e-13}{J} $.
\end{mcq}

\textbf{Đáp án:} A.
}

\viduii{2}
{Vận tốc của electron khi đập vào đối cathode của ống tia X là $8\cdot 10^7 \ \text{m/s}$. Để vận tốc tại đối cathode  giảm $6\cdot 10^6\ \text{m/s}$ thì hiệu điện thế giữa hai cực của ống phải 
\begin{mcq}(2)
	\item Giảm 5200 V.
	\item Tăng 2628 V.
	\item Giảm 2628 V.
	\item Giảm 3548 V.
\end{mcq}
}
{\begin{center}
	\textbf{Hướng dẫn giải}
\end{center}
\begin{itemize}
	\item Vận tốc của electron khi đập vào đối cathode của ống tia X là $8 \cdot 10^7 \ \text{m/s}$
	\begin{equation*}
		\dfrac{1}{2}mv^2_1 =|e| U_1 \Rightarrow U_1=\dfrac{mv^2_1}{2|e|}=18200\ \text{V}.
	\end{equation*}
	\item Vận tốc đối cathode giảm $6 \cdot 10^6\ \text{m/s}$ $v_2=v_1 - 6 \cdot 10^6 = \text{7,4}\cdot 10^7\ \text{m/s}$
	\begin{equation*}
		\dfrac{1}{2}mv^2_2 =|e| U_2 \Rightarrow U_2=\dfrac{mv^2_2}{2|e|}=15572\ \text{V}.
	\end{equation*}
	\item Hiệu điện thế giữa hai cực của ống giảm $\Delta U = U_1 - U_2 =2628\ \text{V}$.
	
	\textbf{Đáp án: C.}
\end{itemize}

\begin{center}
\textbf{Câu hỏi tương tự}
\end{center}

Vận tốc của electron khi đập vào đối cathode của ống tia X là $8\cdot 10^7 \ \text{m/s}$. Để vận tốc tại đối cathode  tăng $6\cdot 10^6\ \text{m/s}$ thì hiệu điện thế giữa hai cực của ống phải 
\begin{mcq}(2)
	\item Giảm 5200 V.
	\item Tăng 2628 V.
	\item Giảm 2628 V.
	\item Tăng 2831 V.
\end{mcq}

\textbf{Đáp án:} D.

}

\end{dang}


\begin{dang}{Tính nhiệt lượng làm nóng đối Cathode.}

\ppgiai{
\begin{description}
	\item [Bước 1] Tìm động năng cực đại 
	\begin{equation*}
		W_{\text{đmax}}=\dfrac{1}{2}mv^2 =|e|U_{\text{AK}}.
	\end{equation*}
	\item [Bước 2] Xác định số electron đến đối cathode
		\begin{equation*}
	N = \dfrac{I \cdot t}{|e|}.
		\end{equation*}
	\item [Bước 3] Áp dụng công thức để tìm nhiệt lượng tỏa ra
	\begin{equation*}
		Q = W = N\cdot W_{\text{đ}} = N\cdot e.
	\end{equation*}
\end{description}
}

\viduii{3}
{Một ống phát tia X có bước sóng ngắn nhất $5\cdot 10^{-10}\ \text{m}$. Bỏ qua vận tốc ban đầu của các electron khi bứt ra khỏi cathode. Giả sử $100\%$ động năng của các electron biến thành nhiệt làm nóng đối cathode và cường độ dòng điện chạy qua ống là $I = 2\ \text{mA}$. Nhiệt lượng tỏa ra trên đối cathode trong 1 phút là 
\begin{mcq}(4)
\item 298,125 J.
\item 29,813 J.
\item 928,125 J.
\item 92,813 J.
\end{mcq}
}
{\begin{center}
	\textbf{Hướng dẫn giải}
\end{center}

\begin{itemize}
	\item Số electron tới đập vào đối cathode trong $1\ \text{s}$ là:
	\begin{equation*}
		N=\dfrac{I}{e} = \text{1,25} \cdot 10^{16}.
	\end{equation*}
	\item Năng lượng của một photon tia X 
	\begin{equation*}
		W=W_{\text{đmax}}=\dfrac{hc}{\lambda_{\text{min}}}=\text{3,975} \cdot 10^{-16}\ \text{J}.
	\end{equation*}
	\item Nhiệt lượng tỏa ra trong 1 phút
	\begin{equation*}
	Q=NW \cdot 60 = \text{298,125}\ \text{J}.
	\end{equation*}
\end{itemize}

\begin{center}
\textbf{Câu hỏi tương tự}
\end{center}

Một ống phát tia X có bước sóng ngắn nhất $5\cdot 10^{-10}\ \text{m}$. Bỏ qua vận tốc ban đầu của các electron khi bứt ra khỏi cathode. Giả sử $98\%$ động năng của các electron biến thành nhiệt làm nóng đối cathode và cường độ dòng điện chạy qua ống là $I = 1,5\ \text{mA}$. Nhiệt lượng tỏa ra trên đối cathode trong 1 phút là 
\begin{mcq}(4)
\item 298,125 J.
\item 29,813 J.
\item 218,708 J.
\item 92,813 J.
\end{mcq}

\textbf{Đáp án:} C.
}
\viduii{3}
{Một ống Ronghen hoạt động với hiệu điện thế $U =\text{2,5}\ \text{kV}$. Cường độ dòng điện qua ống là 0,01 A. Tính số electron đập vào cathode mỗi giây và nhiệt lượng cung cấp cho đối cathode mỗi phút, giả sử toàn bộ động năng của electron dùng để đốt nóng đối âm cực.
}
{\begin{center}
	\textbf{Hướng dẫn giải}
\end{center}   

\begin{itemize}
	\item Động năng ban đầu cực đại
	\begin{equation*}
		W_{\text{đmax}}=\dfrac{1}{2}mv^2 =|e|U_{\text{AK}}= 4 \cdot 10^{-16}\ \text{J}.
	\end{equation*}
	\item Số electron đập vào đối cathode trong thời gian 1 giây
	
	\begin{equation*}
		N = \dfrac{I \cdot t}{|e|}= \text{6,25} \cdot 10^{16}.
	\end{equation*}
	\item Nhiệt lượng tỏa ra cung cấp cho đối cathode
	\begin{equation*}
		Q=NW \cdot 60 = 1500\ \text{J}.
	\end{equation*}
\end{itemize}      
\begin{center}
\textbf{Câu hỏi tương tự}
\end{center}

Một ống Ronghen hoạt động với hiệu điện thế $U =\text{2,5}\ \text{kV}$. Cường độ dòng điện qua ống là 0,02 A. Số electron phát ra từ Cathode trong mỗi giây là
\begin{mcq}(4)
	\item $ \num{1,52 e16} $.
	\item $ \num{1,25 e16} $.
	\item $ \num{1,25 e17} $.
	\item $ \num{1,52 e17} $.
\end{mcq}

\textbf{Đáp án:} C.
}

\end{dang}
