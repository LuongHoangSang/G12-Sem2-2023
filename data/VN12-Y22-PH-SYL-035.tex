
\chapter[Lý thuyết: Đặc trưng sinh lý của âm;\\Lý thuyết và bài tập: Nhạc âm]{Lý thuyết: Đặc trưng sinh lý của âm;\\Lý thuyết và bài tập: Nhạc âm}
\section{Lý thuyết}
\subsection{Đặc trưng sinh lý của âm}
\subsubsection{Độ cao}
\begin{itemize}
	\item
	Độ cao của âm là một đặc trưng sinh lý của âm gắn liền với tần số âm, không phụ thuộc năng lượng âm.
	\item
	Âm có tần số càng lớn thì nghe càng bổng, âm có tần số càng nhỏ thì nghe càng trầm.
	
	\item Nguồn âm có khối lượng càng lớn thì âm phát ra càng thấp và ngược lại
\end{itemize}
\subsubsection{Độ to}
\begin{itemize}
	\item
	Độ to là một khái niệm nói về đặc trưng sinh lý của âm gắn liền với đặc trưng vật lý là tần số âm, năng lượng âm, cường độ âm,  mức cường độ âm.
	\item
	Âm có mức cường độ âm càng lớn thì nghe càng to.
	
	\item Ngưỡng nghe $I_\text{min}$ là cường độ âm bé nhất mà tai người cảm nhận được. $I_\text{min}$ phụ thuộc vào tần số, tần số càng lớn thì $I_\text{min}$ của mỗi tai người càng bé và ngược lại.
	
	\item Ngưỡng đau $I_\text{max}$ là cường độ âm lớn nhất mà tai người còn chịu được. $I_\text{max}$ không phụ thuộc vào tần số, mọi tần số đều có  $I_\text{max}=10\ \text{W/m}^2$.
	
	\item Miền $I$ nằm giữa  $I_\text{max}$ và  $I_\text{min}$ là miền nghe được.
\end{itemize} 
\subsubsection{Âm sắc}	
\begin{itemize}
	\item Âm sắc là một đặc trưng sinh lý của âm giúp ta phân biệt các âm có cùng tần số do các nguồn khác nhau phát ra. 
	
	\item Âm sắc liên quan mật thiết với đồ thị dao động âm.
\end{itemize} 
\subsection{Âm cơ bản và họa âm}
Khi cho một nhạc cụ phát ra một âm có tần số $f_0$ thì bao giờ nhạc cụ đó cũng đồng thời phát ra một loạt âm có tần số $2f_0$; $3f_0$; $4f_0$... có cường độ khác nhau. Trong đó:
\begin{itemize}
	\item Âm có tần số $f_0$ là âm cơ bản hay họa âm thứ nhất.
	\item Các âm có tần số $2f_0$; $3f_0$; $4f_0$... gọi là các họa âm thứ hai, thứ ba, thứ tư...
\end{itemize}

Biên độ của các họa âm lớn, nhỏ không như nhau, tùy thuộc vào chính nhạc cụ đó.

Tập hợp các họa âm tạo thành phổ của nhạc âm nói trên.
\section{Mục tiêu bài học - Ví dụ minh họa}
\begin{dang}{Liệt kê được các đặc trưng sinh lý của âm}
	
	\viduii{2}{ Đặc trưng sinh lý của âm là 
		\begin{mcq}
			\item mức cường độ âm, cường độ âm.
			\item độ to của âm, độ cao của âm.
			\item tần số âm, độ to của âm.
			\item năng lượng âm, biên độ âm.
		\end{mcq}
	}
	{\begin{center}
			\textbf{Hướng dẫn giải}
		\end{center}
		
		Đặc trưng sinh lý của âm là  độ to của âm, độ cao của âm.
		
		
		\textbf{Đáp án: B.}
	}
	\viduii{2}{Âm do một chiếc đàn bầu phát ra
		
		\begin{mcq}
			\item 	nghe càng cao khi biên độ âm càng lớn.
			\item 	có âm sắc phụ thuộc vào dạng đồ thị dao động của âm.
			\item  	có độ cao phụ thuộc vào hình dạng và kích thước hộp cộng hưởng.
			\item 	nghe càng trầm khi tần số âm càng lớn.
		\end{mcq}
	}
	{\begin{center}
			\textbf{Hướng dẫn giải}
		\end{center}
		
		Âm do một chiếc đàn bầu phát ra có âm sắc phụ thuộc vào dạng đồ thị dao động của âm.
		
		\textbf{Đáp án: B.}
	}
	
	
\end{dang}
\begin{dang}{Liên hệ được giữa đặc trưng sinh lý\\ và đặc trưng vật lý của âm}
	
	
	\viduii{2}{Độ cao của âm là đặc tính sinh lí của âm phụ thuộc vào
		\begin{mcq}(2)
			\item năng lượng âm.
			\item biên độ âm.
			\item vận tốc truyền âm.
			\item tần số âm.
		\end{mcq}
	}
	{\begin{center}
			\textbf{Hướng dẫn giải}
		\end{center}
		Độ cao của âm là đặc tính sinh lí của âm phụ thuộc vào tần số âm.
		
		\textbf{Đáp án: D.}
	}
	\viduii{2}{Chọn phát biểu \textbf{sai} khi nói về các đặc tính sinh lí của âm.
		\begin{mcq}
			\item Âm sắc gắn liền với tần số và mức cường độ âm.
			\item Có 3 đặc tính sinh lí: độ cao, độ to và âm sắc.
			\item Độ cao gắn liền với tần số nhưng không tỉ lệ.
			\item Độ to gắn liền với mức cường độ âm nhưng không tỉ lệ.
		\end{mcq}
	}
	{\begin{center}
			\textbf{Hướng dẫn giải}
		\end{center}
		
		Âm sắc có liên quan mật thiết với đồ thị dao động âm.
		
		\textbf{Đáp án: A.}
	}
\end{dang}
\begin{dang}{Sử dụng được công thức tính tần số\\ của âm cơ bản, tần số họa âm}
	
	\viduii{2}{ Một ống sáo (một đầu kín, một đầu hở) phát ra âm cơ bản là nốt Sol có tần số $\SI{392}{Hz}$. Ngoài âm cơ bản, tần số nhỏ nhất của các họa âm do sáo này phát ra là
		\begin{mcq}(4)
			\item $\SI{784}{Hz}$.
			\item $\SI{1176}{Hz}$.
			\item $\SI{1568}{Hz}$.
			\item $\SI{392}{Hz}$.
		\end{mcq}
	}
	{\begin{center}
			\textbf{Hướng dẫn giải}
		\end{center}
		
		Một ống sáo có một đầu kín, một đầu hở nên tần số để trong ống sáo có sóng dừng là
		
		\begin{equation*}
			f=(2n+1)\dfrac{v}{4L} = (2n+1)f_\text{cb}.
		\end{equation*}
		
		Do vậy, ngoài âm cơ bản, tần số nhỏ nhất của các họa âm do sáo này phát ra là
		
		\begin{equation*}
			f_1 =3f_\text{cb} = \SI{1176}{Hz}.
		\end{equation*}
		
		
		\textbf{Đáp án: B.}
	}
	\viduii{3}{	Một dây đàn có chiều dài 70 cm, khi gảy nó phát ra âm cơ bản có tần số $f$. Người chơi bấm phím đàn cho dây ngắn lại để nó phát ra âm mới có họa âm bậc 3 với tần số 3,5$f$. Chiều dài của dây còn lại là
		\begin{mcq}(4)
			\item $\SI{60}{cm}$.
			\item $\SI{30}{cm}$.
			\item $\SI{10}{cm}$.
			\item $\SI{20}{cm}$.
		\end{mcq}
	}
	{\begin{center}
			\textbf{Hướng dẫn giải}
		\end{center}
		
		Ta có: 
		
		\begin{equation*}
			f =\dfrac{v}{2L}; f'_3 = 3\dfrac{v}{2L'}.
		\end{equation*}
		
		Vì $f'_3 =\text{3,5}f$ nên ta có 
		\begin{equation*}
			3\cdot \dfrac{v}{L'} = \text{3,5} \cdot \dfrac{v}{2L}.
		\end{equation*}
		
		Suy ra 
		\begin{equation*}
			L' = \dfrac{3}{\text{3,5}}\cdot L = \SI{60}{cm}.
		\end{equation*}
		
		Chiều dài của dây còn lại là 10 cm.
		
		\textbf{Đáp án: C.}
	}
	
	
\end{dang}


