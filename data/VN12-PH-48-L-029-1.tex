% --- chapter
\newcommand{\chapter}[2][]{
	\newcommand{\chapname}{#2}
	\begin{flushleft}
		\begin{minipage}[t]{\linewidth}
			\includegraphics[height=1cm]{hdht-logo.png}
			\hspace{0pt}	
			\sffamily\bfseries\large Bài  37. Phóng xạ
			\begin{flushleft}
				\huge\bfseries #1
			\end{flushleft}
		\end{minipage}
	\end{flushleft}
	\vspace{1cm}
	\normalfont\normalsize
}
%-----------------------------------------------------
\chapter[Hiện tượng phóng xạ hạt nhân \\ Các loại tia phóng xạ]{Hiện tượng phóng xạ hạt nhân\\ Các loại tia phóng xạ}

\subsection{Định nghĩa hiện tượng phóng xạ}
Phóng xạ là quá trình phân rã tự phát của một hạt nhân không bền vững (tự nhiên hay nhân tạo). Quá trình phân rã này kèm theo sự tạo ra các hạt và có thể kèm theo sự phát ra các bức xạ điện từ. 

Hạt nhân tự phân rã gọi là hạt nhân mẹ, hạt nhân được tạo thành sau phân rã gọi là hạt nhân con.
\subsection{Các loại tia phóng xạ}
\subsubsection{Tia alpha $\alpha$}					
	\begin{enumerate}[label=\alph*)]
		\item Bản chất
		
		Tia $\alpha$ là dòng hạt nhân $^4_2\text{He}$, mang điện tích dương.
		\item Tính chất
		\begin{itemize}
			\item Vận tốc tia $\alpha$ lớn, khoảng $\SI{2e7}{\meter/\second}$;
			\item Tia $\alpha$ lệch về bản tụ âm khi bay vào giữa hai bản của tụ điện;
			\item Tia $\alpha$ ion hóa chất khí mạnh (đi được $\SI{8}{\centi\meter}$ trong không khí);
			\item Khả năng đâm xuyên của tia $\alpha$ yếu (không xuyên qua được tấm bìa dày cỡ $\SI{1}{\milli\meter}$).
		\end{itemize}
	\end{enumerate}
\subsubsection{Tia beta $\beta$}
	\begin{enumerate}[label=\alph*)]
		\item Bản chất
		Tia $\beta$ gồm hai loại:
		\begin{itemize}
			\item Tia $\beta^-$ hay $^{\ 0}_{-1}e$: là loại tia phổ biến, có bản chất là chùm electron mang điện tích $-e$.
			\item Tia $\beta^+$ hay $^{\ 0}_{+1}e$: hiếm hơn $\beta^-$, bản chất là chùm hạt có khối lượng như electron nhưng mang điện tích $+e$, gọi là các pozitron.
		\end{itemize}
		
		\item Tính chất
		
			\begin{itemize}
				\item Vận tốc của $\beta$ rất lớn, có thể đạt xấp xỉ vận tốc ánh sáng;
				\item Tia $\beta^-$ lệch về bản tụ dương, tia $\beta^+$ lệch về bản tụ âm khi bay vào giữa hai bản của tụ điện;
				\item Tia $\beta$ ion hóa chất khí mạnh, nhưng yếu hơn so với tia $\alpha$ (đi được khoảng vài mét trong không khí);
				\item Khả năng đâm xuyên của tia $\beta$ mạnh hơn tia $\alpha$ (có thể xuyên qua được lá nhôm dày cỡ milimet).			
			\end{itemize}
\end{enumerate}
\subsubsection{Tia gama $\gamma$}
	\begin{enumerate}[label=\alph*)]
		\item Bản chất
		
		Tia $\gamma$ có bản chất là sóng điện từ có bước sóng rất ngắn cũng là hạt photon có năng lượng cao.
		
		Trong phân rã $\alpha$ và $\beta$, hạt nhân con có thể ở trong trạng thái kích thích và phóng xạ tia $\gamma$ để trở về trạng thái cơ bản.
		
		\item Tính chất
		\begin{itemize}
			\item Vận tốc của $\gamma$ bằng vận tốc ánh sáng (vì bản chất tia gamma là chùm photon);
			\item Tia $\gamma$ không bị lệch khi bay vào giữa hai bản của tụ điện;
			\item Khả năng đâm xuyên của tia $\gamma$ lớn hơn nhiều so với tia $\alpha$ và tia $\beta$ (khoảng vài mét trong bê tông và vài centimet trong chì).
		\end{itemize}
		
	\end{enumerate}		