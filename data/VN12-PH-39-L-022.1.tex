% --- chapter
\newcommand{\chapter}[2][]{
	\newcommand{\chapname}{#2}
	\begin{flushleft}
		\begin{minipage}[t]{\linewidth}
			\includegraphics[height=1cm]{hdht-logo.png}
			\hspace{0pt}	
			\sffamily\bfseries\large Bài  30. Hiện tượng quang điện. Thuyết lượng tử ánh sáng
			\begin{flushleft}
				\huge\bfseries #1
			\end{flushleft}
		\end{minipage}
	\end{flushleft}
	\vspace{1cm}
	\normalfont\normalsize
}
%-----------------------------------------------------
\chapter[Hiện tượng quang điện và các định luật quang điện]{Hiện tượng quang điện và các định luật quang điện}

\subsection{Định nghĩa hiện tượng quang điện}
Hiện tượng ánh sáng làm bật các electron ra khỏi mặt kim loại gọi là hiện tượng quang điện.
%\manatip{Quang (ánh sáng), điện (điện tử, electron). Quang điện (ánh sáng làm bật các điện tử, electron).}
\luuy{``Hiện tượng quang điện'' được hiểu là ``hiện tượng quang điện ngoài''.}
\subsection{Định luật về giới hạn quang điện}
Đối với mỗi kim loại, ánh sáng kích thích phải có bước sóng $\lambda$ nhỏ hơn hoặc bằng giới hạn quang điện $\lambda_0$ của kim loại đó, mới gây ra được hiện tượng quang điện.
\begin{equation}
	\lambda \leq \lambda_0
\end{equation}
\luuy{Giới hạn quang điện của mỗi kim loại là đặc trưng riêng của kim loại đó.}
\subsection{Lượng tử năng lượng}
\subsubsection{Giả thuyết Planck về lượng tử năng lượng}
Năng lượng mà mỗi lần một nguyên tử hay phân tử hấp thụ hay phát xạ có giá trị hoàn toàn xác định và bằng $hf$; trong đó $f$ là tần số của ánh sáng bị hấp thụ hay được phát ra, còn $h$ là một hằng số.
\subsubsection{Lượng tử năng lượng}
Lượng tử năng lượng được kí hiệu bằng chữ $\varepsilon$:
\begin{equation}
	\varepsilon=hf,
\end{equation}
trong đó $h=\SI{6.625e-34}{\joule . \second}$ được gọi là hằng số Planck.