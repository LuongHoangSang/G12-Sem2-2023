\whiteBGstarBegin
\begin{enumerate}[label=\bfseries Câu \arabic*:]
	
\item \mkstar{2}
	
	\cauhoi
	{	Hai điện tích đặt gần nhau, nếu giảm khoảng cách giữa chúng đi 2 lần thì lực tương tác giữa 2 vật sẽ 
		\begin{mcq}(4)
			\item tăng lên 2 lần. 
			\item giảm đi 2 lần. 
			\item tăng lên 4 lần. 
			\item giảm đi 4 lần. 
		\end{mcq}
	}
	
	\loigiai
	{
		$F=k\dfrac{q_1q_2}{r^2}$. 
		
		$F'=k\dfrac{q_1q_2}{\left( \text{0,5}r\right)^2}$. 
		
		$F'=4F$.
		
		\textbf{Đáp án: C.}
	}

\item \mkstar{2}
	
	\cauhoi
	{
	Bốn quả cầu kim loại kích thước giống nhau mang điện tích $\text{2,3}\ \mu\text{C},\ -264\cdot 10^{-7}\ \text{C}, -5,9\ \mu\text{C},\ \text{3,6}\cdot 10^{-5}\ \text{C}$. Cho 4 quả cầu đồng thời tiếp xúc nhau sau đó tách chúng ra. Tìm điện tích mỗi quả cầu?
	\begin{mcq}(4)
		\item $\text{0,5}\ \mu\text{C}$.
		\item $\text{1,5}\ \mu\text{C}$.
		\item $\text{2,5}\ \mu\text{C}$.
		\item $\text{3,5}\ \mu\text{C}$.
	\end{mcq}
	}

	\loigiai
	{
		Bảo toàn điện tích: $q'_1=q'_2=q'_3=q'_4=\dfrac{q_1+q_2+q_3+q_4}{4}=\text{1,5}\ \mu\text{C}$.
		
		\textbf{Đáp án: B.}
	}

\item \mkstar{3}
	
	\cauhoi
	{Hai điện tích điểm đặt trong không khí cách nhau 12 cm, lực tương tác giữa chúng bằng 10 N. Đặt chúng vào trong dầu cách nhau 8 cm thì lực tương tác giữa chúng vẫn bằng 10 N. Hằng số điện môi của dầu là
		\begin{mcq}(4)
			\item $\text{1,51}$. 
			\item $\text{2,01}$. 
			\item $\text{2,25}$.
			\item $\text{3,41}$. 
		\end{mcq}
	
	}

	\loigiai
	{$F=k\dfrac{q_1q_2}{r^2}$.
		
		$F'=k\dfrac{q_1q_2}{\varepsilon r'^2}$.
		
		Mà $F'=F$ nên $\varepsilon=\dfrac{r^2}{r'^2}=\text{2,25}$.
		
		\textbf{Đáp án: C.}
	}

\item \mkstar{3}
	
	\cauhoi
	{Hai quả cầu giống nhau mang điện tích có độ lớn như nhau, khi đưa chúng lại gần nhau thì chúng hút nhau. Cho chúng tiếp xúc nhau, sau đó tách chúng ra một khoảng nhỏ thì chúng 
		\begin{mcq}(2)
			\item hút nhau.
			\item đẩy nhau.
			\item có thể hút hoặc đẩy nhau.
			\item không còn tương tác hút hay đẩy.
		\end{mcq}
		\begin{mcq}(4)

		\end{mcq}
	}

	\loigiai
	{	Ta có $q_1=-q_2$.
		
		Bảo toàn điện tích: $q'_1=q'_2=\dfrac{q_1+q_2}{2}=0$.
		
		Do đó, khi tách chúng ra một khoảng nhỏ thì chúng không còn tương tác hút hay đẩy.
		
		\textbf{Đáp án: D.}	
	}

\item \mkstar{4}
	
	\cauhoi
	{Hai quả cầu nhỏ giống nhau, có cùng khối lượng $\text{2,5}\ \text{g}$, điện tích $5\cdot 10^{-7}\ \text{C}$ được treo tại cùng một điểm bằng hai dây mảnh. Do lực đẩy tĩnh điện hai quả cầu tách ra xa nhau một đoạn 60 cm, lấy $g=10\ \text{m/s}^2$. Góc lệch của dây so với phương thẳng đứng là
		
		\begin{mcq}(4)
			\item $\alpha= 4^\circ$.	
			\item  $\alpha= 14^\circ$.	
			\item $\alpha= 24^\circ$.	
			\item  $\alpha= 34^\circ$.	
		\end{mcq}
	
	}

	\loigiai
	{Gọi $\alpha$ là góc hợp bởi dây treo và phương thẳng đứng.
		
		Khi cân bằng: $\vec{P}+\vec{F_\text{đ}}+\vec{T}=0$.
		
		Ta có: $\tan \alpha= \dfrac{F_\text{đ}}{P}=\dfrac{k\cdot\dfrac{|q_1q_2|}{r^2}}{mg}=\dfrac{1}{4}$.
		
		Suy ra $\alpha= 14^\circ$.
		
				
		\textbf{Đáp án: B.}
	}

\item \mkstar{2}
	
	\cauhoi
	{
	
	\begin{mcq}(4)

	\end{mcq}
	}

	\loigiai
	{
		
		\textbf{Đáp án: A.}
	}

\item \mkstar{2}
	
	\cauhoi
	{
	
	\begin{mcq}(1)
		
	\end{mcq}	
	}

	\loigiai
	{
		
		\textbf{Đáp án: C.}
	}

\item \mkstar{2}
	
	\cauhoi
	{
		\begin{mcq}(1)
		
		\end{mcq}
	}

	\loigiai
	{
		
		
		\textbf{Đáp án: C.}
	}

\item \mkstar{2}
	
	\cauhoi
	{
		\begin{mcq}(1)
			
		\end{mcq}
	}

	\loigiai
	{
		
		\textbf{Đáp án: D.}
	}

\item \mkstar{2}
	
	\cauhoi
	{
		\begin{mcq}(2)

		\end{mcq}
	}

	\loigiai
	{
	
		\textbf{Đáp án: D.}
	}

\item \mkstar{2}
	
	\cauhoi
	{
		\begin{mcq}(4)

		\end{mcq}
	}
	\loigiai
	{
		
		\textbf{Đáp án: C.}
	}

\item \mkstar{3}
	
	\cauhoi
	{
		\begin{mcq}(4)

		\end{mcq}
	}

	\loigiai
	{		
		\textbf{Đáp án: A.}
	}

\item \mkstar{3}
	
	\cauhoi
	{
		\begin{mcq}(4)

		\end{mcq}
	}

	\loigiai
	{

		
		\textbf{Đáp án: A.}
	}

\item \mkstar{3}
	
	\cauhoi
	{
		\begin{mcq}(2)

		\end{mcq}
	}

	\loigiai
	{
		
		\textbf{Đáp án: A.}
	}

\item \mkstar{3}
	
	\cauhoi
	{
		\begin{mcq}(2)

		\end{mcq}
	}

	\loigiai
	{
		
		\textbf{Đáp án: B.}
	}

\item \mkstar{3}
	
	\cauhoi
	{
		\begin{mcq}(4)

		\end{mcq}
	}

	\loigiai
	{
		
		\textbf{Đáp án: B.}	
	}

\item \mkstar{3}
	
	\cauhoi
	{
		\begin{mcq}(4)

		\end{mcq}
	}

	\loigiai
	{

		\textbf{Đáp án: A.}
	}

\item \mkstar{4}
	
	\cauhoi
	{
		\begin{mcq}(4)

		\end{mcq}
	
	}

	\loigiai
	{
		
		\textbf{Đáp án: A.}
	}

\item \mkstar{4}
	
	\cauhoi
	{
		\begin{mcq}

		\end{mcq}
	}

	\loigiai
	{
		
		\textbf{Đáp án: B.}
	}

\item \mkstar{4}
	
	\cauhoi
	{
		\begin{mcq}(4)

		\end{mcq}
	}

	\loigiai
	{
		
		\textbf{Đáp án: C.}
	}
\end{enumerate}

\whiteBGstarEnd

\loigiai
{
	\begin{center}
	\textbf{BẢNG ĐÁP ÁN}
\end{center}
\begin{center}
	\begin{tabular}{|m{2.8em}|m{2.8em}|m{2.8em}|m{2.8em}|m{2.8em}|m{2.8em}|m{2.8em}|m{2.8em}|m{2.8em}|m{2.8em}|}
		\hline
		1.C  & 2.A  & 3.B  & 4.B  & 5.C  & 6.A  & 7.C  & 8.C  & 9.D  & 10.D  \\
		\hline
		11.C  & 12.A  & 13.A  & 14.A  & 15.B  & 16.B  & 17.A  & 18.A  & 19.B  & 20.C  \\
		\hline
		
	\end{tabular}
\end{center}
}