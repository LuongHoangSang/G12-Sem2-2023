% --- chapter
\newcommand{\chapter}[2][]{
	\newcommand{\chapname}{#2}
	\begin{flushleft}
		\begin{minipage}[t]{\linewidth}
			\includegraphics[height=1cm]{hdht-logo.png}
			\hspace{0pt}	
			\sffamily\bfseries\large Bài  25. Giao thoa ánh sáng
			\begin{flushleft}
				\huge\bfseries #1
			\end{flushleft}
		\end{minipage}
	\end{flushleft}
	\vspace{1cm}
	\normalfont\normalsize
}
%-----------------------------------------------------
\chapter[Giao thoa ánh sáng đơn sắc]{Giao thoa ánh sáng đơn sắc}

\section{Lý thuyết}

\subsection{Vị trí vân sáng, vân tối - khoảng vân}


\begin{itemize}
	\item Hiệu đường đi của hai sóng kết hợp đến 1 điểm trên màn: 
	\begin{equation}
		d_2-d_1=\dfrac{ax}{D}.
	\end{equation}
	\item Khoảng vân: 
	\begin{equation}
		i=\dfrac{\lambda D}{a}.
	\end{equation}
	\item Vân sáng: 
	\begin{equation}
		d_2-d_1=\dfrac{ax}{D}=k\lambda \Leftrightarrow x=k\dfrac{\lambda D}{a}.
	\end{equation}

Vân sáng trung tâm: $d_2 -d_1=0 \lambda \Leftrightarrow x= 0i$.

Vân sáng bậc 1: $d_2 -d_1=\pm  \lambda \Leftrightarrow x= \pm i$.

Vân sáng bậc 2: $d_2 -d_1=\pm  2\lambda \Leftrightarrow x= \pm 2i$.

...

Vân sáng bậc k: $d_2 -d_1=\pm  k\lambda \Leftrightarrow x= \pm ki$.

	\item Vân tối: 
	\begin{equation}
		d_2-d_1=\dfrac{ax}{D}=\left(k'-\dfrac{1}{2}\right) \lambda \Leftrightarrow x=\left(k'-\dfrac{1}{2}\right) i.
	\end{equation}

Vân tối thứ 1: $d_2 -d_1=\pm (1-\text{0,5})  \lambda \Leftrightarrow x= \pm \text{0,5} i$.

Vân tối thứ 2: $d_2 -d_1=\pm (2-\text{0,5})  \lambda \Leftrightarrow x= \pm \text{1,5} i$.

...

Vân tối thứ k': $d_2-d_1= \pm \left(k'-\dfrac{1}{2}\right) \lambda \Leftrightarrow  x=\pm\left(k'-\dfrac{1}{2}\right) i$.
\end{itemize}

\subsection{Xác định số vân trên trường giao thoa}


Vị trí vân sáng:
\begin{equation}
	x=k\dfrac{\lambda D}{a},\ k=\pm 1; \pm 2,...
\end{equation}

Vị trí vân tối: 
\begin{equation}
	x=\left(k'+\dfrac{1}{2}\right) \dfrac{\lambda D}{a},\ k'=0; \pm 1; \pm 2,... 
\end{equation}
trong đó:
\begin{itemize}
	\item $a$ là khoảng cách giữa hai nguồn kết hợp,
	\item $D$ là khoảng cách từ hai nguồn đến màn,
	\item $\lambda$ là bước sóng ánh sáng.
\end{itemize} 

\section{Mục tiêu bài học - Ví dụ minh họa}

\begin{dang}{Vị trí vân sáng, vân tối - khoảng vân.}
\ppgiai{
\begin{description}
	\item [Bước 1] Xác định số khoảng vân khi biết được bậc của vân sáng hoặc thứ của vân tối. 
	\item [Bước 2] Sử dụng công thức tính khoảng vân để tìm các đại lượng để bài yêu cầu.
\end{description}
}

\viduii{2}{Trong thí nghiệm Y-âng về giao thoa ánh sáng, hai khe được chiếu bằng ánh sáng đơn sắc có bước sóng $\lambda$. Nếu tại điểm M trên màn quan sát có vân tối thứ tư (tính vân sáng trung tâm) thì hiệu đường đi của ánh sáng từ hai khe $\text{S}_1, \text{S}_2$ đến M có độ lớn bằng
\begin{mcq}(4)
		\item 3,5 $\lambda$.
		\item 3 $\lambda$.
		\item 2,5 $\lambda$.
		\item 2 $\lambda$.
	\end{mcq}
	}
	{	\begin{center}
			\textbf{Hướng dẫn giải}
		\end{center}
		
		Nếu tại điểm M trên màn quan sát có vân tối thứ tư (tính vân sáng trung tâm) thì hiệu đường đi:
		
		\begin{equation*}
		d_2 - d_1 = (4-\text{0,5})\lambda = \text{3,5} \lambda.
		\end{equation*}
		
		
		\begin{center}
		\textbf{Câu hỏi tương tự}
		\end{center}

Trong thí nghiệm Y-âng về giao thoa ánh sáng, hai khe được chiếu bằng ánh sáng đơn sắc có bước sóng $ \lambda $. Nếu tại điểm $ M $ trên màn quan sát có vân tối thứ ba (tính từ vân trung tâm) thì hiệu đường đi của ánh sáng từ hai khe đến $ M $ có độ lớn bằng
\begin{mcq}(4)
	\item $ 3\lambda $.
	\item $ 2\lambda $.
	\item $ \num{1,5}\lambda $.
	\item $ \num{2,5}\lambda $.
\end{mcq}
	
	
	\textbf{Đáp án:} D.
	}

\viduii{2}
	{
		Trong thí nghiệm Y-âng về giao thoa ánh sáng, các khe hẹp được chiếu sáng bởi ánh sáng đơn sắc. Khoảng vân trên màn là $\text{1,2}\ \text{mm}$. Trong khoảng giữa hai điểm M và N trên màn ở cùng một phía so với vân sáng trung tâm, cách vân trung tâm lần lượt $2 \text{mm}$ và $\text{4,5}\ \text{mm}$, quan sát được
		\begin{mcq}(2)
			\item 2 vân sáng và 2 vân tối.		
			\item 3 vân sáng và 2 vân tối.
			\item 2 vân sáng và 3 vân tối.		
			\item 2 vân sáng và 1 vân tối.
		\end{mcq}
	}
	{
		\begin{center}
			\textbf{Hướng dẫn giải}
		\end{center}
		
		\begin{itemize}
			
			\item Vì hai điểm M và N trên màn ở cùng một phía so với vân sáng trung tâm nên có thể chọn $x_{\text{M}}= 2\ \text{mm}$ và $x_{\text{N}}= \text{4,5} \text{mm}$.
			
			\item Số vân sáng:
			\begin{equation*}
			x_{\text{M}} \leq  ki \leq x_{\text{N}} \Rightarrow 2 \leq \text{1,2} k \leq \text{4,5} \Rightarrow \text{1,67}  \leq k \leq \text{3,75} \Rightarrow k = 2, 3. 
			\end{equation*}
			\item Số vân sáng:
			\begin{equation*}
			x_{\text{M}} \leq  (m+\text{0,5}) \leq x_{\text{N}} \Rightarrow 2 \leq \text{1,2} (m+\text{0,5}) \leq \text{4,5} \Rightarrow \text{1,17}  \leq m \leq \text{3,25} \Rightarrow k = 2, 3. 
			\end{equation*}
			
			Suy ra có thể quan sát được 2 vân sáng và 2 vân tối.
		\end{itemize}		
		
		\begin{center}
			\textbf{Câu hỏi tương tự}
		\end{center}
		
Trong thí nghiệm Y-âng về giao thoa ánh sáng, các khe hẹp được chiếu sáng bởi ánh sáng đơn sắc. Khoảng vân trên màn là $ \SI{1,4}{mm} $. Trong khoảng giữa $ M $ và $ N $ trên màn ở cùng một phía so với vân sáng trung tâm lần lượt $ \SI{2}{mm} $ và $ \SI{4,5}{mm} $, quan sát được	
\begin{mcq}(2)
			\item 2 vân sáng và 2 vân tối.		
			\item 3 vân sáng và 2 vân tối.
			\item 2 vân sáng và 3 vân tối.		
			\item 2 vân sáng và 1 vân tối.
		\end{mcq}	
		\textbf{Đáp án: A.}
	}
\end{dang}

\begin{dang}{Xác định số vân trên trường giao thoa.}
\ppgiai{
Với $L$ là bề rộng trường giao thoa. 

\begin{description}
	
	\item [Bước 1] Tìm khoảng vân $i$. 
	\item [Bước 2] Tính tỉ số $y=\dfrac{L}{2i}$.
	\item [Bước 3] Xác định số vân sáng hoặc số vân tối theo quy tắc:
	
	Số vân sáng $=2[y] +1$, với $[y]$ có nghĩa là lấy phần nguyên của $y$.
		
	Số vân tối $=2\{y\}$, trong đó $\{y\}$ có nghĩa là làm tròn số đến số nguyên gần nhất của $y$.	
\end{description}
}

\viduii{2}
	{Trong một thí nghiệm về Giao thoa ánh sáng bằng khe I âng với ánh sáng đơn sắc $\lambda  = \text{0,7}\ \mu \text{m}$, khoảng cách giữa 2 khe $S_1, S_2$ là $a = \text{0,35}\ \text{mm}$ , khoảng cách từ 2 khe đến màn quan sát là $D = 1\ \text{m}$, bề rộng của vùng có giao thoa là $\text{13,5}\ \text{mm}$. Số vân sáng, vân tối quan sát được trên màn là
		\begin{mcq}(2)
			\item 7 vân sáng, 6 vân tối.        
			\item 6 vân sáng, 7 vân tối.
			\item 6 vân sáng, 6 vân tối.          
			\item 7 vân sáng, 7 vân tối.
		\end{mcq}
	}
	{
		\begin{center}
			\textbf{Hướng dẫn giải}
		\end{center}
		
		
		\begin{itemize}
			\item Khoảng vân: 
			\begin{equation*}
			i=\dfrac{\lambda D}{a}=\dfrac{\text {0,7} \cdot 10^{-6} \cdot 1}{\text{0,35} \cdot 10^{-3}}= 2 \cdot 10^{-3}\ \text{m}.
			\end{equation*}
			\item Xét: $\dfrac{\text{13,5}\cdot 10^{-3}}{2} : (2 \cdot 10^{-3}) =\text{3,375}$.
			\item Số vân sáng: $2 \cdot 3 + 1 = 7$.
			\item Số vân tối: $2 \cdot 3 =6$. 
			
		\end{itemize}
		
		\begin{center}
			\textbf{Câu hỏi tương tự}
		\end{center}
		
Trong một thí nghiệm về giao thoa ánh sáng bằng khe Young với ánh sáng đơn sắc $ \SI{0,74}{\mu m} $, khoảng cách giữa hai khe $ S_{1} S_{2} $ là $ a = \SI{2}{mm} $, khoảng cách từ hai khe đến màn quan sát là $ D = \SI{1}{m} $. Bề rộng của vùng giao thoa là $ \SI{8}{mm} $. Số vân sáng, vân tối quan sát được trên màn là
\begin{mcq}(2)
	\item 11 vân sáng, 10 vân tối.        
	\item 22 vân sáng, 20 vân tối.
	\item 21 vân sáng, 22 vân tối.          
	\item 22 vân sáng, 22 vân tối.
\end{mcq}
		\textbf{Đáp án: C.}
	}
\end{dang}






