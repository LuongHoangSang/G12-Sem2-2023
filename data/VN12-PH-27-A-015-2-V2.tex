% --- chapter
\newcommand{\chapter}[2][]{
	\newcommand{\chapname}{#2}
	\begin{flushleft}
		\begin{minipage}[t]{\linewidth}
			\includegraphics[height=1cm]{hdht-logo.png}
			\hspace{0pt}	
			\sffamily\bfseries\large Bài  20. Mạch dao động
			\begin{flushleft}
				\huge\bfseries #1
			\end{flushleft}
		\end{minipage}
	\end{flushleft}
	\vspace{1cm}
	\normalfont\normalsize
}
%-----------------------------------------------------
\chapter[Bài toán ghép tụ - tụ xoay]{Bài toán ghép tụ - tụ xoay}
\section{Lý thuyết}

\subsection{Công thức ghép tụ}

Ghép thêm tụ điện

Mạch dao động $LC_1$ có chu kỳ $T_1$, tần số . Mạch dao động $LC_2$ có chu kỳ $T_2$, tần số $f_2$

+ Trường hợp 1: $C_1$ mắc nối tiếp với $C_2$. Khi đó:

$$\dfrac{1}{C}= \dfrac{1}{C_1}+ \dfrac{1}{C_2}.$$

$$f_\text{nt}^2=f_1^2+f_2^2.$$

$$\dfrac{1}{T_\text{nt}}= \dfrac{1}{T^2_1}+ \dfrac{1}{T^2_2}.$$

+ Trường hợp 2: $C_1$ mắc song song với $C_2$. Khi đó:


$$C= C_1 + C_2.$$

$$T_\text{ss}^2=T_1^2+T_2^2.$$

$$\dfrac{1}{f_\text{ss}^2}= \dfrac{1}{f^2_1}+ \dfrac{1}{f^2_2}.$$





\subsection{Công thức tụ xoay}

Thường trong mạch có tụ xoay:

- Nếu có $n$ lá thì có $n - 1$ tụ điện phẳng mắc song song.

- Điện dung của tụ phẳng

$$C = \dfrac{S_\text{max}}{k4\pi d}.$$

- Điện dung của tụ điện sau khi quay các lá 1 góc $\alpha$:

 Từ giá trị cực đại:
 
 $$C_{\text V{\alpha}}= C_\text{V\text{max}} - \dfrac{C_\text{V\text{max}}- C_\text{V\text{min}}}{180^\circ}\alpha.$$
 
 Từ giá trị cực tiểu:

$$C_{\text V{\alpha}}= C_\text{V\text{min}} - \dfrac{C_\text{V\text{max}}- C_\text{V\text{min}}}{180^\circ}\beta.$$


\section{Mục tiêu bài học - Ví dụ minh họa}
\begin{dang}{Bài toán ghép tụ điện.}
	\viduii{3}
		{Cho một mạch dao động điện từ gồm một cuộn dây thuần cảm và hai tụ điện có điện dung $C_1 = \SI{2}{nF}$ và $C_2 = \SI{6}{nF}$ mắc song song với nhau. Mạch có tần số là $\SI{4000}{Hz}$. Nếu  tháo dời khỏi mạch tụ điện thứ hai thì mạch còn lại dao động với tần số
		\begin{mcq}(4)
			\item $\SI{2000}{Hz}$. 
			\item $\SI{4000}{Hz}$. 
			\item $\SI{8000}{Hz}$. 
			\item $\SI{16000}{Hz}$. 
		\end{mcq}
	}
	{	\begin{center}
			\textbf{Hướng dẫn giải}
		\end{center}
		
		Điện dung của tụ điện mắc song song cho bởi
		
		$$C=C_{1}+C_{2}= \SI{8}{nF}.$$
		
		Chu kì của mạch khi mắc song song hai tụ là
		
		$$T=\dfrac{1}{f}= \SI{2,5 e-4}{s}.$$
		
		Khi tháo dời khỏi mạch tụ điện $C_{2}$ mạch chỉ còn tụ điện $C_{1}$. Chu kì của mạch chỉ chứa tụ $C_{1}$ và mạch chứa tụ $C$ cho bởi
		
		$$T_{1} =2 \pi \sqrt{L C_{1}}, T =2 \pi \sqrt{L C}.$$
		
		Lập tỉ lệ hai biểu thức trên, ta được:
		
		$$\dfrac{T_{1}}{T}=\sqrt{\dfrac{C_{1}}{C}}\Rightarrow T_{1}= \SI{1,25 e4}{s}.$$
		
		Tần số dao động của mạch chỉ chứa $C_{1}$ là
		
		$$f_{1}=\dfrac{1}{T_{1}}= \SI{8000}{Hz}.$$	
		
		\textbf{Đáp án: C.}
		
		\begin{center}
			\textbf{Câu hỏi tương tự}
		\end{center}
		
	Trong mạch dao động, khi mắc tụ điện có điện dung $C_1$ với cuộn cảm $L$ thì tần số dao động của mạch là $f_1 = \SI{60}{kHz}$. Khi mắc tụ có điện dụng $C_2$ với cuộn cảm $L$ thì tần số dao động của mạch là $f_2 = \SI{80}{kHz}$. Khi mắc $C_1$ song song với $C_2$ rồi mắc vào cuộn cảm $L$ thì tần số dao động của mạch là bao nhiêu?

	\textbf{Đáp án:} $f= \SI{48}{kHz}.$ 
	}
		\viduii{3}{
		
		Một mạch dao động $LC$ lí tưởng gồm cuộn cảm thuần có độ tự cảm không đổi, tụ điện có điện dung $C$ thay đổi. Khi $C = C_1$ thì tần số dao động riêng của mạch là $\SI{7,5}{MHz}$ và khi $C = C_2$ thì tần số dao động riêng của mạch là $\SI{10}{MHz}$. Nếu $C = C_1 + C_2$ thì tần số dao động riêng của mạch là
		\begin{mcq}(4)
			\item $\SI{6,0}{MHz}$. 
			\item $\SI{17,5}{MHz}$. 
			\item $\SI{12,5}{MHz}$. 
			\item $\SI{2,5}{MHz}$. 
		\end{mcq}
		
	}
	{	\begin{center}
			\textbf{Hướng dẫn giải}
		\end{center}
		
		Tần số dao động riêng của mạch cho bởi
		
		$$f=\dfrac{1}{2 \pi \sqrt{L C}} \Rightarrow C \sim \dfrac{1}{f^{2}}.$$
		
		Khi đó biểu thức $C=C_{1}+C_{2}$ trở thành
		
		$$\dfrac{1}{f^{2}}=\dfrac{1}{f_{1}^{2}}+\dfrac{1}{f_{2}^{2}} \Rightarrow  f = \SI{6}{MHz}.$$
		
		\textbf{Đáp án: A.}
		
		\begin{center}
			\textbf{Câu hỏi tương tự}
		\end{center}
		
		Một mạch dao động điện từ có cuộn cảm không đổi $L$. Nếu thay tụ điện $C$ bởi các tụ điện $C_1$, $C_2$. Nếu $C_1$ nối tiếp $C_2$, $C_1$ song song $C_2$ thì chu kì dao động riêng của mạch lần lượt là $T_1$, $T_2$, $T_\text{nt} = \SI{4,8}{\mu s}$, $T_\text{ss} = \SI{10}{\mu s}$. Hãy xác định $T_1$, biết $T_1 >T_2$.
		
		\textbf{Đáp án:} $T_1 = \SI{8}{\mu s}$.
	}

\end{dang}
\begin{dang}{Bài toán tụ xoay.}
	\viduii{3}{
		
		Một mạch dao động gồm một cuộn dây có độ tự cảm $L = \SI{1,5}{mH}$ và một tụ xoay có điện dung biến thiên từ $C_1 = \SI{50}{pF}$ đến $C_2 = \SI{450}{pF}$ khi một trong hai bản tụ xoay từ $0^\circ$ đến $180^\circ$. Biết điện dung của tụ phụ thuộc vào góc xoay theo hàm bậc nhất. Để mạch thu được sóng điện từ có bước sóng $\SI{1200}{m}$ cần xoay bản động của tụ điện một góc bằng bao nhiêu kể từ vị trí mà tụ điện có điện dung cực đại? Cho $\pi^2 = 10$.
		\begin{mcq}(4)
			\item $99^\circ$. 
			\item $81^\circ$. 
			\item $121^\circ$. 
			\item $108^\circ$. 
		\end{mcq}
		
	}
	{	\begin{center}
			\textbf{Hướng dẫn giải}
		\end{center}
		
		Bước sóng của một mạch thu sóng:
		
		$$\lambda=2 \pi c \sqrt{L C}.$$
		
		Khi mạch thu được bước sóng $1200 \mathrm{m}$, ta có:
		
		$$\lambda=2 \pi c \sqrt{L C}  \Rightarrow C = \SI{270}{pF}.
		$$
		Công thức tụ xoay
		 
		$$\dfrac{C-C_{1}}{C_{2}-C_{1}}=\dfrac{\alpha-\alpha_{1}}{\alpha_{2}-\alpha_{1}}  \Rightarrow \alpha=99^{\circ}.$$
		
		\textbf{Đáp án: A.}
		
		\begin{center}
			\textbf{Câu hỏi tương tự}
		\end{center}
		
		Mạch chọn sóng của một máy thu thanh gồm cuộn dây thuần cảm có độ tự cảm $L = \SI{2,9}{\mu H}$ và tụ điện có điện dung $C = \SI{490}{pF}$. Để máy thu được dải sóng từ $\lambda_\text{m} = \SI{10}{m}$ đến $\lambda_\text M = \SI{50}{m}$, người ta ghép thêm một tụ xoay $C_\text V$ biến thiên từ $C_\text m = \SI{10}{pF}$ đến $C_\text M = \SI{490}{pF}$. Muốn mạch thu được sóng có bước sóng $\lambda = \SI{20}{m}$, thì phải xoay các bản di động của tụ $C_\text V$ từ vị trí ứng với điện dung cực đại $C_\text M$ một góc $\alpha$ là bao nhiêu?
		
		\textbf{Đáp án:} $\alpha = 168^\circ.$
	}
	\viduii{3}{
		
		Tụ xoay có điện dung biến thiên liên tục và tỉ lệ thuận với góc quay từ giá trị $C_1 = \SI{10}{pF}$ đến $C_2 = \SI{370}{pF}$ tương ứng khi góc quay của các bản tụ tăng dần từ $0^\circ$ đến $180^\circ$. Tụ điện được mắc với một cuộn dây có hệ số tự cảm $L = \SI{2}{\mu H}$ để tạo thành mạch chọn sóng của máy thu. Để thu được sóng có bước sóng $\lambda = \SI{18,84}{m}$ phải xoay tụ một góc bằng bao nhiêu kể từ khi tụ có điện dung nhỏ nhất.
		\begin{mcq}(2)
		\item $\alpha = 90^\circ$.             
		\item $\alpha = 20^\circ$.
		\item $\alpha = 120^\circ$.              
		\item $\alpha = 30^\circ$.
		\end{mcq}
		
	}
	{	\begin{center}
			\textbf{Hướng dẫn giải}
		\end{center}
		
		Khi tụ xoay từ $0^\circ$ đến $180^\circ$ thì $C$ tăng từ $C_1 = \SI{10}{pF}$ đến $C_2 = \SI{370}{pF}$
		
		Tụ xoay thêm $1^\circ$ thì $C$ tăng thêm một lượng 
		
		$$\dfrac{(370-10)}{180} = \SI{2}{pF}.$$
		
		Lại có 
		
		$$\lambda = cT = c2 \pi \sqrt{LC}.$$
		
		$$\Rightarrow C = \dfrac{\lambda^2}{4\pi^2 c^2 L} = \SI{50}{pF}.$$
		
		Điện dung của tụ cần tăng 
		
		$$\Delta C = 50 - 10 = \SI{40}{pF}.$$
		
		Tụ cần xoay một góc 
		
		$$\alpha = \dfrac{40}{2} = 20^\circ.$$
		
		\textbf{Đáp án: B.}
		
		\begin{center}
			\textbf{Câu hỏi tương tự}
		\end{center}
		
		Một tụ điện xoay có điện dung tỉ lệ thuận với góc quay các bản tụ. Tụ có giá trị điện dung $C$ biến đổi giá trị $C_1 = \SI{10}{pF}$ đến $C_2 = \SI{490}{pF}$ ứng với góc quay của các bản tụ là $\alpha$ các bản tăng dần từ $0^\circ$ đến $180^\circ$. Tụ điện được mắc với một cuộn dây có hệ số tự cảm $L = \SI{2}{\mu H}$ để làm thành mạch dao động ở lối vào của 1 một máy thu vô tuyến điện. Để bắt được sóng $\SI{19,2}{m}$ phải quay các bản tụ một góc $\alpha$ là bao nhiêu tính từ vị trí điện dung $C$ bé nhất.
		
		\textbf{Đáp án:} $\alpha = \text{15,7}^\circ.$
	}
	
\end{dang}
