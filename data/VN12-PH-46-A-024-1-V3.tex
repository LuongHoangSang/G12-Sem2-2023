
\chapter[Độ hụt khối của hạt nhân]{Độ hụt khối của hạt nhân}
\section{Lý thuyết}

Khi một hạt nhân $^A_Z X$ tạo thành từ $Z$ hạt proton và $N$ hạt nơtron thì tổng khối lượng của hạt nuclon riêng rẽ tạo thành hạt nhân là
\begin{equation}
	m_0=Z m_p+N m_n=Z m_p+ (A-Z) m_n.
\end{equation}
trong đó: $m_p$ và $m_n$ lần lượt là khối lượng của các proton và nơtron.

Tuy nhiên, các phép đo chính xác đã chứng tỏ rằng, khối lượng của một hạt nhân luôn nhỏ hơn tổng khối lượng của các nuclon tạo thành hạt nhân đó. Độ chênh lệch giữa hai khối lượng đó được gọi là \textbf{độ hụt khố}i của hạt nhân, kí hiệu là $\Delta m$:
\begin{equation}
	\Delta m = m_0 - m_X = Z m_p+ (A-Z) m_n - m_X,
\end{equation}
trong đó:
\begin{itemize}
	\item $m_0=Z m_p+ (A-Z) m_n$ là tổng khối lượng các nuclon lúc đầu chưa liên kết;
	\item $m_X$ khối lượng của hạt nhân $^A_Z X$;
	\item $m_p$ và $m_n$ lần lượt là khối lượng của các proton và nơtron.
\end{itemize}

\luuy{Khối lượng nguyên tử (ở trạng thái trung hòa):
	\begin{equation}
		m_\text{nt}=m_X+Z m_e.
	\end{equation}
}

\section{Mục tiêu bài học - Ví dụ minh họa}

\begin{dang}{Tìm khối lượng của hạt nhân}
	
	\ppgiai{
		Sử dụng công thức
		\begin{equation*}
			m_X = Z m_p+ (A-Z) m_n - \Delta m.
		\end{equation*}
	}
	
	\vidu{2}{ [THPT QG 2019 - Mã đề 201] Hạt nhân $^9_4\text{Be}$ có độ hụt khối là $\SI{0,0627}{\atomicmassunit}$. Cho khối lượng của proton và nơtron lần lượt là $\SI{1,0073}{\atomicmassunit}$ và $\SI{1,0087}{\atomicmassunit}$. Khối lượng của $^9_4\text{Be}$ là
		\begin{mcq}(4)
			\item $\SI{9,0086}{\atomicmassunit}$.
			\item $\SI{9,0068}{\atomicmassunit}$.
			\item $\SI{9,0020}{\atomicmassunit}$.
			\item $\SI{9,0100}{\atomicmassunit}$.
	\end{mcq}}
	{\begin{center}
			\textbf{Hướng dẫn giải}
		\end{center}
		
		Khối lượng của $^9_4\text{Be}$ là
		\begin{equation*}
			m_\text{Be}= Z m_p+ (A-Z) m_n - \Delta m = 4\cdot \SI{1,0073}{\atomicmassunit} + 5 \cdot \SI{1,0087}{\atomicmassunit} - \SI{0,0627}{\atomicmassunit} = \SI{9,0100}{\atomicmassunit}.
		\end{equation*}
		
		\begin{center}
			\textbf{Câu hỏi tương tự}
		\end{center}
		
		Hạt nhân $ ^7_{3} \text{Li} $ có khối lượng $ \SI{7,0144}{u} $. Khối lượng của proton và neutron lần lượt là $ \SI{1,0073}{u} $ và $ \SI{1,0087}{u} $. Độ hụt khối của hạt nhân là $ ^7_{3} \text{Li} $ là
		\begin{mcq}(4)
			\item $ \SI{0,0401}{u} $.
			\item $ \SI{0,0457}{u} $.
			\item $ \SI{0,0359}{u} $.
			\item $ \SI{0,0423}{u} $.
		\end{mcq}
		\textbf{Đáp án:} D.}
	
\end{dang}

\begin{dang}{Tìm độ hụt khối của hạt nhân.}
	
	\ppgiai{
		Sử dụng công thức
		\begin{equation*}
			\Delta m = Z m_p+ (A-Z) m_n - m_X.
		\end{equation*}
	}
	
	\viduii{2}{ [THPT QG 2015 - Mã đề 138] Cho khối lượng của hạt nhân $^{107}_{\ 47}\text{Ag}$ là $\SI{106,8783}{\atomicmassunit}$; của nơtron là $\SI{1,0087}{\atomicmassunit}$; của proton là $\SI{1,0073}{\atomicmassunit}$. Độ hụt khối của hạt nhân $^{107}_{\ 47}\text{Ag}$ là 
		\begin{mcq}(4)
			\item $\SI{0,9868}{\atomicmassunit}$.
			\item $\SI{0,9868}{\atomicmassunit}$.
			\item $\SI{0,6868}{\atomicmassunit}$.
			\item $\SI{0,9686}{\atomicmassunit}$.
	\end{mcq}}
	{\begin{center}
			\textbf{Hướng dẫn giải}
		\end{center}
		
		Độ hụt khối của hạt nhân $^{107}_{\ 47}\text{Ag}$ là 
		\begin{equation*}
			\Delta m = Z m_p+ (A-Z) m_n - m_\text{Ag} = 47\cdot \SI{1,0073}{\atomicmassunit} + 60 \cdot \SI{1,0087}{\atomicmassunit} - \SI{106,8783}{\atomicmassunit} = \SI{0,9868}{\atomicmassunit}.
		\end{equation*}
		
		\begin{center}
			\textbf{Câu hỏi tương tự}
		\end{center}
		
		[THPT QG 2018 - Mã đề 209] Hạt nhân $^7_3\text{Li}$ có khối lượng $\SI{7,0144}{\atomicmassunit}$. Cho khối lượng của proton và nơtron lần lượt là $\SI{1,0073}{\atomicmassunit}$ và $\SI{1,0087}{\atomicmassunit}$. Độ hụt khối của hạt nhân $^7_3\text{Li}$ là
		\begin{mcq}(4)
			\item $\SI{0,0457}{\atomicmassunit}$.
			\item $\SI{0,0423}{\atomicmassunit}$.
			\item $\SI{0,0359}{\atomicmassunit}$.
			\item $\SI{0,0401}{\atomicmassunit}$.
		\end{mcq}
		
		\textbf{Đáp án:} B.}
	
	\viduii{2}
	{Khối lượng của nguyên tử nhôm $ \text{Al}_{13}^{27} $ là $ \SI{26,9803}{u} $. Khối lượng của nguyên tử $ H_{1}^{1} $ là $ \SI{1,007825}{u} $, khối lượng của proton là $ \SI{1,00728}{u} $ và khối lượng của neutron là $ \SI{1,00866}{u} $. Độ hụt khối của hạt nhân nhôm là
		\begin{mcq}(2)
			\item $ \SI{0,242665}{u} $.
			\item $ \SI{0,23558}{u} $.
			\item $ \SI{0,023548}{u} $.
			\item $ \SI{0,023544}{u} $.
		\end{mcq}
	}
	{
		\begin{center}
			\textbf{Hướng dẫn giải}
		\end{center}
		Ta có:
		$$
		\Delta m = 13m_{H} + 14m_{n} - m_{Al} = \SI{0,242665}{u}.
		$$
		\begin{center}
			\textbf{Câu hỏi tương tự}
		\end{center}
		
		Xét đồng vị Côban $ \text{Co}_{27}^{60} $ hạt nhân có khối lượng $ m_{Co} = \SI{59,934}{u} $. Biết khối lượng của các hạt $ m_{p} = \SI{1,007276}{u} $ và $ m_{n} = \SI{1,008665}{u} $. Độ hụt khối của hạt nhân đó là
		\begin{mcq}(4)
			\item $ \SI{0,401}{u} $.
			\item $ \SI{0,302}{u} $.
			\item $ \SI{0,548}{u} $.
			\item $ \SI{0,544}{u} $.
		\end{mcq}
		\textbf{Đáp án:} C.
	}
	
\end{dang}

\section{Bài tập tự luyện}
\begin{enumerate}[label=\bfseries Câu \arabic*:]
	\item \mkstar{1} [12]
	\cauhoi
	{Các hạt nhân có cùng độ hụt khối thì
		\begin{mcq}
			\item năng lượng liên kết như nhau. 
			\item cùng năng lượng liên kết riêng.
			\item có năng lượng liên kết riêng lớn nếu số khối lớn.
			\item có cùng khối lượng. 
		\end{mcq}
	}
	
	\loigiai
	{		\textbf{Đáp án: A.}
		
		Các hạt nhân có cùng độ hụt khối thì năng lượng liên kết như nhau:
		$$E_\text{lk} = \Delta m c^2$$
		
	}
	
	\item \mkstar{1} [5]
	\cauhoi
	{Gọi $m_p$, $m_n$, $m_X$ lần lượt là khối lượng của hạt proton, nơtron và hạt nhân $\ce{^A_Z X}$. Độ hụt khối khi các nuclon ghép lại tạo thành hạt nhân $\ce{^A_Z X}$ là $\Delta m$ được tính bằng biểu thức
		\begin{mcq}(2)
			\item $\Delta m = Z m_p + (A-Z) m_n - m_X$. 
			\item $\Delta m = Z m_p + (A-Z) m_n + Am_X$. 
			\item $\Delta m = Z m_p + (A-Z) m_n + m_X$.
			\item $\Delta m = Z m_p + (A-Z) m_n - Am_X$.
		\end{mcq}
	}
	
	\loigiai
	{		\textbf{Đáp án: A.}
		
		Độ hụt khối khi các nuclon ghép lại tạo thành hạt nhân $\ce{^A_Z X}$ là $\Delta m$ được tính bằng biểu thức $\Delta m = Z m_p + (A-Z) m_n - m_X$. 
		
	}
	\item \mkstar{1} [7]
	\cauhoi
	{Đơn vị nào dưới đây \textbf{không phải} đơn vị khối lượng hạt nhân?
		\begin{mcq}(4)
			\item $\SI{}{kg}$. 
			\item $\SI{}{MeV/c^2}$.
			\item $\SI{}{u}$.
			\item $\SI{}{MeV/c}$.
		\end{mcq}
	}
	
	\loigiai
	{		\textbf{Đáp án: D.}
		
		Đơn vị khối lượng hạt nhân gồm $\SI{}{kg}$, $\SI{}{MeV/c^2}$, $\SI{}{u}$. Đơn vị $\SI{}{MeV/c}$ không phải là đơn vị khối lượng hạt nhân.
		
	}
	
	\item \mkstar{3} [5]
	\cauhoi
	{Một hạt nhân $\ce{^5_3 Li}$ có năng lượng liên kết bằng $\SI{26.3}{MeV}$. Biết khối lượng proton $m_p=\SI{1.0073}{u}$, khối lượng nơtron $m_n=\SI{1.0087}{u}$, $\SI{1}{u}=\SI{931}{MeV/c^2}$. Khối lượng nghỉ của hạt nhân $\ce{^5_3 Li}$ bằng
		\begin{mcq}(4)
			\item $\SI{5.0111}{u}$.
			\item $\SI{5.0675}{u}$.
			\item $\SI{4.7179}{u}$.
			\item $\SI{4.6916}{u}$.
		\end{mcq}
	}
	
	\loigiai
	{		\textbf{Đáp án: A.}
		
		Độ hụt khối của hạt nhân $\ce{^5_3 Li}$:
		$$E_\text{lk} = \Delta m c^2 \Rightarrow \Delta m = \dfrac{\SI{26.3}{MeV}}{\SI{931}{MeV/c^2}} = \SI{0.02825}{u}$$
		
		Khối lượng nghỉ của hạt nhân:
		$$\Delta m = Z m_p + (A-Z)m_n - m_X \Rightarrow m_X = \SI{5.0111}{u}$$
		
	}
	
	\item \mkstar{3} [1]
	\cauhoi
	{Theo thuyết tương đối, một vật có khối lượng nghỉ $\SI{100}{g}$ chuyển động với tốc độ $\SI{2.4e8}{m/s}$ thì có động năng bằng
		\begin{mcq}(4)
			\item $\SI{5.76e15}{J}$.
			\item $\SI{2.88e15}{J}$.
			\item $\SI{6e15}{J}$.
			\item $\SI{15e15}{J}$.
		\end{mcq}
	}
	
	\loigiai
	{		\textbf{Đáp án: C.}
		
		Động năng bằng hiệu giữa $E$ và $E_0$:
		$$W_\text{đ} = E-E_0 = mc^2 - m_0 c^2 = m_0 c^2 \left(\dfrac{1}{\sqrt{1-\dfrac{v^2}{c^2}}-1}\right) = \SI{6e15}{J}$$
		
	}
	
	
\end{enumerate}
