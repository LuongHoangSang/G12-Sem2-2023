\chapter{Luyện tập: Tán sắc ánh sáng}
\begin{enumerate}
	\item
	{
		Gọi $n_\text{đ}$, $n_\text{t}$ và $n_\text{v}$ lần lượt là chiết suất của một môi trường trong suốt đối với các ánh sáng đơn sắc đỏ, tím và vàng. Sắp xếp nào sau đây đúng?
		\begin{mcq}(2)
			\item{$n_\text{đ}<n_\text{v}<n_\text{t}$.}
			\item{$n_\text{v}>n_\text{đ}>n_\text{t}$.}
			\item{$n_\text{đ}>n_\text{t}>n_\text{v}$.}
			\item{$n_\text{t}>n_\text{đ}>n_\text{v}$.}
		\end{mcq}
	}
	\item
	{
		Bước sóng của ánh sáng đỏ trong không khí là $\SI{0.64}{\micro \meter}$. Tính bước sóng của ánh sáng đỏ trong nước biết chiết suất của nước đối với ánh sáng đỏ là $\dfrac{4}{3}$.
		\begin{mcq}(4)
			\item{$\SI{0.24}{\micro \meter}$.}
			\item{$\SI{0.48}{\micro \meter}$.}
			\item{$\SI{0.36}{\micro \meter}$.}
			\item{$\SI{0.54}{\micro \meter}$.}
		\end{mcq}
	}
	\item
	{
		Một lăng kính có góc chiết quang là $\ang{60}$. Biết chiết suất của lăng kính đối với ánh sáng đỏ là $\SI{1.5}{}$. Chiếu tia sáng màu đỏ vào mặt bên của lăng kính với góc tới $\ang{60}$. Góc lệch của tia ló và tia tới là
		\begin{mcq}(4)
			\item{$\ang{\SI{60.0}{}}$.}
			\item{$\ang{\SI{40.0}{}}$.}
			\item{$\ang{\SI{38.8}{}}$.}
			\item{$\ang{\SI{42.1}{}}$.}
		\end{mcq}
	}
	\item
	{
		Chiếu một tia sáng đơn sắc màu vàng từ không khí (chiết suất coi như bằng 1 đối với mọi ánh sáng) vào mặt phẳng phân cách của một khối chất rắn trong suốt với góc tới $\ang{60}$ thì thấy tia phản xạ trở lại không khí vuông góc với tia khúc xạ đi vào khối chất rắn. Chiết suất của chất rắn trong suốt đó đối với ánh sáng màu vàng là
		\begin{mcq}(4)
			\item{$1$.}
			\item{$\sqrt 5$.}
			\item{$\sqrt 2$.}
			\item{$\sqrt 3$.}
		\end{mcq}	
	}
	\item
	{
		Một tia sáng trắng chiếu tới mặt bên của một lăng kính thuỷ tinh tam giác đều. Tia ló màu vàng qua lăng kính có góc lệch cực tiểu. Biết chiết suất của lăng kính đối với ánh sáng vàng, ánh sáng tím lần lượt là $n_\text{v} = \SI{1.50}{}; n_\text{t}=\SI{1.52}{}$. Góc tạo bởi tia ló màu vàng và tia ló màu tím có giá trị xấp xỉ bằng
		\begin{mcq}(4)
			\item{$\ang{\SI{0.77}{}}$.}
			\item{$\ang{\SI{48.59}{}}$.}
			\item{$\ang{\SI{4.46}{}}$.}
			\item{$\ang{\SI{1.73}{}}$.}
		\end{mcq}	
	}
	\item
	{
		Một lăng kính có góc chiết quang $\ang{\SI{6.0}{}}$ (coi là góc nhỏ) được đặt trong không khí. Chiếu một chùm ánh sáng trắng song song, hẹp vào mặt bên của lăng kính theo phương vuông góc với mặt phẳng phân giác của góc chiết quang, rất gần cạnh của lăng kính. Đặt một màn ảnh E sau lăng kính, vuông góc với phương của chùm tia tới và cách mặt phẳng phân giác của góc chiết quang $\SI{1.2}{\meter}$. Chiết suất của lăng kính đối với ánh sáng đỏ là $n_\text{đ}=\SI{1.642}{}$ và đối với ánh sáng tím là $n_\text{t}=\SI{1.685}{}$. Độ rộng từ màu đỏ đến màu tím của quang phổ liên tục quan sát được trên màn là
		\begin{mcq}(4)
			\item{$\SI{5.4}{\milli \meter}$.}
			\item{$\SI{36.9}{\milli \meter}$.}
			\item{$\SI{4.5}{\milli \meter}$.}
			\item{$\SI{10.1}{\milli \meter}$.}
		\end{mcq}	
	}
	\item
	{
		Phát biểu nào trong các phát biểu dưới đây là đúng khi nói về hiện tượng tán sắc ánh sáng và ánh sáng đơn sắc?
		\begin{mcq}(1)
			\item{Hiện tượng tán sắc ánh sáng là hiện tượng khi qua lăng kính, chùm ánh sáng trắng không những là bị lệch về phía đáy mà còn bị tách ra thành chiều chùm sáng có màu sắc khác nhau.}
			\item{Mỗi ánh sáng đơn sắc có một màu nhất định.}
			\item{Trong quang phổ của ánh sáng trắng có vô số các ánh sáng đơn sắc khác nhau.}
			\item{Cả A, B, C đều đúng.}
		\end{mcq}	
	}
	\item
	{
		Phát biểu nào sau đây là \textbf{không đúng}?
		\begin{mcq}(1)
			\item{Ánh sáng trắng là tập hợp của vô số các ánh sáng đơn sắc có màu biến đổi liên tục từ đỏ đến tím.}
			\item{Chiết suất của chất làm lăng kính đối với các ánh sáng đơn sắc là khác nhau.}
			\item{Ánh sáng đơn sắc không bị tán sắc khi đi qua lăng kính.}
			\item{Khi chiếu một chùm ánh sáng Mặt Trời đi qua một cặp hai môi trường trong suốt thì tia tím bị lệch về phía mặt phân cách hai môi trường nhiều hơn tia đỏ.}
		\end{mcq}	
	}
	\item
	{
		Ánh sáng trắng hợp bởi
		\begin{mcq}(2)
			\item{bảy màu đơn sắc.}
			\item{vô số màu đơn sắc.}
			\item{các màu đơn sắc từ đỏ đến tím.}
			\item{Tất cả đều đúng.}
		\end{mcq}	
	}
	\item
	{
		Đặc trưng cho sóng ánh sáng đơn sắc là
		\begin{mcq}(2)
			\item{màu sắc.}
			\item{tần số sóng.}
			\item{vận tốc truyền sóng.}
			\item{chiết suất lăng kính đối với ánh sáng đó.}
		\end{mcq}	
	}
	\item
	{
		Khi ánh sáng trắng bị tán sắc thì
		\begin{mcq}(2)
			\item{màu đỏ lệch nhiều nhất.}
			\item{màu tím lệch nhiều nhất.}
			\item{màu tím lệch ít nhất.}
			\item{ánh sáng trắng tách ra thành 7 màu.}
		\end{mcq}	
	}
\end{enumerate}
\textbf{Đáp án}
\begin{center}
	\begin{tabular}{|m{2.8em}|m{2.8em}|m{2.8em}|m{2.8em}|m{2.8em}|m{2.8em}|m{2.8em}|m{2.8em}|m{2.8em}|m{2.8em}|}
		\hline
		1. A & 2. D & 3. C & 4. D & 5.  & 6. A & 7. D & 8. D & 9. B & 10. B \\
		\hline
		11. B &&&&&&&&& \\
		\hline
	\end{tabular}
\end{center}