\begin{enumerate}[label=\bfseries Câu \arabic*:]
\item \mkstar{1}

\cauhoi{ 
	
	Trên một dây có hiện tượng sóng dừng thì 
	
	
	\begin{mcq}
		\item tất cả các điểm trên dây đều chuyển động với cùng một tốc độ.
		\item tất cả các điểm trên dây đều dao động với biên độ cực đại.
		\item tất cả phần tử trên dây đều đứng yên.
		\item xuất hiện trên dây có những bụng sóng xen kẽ với nút sóng.
	\end{mcq}
	
}
\loigiai{\textbf{Đáp án: D.}

Trên một dây có hiện tượng sóng dừng thì xuất hiện trên dây có những bụng sóng xen kẽ với nút sóng.
}	
	\item \mkstar{2}

\cauhoi{
	
	Chọn câu trả lời \textbf{đúng}. Người ta nói sóng dừng là một trường hợp đặc biệt của giao thoa sóng vì
	
	\begin{mcq}
		\item sóng dừng là sự giao thoa của các sóng trên cùng một phương truyền sóng.  
		\item sóng dừng là sự chồng chất của các sóng trên cùng một phương truyền sóng.
		\item sóng dừng xảy ra khi có sự giao thoa của sóng tới và sóng phản xạ trên cùng một phương truyền sóng. 
		\item sóng dừng là sự giao thoa của các sóng trên cùng một phương truyền sóng. 
	\end{mcq}
	
}

\loigiai{
	\textbf{Đáp án: C.}
	
	Người ta nói sóng dừng là một trường hợp đặc biệt của giao thoa sóng vì sóng dừng xảy ra khi có sự giao thoa của sóng tới và sóng phản xạ trên cùng một phương truyền sóng. 
}

\item \mkstar{2}

\cauhoi{
	
	Một sợi dây đàn hồi AB có chiều dài 120 cm, có đầu B cố định, đầu A được gắn với một bản rung tần số $f$. Trên dây có sóng dừng với 4 bụng sóng. Biên độ tại bụng là 5 cm. Tại điểm C trên dây gần B nhất có biên độ dao động là 2,5 cm. Hỏi CB có giá trị bao nhiêu?
	\begin{mcq}(4)
		\item 7,5 cm.
		\item 5 cm.
		\item 35 cm.
		\item 25 cm.
	\end{mcq}
	
}

\loigiai{
	\textbf{Đáp án B.}
	
	Ta có $4\dfrac{\lambda}{2}=120 \Rightarrow \lambda =\SI{60}{cm}$.
	
	Số bụng gần nút B nhất cách B một khoảng $\dfrac{\lambda}{4} =\SI{15}{cm}$.
	
	$\text{CB} =\SI{5}{cm}$.
}
\item \mkstar{2}

\cauhoi{
	
	Một sợi dây mảnh đàn hồi AB dài 2,5 m được căng theo phương ngang, trong đó đầu B cố định, đầu A được rung nhờ một dụng cụ để tạo sóng dừng trên dây. Tần số rung $f$ có thể thay đổi được giá trị trong khoảng 93 Hz đến 100 Hz. Biết tốc độ truyền sóng trên dây là 24 m/s. Hỏi tần số $f$ phải nhận giá trị nào dưới đây để trên dây có sóng dừng?
	\begin{mcq}(4)
		\item 94 Hz.
		\item 96 Hz.
		\item 98 Hz.
		\item 100 Hz.
	\end{mcq}
	
}

\loigiai{
	\textbf{Đáp án B.}
	
	Ta có sóng dừng 2 đầu cố định $l=k\dfrac{\lambda}{2} = \dfrac{kv}{2f} \Rightarrow f = \dfrac{kv}{2l} = \text{9,6} k.$
	
	Theo đề bài $93 <f<100 \Rightarrow \text{9,68} <k<\text{10,41}$.
	
	$k$ là số nguyên nên $k=10$, $f=\SI{96}{Hz}$.
}
\item \mkstar{2}

\cauhoi{
	
	Một sóng cơ truyền trên một sợi dây rất dài thì một điểm M trên sợi dây có vận tốc dao động biến thiên theo phương trình $v_\text{M}= 20\pi \cos (10\pi t + \varphi)\ \text{cm/s}$. Giữ chặt một điểm trên dây sao cho trên dây hình thành sóng dừng, khi đó bề rộng một bụng sóng có độ lớn là
	\begin{mcq}(4)
		\item 8 cm.
		\item 6 cm.
		\item 10 cm.
		\item 4 cm.
	\end{mcq}
	
}

\loigiai{
	\textbf{Đáp án A.}
	
	Ta có phương trình vận tốc $v_\text{M}= 20\pi \cos (10\pi t + \varphi)\ \text{cm/s}$.
	
	Suy ra $A=\SI{2}{cm}$.
	
	Bề rộng bụng sóng là $4A =\SI{8}{cm}$.
}
\item \mkstar{2}

\cauhoi{
	
	Một sợi dây đàn hồi dài $\SI{90}{cm}$ có một đầu cố định và một đầu tự do đang có sóng dừng. Kể cả đầu cố định, trên dây có 8 nút. Biết rằng khoảng thời gian giữa 6 lần liên tiếp sợi dây duỗi thẳng là $\SI{0.25}{\second}$. Tốc độ truyền sóng trên dây là
	\begin{mcq} (4)
		\item $\SI{1.2}{\meter/\second}$.
		\item $\SI{2.9}{\meter/\second}$.
		\item $\SI{2.4}{\meter/\second}$.
		\item $\SI{2.6}{\meter/\second}$.
	\end{mcq}
	
}

\loigiai{
	\textbf{Đáp án: C.}
	
	Khoảng thời gian giữa 6 lần liên tiếp sợi dây duỗi thẳng là $\SI{0.25}{\second}$ nên
	\begin{equation*}
		5\cdot\dfrac{T}{2}=\SI{0.25}{\second}\Rightarrow T=\SI{0.1}{\second}.
	\end{equation*}
	
	Điều kiện để có sóng dừng trên một sơi dây có một đầu cố định, một đầu tự do là chiều dài của sợi dây phải bằng một số lẻ lần $\dfrac{\lambda}{4}$
	\begin{equation*}
		l=(2k+1)\dfrac{\lambda}{4},\qquad(k=0,1,2,3,\ldots),
	\end{equation*}
	trong đó: $k$ là số bó sóng nguyên.
	
	Dây dài $\SI{90}{cm}$, gồm một đầu cố định và một đầu tự do. Kể cả đầu cố định, trên dây có 8 nút nên trên dây có 7 bó sóng nguyên
	\begin{equation*}
		l=(2\cdot 7 +1)\cdot\dfrac{\lambda}{4}=\SI{0.9}{\meter}\Rightarrow\lambda=\SI{0.24}{\meter}.
	\end{equation*}
	
	Tốc độ truyền sóng trên dây là
	\begin{equation*}
		v=\dfrac{\lambda}{T}=\dfrac{\SI{0.24}{\meter}}{\SI{0.1}{\second}}=\SI{2.4}{\meter/\second}.
	\end{equation*}
	
	
}
\item \mkstar{2}

\cauhoi{
	
	Trên một sợi dây đàn hồi dài có sóng dừng với bước sóng $\SI{1.4}{cm}$. Trên dây có hai điểm A và B cách nhau $\SI{5.6}{cm}$. Biết A, B là một bụng. Số nút sóng và bụng sóng trên đoạn dây AB là
	\begin{mcq} (2)
		\item 8 nút và 9 bụng.
		\item 9 nút và 8 bụng.
		\item 9 nút và 10 bụng.
		\item 10 nút và 9 bụng.
	\end{mcq}	
}

\loigiai{
	\textbf{Đáp án A.}
	
	Ta có: $\text{AB}=4\lambda=8\cdot\dfrac{\lambda}{2}.$
	
	Vì $k=8$ nên trên dây có 8 bó sóng nguyên và A,B là bụng sóng nên trên dây AB có 8 nút sóng và 9 bụng sóng.
}
\item \mkstar{2}

\cauhoi{
	
	Trên một sợi dây đàn hồi dài có sóng dừng với bước sóng $\SI{1.2}{cm}$. Trên dây có hai điểm A và B cách nhau $\SI{6.1}{cm}$, tại A là một nút sóng. Số nút sóng và bụng sóng trên đoạn dây AB là
	\begin{mcq}(2)
		\item 10 nút và 9 bụng.
		\item 9 nút và 10 bụng.
		\item 10 nút và 11 bụng.
		\item 11 nút và 10 bụng.
	\end{mcq}
	
}

\loigiai{
	\textbf{Đáp án D.}
	
	Ta có: $\text{AB}=10\cdot\dfrac{\lambda}{2}+0,1.$
	
	Vì $k=10$ nên trên AB có 10 bó sóng nguyên mà $\SI{0,1}{cm}<\dfrac{\lambda}{4}$ nên trên đoạn dây AB ta có 11 nút sóng và 10 bụng sóng.
	
}

\item \mkstar{2}

\cauhoi{
	
	Thực hiện thí nghiệm sóng dừng trên sợi dây có hai đầu cố định có chiều dài 90 cm. Tần số của nguồn sóng là 10 Hz thì thấy trên dây có 2 bụng sóng. Xác định vận tốc truyền sóng trên dây
	
	\begin{mcq}(4)
		\item 9 m/s.
		\item 8 m/s.
		\item 4,5 m/s.
		\item 90 cm/s.
	\end{mcq}
	
}

\loigiai{
	\textbf{Đáp án A.}
	
	$l =k\dfrac{\lambda}{2} \Rightarrow \lambda =\SI{90}{cm}$.
	
	$v=\lambda f =9\ \text{m/s}$.
}
\item \mkstar{2}

\cauhoi{
	
	Một sợi dây đàn hồi, hai tần số liên tiếp có sóng dừng trên dây là 50 Hz và 70 Hz. Hãy xác định tần số nhỏ nhất có sóng dừng trên dây.
	\begin{mcq}(4)
		\item 10 Hz.
		\item 20 Hz.
		\item 30 Hz.
		\item 40 Hz.
	\end{mcq}
	
}

\loigiai{
	\textbf{Đáp án B.}
	
	Tần số $f=\dfrac{kv}{2l} \Rightarrow f_{k+1} -f_k = f_\text{min} = \SI{20}{Hz}$.
	
}
\item \mkstar{2}

\cauhoi{
	
	Một dây AB dài 90 cm có đầu B thả tự do. Tạo ở đầu A một dao động điều hòa ngang có tần số 100 Hz ta có sóng dừng, trên dây có 4 múi nguyên. Vận tốc truyền sóng trên dây có giá trị bao nhiêu?
	\begin{mcq}(4)
		\item 20 m/s.
		\item 40 m/s.
		\item 30 m/s.
		\item 60 m/s.
	\end{mcq}
	
}

\loigiai{
	\textbf{Đáp án B.}
	
	Ta có: $l = \left(k+\dfrac{1}{2}\right) \dfrac{\lambda}{2}$.
	
	$k=4$ suy ra $\lambda =\SI{40}{cm}$.
	
	Suy ra vận tốc $v=\lambda f = 40\ \text{m/s}$.
	
}
\item \mkstar{3}

\cauhoi{
	
	Một sợi dây CD dài 1 m, đầu C cố định, đầu D gắn với cần rung với tần số thay đổi được. D được coi là nút sóng. Ban đầu trên dây có sóng dừng. Khi tần số tăng thêm 20 Hz thì số nút trên dây tăng thêm 7 nút. Sau khoảng thời gian bằng bao nhiêu sóng phản xạ từ C truyền hết một lần chiều dài sợi dây?
	\begin{mcq}(4)
		\item 0,175 s.
		\item 0,07 s.
		\item 0,5 s.
		\item 1,2 s.
	\end{mcq}
	
}

\loigiai{
	\textbf{Đáp án A.}
	
	Ta có $l=k\dfrac{\lambda}{2} = \dfrac{kv}{2f} = \dfrac{(k+7)v}{f+20} \Rightarrow \dfrac{7}{k} =\dfrac{20}{f} \Rightarrow k = \dfrac{7f}{20} \Rightarrow l=\dfrac{7v}{40}$.
	
	Thời gian sóng truyền hết sợi dây
	
	$t =\dfrac{l}{v}=\SI{0,175}{s}$.
}

\item \mkstar{3}

\cauhoi{
	
	Một sợi dây đàn hồi căng thẳng đứng đầu dưới cố định đầu trên gắn với một nhánh của âm thoa dao động với tần số 12 Hz, thấy trên dây xảy ra sóng dừng với 7 nút sóng. Thả cho đầu dưới của dây tự do để trên dây vẫn xảy ra sóng dừng với 7 nút sóng thì tần số của âm thoa phải
	\begin{mcq}(2)
		\item tăng lên 1 Hz.
		\item giảm xuống 1 Hz.
		\item giảm xuống 1,5 Hz.
		\item tăng lên 1,5 Hz.
	\end{mcq}
	
}

\loigiai{
	\textbf{Đáp án A.}
	
	Khi 2 đầu cố định $l = 6 \dfrac{v}{2\cdot 12} (1)$.
	
	Khi 1 đầu cố định $l=\text{6,5} \dfrac{v}{2f}(2)$.
	
	Chia (1) cho (2) ta được $f=\SI{13}{Hz}$.
	
	Vậy phải tăng thêm 1 Hz.
}
\item \mkstar{3}

\cauhoi{
	
	Trên dây AB xảy ra sóng dừng. Đầu A gắn vào 1 âm thoa, đầu B để tự do. Chiều dài dây $L$. Quan sát trên dây thấy có 5 bụng sóng. Tổng độ dài của các phần tử dây dao động ngược pha với điểm B là 
	\begin{mcq}(4)
		\item $\dfrac{5L}{9}$.
		\item $\dfrac{5L}{4}$.
		\item $\dfrac{4L}{9}$.
		\item $\dfrac{4L}{5}$.
	\end{mcq}
	
}

\loigiai{
	\textbf{Đáp án C.}
	
	Đối với 1 đầu cố định 1 đầu tự do $\text{4,5} \dfrac{\lambda}{2} = L \Rightarrow \dfrac{\lambda}{2} =\dfrac{L}{\text{4,5}}$.
	
	Các phần tử dây dao động ngược pha với B có tổng độ dài bằng
	
	$2\cdot \dfrac{\lambda}{2} = \dfrac{2L}{\text{4,5}} =\dfrac{4L}{9}$.
}


\item \mkstar{3}

\cauhoi{
	
	Một sợi dây đàn hồi dài 1 m được treo lơ lửng lên một cần rung, cần có thể rung theo phương ngang với tần số thay đổi được từ 100 Hz đến 120 Hz. Vận tốc truyền sóng trên dây là 8 m/s. Trong quá trình thay đổi tần số rung của cần, có thể tạo ra được bao nhiêu lần sóng dừng trên dây với số bụng khác nhau?
	\begin{mcq}(4)
		\item 7.
		\item 4.
		\item 5.
		\item 6.
	\end{mcq}
	
}

\loigiai{
	\textbf{Đáp án C.}
	
	Dây treo lơ lững tức 1 đầu cố định, 1 đầu tự do.
	
	$l=(2k+1) \dfrac{\lambda}{4} = (2k+1) \dfrac{v}{4f}$.
	
	$\Rightarrow 100 \leq \dfrac{(2k+1)v}{4l} \leq 120.$
	
	$\Rightarrow \text{24,5} \leq k \leq \text{29,5}$.
	
	Có 5 giá trị của $k$ thỏa mãn hay có thể tạo ra được 5 lần sóng dừng.
}
\item \mkstar{3}

\cauhoi{
	
	Một sợi dây đàn hồi căng ngang, đang có sóng dừng ổn định. Khoảng thời gian giữa hai lần liên tiếp sợi dây duỗi thẳng là 0,1 s, tốc độ truyền sóng trên dây là 3 m/s. Khoảng cách giữa hai điểm gần nhau nhất trên sợi dây dao động cùng pha và có biên độ dao động bằng một nửa biên độ của bụng sóng là
	\begin{mcq}(4)
		\item 10 cm.
		\item 8 cm.
		\item 20 cm.
		\item 30 cm.
	\end{mcq}
	
}

\loigiai{
	\textbf{Đáp án C.}
	
	Khoảng thời gian giữa hai lần liên tiếp sợi dây duỗi thẳng là $\dfrac{T}{2} =\SI{0,1}{s} \Rightarrow T=\SI{0,2}{s} \Rightarrow \lambda =vT=\SI{0,6}{m}$.
	
	Khoảng cách từ một nút đến vị trí có biên độ bằng nửa biên độ cực đại $\dfrac{\lambda}{12}$.
	
	Hai điểm gần nhau nhất trên sợi dây dao động cùng pha suy ra 2 điểm thuộc cùng 1 bó sóng. 
	
	Khoảng cách cần tìm $\dfrac{\lambda}{2} - \dfrac{\lambda}{12}- \dfrac{\lambda}{12} =\dfrac{\lambda}{3} = \SI{20}{cm}$.
}
\item \mkstar{3}

\cauhoi{
	
	Trên một sợi dây có sóng dừng, điểm bụng M cách nút gần nhất N một đoạn 10 cm, khoảng thời gian giữa hai lần liên tiếp trung điểm P của đoạn MN có cùng li độ với điểm M là 0,1 s. Tốc độ truyền sóng trên dây là
	\begin{mcq}(4)
		\item 400 cm/s.
		\item 200 cm/s.
		\item 100 cm/s.
		\item 300 cm/s.
	\end{mcq}
	
}

\loigiai{
	\textbf{Đáp án B.}
	
	Bụng cách nút gần nhất $\dfrac{\lambda}{4} \Rightarrow \lambda = \SI{40}{cm}$.
	
	Khoảng thời gian giữa hai lần liên tiếp trung điểm P của đoạn MN có cùng li độ với M là:
	
	$\Delta t =\dfrac{T}{2} =\SI{0,1}{s} \Rightarrow T =\SI{0,2}{s}$.
	
	Tốc độ truyền sóng trên dây là 
	
	$v=\dfrac{\lambda}{T} = 2\ \text{m/s}$.
}
\item \mkstar{3}

\cauhoi{
	
	Trên một sợi dây dài 16 cm được tạo ra sóng dừng nhờ nguồn có biên độ 4 mm. Biên độ không đổi trong quá trình truyền sóng. Người ta đếm được trên sợi dây có 22 điểm dao động với biên độ 6 mm. Biết hai đầu sợi dây là hai nút. Số nút và bụng sóng trên dây là
	\begin{mcq}(2)
		\item 22 bụng và 23 nút.
		\item 8 bụng và 9 nút.
		\item 11 bụng và 12 nút.
		\item 23 bụng và 22 nút.
	\end{mcq}
	
}

\loigiai{
	\textbf{Đáp án C.}
	
	Trên mỗi bó có 2 điểm dao động với biên độ là 6 cm.
	
	Trên dây có 11 bó sóng.
	
	Trên dây có 11 bụng và 12 nút.
}
\item \mkstar{4}

\cauhoi{
	
	Một sợi dây đang có sóng dừng ổn định. Sóng truyền trên dây có tần số $\SI{10}{\hertz}$ và bước sóng $\SI{6}{\centi\meter}$. Trên dây, hai phần tử M và N có vị trí cân bằng cách nhau $\SI{8}{\centi\meter}$, M thuộc một bụng sóng dao
	động điều hoà với biên độ $\SI{6}{\milli\meter}$. Lấy $\pi^2=10$. Tại thời điểm $t$, phần tử M đang chuyển động với tốc độ $6\pi\,\text{cm/s}$ thì phần tử N chuyển động với gia tốc có độ lớn là
	\begin{mcq} (4)
		\item $6\sqrt{3}\,\text{m/}\text{s}^2$.
		\item $6\,\text{m/}\text{s}^2$.
		\item $6\sqrt{2}\,\text{m/}\text{s}^2$.
		\item $3\,\text{m/}\text{s}^2$.
	\end{mcq}
	
}

\loigiai{
	\textbf{Đáp án A.}
	
	Tần số góc của dao động là
	\begin{equation*}
		\omega=2\pi f=2\pi\cdot \SI{10}{\hertz} =20\pi\,\text{rad/s}.
	\end{equation*}
	
	Biên độ dao động của N là
	\begin{equation*}
		A_\text{M}= A_\text{b}\left|\cos\left( \dfrac{2\pi d}{\lambda}\right)\right|
		=\SI{0,006}{\meter}\cdot\left|\cos\left( \dfrac{2\pi\cdot \SI{0,08}{\meter}}{\SI{0,06}{\meter}}\right)\right|=\SI{0,003}{\meter}.
	\end{equation*}
	
	Tại thời điểm M chuyển động với tốc độ $6\pi\,\text{cm/s}$ thì đang có li độ là
	\begin{equation*}
		\left(\dfrac{x_\text{M}}{A_\text{M}}\right)^2+\left(\dfrac{v_\text{M}}{v_\textrm{M max}}\right)^2=1
		\Rightarrow \left(\dfrac{x}{\SI{0,006}{\meter}}\right)^2+\left(\dfrac{6\pi\cdot10^{-2}\,\text{m/s}}{20\pi\,\text{rad/s}\cdot \SI{0,006}{\meter}}\right)^2=1
		\Rightarrow x_\text{M}=\dfrac{3\sqrt{3}}{1000}\,\text{m}.
	\end{equation*}
	
	Gia tốc của điểm N có thể được suy ra bằng cách lập tỉ số giữa gia tốc của M và gia tốc của N
	\begin{equation*}
		\left|\dfrac{a_\text{N}}{a_\text{M}}\right| = \left|\dfrac{a_\text{N}}{\omega^2 x_\text{M}}\right| = \dfrac{A_\text{N}}{A_\text{M}}
		\Rightarrow \left|\dfrac{a_\text{N}}{(20\pi\,\text{rad/s})^2\cdot \dfrac{3\sqrt{3}}{1000}\,\text{m}}\right| = \dfrac{\SI{0,003}{\meter}}{\SI{0,006}{\meter}}
		\Rightarrow \left|a_\text{N}\right|=6\sqrt{3}\,\text{m/}\text{s}^2.
	\end{equation*}
}
\end{enumerate}
\loigiai{\textbf{Đáp án}
	\begin{center}
		\begin{tabular}{|m{2.8em}|m{2.8em}|m{2.8em}|m{2.8em}|m{2.8em}|m{2.8em}|m{2.8em}|m{2.8em}|m{2.8em}|m{2.8em}|}
			\hline
			1. D & 2. C & 3. A & 4. B & 5. B & 6. A  & 7. C  & 8. A & 9. D & 10. A\\
			\hline
			11. B & 12. B & 13. A & 14. A & 15. C & 16. C  & 17. C  & 18. B & 19. C & 20. A\\
			\hline
		\end{tabular}
\end{center}}