\whiteBGstarBegin
\setcounter{section}{0}
\section{Lý thuyết: Hiện tượng quang điện và các định luật quang điện}
\begin{enumerate}[label=\bfseries Câu \arabic*:]
	
%=======================================
\item \mkstar{1} [1]

\cauhoi
{Kim loại nào sau đây có giới hạn quang điện thuộc vùng ánh sáng nhìn thấy
	\begin{mcq}(4)
		\item Zn. 
		\item Cu. 
		\item Ag. 
		\item Ca. 
	\end{mcq}
}

\loigiai
{		\textbf{Đáp án: D.}
	
	Kim loại kiềm có giới hạn quang điện thuộc vùng ánh sáng nhìn thấy
}

%=======================================
\item \mkstar{1} [15]

\cauhoi
{Giới hạn quang điện của kim loại
	\begin{mcq}(1)
		\item là bước sóng dài nhất của ánh sáng kích thích gây ra hiện tượng quang điện. 
		\item không phụ thuộc vào bản chất kim loại. 
		\item phụ thuộc vào ánh sáng kích thích chiếu vào kim loại. 
		\item tỉ lệ với cường độ chùm sáng thích hợp chiếu vào kim loại. 
	\end{mcq}
}

\loigiai
{		\textbf{Đáp án: A.}
	
	Giới hạn quang điện của kim loại là bước sóng dài nhất của ánh sáng kích thích gây ra hiện tượng quang điện. 
}

%=======================================
\item \mkstar{1} [2]

\cauhoi
{Giới hạn quang điện của các kim loại kiềm như Ca, Na, K, Cs nằm trong 
	\begin{mcq}(1)
		\item vùng tử ngoại và ánh sáng nhìn thấy.
		\item vùng tử ngoại. 
		\item vùng hồng ngoại.
		\item ánh sáng nhìn thấy được. 
	\end{mcq}
}

\loigiai
{		\textbf{Đáp án: D.}
	
	Giới hạn quang điện của các kim loại kiềm như Ca, Na, K, Cs nằm trong  ánh sáng nhìn thấy được.
}

%=======================================
\item \mkstar{1} [2]

\cauhoi
{Công thoát electron của một kim loại là $A = \SI{7,64e-19}{J}$. Chiếu lần lượt vào tấm kim loại này các bước sóng lần lượt là $\lambda_{1} = \SI{0,18}{\mu m}$, $\lambda_{2} = \SI{0,21}{\mu m}$, $\lambda_{3} = \SI{0,35}{\mu m}$. Bức xạ nào gây ra được hiện tượng quang điện đối với tấm kim loại đó?
	\begin{mcq}(1)
		\item Hai bức xạ $\lambda_{1}$ và $\lambda_{2}$. 
		\item Không có bức xạ nào trong ba bức xạ trên. 
		\item Cả ba bức xạ trên. 
		\item Chỉ có bức xạ $\lambda_{1}$. 
	\end{mcq}
}

\loigiai
{		\textbf{Đáp án: A.}
	
	Giới hạn quang điện của tấm kim loại trên là
	$$
	\lambda_{0} = \dfrac{hc}{A} = \SI{0,26}{\mu m}.
	$$
	Để gây ra được hiện tượng quang điện thì
	$$
	\lambda \leq \lambda_{0}
	$$
	Vậy chỉ có bức xạ $\lambda_{1}$ và $\lambda_{2}$.
}

%=======================================
\item \mkstar{1} [2]

\cauhoi
{Một kim loại có giới hạn quang điện là $\SI{0,3}{\mu m}$. Cho hằng số Plack $h = \SI{6,625e-34}{J.s}$ và tốc độ ánh sáng trong chân không $c = \SI{3e8}{m/s}$. Năng lượng cần thiết để bứt electron ra khỏi bề mặt kim loại là
	\begin{mcq}(4)
		\item $\SI{4,64}{eV}$. 
		\item $\SI{4,14}{eV}$. 
		\item $\SI{4,41}{eV}$. 
		\item $\SI{6,625}{eV}$. 
	\end{mcq}
}

\loigiai
{		\textbf{Đáp án: B.}
	
	Năng lượng cần thiết để bức electron ra khỏi bề mặt kim loại chính là công thoát bằng
	$$
	A = \dfrac{hc}{\lambda_{0}} = \SI{4,14}{eV}.
	$$
}

%=======================================
\item \mkstar{1} [2]

\cauhoi
{Theo thuyết lượng tử ánh sáng, phát biểu nào sau đây \textbf{không} đúng?
	\begin{mcq}(1)
		\item Photon có thể chuyển động hay đứng yên, phụ thuộc vào nguồn sáng chuyển động hay đứng yên. 
		\item Trong chân không, photon bay với tốc độ $\SI{3e8}{m/s}$ dọc theo các tia sáng. 
		\item Ánh sáng được tạo thành từ các hạt gọi là photon. 
		\item Năng lượng photon ánh sáng đỏ nhỏ hơn năng lượng photon ánh sáng tím. 
	\end{mcq}
}

\loigiai
{		\textbf{Đáp án: A.}
	
	Photon luôn chuyển động.
}

%=======================================
\item \mkstar{1} [2]

\cauhoi
{Chọn câu \textbf{đúng} khi nói về photon
	\begin{mcq}(2)
		\item Photon có năng lượng. 
		\item Photon có điện tích. 
		\item Photon có khối lượng. 
		\item Photon có kích thước xác định. 
	\end{mcq}
}

\loigiai
{		\textbf{Đáp án: A.}
	
	Photon là hạt mang năng lượng.
}

%=======================================
\item \mkstar{1} [3]

\cauhoi
{Theo thuyết lượng tử ánh sáng, phát biểu nào sau đây là \textbf{sai}?
	\begin{mcq}(1)
		\item Photon chỉ tồn tại ở trạng thái chuyển động. Không có photon đứng yên. 
		\item Ánh sáng được tạo thành bởi các hạt gọi là photon. 
		\item Năng lượng của các photon ứng với các đơn sắc khác nhau là như nhau. 
		\item Trong chân không, các photon bay dọc theo các tia sáng với tộc độ $\SI{3e8}{m/s}$. 
	\end{mcq}
}

\loigiai
{		\textbf{Đáp án: C.}
	
	Năng lượng của các photon ứng với các đơn sắc khác nhau là khác nhau. 
}

%=======================================
\item \mkstar{1} [13]

\cauhoi
{Giới hạn quang điện của bạc, đồng, kẽm lần lượt là $\SI{0,26}{\mu m}$, $\SI{0,30}{\mu m}$, $\SI{0,35}{\mu m}.$ Khi đó giới hạn quang điện của hợp kim bạc, đồng, kẽm sẽ là
	\begin{mcq}(4)
		\item $\SI{0,26}{\mu m}$. 
		\item $\SI{0,40}{\mu m}$. 
		\item $\SI{0,30}{\mu m}$. 
		\item $\SI{0,35}{\mu m}$. 
	\end{mcq}
}

\loigiai
{		\textbf{Đáp án: A.}
	
	Giới hạn quang điện của hợp kim là giới hạn quang điện nhỏ nhất.
}

%=======================================
\item \mkstar{1} [4]

\cauhoi
{Để gây được hiện tượng quang điện ngoài, bức xạ chiếu vào kim loại phải có
	\begin{mcq}(1)
		\item tần số lớn hơn giới hạn quang điện. 
		\item tần số nhỏ hơn giới hạn quang điện. 
		\item bước sóng nhỏ hơn giới hạn quang điện. 
		\item bước sóng lớn hơn giới hạn quang điện. 
	\end{mcq}
}

\loigiai
{		\textbf{Đáp án: C.}
	
	Để gây được hiện tượng quang điện ngoài, bức xạ chiếu vào kim loại phải có bước sóng nhỏ hơn giới hạn quang điện. 
}

%=======================================
\item \mkstar{1} [4]

\cauhoi
{Bức xạ màu vàng của Natri có bước sóng $\SI{0,59}{\mu m}$. Năng lượng của photon tương ứng có giá trị là
	\begin{mcq}(4)
		\item $\SI{2,1}{eV}$. 
		\item $\SI{2,0}{eV}$. 
		\item $\SI{2,3}{eV}$. 
		\item $\SI{2,2}{eV}$. 
	\end{mcq}
}

\loigiai
{		\textbf{Đáp án: A.}
	
	Năng lượng tương ứng của photon là
	$$
	\varepsilon = \dfrac{hc}{\lambda} = \SI{2,1}{eV}.
	$$
}

%=======================================
\item \mkstar{1} [4]

\cauhoi
{Theo quang điểm của thuyết lượng tử ánh sáng, phát biểu nào sau đây \textbf{sai}?
	\begin{mcq}(1)
		\item Các photon cùng một ánh sáng đơn sắc đều mang năng lượng như nhau. 
		\item Khi ánh sáng truyền đi xa, năng lượng photon giảm dần. 
		\item Photon chỉ tồn tại trong trạng thái chuyển động. 
		\item Ánh sáng được tạo bởi các hạt gọi là photon. 
	\end{mcq}
}

\loigiai
{		\textbf{Đáp án: B.}
	
	Khi ánh sáng truyền đi xa, năng lượng photon không đổi.
}

%=======================================
\item \mkstar{1} [10]

\cauhoi
{Dùng thuyết lượng tử ánh sáng\textbf{ không} giải thích được
	
	\begin{mcq}(1)
		\item hiện tượng quang – phát quang. 
		\item hiện tượng giao thoa ánh sáng. 
		\item nguyên tắc hoạt động của pin quang điện. 
		\item hiện tượng quang điện ngoài. 
	\end{mcq}
}

\loigiai
{		\textbf{Đáp án: B.}
	
	Thuyết lượng tử ánh sáng không giải thích được hiện tượng giao thoa ánh sáng.
}

%=======================================
\item \mkstar{1} [12]

\cauhoi
{Nếu giảm bước sóng chiếu vào một tấm kim loại xuống 2 lần so với ban đầu thì công thoát của kim loại  
	
	\begin{mcq}(4)
		\item giảm 2 lần. 
		\item không đổi. 
		\item tăng 4 lần. 
		\item tăng 2 lần. 
	\end{mcq}
}

\loigiai
{		\textbf{Đáp án: B.}
	
	Công thoát kim loại không phụ thuộc vào bước sóng.
}

%=======================================
\item \mkstar{1} [12]

\cauhoi
{Biết tốc độ ánh sáng là c, hằng số Plăng là h. Một phôtôn có năng lượng $\varepsilon$ thì bước sóng của nó bằng
	\begin{mcq}(4)
		\item $\dfrac{c \varepsilon}{h}$. 
		\item $\dfrac{hc}{\varepsilon}$. 
		\item $\dfrac{h \varepsilon}{c}$. 
		\item $\dfrac{\varepsilon}{hc}$. 
	\end{mcq}
}

\loigiai
{		\textbf{Đáp án: B.}
	
	Bước sóng của photon có năng lượng $\varepsilon$ thì cho bởi
	$$
	\lambda = \dfrac{hc}{\varepsilon}.
	$$
}

%=======================================
\item \mkstar{1} [12]

\cauhoi
{Khi nói về phôtôn, phát biểu nào dưới đây \textbf{không} đúng?
	\begin{mcq}(1)
		\item Phôtôn của mọi ánh sáng đều có năng lượng như nhau. 
		\item Trong chân không, tốc độ của các phôtôn là $\SI{3e8}{m/s}$. 
		\item Phôtôn chỉ tồn tại trong trạng thái chuyển động. 
		\item Chùm sáng chính là chùm hạt các phôtôn. 
	\end{mcq}
}

\loigiai
{		\textbf{Đáp án: A.}
	
	Photon của các ánh sáng có bước sóng khác nhau là khác nhau.
}

%=======================================
\item \mkstar{1} [13]

\cauhoi
{Chọn câu đúng nhất. Hiện tượng quang điện là hiện tượng electron bị bật ra khỏi bề mặt kim loại khi kim loại đó bị
	\begin{mcq}(2)
		\item chiếu bởi tia tử ngoại. 
		\item nung nóng. 
		\item ion đập vào. 
		\item chiếu bởi ánh sáng thích hợp. 
	\end{mcq}
}

\loigiai
{		\textbf{Đáp án: D.}
	
	
}

%=======================================
\item \mkstar{1} [13]

\cauhoi
{Công thoát electron của kim loại là năng lượng
	\begin{mcq}(1)
		\item tối thiểu để bứt nguyên tử ra khỏi kim loại. 
		\item tối thiểu để bứt electron ra khỏi kim loại. 
		\item của photon để cung cấp cho hạt nhân nguyên tử. 
		\item cần thiết để bứt ion ra khỏi kim loại. 
	\end{mcq}
}

\loigiai
{		\textbf{Đáp án: A.}
	
	Công thoát electron của kim loại là năng lượng tối thiểu để bứt nguyên tử ra khỏi kim loại. 
}

%=======================================
\item \mkstar{1} [13]

\cauhoi
{Công thoát electron ra khỏi kim loại là $A = \SI{3,61e-19}{J}$. Khi chiếu lần lượt vào kim loại này hai bức xạ có bước sóng $\lambda_{1} = \SI{0,6}{\mu m}$ và $\lambda_{2} = \SI{0,4}{\mu m}$ thì hiện tượng quang điện
	\begin{mcq}(1)
		\item xảy ra với bức xạ $\lambda_{2}$, không xảy ra với bức xạ $\lambda_{1}$. 
		\item xảy ra với bức xạ $\lambda_{1}$, không xảy ra với bức xạ $\lambda_{2}$. 
		\item xảy ra với cả hai bức xạ. 
		\item không xảy ra với cả hai bức xạ. 
	\end{mcq}
}

\loigiai
{		\textbf{Đáp án: A.}
	
	Giới hạn quang điện của kim loại trên là
	$$
	\lambda_{0} = \dfrac{hc}{A} = \SI{0,55}{\mu m}.
	$$
	Để xảy ra hiện tượng quang điện thì $\lambda \leq \lambda_{0}$. Vậy nên chỉ xảy ra với bức xạ $\lambda_{2}$, không xảy ra với bức xạ $\lambda_{1}$.
}

%=======================================
\item \mkstar{1} [13]

\cauhoi
{Công thoát của electron ra khỏi kim loại là $A = \SI{3,3125}{eV}$. Giới hạn quang điện của kim loại đó bằng
	\begin{mcq}(4)
		\item $\SI{0,331}{\micro m}$. 
		\item $\SI{0,518}{\micro m}$. 
		\item $\SI{0,600}{\micro m}$. 
		\item $\SI{0,375}{\micro m}$. 
	\end{mcq}
}

\loigiai
{		\textbf{Đáp án: D.}
	
	Giới hạn quang điện của kim loại đó là
	$$
	\lambda_{0} = \dfrac{hc}{A} = \SI{0,375}{\mu m}.
	$$
}

%=======================================
\item \mkstar{1} [13]

\cauhoi
{Năng lượng của photon có bước sóng $\SI{5e-12}{m}$ là
	\begin{mcq}(4)
		\item $\SI{4,587e-14}{J}$. 
		\item $\SI{3,975e-14}{J}$. 
		\item $\SI{3,645e-14}{J}$. 
		\item $\SI{4,245e-14}{J}$. 
	\end{mcq}
}

\loigiai
{		\textbf{Đáp án: B.}
	
	Năng lượng của photon cho bởi
	$$
	\varepsilon = \dfrac{hc}{\lambda} = \SI{3,975e-14}{J}. 
	$$
}

%=======================================
\item \mkstar{1} [13]

\cauhoi
{Năng lượng của photon ứng với ánh áng tím bằng $\SI{3,11}{eV}$, tần số của ánh sáng này bằng
	\begin{mcq}(4)
		\item $\SI{4,63e14}{Hz}$. 
		\item $\SI{3,75e14}{Hz}$. 
		\item $\SI{2,31e14}{Hz}$. 
		\item $\SI{7,51e14}{Hz}$. 
	\end{mcq}
}

\loigiai
{		\textbf{Đáp án: D.}
	
	Tần số của ánh sáng tím cho bởi
	$$
	f = \dfrac{\varepsilon}{h} = \SI{7,51e14}{Hz}.
	$$
}

%=======================================
\item \mkstar{1} [13]

\cauhoi
{Giới hạn quang điện của kẽm là $\SI{0,35}{\mu m}$, công thoát electron của kẽm lớn hơn công thoát của natri 1,4 lần. Giới hạn quang điện của natri là
	\begin{mcq}(4)
		\item $\SI{0,49}{\mu m}$. 
		\item $\SI{0,25}{\mu m}$. 
		\item $\SI{1,05}{\mu m}$. 
		\item $\SI{0,75}{\mu m}$. 
	\end{mcq}
}

\loigiai
{		\textbf{Đáp án: A.}
	
	Công thoát electron của kẽm lớn hơn công thoát của natri 1,4 lần. Nên giới hạn quang điện của kẽm nhỏ hơn giới hạn quang điện của natri 1,4 lần. Suy ra
	$$
	\lambda_{Na} = \num{1,4} \cdot \lambda_{Zn} = \SI{0,49}{\mu m}.
	$$
}

%=======================================
\item \mkstar{1} [5]

\cauhoi
{Giới hạn quang điện của mỗi kim loại được hiểu là
	\begin{mcq}(1)
		\item công thoát của electron đối với kim loại đó. 
		\item bước sóng của riêng kim loại đó. 
		\item bước sóng của ánh sáng chiếu vào kim loại. 
		\item một đại lượng đặc trưng cho kim loại, tỉ lệ nghịch với công thoát A đối với kim loại đó. 
	\end{mcq}
}

\loigiai
{		\textbf{Đáp án: D.}
	
	Giới hạn quang điện của mỗi kim loại được hiểu là một đại lượng đặc trưng cho kim loại, tỉ lệ nghịch với công thoát A đối với kim loại đó.
}

%=======================================
\item \mkstar{1} [5]

\cauhoi
{Công thoát là
	\begin{mcq}(1)
		\item năng lượng cung cấp cho các electron để cho chúng thoát ra khỏi mạng tinh thể kim loại. 
		\item năng lượng cần thiết để cung cấp cho các electron nằm sâu trong tinh thể kim loại để chúng thoát ra khỏi tinh thể. 
		\item động năng ban đầu của các electron quang điện. 
		\item năng lượng tối thiếu của photon bức xạ kích thích để có thể gây ra hiện tượng quang điện. 
	\end{mcq}
}

\loigiai
{		\textbf{Đáp án: D.}
	
	Công thoát là năng lượng tối thiếu của photon bức xạ kích thích để có thể gây ra hiện tượng quang điện. 
}

%=======================================
\item \mkstar{1} [5]

\cauhoi
{Cần chiếu bước sóng dài nhất là $\SI{0,49}{\mu m}$ để gây ra hiện tượng quang điện trên mặt lớp vônfram. Công thoát của electron ra khỏi vônfram là 
	\begin{mcq}(4)
		\item $\SI{2,5}{eV}$. 
		\item $\SI{3,3}{eV}$. 
		\item $\SI{5,2}{eV}$. 
		\item $\SI{3,1}{eV}$. 
	\end{mcq}
}

\loigiai
{		\textbf{Đáp án: A.}
	
	Công thoát của electron ra khỏi vônfram là 
	$$
	\varepsilon = \dfrac{hc}{\lambda_{0}} = \SI{2,5}{eV}.
	$$
}

%=======================================
\item \mkstar{1} [5]

\cauhoi
{Biết các kim loại như bạc, đồng, kẽm và nhôm có giới hạn quang điện là $\SI{0,26}{\mu m}$, $\SI{0,3}{\mu m}$, $\SI{0,35}{\mu m}$ và $\SI{0,36}{\mu m}$. Chiếu ánh sáng nhìn thấy lần lượt vào bốn tấm kim loại trên. Hiện tượng quang điện sẽ không xảy ra ở kim loại
	\begin{mcq}(2)
		\item bạc, đồng, kẽm. 
		\item bạc, đồng, kẽm, nhôm. 
		\item bạc. 
		\item bạc, đồng. 
	\end{mcq}
}

\loigiai
{		\textbf{Đáp án: B.}
	
	Ánh sáng nhìn thấy chỉ có thể gây ra hiện tượng quang điện đối với kim loại kiềm hoặc kiềm thổ. Vậy nên là hiện tượng quang điện chẳng xảy ra với kim loại nào trong bốn kim loại trên cả.
}

%=======================================
\item \mkstar{1} [7]

\cauhoi
{ Khi nói về thuyết lượng tử ánh sáng, phát biểu nào sau đây là \textbf{ đúng}?
	\begin{mcq}(1)
		\item Năng lượng phôtôn càng nhỏ khi cường độ chùm ánh sáng càng nhỏ. 
		\item Ánh sáng được tạo bởi các hạt gọi là phôtôn. 
		\item Năng lượng của phôtôn càng lớn khi tần số của ánh sáng ứng với phôtôn đó càng nhỏ. 
		\item Phôtôn có thể chuyển động hay đứng yên tùy thuộc vào nguồn sáng chuyển động hay đứng yên. 
	\end{mcq}
}

\loigiai
{		\textbf{Đáp án: B.}
	
	Ánh sáng được tạo bởi các hạt gọi là phôtôn.
}

%=======================================
\item \mkstar{1} [9]

\cauhoi
{Giới hạn quang điện của một kim loại là
	\begin{mcq}(1)
		\item điện tích tối đa kim loại tích được khi chiếu ánh sáng thích hợp vào. 
		\item điện thế làm ngưng hiện tượng quang điện. 
		\item bước sóng dài nhất ánh sáng chiếu vào tạo được hiện tượng quang điện với kim loại đó. 
		\item bước sóng của ánh sáng chiếu vào tạo ra được hiện tượng quang điện với kim loại đó. 
	\end{mcq}
}

\loigiai
{		\textbf{Đáp án: C.}
	
	Giới hạn quang điện của một kim loại là bước sóng dài nhất ánh sáng chiếu vào tạo được hiện tượng quang điện với kim loại đó.
}

%=======================================
\item \mkstar{1} [9]

\cauhoi
{Chiếu một chùm bức xạ đơn sắc vào tấm kẽm có giới hạn quang điện $\SI{0,35}{\mu m}$. Hiện tượng quang điện sẽ \textbf{không} xảy ra khi chùm bức xạ có bước sóng
	\begin{mcq}(4)
		\item $\SI{0,2}{\mu m}$. 
		\item $\SI{0,4}{\mu m}$. 
		\item $\SI{0,1}{\mu m}$. 
		\item $\SI{0,3}{\mu m}$. 
	\end{mcq}
}

\loigiai
{		\textbf{Đáp án: B.}
	
	Hiện tượng quang điện chỉ xảy ra khi $\lambda \leq \lambda_{0}$ nên hiện tượng quang điện sẽ không xảy ra với bước sóng $\SI{0,4}{\mu m}$. 
}

%=======================================
\item \mkstar{1} [9]

\cauhoi
{Công thoát của một kim loại là $\SI{4,5}{eV}$. Chiếu vào kim loại đó lần lượt các bức xạ có bước sóng $\lambda_{1} = \SI{0,16}{\mu m}$, $\lambda_{2} = \SI{0,2}{\mu m}$, $\lambda_{3} = \SI{0,25}{\mu m}$, $\lambda_{4} = \SI{0,3}{\mu m}$, $\lambda_{5} = \SI{0,36}{\mu m}$, $\lambda_6 = \SI{0,4}{\mu m}$. Các bức xạ gây ra hiện tượng quang điện với kim loại đó là
	\begin{mcq}(4)
		\item $\lambda_{1}, \lambda_{2}, \lambda_{3}$. 
		\item $\lambda_{1}, \lambda_{2}$. 
		\item $\lambda_{4}, \lambda_{5}, \lambda_{6}$. 
		\item $\lambda_{2}, \lambda_{3}, \lambda_{4}$. 
	\end{mcq}
}

\loigiai
{		\textbf{Đáp án: A.}
	
	Giới hạn quang điện của kim loại này là
	$$
	\lambda_{0} = \dfrac{hc}{A} = \SI{0,276}{\mu m}.
	$$
	
	Vậy các bức xạ có thể gây ra hiện tượng quang điện với kim loại này là $\lambda_{1}, \lambda_{2}, \lambda_{3}$. 
}

\end{enumerate}

\loigiai
{
	\begin{center}
		\textbf{BẢNG ĐÁP ÁN}
	\end{center}
	\begin{center}
		\begin{tabular}{|m{2.8em}|m{2.8em}|m{2.8em}|m{2.8em}|m{2.8em}|m{2.8em}|m{2.8em}|m{2.8em}|m{2.8em}|m{2.8em}|}
			\hline
			01.D  & 02.A  & 03.D  & 04.A  & 05.B  & 06.A  & 07.A & 08.C & 09.A & 10.C \\
			\hline
			11.A  & 12.B  & 13.B  & 14.B  & 15.B  & 16.A  & 17.D & 18.A & 19.A & 20.D \\
			\hline
			21.B  & 22.D  & 23.A  & 24.D  & 25.D  & 26.A  & 27.B & 28.B & 29.C & 30.B \\
			\hline
			31.A  &  &  &  &  &  & & &  & \\
			\hline
			
		\end{tabular}
	\end{center}
}

\section{Dạng bài: Hệ thức Einstein trong hiện tượng quang điện}
\begin{enumerate}[label=\bfseries Câu \arabic*:]
	
%=======================================
\item \mkstar{3} [5]

\cauhoi
{Trong các công thức nêu dưới đây, công thức nào là công thức Anhxtanh?
	\begin{mcq}(2)
		\item $hf = A + \dfrac{mv_{0}^2}{2}$. 
		\item $hf = A + \dfrac{mv^2}{2}$. 
		\item $hf = A - \dfrac{mv^2}{2}$. 
		\item $hf = A - \dfrac{mv_{0}^2}{2}$. 
	\end{mcq}
}

\loigiai
{		\textbf{Đáp án: A.}
	
	Hệ thức Anhxtanh là
	$$
	hf = A + \dfrac{mv_{0}^2}{2}.
	$$
}

%=======================================
\item \mkstar{3} [9]

\cauhoi
{Chiếu một bức xạ có bước sóng $\lambda = \SI{0,18}{\mu m}$ vào một tấm kim loại có giới hạn quang điện $\lambda_{0} = \SI{0,3}{\mu m}$. Năng lượng mà mỗi electron kim loại hấp thụ từ photon ánh sáng kích thích một phần dùng để giải phóng nó, phần còn lại dùng để tạo động năng ban đầu cực đại cho electron. Vận tốc ban đầu cực đại của các quang electron là
	\begin{mcq}(2)
		\item $\SI{0,0985e5}{m/s}$. 
		\item $\SI{98,5e5}{m/s}$. 
		\item $\SI{0,985e5}{m/s}$. 
		\item $\SI{9,85e5}{m/s}$. 
	\end{mcq}
}

\loigiai
{		\textbf{Đáp án: D.}
	
	Hệ thức Anhxtanh cho ta
	$$
	\dfrac{hc}{\lambda} = \dfrac{hc}{\lambda_{0}} + \dfrac{mv_{0}^2}{2} \rightarrow v_{0} = \SI{9,85e5}{m/s}.
	$$
}

	
\end{enumerate}

\loigiai
{
	\begin{center}
		\textbf{BẢNG ĐÁP ÁN}
	\end{center}
	\begin{center}
		\begin{tabular}{|m{2.8em}|m{2.8em}|m{2.8em}|m{2.8em}|m{2.8em}|m{2.8em}|m{2.8em}|m{2.8em}|m{2.8em}|m{2.8em}|}
			\hline
			01.A  & 02.D  &  &  &  &  & & & & \\
			\hline
			
		\end{tabular}
	\end{center}
}

\section{Lý thuyết: Thuyết lượng tử ánh sáng và lưỡng tính sóng hạt của ánh sáng}
\begin{enumerate}[label=\bfseries Câu \arabic*:]
	
	%=======================================
	\item \mkstar{1} [1]
	
	\cauhoi
	{Bức xạ đơn sắc có tần số $\SI{6e14}{Hz}$ có năng lượng photon bằng 
		\begin{mcq}(4)
			\item $\SI{3,975e-19}{J}$. 
			\item $\SI{7,950e-19}{J}$. 
			\item $\SI{5,9625e-19}{J}$. 
			\item $\SI{1,9875e-19}{J}$. 
		\end{mcq}
	}
	
	\loigiai
	{		\textbf{Đáp án: A.}
		
		Năng lượng photon cho bởi
		$$
		\varepsilon = hf = \SI{3,975e-19}{J}.
		$$
	}

	
\end{enumerate}

\loigiai
{
	\begin{center}
		\textbf{BẢNG ĐÁP ÁN}
	\end{center}
	\begin{center}
		\begin{tabular}{|m{2.8em}|m{2.8em}|m{2.8em}|m{2.8em}|m{2.8em}|m{2.8em}|m{2.8em}|m{2.8em}|m{2.8em}|m{2.8em}|}
			\hline
			01.A  &  &  &  &  &  & & & & \\
			\hline
			
		\end{tabular}
	\end{center}
}

\whiteBGstarEnd