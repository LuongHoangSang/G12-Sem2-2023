
\chapter[Tia tử ngoại]{Tia tử ngoại}
\section{Lý thuyết}

\subsection {Bản chất}

\begin{itemize}
	\item Tia tử ngoại là những bức xạ không nhìn thấy được, có bước sóng nhỏ hơn bước sóng của ánh sáng tím ($< \text{0,38}\ \mu \text{m}$). 
	\item Thu được cùng với các tia sáng thông thường.
	\item Có bản chất với ánh sáng.
\end{itemize}

\subsection{Tính chất chung}
\begin{itemize}
	\item Tuân theo các định luật: truyền thẳng, phản xạ, khúc xạ.
	\item Gây được hiện tượng nhiễu xạ, giao thoa như ánh sáng thông thường. 
\end{itemize}

\subsection{Nguồn phát}

\begin{itemize}
	\item Tất cả những vật có nhiệt độ trên $2000^\circ \text{C}$ thì phát được tia tử ngoại.
	\item Nhiệt độ của vật càng cao thì phổ tử ngoại của vật trải càng dài hơn về phía sóng ngắn
	\item Nguồn phát tia tử ngoại: hồ quang điện, mặt trời, đèn hơi thủy ngân,...
\end{itemize}

\subsection{Tính chất và công dụng}

\subsubsection{Tính chất}

\begin{itemize}
	\item  Tác dụng hóa học: gây phản ứng trên phim ảnh, kích thích nhiều phản ứng hóa học.
	\item  Tác dụng sinh học: hủy diệt tế bào, diệt khuẩn nấm mốc, là tiền tố tổng hợp vitamin D.
	\item Kích thích sự phát quang của nhiều chất.
	\item Làm ion hóa không khí và nhiều chất khí khác. \item Gây ra hiện tượng quang điện.
	\item Bị nước và thủy tinh hấp thụ rất mạnh nhưng lại có thể truyền qua được thạch anh. Ngoài ra tầng ozon hấp thụ hết các tia có bước sóng dưới $300\ \text{nm}$ và là tấm áo giáp bảo vệ sinh vật trên Trái Đất.
\end{itemize}

\subsubsection{Công dụng}

\begin{itemize}
	\item Y học: dùng để tiệt trùng dụng cụ phẫu thuật, chữa bệnh còi xương.
	\item Công nghiệp thực phẩm: tiệt trùng thực phẩm.
	\item Công nghiệp cơ khí: tìm về nứt (khuyết tật) trên bề mặt sản phẩm.
\end{itemize}
\section{Bài tập tự luyện}
\begin{enumerate}[label=\bfseries Câu \arabic*:]
	
	%=======================================
	\item \mkstar{1} [7]
	\cauhoi
	{Tia tử ngoại có ứng dụng nào sau đây?
		\begin{mcq}(1)
			\item Phát hiện các vết nứt trên bề mặt sản phẩm kim loại. 
			\item Kiểm tra hành lý khách hàng đi máy bay. 
			\item Chụp ảnh bề mặt trái đất từ vệ tinh. 
			\item Chiếu điện, chụp điện. 
		\end{mcq}
	}
	
	\loigiai
	{		\textbf{Đáp án: A.}
		
		Tia tử ngoại có công dụng phát hiện các vết nứt trên bề mặt sản phẩm kim loại. 
	}
	
	%=======================================
	\item \mkstar{1}
	{Nước hấp thụ được tia nào sau đây?
		\begin{mcq}(4)
			\item Tia tử ngoại. 
			\item Tia hồng ngoại. 
			\item Tia X. 
			\item Tia gamma. 
		\end{mcq}
	}
	
	\loigiai
	{		\textbf{Đáp án: A.}
		
		Nước hấp thụ tốt tia tử ngoại.
	}
	
	%=======================================
	\item \mkstar{1} [2]
	\cauhoi
	{Tia được dùng để khử khuẩn, tiệt trùng dụng cụ y tế
		\begin{mcq}(4)
			\item Tia tử ngoại. 
			\item Tia hồng ngoại. 
			\item Tia X. 
			\item Tia laser.
		\end{mcq}
	}
	
	\loigiai
	{		\textbf{Đáp án: A.}
		
		Tia được dùng để khử khuẩn, tiệt trùng dụng cụ y tế là tia tử ngoại.
	}
	
	%=======================================
	\item \mkstar{1} [31]
	\cauhoi
	{Để kiểm tra vết nứt trên bề mât sản phẩm kim loại, người ta dùng
		\begin{mcq}(2)
			\item tia tử ngoại. 
			\item tia Rơnghen. 
			\item tia hồng ngoại. 
			\item ánh sáng nhìn thấy. 
		\end{mcq}
	}
	
	\loigiai
	{		\textbf{Đáp án: A.}
		
		Để kiểm tra vết nứt trên bề mât sản phẩm kim loại, người ta dùng tia tử ngoại.
	}
	
	%=======================================
	\item \mkstar{1} [4]
	\cauhoi
	{Tầng ozon là tầng áo giáp bảo vệ cho con người và sinh vật trên mặt đất khỏi bị tác dụng hủy diệt của
		\begin{mcq}(1)
			\item tia tử ngoại trong ánh sáng mặt trời.
			\item tia hồng ngoại trong ánh sáng mặt trời. 
			\item tia đơn sắc màu đỏ trong ánh sáng mặt trời. 
			\item tia đơn sắc màu tím trong ánh sáng mặt trời. 
		\end{mcq}
	}
	
	\loigiai
	{		\textbf{Đáp án: A.}
		
		Tầng ozon là tầng áo giáp bảo cho con người và sinh vật trên mặt đất khỏi bị tác dụng hủy diệt của tia tử ngoại trong ánh sáng mặt trời.
	}
	
	%=======================================
	\item \mkstar{1} [4]
	\cauhoi
	{Phát biểu nào sau đây là \textbf{đúng}?
		\begin{mcq}(1)
			\item Tia hồng ngoại có tần số cao hơn tần số của tia sáng vàng. 
			\item Tia tử ngoại có bước sóng lớn hơn bước sóng của tia sáng đỏ. 
			\item Bức xạ tử ngoại có tần số cao hơn bức xạ hồng ngoại. 
			\item Bức xạ tử ngoại có chu kì lớn hơn chu kì của bức xạ tử ngoại. 
		\end{mcq}
	}
	
	\loigiai
	{		\textbf{Đáp án: C.}
		
		Phát biểu \textbf{đúng} là bức xạ tử ngoại có tần số cao hơn bức xạ hồng ngoại. 
	}
	
	%=======================================
	
		\item \mkstar{1} 
	\cauhoi
	{Khi nói về tia tử ngoại, phát biểu nào dưới đây là sai?
		
		\begin{mcq}(1)
			\item Tia tử ngoại có tác dụng mạnh lên kính ảnh.
			
			\item Tia tử ngoại có bản chất là sóng điện từ.
			
			\item Tia tử ngoại có bước sóng lớn hơn bước sóng của ánh sáng tím.
			
			\item Tia tử ngoại bị thuỷ tinh hấp thụ mạnh và làm ion hoá không khí.
			
		\end{mcq}
	}
	
	\loigiai
	{		\textbf{Đáp án: C.}
		
		Tia tử ngoại có bước sóng nhỏ hơn bước sóng của ánh sáng tím.
	
	}
		\item \mkstar{1} 
	\cauhoi
	{Tìm phát biểu sai về tác dụng và công dụng của tia tử ngoại. Tia tử ngoại
		
		\begin{mcq}(1)
			\item có tác dụng rất mạnh lên kính ảnh.
			
			\item có thể gây ra các hiệu ứng quang hoá, quang hợp.
			
			\item có tác dụng sinh học, huỷ diệt tế bào, khử trùng.
			
			\item trong công nghiệp được dùng để sấy khô các sản phẩm nông - công nghiệp.
			
		\end{mcq}
	}
	
	\loigiai
	{		\textbf{Đáp án: D.}
		
		Tác dụng sấy khô sản phầm nông - công nghiệp là tia hồng ngoại. 
	}
		\item \mkstar{1} 
	\cauhoi
	{Tìm nhận định sai khi nói về ứng dụng của tia tử ngoại?
		
		\begin{mcq}(1)
			\item Tiệt trùng.
			\item Kiểm tra vết nứt trên bề mặt kim loại.
			
			\item Xác định tuổi của cổ vật. 
			\item Chữa bệnh còi xương.
			
		\end{mcq}
	}
	
	\loigiai
	{		\textbf{Đáp án: C.}
		
		Tia tử ngoại không có ứng dụng xác định tuổi cổ vật.
	}
		\item \mkstar{1} 
	\cauhoi
	{Dựa vào tác dụng nào của tia tử ngoại mà người ta có thể tìm được vết nứt trên bề mặt sản phẩm bằng kim loại?
		
		\begin{mcq}(2)
			\item  kích thích phát quang.
			\item nhiệt.
			
			\item hủy diệt tế bào.
			\item gây ra hiện tượng quang điện.
			
		\end{mcq}
	}
	
	\loigiai
	{		\textbf{Đáp án: A.}
		
		Bôi dung dịch phát quang lên bề mặt sản phẩm, dung dịch ngấm vào kẻ nứt $\Rightarrow$ Tia tử ngoại làm chỗ nứt sáng lên. 
	}

\end{enumerate}

