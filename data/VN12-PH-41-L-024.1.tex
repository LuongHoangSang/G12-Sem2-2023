% --- chapter
\newcommand{\chapter}[2][]{
	\newcommand{\chapname}{#2}
	\begin{flushleft}
		\begin{minipage}[t]{\linewidth}
			\includegraphics[height=1cm]{hdht-logo.png}
			\hspace{0pt}	
			\sffamily\bfseries\large Bài  32. Hiện tượng quang - phát quang
			\begin{flushleft}
				\huge\bfseries #1
			\end{flushleft}
		\end{minipage}
	\end{flushleft}
	\vspace{1cm}
	\normalfont\normalsize
}
%-----------------------------------------------------
\chapter[Hiện tượng quang - phát quang]{Hiện tượng quang - phát quang}

\subsection{Hiện tượng quang phát quang}
\subsubsection{Khái niệm về sự phát quang}
	Một số chất trong tự nhiên có khả năng tự phát sáng, những chất đó gọi là chất phát quang và sự phát sáng đó gọi là sự phát quang.
	
	Ví dụ: đom đóm (hóa phát quang), đèn LED (điện phát quang), lớp huỳnh quang ở đèn ống (quang phát quang).
\subsubsection{Hiện tượng quang phát quang}
\begin{itemize}
	\item Hiện tượng quang - phát quang là hiện tượng một số chất có khả năng hấp thụ ánh sáng có bước sóng này để phát ra ánh sáng có bước sóng khác.
	
	Ví dụ: khi chiếu chùm tia tử ngoại vào dung dịch fluorescein (màu vàng nhạt) thì sẽ phát ánh sáng màu xanh lục.
	\item Đặc điểm quan trọng của sự phát quang là nó còn kéo dài một thời gian sau khi tắt ánh sáng kích thích. Thời gian này dài hay ngắn khác nhau phụ thuộc vào chất phát quang.
\end{itemize}
\subsubsection{Huỳnh quang và lân quang}
\begin{itemize}
	\item Huỳnh quang là sự phát quang của chất lỏng và chất khí, có đặc điểm là ánh sáng phát quang bị tắt rất nhanh sau khi tắt ánh sáng kích thích.
	\item Lân quang là sự phát quang của chất rắn, có đặc điểm là ánh sáng phát quang có thể kéo dài một khoảng thời gian nào đó sau khi tắt ánh sáng kích thích.
\end{itemize}
\subsection{Đặc điểm của ánh sáng huỳnh quang}
Ánh sáng huỳnh quang có bước sóng dài hơn bước sóng ánh sáng kích thích.
\begin{equation}
	\lambda_\text{hq} > \lambda_\text{kt}
\end{equation}
