
\chapter[Khối lượng và năng lượng của hạt nhân]{Khối lượng và năng lượng của hạt nhân}
\section{Lý thuyết}

\subsection{Khối lượng và năng lượng}
Hệ thức Einstein:
\begin{equation}
	E=mc^2
\end{equation}
trong đó:
\begin{itemize}
	\item $E$ là năng lượng;
	\item $m$ là khối lượng;
	\item $c=\SI{3e8}{\meter/\second}$ là tốc độ ánh sáng trong chân không.
\end{itemize}
Năng lượng (tính ra eV) tương ứng với khối lượng $\SI{1}{u}$ được xác định:
\begin{equation}
	E=uc^2\approx\SI{931,5}{\MeV}
	\Rightarrow 1\text{u}=\SI{931,5}{\MeV/c^2}.
\end{equation}
\subsection{Khối lượng tương đối}
Một vật có khối lượng $m_0$ khi ở trạng thái nghỉ thì khi chuyển động với vận tốc $v$, khối lượng sẽ tăng lên thành $m$ với
\begin{equation}
	m=\dfrac{m_0}{\sqrt{1-\dfrac{v^2}{c^2}}},
\end{equation}
trong đó:
\begin{itemize}
	\item $m_0$ là khối lượng nghỉ;
	\item $m$ là khối lượng động;
	\item $v$ là vận tốc chuyển động của vật;
	\item $c=\SI{3e8}{\meter/\second^2}$ là tốc độ ánh sáng trong chân không.
\end{itemize}
\subsection{Năng lượng toàn phần}
Khi một vật có khối lượng nghỉ $m_0$ chuyển động với tốc độ $v$ thì khối lượng tăng lên thành $m={m_0}/{\sqrt{1-{v^2}/{c^2}}}$; khi đó năng lượng của vật (gọi là năng lượng toàn phần) cho bởi công thức
\begin{equation}
	E=mc^2=\dfrac{m_0c^2}{\sqrt{1-\dfrac{v^2}{c^2}}},
\end{equation}
trong đó:
\begin{itemize}
	\item $E$ là năng lượng toàn phần;
	\item $E_0=m_0c^2$ là năng lượng nghỉ.
\end{itemize}

Hiệu $E-E_0$ chính là động năng của vật, kí hiệu là $W_\text{đ}$ và có công thức là
\begin{equation}
	W_\text{đ}=E-E_0=(m-m_0)c^2.
\end{equation}

\section{Mục tiêu bài học - Ví dụ minh họa}

\begin{dang}{Khối lượng và năng lượng}
	
	\viduii{2}
	{[Đề thi đại học khối A, A1 2013] Một hạt có khối lượng nghỉ $m_0$. Theo thuyết tương đối, khối lượng động (khối lượng tương đối tính) của hạt này khi chuyển động với tốc độ $0,6c$ ($c$là tốc độ ánh sáng trong chân không) là
		\begin{mcq}(4)
			\item $1,75 m_0$.
			\item $1,25 m_0$.
			\item $0,36 m_0$.
			\item $0,25 m_0$.
	\end{mcq}}
	{\begin{center}
			\textbf{Hướng dẫn giải}
		\end{center}
		Khối lượng động của hạt này khi chuyển động với tốc độ $0,6c$ là
		\begin{equation*}
			m=\dfrac{m_0}{\sqrt{1-\dfrac{v^2}{c^2}}}=\dfrac{m_0}{\sqrt{1-\dfrac{(0,6c)^2}{c^2}}}=1,25 m_0.
		\end{equation*}
		
		\begin{center}
			\textbf{Câu hỏi tương tự}
		\end{center}
		
		Một hạt có khối lượng nghỉ $m_0$. Theo thuyết tương đối, khối lượng động (khối lượng tương đối tính) của hạt này khi chuyển động với tốc độ $0,8c$ ($c$là tốc độ ánh sáng trong chân không) là
		\begin{mcq}(4)
			\item $\dfrac{2}{3} m_0$.
			\item $\dfrac{3}{2} m_0$.
			\item $\dfrac{3}{5} m_0$.
			\item $\dfrac{5}{3} m_0$.
		\end{mcq}
		\textbf{Đáp án:} D.
	}
	
	\viduii{2}
	{
		Khối lượng của vật tăng thêm bao nhiêu lần nếu vận tốc của nó tăng từ $ 0 $ đến $ \num{0,9} $ lần tốc độ ánh sáng.
		\begin{mcq}(4)
			\item $ \num{2,3} $.
			\item $ \num{3} $.
			\item $ \num{3,2} $.
			\item $ \num{2,4} $.
		\end{mcq}
	}
	{
		\begin{center}
			\textbf{Hướng dẫn giải}
		\end{center}
		Ta có:
		$$
		\dfrac{m}{m_{0}} = \dfrac{1}{\sqrt{1-\dfrac{v^{2}}{c^{2}}}} \approx \num{2,3}.
		$$
		\begin{center}
			\textbf{Câu hỏi tương tự}
		\end{center}
		Coi tốc độ ánh sáng trong chân không là $ \SI{3 e8}{m/s} $. Năng lượng của vật biến thiên bao nhiêu nếu khối lượng của vật biến thiên một lượng bằng khối lượng của electron $ \SI{9,1 e-31}{kg} $.
		\begin{mcq}(2)
			\item $ \SI{8,2 e-14}{J} $.
			\item $ \SI{8,7 e-14}{J} $.
			\item $ \SI{8,2 e-16}{J} $.
			\item $ \SI{8,7 e-16}{J} $.
		\end{mcq}
		\textbf{Đáp án;} A.
	}
	
\end{dang}

\begin{dang}{Động năng}
	
	\viduii{3}
	{[Đề thi đại học khối A 2011] Theo thuyết tương đối, một electron có động năng bằng một nửa năng lượng nghỉ của nó thì electron này chuyển động với tốc độ bằng 
		\begin{mcq}(2)
			\item $\SI{2,41e8}{\meter/\second}$.
			\item $\SI{2,75e8}{\meter/\second}$. 
			\item $\SI{1,67e8}{\meter/\second}$. 
			\item $\SI{2,24e8}{\meter/\second}$. 
	\end{mcq}}
	{\begin{center}
			\textbf{Hướng dẫn giải}
		\end{center}
		Tốc độ của electron này là
		\begin{equation*}
			E=E_0+W_\text{đ}=E_0+\dfrac{E_0}{2}=\dfrac{3E_0}{2}\Rightarrow \dfrac{m_0c^2}{\sqrt{1-\dfrac{v^2}{c^2}}} = \dfrac{3}{2}m_0c^2\Rightarrow v=\dfrac{\sqrt{5}}{3}c\approx\SI{2,24e8}{\meter/\second}.
		\end{equation*}
		
		\begin{center}
			\textbf{Câu hỏi tương tự}
		\end{center}
		
		Một hạt có khối lượng $ m_{0} $. Theo thuyết tương đối, động năng của hạt này khi chuyển động với tốc độ $ \num{0,6}c $ ($ c $ là tốc độ ánh sáng trong chân không) là
		\begin{mcq}(2)
			\item $ \num{0,36} m_{0}c^{2} $.
			\item $ \num{1,25} m_{0}c^{2} $.
			\item $ \num{0,225} m_{0}c^{2} $.
			\item $ \num{0,25} m_{0}c^{2} $.
		\end{mcq}
		
		\textbf{Đáp án:} D.
	}
	
	\viduii{3}
	{
		Khối lượng của electron chuyển động bằng hai lần khối lượng nghỉ của nó. Tìm tốc độ chuyển động của electron. Coi tốc độ ánh sáng trong chân không là $ \SI{3 e8}{m/s} $.
		\begin{mcq}(2)
			\item $ \SI{0,4 e8}{m/s} $.
			\item $ \SI{2,59 e8}{m/s} $.
			\item $ \SI{1,2 e8}{m/s} $.
			\item $ \SI{2,985 e8}{m/s} $.
		\end{mcq}
	}
	{
		\begin{center}
			\textbf{Hướng dẫn giải}
		\end{center}
		Ta có:
		$$
		m = \dfrac{m_{0}}{\sqrt{1-\dfrac{v^{2}}{c^{2}}}} = 2m_{0} \rightarrow v = \dfrac{c\sqrt{3}}{2} = \SI{2,59 e8}{m/s}.
		$$
		
		\begin{center}
			\textbf{Câu hỏi tương tự}
		\end{center}
		
		Biết khối lượng của electron $ \SI{9,1 e-31}{kg} $ và tốc độ ánh sáng trong chân không $ c = \SI{3 e8}{m/s} $. Động năng của electron chuyển động với tốc độ $ \SI{e8}{m/s} $ là
		\begin{mcq}(2)
			\item $ \SI{4,968 e-15}{J} $.
			\item $ \SI{4,550 e-15}{J} $.
			\item $ \SI{4,267 e-15}{J} $.
			\item $ \SI{4,765 e-15}{J} $.
		\end{mcq}
		\textbf{Đáp án:} A.
	}


\end{dang}

\section{Bài tập tự luyện}
\begin{enumerate}[label=\bfseries Câu \arabic*:]
	
		\item \mkstar{1} 
		\cauhoi
{Theo định nghĩa về đơn vị khối lượng nguyên tử thì $1\ \text u$ bằng
	\begin{mcq}
		\item khối lượng của một nguyên tử hydro $^1_1 \text H$. 
		\item khối lượng của một hạt nhân nguyên tử cacbon $^{12}_6 \text C$.
		\item 1/12 khối lượng hạt nhân nguyên tử của đồng vị cacbon $^{12}_6 \text C$.
		\item 1/12 khối lượng của đồng vị nguyên tử Oxi.
		
	\end{mcq}
}

\loigiai
{		\textbf{Đáp án: C.}
	
	
	
}
\item \mkstar{1}
\cauhoi
{Khối lượng proton $m_p = \text{1,007276}\ \text{u}$. Khi tính theo đơn vị kg thì
	
	\begin{mcq}(4)
		\item $m_p= \text{1,762}\cdot 10^{-27}\ \text{kg}.$
		\item $m_p= \text{1,672}\cdot 10^{-27}\ \text{kg}.$
		\item $m_p= \text{16,72}\cdot 10^{-27}\ \text{kg}.$
		\item $m_p= \text{167,2}\cdot 10^{-27}\ \text{kg}.$
	\end{mcq}
}

\loigiai
{		\textbf{Đáp án: B.}
	
	Do $1\ \text u = \text{1,66055} \cdot 10^{-27}\ \text{kg}.$
	
	Khối lượng proton bằng
	
	$$m_p =  \text{1,007276}\ \text{u} \approx \text{1,672}\cdot 10^{-27}\ \text{kg}.$$
	
}
\item \mkstar{1}
\cauhoi
{Với $c$ là vận tốc ánh sáng trong chân không, hệ thức Einstein giữa năng lượng $E$ và khối lượng $m$ của vật là
	
	\begin{mcq}(4)
		\item $E = mc^2$.
		\item $E = m^2c$.
		\item $E = 2mc^2$.
		\item $E = 2mc$.
	\end{mcq}
}

\loigiai
{		\textbf{Đáp án: A.}
	
	
	
}

\item \mkstar{1} 
\cauhoi
{Gọi $m_0$ là khối lượng nghỉ của vật ; $m, v$ lần lượt là khối lượng và vận tốc khi vật chuyển động. Biểu thức nào sau đây không phải là biểu thức tính năng lượng toàn phần của một hạt tương đối tính ?
	
	\begin{mcq}(4)
		\item $E = mc^2$.
		\item $E = E_0 + W_\text{đ}$.
		\item $E = \dfrac{m_0c^2}{\sqrt{1- \dfrac{v^2}{c^2}}}.$
		\item $E = m_0c^2$.
	\end{mcq}
}

\loigiai
{		\textbf{Đáp án: D.}
	
	$$E = mc^2 = \dfrac{mc^2}{\sqrt{1- \dfrac{v^2}{c^2}}} = E_0 + W_\text{đ}.$$
	
}
\item \mkstar{1} 
\cauhoi
{Một hạt có khối lượng nghỉ $m_0$, chuyển động với tốc độ $v$ thì theo thuyết tương đối, động năng của hạt được định bởi công thức:
	
	\begin{mcq}(2)
		\item $\dfrac{m_0c^2}{\sqrt{1- \dfrac{v^2}{c^2}}}.$
		\item $m_0c^2 \left(\dfrac{1}{\sqrt{1-\dfrac{v^2}{c^2}}}-1\right)$.
		\item $\dfrac{2m_0c^2}{\sqrt{1- \dfrac{v^2}{c^2}}}.$
		\item $2m_0c^2 \left(\dfrac{1}{\sqrt{1-\dfrac{v^2}{c^2}}}-1\right)$.
	\end{mcq}
}

\loigiai
{		\textbf{Đáp án: B.}
	
	Theo thuyết tương đối, động năng của hạt là 
	
	$$W_\text{đ} = (m - m_0)c^2 = m_0c^2 \left(\dfrac{1}{\sqrt{1-\dfrac{v^2}{c^2}}}\right).$$
	
}
\item \mkstar{2} 
\cauhoi
{Một hạt có khối lượng nghỉ $m_0$, chuyển động với tốc độ $v = \dfrac{\sqrt 3}{2}$  (với $c$ là tốc độ ánh sáng trong chân không). Theo thuyết tương đối, năng lượng toàn phần của hạt sẽ
	
	\begin{mcq}(2)
		\item gấp 2 lần động năng của hạt.
		\item gấp bốn lần động năng của hạt.
		\item gấp $\sqrt 3$ lần động năng của hạt.
		\item gấp $\sqrt 2$ lần động năng của hạt.
	\end{mcq}
}

\loigiai
{		\textbf{Đáp án: A.}
	
	Tỉ số giữa năng lượng toàn phần và động năng của vật là 
	
	$$\dfrac{mc^2}{(m-m_0)c^2} = \dfrac{1}{1- \dfrac{m_0}{m}} = \dfrac{1}{1- \sqrt{1-\dfrac{v^2}{c^2}}} = 2.$$
	
	
	
}
\item \mkstar{2} 
\cauhoi
{Kí hiệu $E_0$, $E$ là năng lượng nghỉ và năng lượng toàn phần của một hạt có khối lượng nghỉ $m_0$, chuyển động với vận tốc $v = 0,8c$. Theo thuyết tương đối, năng lượng nghỉ $E_0$ của hạt bằng
	
	\begin{mcq}(4)
		\item 0,5$E$.
		\item 0,6$E$.
		\item 0,25$E$.
		\item 0,8$E$.
	\end{mcq}
}

\loigiai
{		\textbf{Đáp án: B.}
	
	Theo thuyết tương đối, năng lượng nghỉ:
	
	$$E_0 = m_0c^2 = mc^2 \sqrt{1- \dfrac{v^2}{c^2}} = \text{0,6}E.$$
	
}
\item \mkstar{2}
\cauhoi
{Do sự phát bức xạ nên mỗi ngày ($\SI{86400}{s}$) khối lượng Mặt Trời giảm một lượng $\text{3,744}\cdot 10^{14}\ \text{kg}$. Biết vận tốc ánh sáng trong chân không là $3\cdot 10^8\  \text{m/s}$. Công suất bức xạ (phát xạ) trung bình của Mặt Trời bằng
	
	\begin{mcq}(4)
		\item $\text{6,9}\cdot 10^{15}\ \text{MW}$.
		\item $\text{3,9}\cdot 10^{15}\ \text{MW}$.
		\item $\text{4,9}\cdot 10^{15}\ \text{MW}$.
		\item $\text{5,9}\cdot 10^{15}\ \text{MW}$.
	\end{mcq}
}

\loigiai
{		\textbf{Đáp án: B.}
	
	Công suất bức xạ trung bình của mặt trời:
	
	$$P = \dfrac{E}{t} = \dfrac{mc^2}{t} = \text{3,9}\cdot 10^{20}\ \text{MW}.$$
	
}
\item \mkstar{2}
\cauhoi
{Theo thuyết tương đối, một electron có động năng bằng một nửa năng lượng nghỉ của nó thì electron này chuyển động với tốc độ bằng
	
	\begin{mcq}(4)
		\item $\text{2,41}\cdot 10^8 \ 
		\text{m/s}$.
		\item $\text{2,75}\cdot 10^8 \ 
		\text{m/s}$.
		\item $\text{1,67}\cdot 10^8 \ 
		\text{m/s}$.
		\item $\text{2,59}\cdot 10^8 \ 
		\text{m/s}$.
	\end{mcq}
}

\loigiai
{		\textbf{Đáp án: D.}
	
	Động năng của electron khi chuyển động bằng một nửa năng lượng nghỉ của nó:
	
	$$W_\text{đ} = (m - m_0)c^2 = \text{0,5}mc^2 \Rightarrow m = 2m_0.$$
	
	Mà
	
	$$m = \dfrac{m_0}{\sqrt{1- \dfrac{v^2}{c^2}}} \Rightarrow \sqrt{1- \dfrac{v^2}{c^2}} = \dfrac{1}{2} \Rightarrow v = \dfrac{c\sqrt 3}{2} \approx \text{2,59} \cdot 10^8\ \text{m/s}.$$
	
}
\item \mkstar{3} 
\cauhoi
{Một hộ gia đình trung bình mỗi tháng sử dụng hết một lượng điện năng là 250 kWh. Nếu có cách chuyển hoàn toàn một chiếc móng tay nặng 0,05 g thành năng lượng điện thì sẽ đủ cho hộ gia đình đó dùng trong
	
	\begin{mcq}(4)
		\item 104 năm.	
		\item 208,3 năm.
		\item 416,6 năm.	
		\item 832,5 năm.
		
	\end{mcq}
}

\loigiai
{		\textbf{Đáp án: C.}
	
	Lượng điện năng thu được khi chuyển hóa hoàn toàn một chiếc móng tay là
	
	$$E = mc^2 = \text{4,5} \cdot 10^{12}\ \text J.$$
	
	Mỗi tháng sử dụng hết $\SI{250}{kWh} = 9\cdot 10^9\ \text J$ thì sau $\dfrac{\text{4,5} \cdot 10^{12}}{\text{0,9}\cdot 10^9} = 5000$ tháng = 416,6 năm mới sử dụng hết năng lượng của chiếc móng tay chuyển hóa thành.
	
	
}

\end{enumerate}


