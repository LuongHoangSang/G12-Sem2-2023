
\chapter[Sơ lược về LASER]{Sơ lược về LASER}
\section{Lý thuyết}
\subsection{Thuật ngữ LASER}
LASER (Light Amplifier by Stimulated Emission of Radiation) là một nguồn sáng phát ra một chùm sáng cường độ lớn dựa trên việc ứng dụng hiện tượng phát xạ cảm ứng.
\luuy{Chùm bức xạ phát ra từ nguồn phát LASER được gọi là tia LASER}
\subsection{Hiện tượng phát xạ cảm ứng}
Nguồn phát LASER hoạt động dựa trên hiện tượng phát xạ cảm ứng: nếu một nguyên tử đang ở trong trạng thái kích thích, sẵn sàng phát ra một phôtôn có năng lượng $\varepsilon = hf$, bắt gặp một phôtôn có năng lượng đúng bằng $hf$ bay lướt qua nó, thì lập tức nguyên tử này cũng phát ra phôtôn $\varepsilon$. Kết quả là sẽ có 2 phôtôn được phát ra, thay vì chỉ có 1 phôtôn như phát xạ thông thường.

Các phôtôn phát ra có cùng tần số, cùng phương, cùng pha. Ngoài ra, vì số phôtôn bay theo cùng một hướng nên cường độ của chùm sáng rất lớn.
\subsection{Tính chất của tia LASER}
Do các đặc trưng của chùm phôtôn phát ra nên tia LASER có những tính chất sau:
\begin{itemize}
	\item Tính đơn sắc cao;
	\item Tính kết hợp cao;
	\item Tính định hướng cao;
	\item Cường độ lớn.
\end{itemize}
\subsection{Phân loại}
Dựa vào trạng thái của môi trường hoạt chất (môi trường xảy ra hiện tượng phát xạ cảm ứng), LASER được chia thành 3 loại:
\begin{itemize}
	\item LASER rắn (hồng ngọc);
	\item LASER khí (He-Ne, CO$_2$, Ar, N$_2$);
	\item LASER bán dẫn;
\end{itemize}
\subsection{Một số ứng dụng của tia LASER}
\begin{itemize}
	\item Trong y học: làm dao mổ trong phẫu thuật mắt, chữa một số bệnh ngoài da;
	\item Trong thông tin liên lạc: vô tuyến định vị, truyền thông tin bằng cáp quang;
	\item Trong công nghiệp: khoan, cắt chính xác;
	\item Trong trắc địa: đo khoảng cách;
	\item Trong đầu đọc đĩa CD, bút trỏ bảng, \ldots
\end{itemize}
\section{Bài tập tự luyện}
\begin{enumerate}[label=\bfseries Câu \arabic*:]
	\item \mkstar{1} [3]
	\cauhoi
	{Chùm ánh sáng laze được ứng dụng 
		\begin{mcq}(2)
			\item trong truyền tin bằng vệ tinh. 
			\item làm nguồn phát siêu âm. 
			\item làm dao mổ trong y học. 
			\item trong đầu đọc USB. 
		\end{mcq}
	}
	
	\loigiai
	{		\textbf{Đáp án: C.}
		
		Chùm ánh sáng laze được ứng dụng làm dao mổ trong y học. 
		
	}
	
	\item \mkstar{1} [12]
	\cauhoi
	{ Laze không có đặc điểm nào dưới đây? 
		
		\begin{mcq}(2)
			\item Tính định hướng cao.   
			\item Cường độ lớn. 
			\item Chùm tia laze là chùm phân kì.
			\item Chùm tia laze có tính đơn sắc cao.  
		\end{mcq}
	}
	
	\loigiai
	{		\textbf{Đáp án: C.}
		
		Laze là chùm tia gần như song song.
		
		
	}
	
	\item \mkstar{1} [5]
	\cauhoi
	{Tia laze không có tính chất nào dưới đây:
		\begin{mcq} (2)
			\item Tia laze có công suất lớn. 
			\item Tia laze là chùm sáng kết hợp. 
			\item Tia laze có tính đơn sắc cao. 
			\item Tia laze có tính định hướng cao. 
		\end{mcq}
	}
	
	\loigiai
	{		\textbf{Đáp án: A.}
		
		Tia laze không phải lúc nào cũng có công suất lớn.
		
	}
	
	\item \mkstar{1} [5]
	\cauhoi
	{Bút laze ta dùng để chỉ bảng thuộc loại laze
		\begin{mcq}(4)
			\item rắn. 
			\item bán dẫn. 
			\item lỏng. 
			\item khí. 
		\end{mcq}
	}
	
	\loigiai
	{		\textbf{Đáp án: B.}
		
		Bút laze ta dùng để chỉ bảng thuộc loại laze bán dẫn.
		
	}
		\item \mkstar{1} [3]
	\cauhoi
	{Chùm ánh sáng laze được ứng dụng
		\begin{mcq}(2)
			\item trong truyền tin bằng vệ tinh. 
			\item làm nguồn phát siêu âm. 
			\item làm dao mổ trong y học. 
			\item trong đầu đọc USB. 
		\end{mcq}
	}
	
	\loigiai
	{		\textbf{Đáp án: C.}
		
		Chùm ánh sáng laze được ứng dụng làm dao mổ trong y học. 
	}
	
\end{enumerate}

