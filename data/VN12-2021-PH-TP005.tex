\begin{enumerate}[label=\bfseries Câu \arabic*:]
	
	\item \mkstar{1}

\cauhoi{
	
	Chọn phương án \textbf{sai} khi nói về dao động cưỡng bức.
	\begin{mcq}
		\item Dao động với biên độ thay đổi theo thời gian.
		\item Dao động điều hòa.
		\item Dao động với tần số bằng tần số của ngoại lực.
		\item Dao động với biên độ không đổi.
	\end{mcq}
}
\loigiai{
	\textbf{Đáp án A.}
	
	Dao động cưỡng bức dao động với biên độ không đổi.
	
}
\item \mkstar{2}

\cauhoi{
	
	Hai con lắc làm bằng hai hòn bi có bán kính bằng nhau, treo trên hai sợi dây có cùng độ dài. Khối lượng của hai hòn bi khác nhau. Hai con lắc cùng dao động trong một môi trường với li độ ban đầu như nhau và vận tốc ban đầu đều bằng 0 thì
	\begin{mcq}
		\item con lắc nhẹ tắt nhanh hơn.
		\item con lắc nặng tắt nhanh hơn hay con lắc nhẹ tắt nhanh hơn còn tùy thuộc vào gia tốc trọng trường.
		\item hai con lắc tắt cùng một lúc.
		\item con lắc nặng tắt nhanh hơn.
	\end{mcq}
}
\loigiai{
	\textbf{Đáp án A.}
	
	Hai con lắc làm bằng hai hòn bi có bán kính bằng nhau, treo trên hai sợi dây có cùng độ dài. Khối lượng của hai hòn bi khác nhau. Hai con lắc cùng dao động trong một môi trường với li độ ban đầu như nhau và vận tốc ban đầu đều bằng 0 thì con lắc nhẹ tắt nhanh hơn.
}
\item \mkstar{2}

\cauhoi{ 
	
	Để duy trì dao động cho một cơ hệ mà không làm thay đổi chu kỳ riêng của nó ta phải
	
	\begin{mcq}
		\item làm nhẵn, bôi trơn để giảm ma sát.
		\item tác dụng ngoại lực vào vật dao động cùng chiều với chuyển động trong một phần của từng chu kỳ.
		\item tác dụng vào vật dao động một ngoại lực biến thiên tuần hoàn theo thời gian.
		\item tác dụng vào vật dao động một ngoại lực không đổi theo thời gian.
	\end{mcq}
}
\loigiai{
	\textbf{Đáp án B.}
	
	Để duy trì dao động cho một cơ hệ mà không làm thay đổi chu kỳ riêng của nó ta phải tác dụng ngoại lực vào vật dao động cùng chiều với chuyển động trong một phần của từng chu kỳ.
	
}
\item \mkstar{2}

\cauhoi{
	
	Hiện tượng cộng hưởng thể  hiện càng rõ nét khi
	\begin{mcq}(2)
		\item độ nhớt của môi trường càng nhỏ.
		\item tần số của lực cưỡng bức lớn.
		\item độ nhớt của môi trường càng lớn.
		\item biên độ của lực cưỡng bức nhỏ.
	\end{mcq}
	
}
\loigiai{
	\textbf{Đáp án A.}
	
	
	Hiện tượng cộng hưởng thể  hiện càng rõ nét khi độ nhớt của môi trường càng nhỏ.
}
\item \mkstar{2}

\cauhoi{
	
	Hai con lắc làm bằng hai hòn bi có bán kính bằng nhau, treo trên hai sợi dây có cùng độ dài. Khối lượng của hai hòn bi bằng nhau. Hai con lắc cùng dao động trong một môi trường với li độ ban đầu như nhau và vận tốc ban đầu đều bằng 0 thì
	\begin{mcq}
		\item con lắc nhẹ sẽ tắt nhanh hơn so với con lắc nặng.
		\item con lắc nặng sẽ tắt nhanh hơn so với con lắc nhẹ.
		\item hai con lắc tắt dần như nhau.
		\item chưa đủ dữ kiện để kết luận.
	\end{mcq}
}
\loigiai{
	\textbf{Đáp án C.}
	
	
	Vì hai con lắc có cùng mọi thông số và cùng một môi trường nên tắt dần như nhau.
}

	\item \mkstar{3}
	
	\cauhoi{
		
		Một con lắc dài $\SI{44}{\centi \meter}$ được treo vào trần của một toa xe lửa. Con lắc bị kích động mỗi khi bánh của toa xe gặp chỗ nối nhau của đường ray. Hỏi tàu chạy thẳng đều với tốc độ bằng bao nhiêu thì biên độ dao động của con lắc sẽ lớn nhất? Cho biết chiều dài của mỗi đường ray là $\SI{12.5}{\meter}$. Lấy $g=\SI{9.8}{\meter / \second ^2}$.
		\begin{mcq}(4)
			\item $\SI{10.7}{\kilo \meter / \hour}$.
			\item $\SI{34}{\kilo \meter / \hour}$.
			\item $\SI{106}{\kilo \meter / \hour}$.
			\item $\SI{45}{\kilo \meter / \hour}$.
		\end{mcq}
		
	}
	\loigiai{
		\textbf{Đáp án B.}
		
		Dao động của con lắc là dao động cưỡng bức, biên độ của dao động cưỡng bức sẽ lớn nhất khi xảy ra hiện tượng cộng hưởng.
		
		Chu kì dao động riêng của con lắc:
		$$T_0=2\pi \sqrt {\dfrac {l}{g}} = 2\pi \sqrt {\dfrac{\SI{0.44}{\meter}}{\SI{9.8}{\meter / \second ^2}}} = \SI{1.33}{\second}.$$
		
		Khi xảy ra hiện tượng cộng hưởng, tốc độ của đoàn tàu là
		$$T=\dfrac{l}{v} = T_0 \Leftrightarrow v= \dfrac{l}{T_0}=\dfrac {\SI{12.5}{\meter}}{\SI{1.33}{\second}} = \SI{9.4}{\meter / \second} = \SI{34}{\kilo \meter / \hour}.$$
		
		Vậy khi tàu chạy thẳng đều với tốc độ $v=\SI{34}{\kilo \meter / \hour}$ thì biên độ dao động của con lắc sẽ lớn nhất.
	}
	\item \mkstar{3}

\cauhoi{
	
	Một xe ô tô chạy trên đường, cứ cách $\SI{8}{\meter}$ lại có một cái mô nhỏ. Chu kì dao động tự do của khung xe trên các lò xo là $\SI{1.5}{\second}$. Xe chạy với vận tốc nào thì bị rung mạnh nhất?
	\begin{mcq}(4)
		\item $\SI{19.1}{\kilo \meter / \hour}$.
		\item $\SI{20.1}{\kilo \meter / \hour}$.
		\item $\SI{21.1}{\kilo \meter / \hour}$.
		\item $\SI{22.1}{\kilo \meter / \hour}$.
	\end{mcq}
	
}
\loigiai{
	\textbf{Đáp án A.}
	
	Dao động của khung xe là dao động cưỡng bức, biên độ của dao động cưỡng bức sẽ lớn nhất khi xảy ra hiện tượng cộng hưởng. Tốc độ của xe để xảy ra hiện tượng cộng hưởng là
	\begin{equation*}
		T=\dfrac{l}{v} = T_0 \Leftrightarrow v= \dfrac{l}{T_0}=\dfrac {\SI{8}{\meter}}{\SI{1.5}{\second}} = \SI{5.3}{\meter / \second} = \SI{19.1}{\kilo \meter / \hour}.
	\end{equation*}
	Vậy khi ô tô chạy thẳng đều với tốc độ $v=\SI{19.1}{\kilo \meter / \hour}$ thì bị rung mạnh nhất.
}
	\item \mkstar{3}

\cauhoi{
	
	Một người đèo hai thùng nước ở phía sau xe đạp và đạp xe trên con đường lát bêtông. Cứ cách 
	3m, trên đường lại có một rãnh nhỏ. Đối với người đó tốc độ nào là không có lợi? Cho biết chu kì dao động riêng của nước trong thùng là $\text{0,6}\ \text{s}$.
	
	\begin{mcq}(4)
		\item $\SI{13}{\meter / \second}$.
		\item $\SI{14}{\meter / \second}$.
		\item $\SI{5}{\meter / \second}$.
		\item $\SI{6}{\meter / \second}$.
	\end{mcq}
	
}
\loigiai{
	\textbf{Đáp án C.}
	
	Khi chu kì dao động riêng $T_0$ của nước bằng chu kì dao động cưỡng bức $T$ thì nước trong thùng dao động mạnh nhất, dễ đổ ra ngoài nhất nên không có lợi.
	\begin{equation*}
		T_0=T=\dfrac{s}{v}\Rightarrow v=\dfrac{s}{T_0}=\SI{5}{\meter / \second}.
	\end{equation*}
}
\item \mkstar{4}

\cauhoi{
	
	Một con lắc dao động tắt dần. Cứ sau mỗi chu kì, biên độ giảm $\SI{3}{\percent}$. Phần năng lượng của con lắc bị mất đi trong một dao động toàn phần là bao nhiêu?
	\begin{mcq}(4)
		\item $\SI{3}{\percent}$.
		\item $\SI{6}{\percent}$.
		\item $\SI{4.5}{\percent}$.
		\item $\SI{9}{\percent}$.
	\end{mcq}
	
}
\loigiai{
	\textbf{Đáp án B.}
	
	Giả sử biên độ ban đầu là $A$.
	
	Sau một chu kỳ biên độ con lắc còn $A_1=\text{0,97}A$.
	\begin{equation*}
		\Rightarrow \dfrac{W_\text{1}}{W}=\left( \dfrac{A_\text{1}}{A}\right) ^2=\text{0,97}^2.
	\end{equation*}
	Phần trăm cơ năng con lắc bị mất đi trong hai dao động toàn phần liên tiếp là
	\begin{equation*}
		\dfrac{\Delta W_\text{1}}{W}=1-\text{0,97}^2\approx 6\%.
	\end{equation*}
}
\item \mkstar{4}

\cauhoi{
	
	Một con lắc lò xo dao động tắt dần trên mặt phẳng nằm ngang. Cứ sau mỗi chu kì biên độ giảm 2\%. Gốc thế năng tại vị trí mà lò xo không bị biến dạng. Phần trăm cơ năng con lắc bị mất đi trong hai dao động toàn phần liên tiếp có giá trị gần nhất với giá trị nào sau đây?
	\begin{mcq}(4)
		\item  7\%.
		\item  4\%.
		\item 10\%.
		\item 8\%.
	\end{mcq}
	
}
\loigiai{
	\textbf{Đáp án D.}
	
	Giả sử biên độ ban đầu là $A$.
	
	Sau một chu kỳ biên độ con lắc còn
	\begin{equation*}
		A_1=\text{0,98}A.
	\end{equation*}
	Sau hai chu kỳ biên độ con lắc còn 
	\begin{equation*}
		A_2=\text{0,98}A_1=\text{0,98}^2A.
	\end{equation*}
	\begin{equation*}
		\Rightarrow \dfrac{W_\text{2}}{W}=\left( \dfrac{A_\text{n}}{A}\right) ^2=\text{0,98}^4.
	\end{equation*}
	Phần trăm cơ năng con lắc bị mất đi trong hai dao động toàn phần liên tiếp là:
	\begin{equation*}
		\dfrac{\Delta W_\text{2}}{W}=1-\text{0,98}^4\approx 8\%.
	\end{equation*}
}
	
\end{enumerate}
\loigiai{\textbf{Đáp án}
	\begin{center}
		\begin{tabular}{|m{2.8em}|m{2.8em}|m{2.8em}|m{2.8em}|m{2.8em}|m{2.8em}|m{2.8em}|m{2.8em}|m{2.8em}|m{2.8em}|}
			\hline
			1. A & 2. A & 3. B & 4. A & 5. C & 6. B & 7. A & 8. C & 9. B & 10. D \\
			\hline
		\end{tabular}
\end{center}}