% --- chapter
\newcommand{\chapter}[2][]{
	\newcommand{\chapname}{#2}
	\begin{flushleft}
		\begin{minipage}[t]{\linewidth}
			\includegraphics[height=1cm]{hdht-logo.png}
			\hspace{0pt}	
			\sffamily\bfseries\large Bài 35. Tính chất và cấu tạo hạt nhân
			\begin{flushleft}
				\huge\bfseries #1
			\end{flushleft}
		\end{minipage}
	\end{flushleft}
	\vspace{1cm}
	\normalfont\normalsize
}
%-----------------------------------------------------
\chapter[Tính chất và cấu tạo hạt nhân]{Tính chất và cấu tạo hạt nhân}

\subsection{Cấu tạo hạt nhân}
Hạt nhân có kích thước rất nhỏ (nhỏ hơn kích thước nguyên tử $10^4\div 10^5$ lần) và được cấu tạo bởi những hạt nhỏ hơn gọi là nuclon. Có hai loại nuclon là: proton và nơtron.
\begin{itemize}
	\item proton mang điện tích dương $+e=\SI{1,6e-19}{\coulomb}$, có kí hiệu là $p$.
	\item nơtron không mang điện, có kí hiệu là $n$.
\end{itemize}
\subsection{Kí hiệu hạt nhân}
Hạt nhân được kí hiệu như sau:
\begin{equation}
^A_Z X,
\end{equation}
trong đó:
\begin{itemize}
	\item $X$ là tên hạt nhân;
	\item $A$ = số nuclon: số khối;
	\item $Z$ = số proton = điện tích hạt nhân (nguyên tử số);
	\item $N =A Z$: số nơtron.
\end{itemize}
\subsection{Bán kính hạt nhân}
Bán kính hạt nhân được xác định bởi:
\begin{equation}
R=1,2\cdot 10^{-15} \cdot A^{\frac{1}{3}}\, \text{m},
\end{equation}
trong đó:
\begin{itemize}
	\item R là bán kính hạt nhân;
	\item A số khối.
\end{itemize}
Ví dụ: Bán kính hạt nhân $^{27}_{13}\text{Al}$ là $R=1,2\cdot 10^{-15} \cdot 27^{\frac{1}{3}}\, \text{m}=\SI{3,6e-15}{\meter}$.

\subsection{Lực hạt nhân}
Hạt nhân được cấu tạo bởi hạt mang điện và không mang điện nhưng chúng vô cùng bền vững,
chứng tỏ rằng các nuclon liên kết với nhau bởi lực rất mạnh, gọi là lực hạt nhân.
\begin{itemize}
	\item Lực hạt nhân có tác dụng liên kết các nuclon với nhau.
	\item Lực hạt nhân không phải lực tĩnh điện, không phụ thuộc vào điện tích của các nuclon.
	\item Lực hạt nhân là lực hút, có bán kính tác dụng trong phạm vi hạt nhân nguyên tử (khoảng $\SI{e-15}{\meter}$) và cường độ rất lớn (còn gọi là lực tương tác mạnh) so với lực điện từ, lực hấp dẫn.
\end{itemize}

\subsection{Đồng vị}
Các hạt nhân đồng vị là những hạt nhân có cùng số proton ($Z$), nhưng khác số nơtron ($N$), dẫn tới khác số nuclon ($A$).

Ví dụ: hiđrô có ba đồng vị $^1_1\text{H}$; $^2_1\text{H}$ ($^2_1\text{D}$); $^3_1\text{H}$ ($^3_1\text{T}$).

\subsection{Đơn vị khối lượng hạt nhân}
Để thuận tiện tính toán ở cấp độ hạt nhân, người ta định nghĩa đơn vị khối lượng nguyên tử, ký hiệu là u và có giá trị bằng $\frac{1}{12}$ khối lượng đồng vị cacbon $^{12}_{\ 6}\text{C}$
\begin{equation}
1\text{u}=\SI{1,6605e-27}{\kilogram}.
\end{equation}
