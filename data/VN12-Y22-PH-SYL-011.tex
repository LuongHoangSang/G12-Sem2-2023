
\chapter[Dạng bài: Mối liên hệ giữa các đại lượng trong dao động điều hòa]{Dạng bài: Mối liên hệ giữa các đại lượng trong dao động điều hòa}
\section{Lý thuyết}
\subsection{Mối liên hệ giữa chu kì, tần số, tần số góc của dao động điều hòa}
\begin{equation*}
	\omega=\dfrac{2\pi}{T}=2\pi f
\end{equation*}
\subsection{Mối liên hệ giữa li độ, vận tốc và gia tốc trong dao động điều hòa}
\subsubsection{Mối liên hệ giữa li độ và vận tốc trong dao động điều hòa}
\begin{equation*}
	x^2+\dfrac{v^2}{\omega^2}=A^2,
\end{equation*}
trong đó:
\begin{itemize}
	\item $x$ là li độ của dao động tại thời điểm $t$;
	\item $v$ là vận tốc của dao động tại thời điểm $t$;
	\item $\omega$ là tần số góc của dao động;
	\item $A$ là biên độ dao động.
\end{itemize}
\subsubsection{Mối liên hệ giữa vận tốc và gia tốc trong dao động điều hòa}
\begin{equation*}
	v^2+\dfrac{a^2}{\omega^2}=\omega^2A^2=v^2_\text{max},
\end{equation*}
trong đó:
\begin{itemize}
	\item $v$ là vận tốc của dao động tại thời điểm $t$;
	\item $a$ là gia tốc của dao động tại thời điểm $t$;
	\item $\omega$ là tần số góc của dao động;
	\item $A$ là biên độ dao động;
	\item $v_\text{max}$ là giá trị vận tốc cực đại.
\end{itemize}
\subsubsection{Mối liên hệ giữa gia tốc và li độ trong dao động điều hòa}
\begin{equation*}
	a=-\omega^2x,
\end{equation*}
trong đó:
\begin{itemize}
	\item $a$ là gia tốc của dao động tại thời điểm $t$;
	\item $\omega$ là tần số góc của dao động;
	\item $x$ là li độ của dao động tại thời điểm $t$.
\end{itemize}
\subsection{Bảng tóm tắt các đại lượng trong dao đông điều hòa đã học}
\begin{center}
	\includegraphics[scale=0.5]{../figs/VN12-PH-02-A-001-3-V2-1.png}
\end{center}
\section{Mục tiêu bài học - Ví dụ minh họa}
\begin{dang}{Xây dựng được mối liên hệ\\ giữa các đại lượng\\ trong dao động điều hòa}
	\viduii{3}{Một vật dao động điều hòa với biên độ $A=\SI{5}{cm}$. Trong 10 giây vật thực hiện được 20 dao động. Xác định phương trình dao động của vật, biết rằng tại thời điểm ban đầu vật chuyển động qua vị trí cân bằng theo chiều dương.
		\begin{mcq}(2)
			\item $x= 5\cos \left (4\pi t +\dfrac{\pi}{2}\right)\ \text{cm}$.
			\item $x= 5\cos \left (4\pi t -\dfrac{\pi}{2}\right)\ \text{cm}$.
			\item $x= 5\cos \left (2\pi t +\dfrac{\pi}{2}\right)\ \text{cm}$.
			\item $x= 5\cos \left (2\pi t +\dfrac{\pi}{2}\right)\ \text{cm}$.
		\end{mcq}
	}
	{\begin{center}
			\textbf{Hướng dẫn giải}
		\end{center}
		
		Ta có
		\begin{equation*}
			f = \dfrac{N}{t} = \SI{2}{Hz} \Rightarrow \omega =2\pi f = 4\pi\ \text{rad/s}.
		\end{equation*}
		
		Biên độ dao động
		\begin{equation*}
			A = \SI{5}{cm}.
		\end{equation*}
		Khi $t=0$ vật đang ở vị trí cân bằng theo chiều dương.
		
		\begin{equation*}
			x=5\cos \varphi =0; v>0.
		\end{equation*}
		
		Suy ra
		\begin{equation*}
			\varphi = - \dfrac{\pi}{2}.
		\end{equation*}
		
		Phương trình dao động của vật là
		
		\begin{equation*}
			x=5\cos \left(4\pi t - \dfrac{\pi}{2}\right)\ \text{cm}.
		\end{equation*}
		
		\textbf{Đáp án: B.}
	}
	\viduii{2}{Trong dao động điều hòa, gia tốc biến đổi
		\begin{mcq}(2)
			\item cùng pha so với li độ.
			\item ngược pha so với vận tốc.
			\item sớm pha $\pi / 2$ so với li độ.
			\item sớm pha $\pi / 2$ so với vận tốc.
		\end{mcq}
	}
	{\begin{center}
			\textbf{Hướng dẫn giải}
		\end{center}
		
		Phương trình tổng quát của vận tốc trong dao động điều hòa là $$v=v_\text{max}\cos \left( \omega t+\varphi+\dfrac{\pi}{2}\right)$$
		Phương trình tổng quát của gia tốc trong dao động điều hòa là $$a=a_\text{max}\cos \left( \omega t+\varphi+ \pi\right)$$
		
		Do đó gia tốc biến đổi sớm pha $\pi / 2$ so với vận tốc.
		
		\textbf{Đáp án: D.}
	}
	
\end{dang}
\begin{dang}{Sử dụng được các hệ thức\\ độc lập thời gian}
	\viduii{3}{Một chất điểm dao động điều hòa trên trục $\text{O}x$ với chu kì $\SI{0,2}{\second}$. Lấy gốc thời gian là lúc chất điểm đi qua vị trí có li độ $\SI{2}{\centi\meter}$ ngược chiều dương với tốc độ $20\pi\, \text{cm/s}$. Phương trình dao động của chất điểm là
		\begin{mcq}(2)
			\item $x=2\sqrt{2}\cos\left(10\pi t+\dfrac{\pi }{4} \right)\,\text{cm}$.
			\item $x=2\sqrt{2}\cos \left(10\pi t-\dfrac{\pi }{4} \right)\,\text{cm}$.
			\item $x=2\sqrt{2}\cos \left(10\pi t+\dfrac{3\pi }{4} \right)\,\text{cm}$.
			\item $x=2\sqrt{2}\cos \left(10\pi t+\dfrac{3\pi }{4} \right)\,\text{cm}$.
		\end{mcq}
		
	}
	{\begin{center}
			\textbf{Hướng dẫn giải}
		\end{center}
		
		\begin{description}
			\item[Bước 1:] Xác định tần số góc $\omega$
			
			\begin{equation*}
				\omega =\frac{2\pi }{T}=\frac{2\pi }{\SI{0,2}{\second}}=10\pi \,\text{rad/s}.
			\end{equation*}
			
			\item[Bước 2:] Xác định biên độ $A$
			
			Gốc thời gian là lúc chất điểm đi qua vị trí có li độ $\SI{0,2}{\centi\meter}$ ngược chiều dương với tốc độ $20\pi\, \text{cm/s}$ nên $x_0=\SI{2}{\centi\meter}$ và $v_0=-20\pi\, \text{cm/s}$.
			
			Áp dụng công thức độc lập thời gian
			\begin{equation*}
				A=\sqrt{x^2+\dfrac{v^2}{\omega ^2}}=\sqrt{(\SI{2}{\centi\meter})^2+\dfrac{(-20\pi\, \text{cm/s} )^2}{(10\pi \,\text{rad/s} )^2}}=2\sqrt{2}\,\text{cm}.
			\end{equation*}
			
			\item[Bước 3:] Xác định pha ban đầu $\varphi$
			
			\begin{equation*}
				\cos\varphi =\dfrac{x_{0}}{A}=\frac{1}{\sqrt{2}}\Rightarrow\varphi =\pm\dfrac{\pi }{4}
			\end{equation*}
			
			Vì tại thời điểm $t=0$ chất điểm đi theo chiều âm nên loại $\varphi =-\dfrac{\pi }{4}$.
			
			Do đó, $\varphi =\dfrac{\pi }{4}$.
		\end{description}
		
		Vậy phương trình dao động của chất điểm là
		\begin{equation*}
			x=2\sqrt{2}\cos\left(10\pi t+\dfrac{\pi }{4} \right)\,\text{cm}.
		\end{equation*}
		
		\textbf{Đáp án: A.}
	}
	
	\viduii{3}{Một vật dao động điều hòa với tần số góc $\omega =10\sqrt 2\ \text{rad/s}$. Chọn gốc thời gian $t=0$ lúc vật có li độ $x=2\sqrt 3\ \text{cm}$ với vận tốc $\text{0,2}\sqrt 2\ \text{m/s}$ theo chiều dương. Lấy $g=10\ \text{m/s}^2$. Phương trình dao động của vật có dạng
		\begin{mcq}(2)
			\item $x=4\cos \left(10\sqrt 2 t+\dfrac{\pi }{6} \right)\,\text{cm}$.
			\item $x=4\cos \left(10\sqrt 2 t+\dfrac{2\pi }{3} \right)\,\text{cm}$.
			\item $x=4\cos \left(10\sqrt 2 t-\dfrac{\pi }{6} \right)\,\text{cm}$.
			\item $x=4\cos \left(10\sqrt 2 t+\dfrac{\pi }{3} \right)\,\text{cm}$.
		\end{mcq}
		
	}
	{\begin{center}
			\textbf{Hướng dẫn giải}
		\end{center}
		
		\begin{description}
			\item[Bước 1:] Xác định tần số góc $\omega$
			
			\begin{equation*}
				\omega =10\sqrt 2 \,\text{rad/s}.
			\end{equation*}
			
			\item[Bước 2:] Xác định biên độ $A$
			
			Lúc $t=0$ vật ở vị trí M$_0$ có $x=2\sqrt 3\ \text{cm}$; $v=20\sqrt 2\ \text{cm/s}$.
			
			Áp dụng công thức độc lập thời gian
			\begin{equation*}
				A=\sqrt{x^2+\dfrac{v^2}{\omega ^2}}=4\,\text{cm}.
			\end{equation*}
			
			\item[Bước 3:] Xác định pha ban đầu $\varphi$
			
			\begin{equation*}
				\cos\varphi =\dfrac{x}{A}=\dfrac{2\sqrt 3}{4}\Rightarrow\varphi =\pm\dfrac{\pi }{6}
			\end{equation*}
			
			Vì tại thời điểm $t=0$ chất điểm đi theo chiều dương nên loại $\varphi =\dfrac{\pi }{6}$.
			
			Do đó, $\varphi =-\dfrac{\pi }{6}$.
		\end{description}
		
		Vậy phương trình dao động của chất điểm là
		\begin{equation*}
			x=4\cos \left(10\sqrt 2 t-\dfrac{\pi }{6} \right)\,\text{cm}.
		\end{equation*}
		
		\textbf{Đáp án: C.}
	}
\end{dang}