\whiteBGstarBegin
\setcounter{section}{0}
\section{Lý thuyết: Hiện tượng quang - phát quang}
\begin{enumerate}[label=\bfseries Câu \arabic*:]

% Câu No 
		\item \mkstar{3} [13] 
	
		\cauhoi
		{Một chất có khả năng phát ra ánh sáng phát quang với tần số $ \SI{6e14}{Hz} $. Khi dùng các ánh sáng có bước sóng nào dưới đây để kích thích thì chất này \textbf{không} thể phát quang
		\begin{mcq}(2)
			\item $ \lambda_{1} = \SI{0,58}{\mu m}; \lambda_{2} = \SI{0,76}{\mu m} $. 
			\item $ \lambda_{1} = \SI{0,42}{\mu m}; \lambda_{2} = \SI{0,46}{\mu m} $.
			\item $ \lambda_{1} = \SI{0,38}{\mu m}; \lambda_{2} = \SI{0,46}{\mu m} $. 
			\item $ \lambda_{1} = \SI{0,40}{\mu m}; \lambda_{2} = \SI{0,45}{\mu m} $. 
		\end{mcq}
		}
	
		\loigiai
		{		\textbf{Đáp án: A.}
		
Để không xảy ra hiện tượng phát quang thì $ \lambda \geq \lambda_{0} $. Với $ \lambda_{0} $ cho bởi:
$$
	\lambda_{0} = \dfrac{c}{f_{0}} = \SI{0,5}{\mu m}.
$$	
		}
		
	
\end{enumerate}

\loigiai
{
	\begin{center}
		\textbf{BẢNG ĐÁP ÁN}
	\end{center}
	\begin{center}
		\begin{tabular}{|m{2.8em}|m{2.8em}|m{2.8em}|m{2.8em}|m{2.8em}|m{2.8em}|m{2.8em}|m{2.8em}|m{2.8em}|m{2.8em}|}
			\hline
			01.A  & &  &  &  &  & & &  &  \\
			\hline
			
		\end{tabular}
	\end{center}
}

\section{Dạng bài: Hiệu suất phát quang, hiệu suất lượng tử}
\begin{enumerate}[label=\bfseries Câu \arabic*:]

% Câu No 
		\item \mkstar{3} [3] 
	
		\cauhoi
		{Một chất phát quang được kích thích bằng ánh sáng có bước sóng $ \SI{0,32}{\mu m} $ thì phát ra ánh sáng có bước sóng $\lambda $. Giả sử công suất chùm sáng phát quang bằng 20\% công suất chùm sáng kích thích. Biết tỉ số giữa photon ánh sáng phát quang và photon ánh sáng kích thích trong cùng một khoảng thời gian là $ \num{0,4} $. Bước sóng của ánh sáng phát quang $ \lambda $ là
		\begin{mcq}(4)
			\item $ \SI{0,32}{\mu m} $. 
			\item $ \SI{1,28}{\mu m} $.
			\item $ \SI{0,64}{\mu m} $. 
			\item $ \SI{0,16}{\mu m} $. 
		\end{mcq}
		}
	
		\loigiai
		{		\textbf{Đáp án: C.}
		
Ta có hiệu suất phát quang cho bởi
$$
	H = \dfrac{n' \varepsilon'}{n \varepsilon} = \dfrac{n'}{n} \cdot \dfrac{\lambda}{\lambda'}.
$$	
Trong đó $ H = 0,2 $; $ \dfrac{n'}{n} = \num{0,4}$ và $ \lambda = \SI{0,32}{\mu m}$.
Từ đó, suy ra $ \lambda' = \SI{0,64}{\mu m}$.
		}
		
% Câu No 
		\item \mkstar{3}[13]
	
		\cauhoi
		{Một đèn laze phát ánh sáng đơn sắc có bước sóng $ \SI{0,7}{\mu m} $, có công suất phát sáng là $ \SI{1}{W} $. Số photon đèn phát ra trong 1 giây bằng
		\begin{mcq}(4)
			\item $ \num{3,52e18} $. 
			\item $ \num{2,84e19} $.
			\item $ \num{2,84e-19} $. 
			\item $ \num{3,52e19} $. 
		\end{mcq}
		}
	
		\loigiai
		{		\textbf{Đáp án: A.}
		
Số photon đèn phát ra trong một giây bằng
$$
	N = \dfrac{P}{\varepsilon} = \dfrac{P \lambda}{hc} = \num{3,52e18}.
$$	
		}
		
	
\end{enumerate}

\loigiai
{
	\begin{center}
		\textbf{BẢNG ĐÁP ÁN}
	\end{center}
	\begin{center}
		\begin{tabular}{|m{2.8em}|m{2.8em}|m{2.8em}|m{2.8em}|m{2.8em}|m{2.8em}|m{2.8em}|m{2.8em}|m{2.8em}|m{2.8em}|}
			\hline
			01.C  & 02.A  &  &  &  &  & &  & &  \\
			\hline
			
		\end{tabular}
	\end{center}
}


\whiteBGstarEnd