
\chapter[Lý thuyết: Sóng âm;\\
Lý thuyết: Âm thanh nghe được, hạ âm, siêu âm]{Lý thuyết: Sóng âm;\\Lý thuyết: Âm thanh nghe được, hạ âm, siêu âm}
\section{Lý thuyết}
\subsection{Sóng âm. Nguồn âm}
\subsubsection{Sóng âm}
Sóng âm là những sóng cơ truyền trong các môi trường rắn, lỏng, khí. 
\subsubsection{Nguồn âm}			
Một vật dao động phát ra âm gọi là nguồn âm.
\subsection{Sự truyền âm}
\subsubsection{Môi trường truyền âm}
\begin{itemize}
	\item Âm truyền được qua các chất rắn, lỏng, và khí.
	\item Âm không truyền được trong chân không.
\end{itemize}
\subsubsection{Tốc độ truyền âm}
\begin{itemize}
	\item Trong mỗi môi trường, âm truyền với một tốc độ hoàn toàn xác định.
	\item Vận tốc truyền âm trong chất rắn, lớn hơn vận tốc truyền âm trong chất lỏng, lớn hơn vận tốc truyền âm trong chất khí.
\end{itemize}
\subsection{Những đặc trưng vật lý của âm} 
\subsubsection{Tần số âm}	
Tần số là một trong những đặc trưng vật lý quan trọng nhất của âm. Trong đó:
\begin{itemize}
	\item Nhạc âm là âm có tần số xác định;
	\item Tạp âm là âm không có tần số xác định.
\end{itemize}
\subsubsection{Cường độ âm và mức cường độ âm}
\begin{enumerate}[label=\alph*)]
	\item \textbf{Cường độ âm}	\\
	Cường độ âm $I$ tại một điểm là đại lượng đo bằng năng lượng mà sóng âm tải qua một đơn vị diện tích đặt tại đó, vuông góc với phương truyền sóng trong một đơn vị thời gian.
	
	Đơn vị cường độ âm là oát trên mét vuông, kí hiệu là $\SI{}{W/\meter^2}$.
	\item \textbf{Mức cường độ âm}	
	
	Để so sánh độ to của một âm với độ to âm chuẩn, người ta dùng đại lượng mức cường độ âm. Mức cường độ âm được định nghĩa bằng công thức
	\begin{equation*}
		L=\log\dfrac{I}{I_0},
	\end{equation*}
	trong đó:
	\begin{itemize}
		\item $L$ là mức cường độ âm, có đơn vị là Ben (B),
		\item $I$ là cường độ âm tại điểm đang xét,
		\item $I_0=\SI{e-12}{W/\meter^2}$ ;à cường độ âm chuẩn.
	\end{itemize}
	Trong thực tế, người ta thường đùng đơn vị đêxiben (dB), với $\SI{1}{B}=\SI{10}{dB}$. Công thức tính mức cường độ âm theo đơn vị đêxiben là
	\begin{equation*}
		L=10\log\dfrac{I}{I_0}.
	\end{equation*}	
\end{enumerate}
\subsubsection {Đồ thị dao động âm}
\begin{itemize}
	\item Tổng hợp đồ thị dao động của tất cả các họa âm trong một nhạc âm ta được đồ thị dao động của nhạc âm đó.
	\item Đồ thị dao động của cùng một nhạc âm do các nhạc cụ khác phát ra thì hoàn toàn khác nhau.
\end{itemize}	
\subsection{Âm nghe được}
Âm nghe được là các âm có tần số nằm trong khoảng từ $\SI{16}{Hz}$ đến $\SI{20000}{Hz}$. Các sóng cơ có tần số trong khoảng này gây cảm giác âm đối với tai người.
\subsection{Hạ âm}
Hạ âm là các âm có tần số nhỏ hơn $\SI{16}{Hz}$, tai người không nghe được. Một số loài vật như voi, chim bồ câu,... có thể "nghe" được hạ âm.
\subsection{Siêu âm}
Siêu âm là những âm có tần số lớn hơn $\SI{20000}{Hz}$, tai người không nghe được. Một số loài vật như dơi, chó, cá heo,... có thể "nghe" được siêu âm.
\section{Mục tiêu bài học - Ví dụ minh họa}
\begin{dang}{Ghi nhớ được định nghĩa sóng âm}
	\viduii{1}{Chọn câu trả lời \textbf{sai}.
		\begin{mcq}
			\item Sóng âm là những sóng cơ học dọc lan truyền trong môi trường vật chất.
			\item Sóng âm, sóng siêu âm, sóng hạ âm về phương diện vật lí có cùng bản chất.
			\item Sóng âm truyền được trong mọi môi trường vật chất đàn hồi kể cả chân không.
			\item Vận tốc truyền âm trong chất rắn thường lớn hơn trong chất lỏng và trong chất khí.
	\end{mcq}}	
	{\begin{center}
			\textbf{Hướng dẫn giải}
		\end{center}
		
		Phát biểu sai: Sóng âm truyền được trong mọi môi trường vật chất đàn hồi kể cả chân không. Lý do: sóng âm không truyền được trong chân không.
		
		\textbf{Đáp án: C.}
		
	}
	\viduii{1}{Khi nói về sóng âm, phát biểu nào sau đây là \textbf{sai}?
		\begin{mcq}
			\item Sóng âm truyền được cả trong chân không.
			\item Sóng âm truyền được trong các môi trường rắn, lỏng, khí.
			\item Sóng âm là sóng cơ học.
			\item Sóng âm trong không khí là sóng dọc.
	\end{mcq}}	
	{\begin{center}
			\textbf{Hướng dẫn giải}
		\end{center}
		
		Phát biểu sai: Sóng âm truyền được cả trong chân không. Lý do: sóng âm không truyền được trong chân không.
		
		\textbf{Đáp án: A.}
		
	}
\end{dang}
\begin{dang}{Liệt kê được những đặc trưng vật lý\\ của sóng âm}
	\viduii{1}{Đơn vị đo mức cường độ âm là
		\begin{mcq}(2)
			\item Ben $(\text B)$.
			\item Niu-tơn trên mét vuông $(\SI{}{N/m^2})$.
			\item Oát trên mét $(\SI{}{W/m})$.
			\item Oát trên mét vuông $(\SI{}{W/m^2})$.
	\end{mcq}}	
	{\begin{center}
			\textbf{Hướng dẫn giải}
		\end{center}
		Đơn vị đo mức cường độ âm là Ben $(\text B)$.
		
		\textbf{Đáp án: A.}
		
	}
	\viduii{1}{Đại lượng nào sau đây \textbf{không} phải là đặc trưng vật lý của âm?
		\begin{mcq}(2)
			\item Cường độ âm.
			\item Mức cường độ âm.
			\item Độ cao của âm.
			\item Tần số âm.
	\end{mcq}}
	{\begin{center}
			\textbf{Hướng dẫn giải}
		\end{center}
		Độ cao của âm không phải là đặc trưng vật lý của âm.
		
		Độ cao là đặc trưng sinh lý của âm.
		
		\textbf{Đáp án: C.}
	}
\end{dang}
\begin{dang}{Nhận biết được đặc điểm của sóng âm\\ khi truyền trong các môi trường}
	\viduii{1}{Sóng âm lần lượt truyền trong các môi trường: kim loại, nước và không khí. Tốc độ truyền âm có giá trị
		\begin{mcq}
			\item lớn nhất khi truyền trong nước và nhỏ nhất khi truyền trong không khí.
			\item lớn nhất khi truyền trong kim loại và nhỏ nhất khi truyền trong không khí.
			\item lớn nhất khi truyền trong nước và nhỏ nhất khi truyền trong không khí.
			\item như nhau khi truyền trong ba môi trường.
	\end{mcq}}	
	{\begin{center}
			\textbf{Hướng dẫn giải}
		\end{center}
		Tốc độ truyền âm có giá trị lớn nhất khi truyền trong kim loại và nhỏ nhất khi truyền trong không khí.
		
		\textbf{Đáp án: B.}
		
	}
	\viduii{2}{Sóng âm lần lượt truyền trong các môi trường: kim loại, nước và không khí. Tần số âm có giá trị
		\begin{mcq}
			\item lớn nhất khi truyền trong nước và nhỏ nhất khi truyền trong không khí.
			\item lớn nhất khi truyền trong kim loại và nhỏ nhất khi truyền trong không khí.
			\item lớn nhất khi truyền trong nước và nhỏ nhất khi truyền trong không khí.
			\item như nhau khi truyền trong ba môi trường.
	\end{mcq}}	
	{\begin{center}
			\textbf{Hướng dẫn giải}
		\end{center}
		Tần số âm có giá trị như nhau khi truyền trong ba môi trường.
		
		\textbf{Đáp án: D.}
		
	}
\end{dang}
\begin{dang}{Ghi nhớ được định nghĩa âm nghe được, hạ âm, siêu âm}
	\viduii{1}{Âm nghe được là sóng cơ học có tần số khoảng
		\begin{mcq} (2)
			\item $\SI{16}{\hertz}$ đến $\SI{20}{\kilo\hertz}$.
			\item $\SI{16}{\hertz}$ đến $\SI{20}{\mega\hertz}$.
			\item $\SI{16}{\hertz}$ đến $\SI{200}{\kilo\hertz}$.
			\item $\SI{16}{\hertz}$ đến $\SI{2}{\kilo\hertz}$.
	\end{mcq}}	
	{\begin{center}
			\textbf{Hướng dẫn giải}
		\end{center}
		
		Âm nghe được là các âm có tần số nằm trong khoảng từ $\SI{16}{Hz}$ đến $\SI{20000}{Hz}$.
		
		\textbf{Đáp án: A.}
		
	}
	\viduii{1}{Sự phân biệt âm thanh với hạ âm và siêu âm dựa trên
		\begin{mcq}
			\item bản chất vật lí của chúng khác nhau.
			\item bước sóng và biên độ dao động của chúng.
			\item khả năng cảm thụ sóng âm của tai người.
			\item một lí do khác.
	\end{mcq}}	
	{\begin{center}
			\textbf{Hướng dẫn giải}
		\end{center}
		
		Sự phân biệt âm thanh với hạ âm và siêu âm dựa trên khả năng cảm thụ sóng âm của tai người.
		
		\textbf{Đáp án: C.}
		
	}
\end{dang}
\begin{dang}{Phát hiện được âm thanh nghe được,\\ hạ âm, siêu âm}
	\viduii{3}{Một âm có bước sóng $\SI{1}{cm}$ lan truyền trong không khí với tốc độ $\SI{330}{m/s}$. Âm đó là
		\begin{mcq}(2)
			\item âm nghe được.
			\item sóng ngang.
			\item hạ âm.
			\item siêu âm.
	\end{mcq}}	
	{\begin{center}
			\textbf{Hướng dẫn giải}
		\end{center}
		
		Tần số âm là
		$$f=\dfrac{v}{\lambda} = \SI{33000}{Hz}$$
		
		Âm đó là siêu âm
		
		\textbf{Đáp án: D.}
		
	}
	\viduii{3}{Sóng cơ truyền trong không khí với cường độ đủ lớn, tai ta có thể cảm thụ được sóng cơ học nào sau đây?
		\begin{mcq}(2)
			\item Sóng có tần số $\SI{30000}{Hz}$.
			\item Sóng có chu kì $\SI{2}{\micro s}$.
			\item Sóng có chu kì $\SI{2}{ms}$.
			\item Sóng có tần số $\SI{10}{Hz}$.
	\end{mcq}}	
	{\begin{center}
			\textbf{Hướng dẫn giải}
		\end{center}
		Sóng có tần số $\SI{30000}{Hz}$ tai người không nghe được.
		
		Sóng có tần số $\SI{10}{Hz}$ tai người không nghe được.
		
		Sóng có chu kì $\SI{2}{\micro s}$ thì tần số $f=\dfrac{1}{T}=\SI{500000}{Hz}$ tai người không nghe được.
		
		Sóng có chu kì $\SI{2}{ms}$ thì tần số $f=\dfrac{1}{T}=\SI{500}{Hz}$ tai người nghe được.
		
		\textbf{Đáp án: C.}
		
	}
\end{dang}