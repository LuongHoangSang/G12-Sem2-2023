
%-----------------------------------------------------
\chapter[Hiện tượng quang - phát quang]{Hiện tượng quang - phát quang}
\section{Lý thuyết}
\subsection{Hiện tượng quang phát quang}
\subsubsection{Khái niệm về sự phát quang}
Một số chất trong tự nhiên có khả năng tự phát sáng, những chất đó gọi là chất phát quang và sự phát sáng đó gọi là sự phát quang.

Ví dụ: đom đóm (hóa phát quang), đèn LED (điện phát quang), lớp huỳnh quang ở đèn ống (quang phát quang).
\subsubsection{Hiện tượng quang phát quang}
\begin{itemize}
	\item Hiện tượng quang - phát quang là hiện tượng một số chất có khả năng hấp thụ ánh sáng có bước sóng này để phát ra ánh sáng có bước sóng khác.
	
	Ví dụ: khi chiếu chùm tia tử ngoại vào dung dịch fluorescein (màu vàng nhạt) thì sẽ phát ánh sáng màu xanh lục.
	\item Đặc điểm quan trọng của sự phát quang là nó còn kéo dài một thời gian sau khi tắt ánh sáng kích thích. Thời gian này dài hay ngắn khác nhau phụ thuộc vào chất phát quang.
\end{itemize}
\subsubsection{Huỳnh quang và lân quang}
\begin{itemize}
	\item Huỳnh quang là sự phát quang của chất lỏng và chất khí, có đặc điểm là ánh sáng phát quang bị tắt rất nhanh sau khi tắt ánh sáng kích thích.
	\item Lân quang là sự phát quang của chất rắn, có đặc điểm là ánh sáng phát quang có thể kéo dài một khoảng thời gian nào đó sau khi tắt ánh sáng kích thích.
\end{itemize}
\subsection{Đặc điểm của ánh sáng huỳnh quang}
Ánh sáng huỳnh quang có bước sóng dài hơn bước sóng ánh sáng kích thích.
\begin{equation}
	\lambda_\text{hq} > \lambda_\text{kt}
\end{equation}
\section{Bài tập tự luyện}
\begin{enumerate}[label=\bfseries Câu \arabic*:]
	
	% Câu No 
	\item \mkstar{3} [13] 
	\cauhoi
	{Một chất có khả năng phát ra ánh sáng phát quang với tần số $ \SI{6e14}{Hz} $. Khi dùng các ánh sáng có bước sóng nào dưới đây để kích thích thì chất này \textbf{không} thể phát quang
		\begin{mcq}(2)
			\item $ \lambda_{1} = \SI{0,58}{\mu m}; \lambda_{2} = \SI{0,76}{\mu m} $. 
			\item $ \lambda_{1} = \SI{0,42}{\mu m}; \lambda_{2} = \SI{0,46}{\mu m} $.
			\item $ \lambda_{1} = \SI{0,38}{\mu m}; \lambda_{2} = \SI{0,46}{\mu m} $. 
			\item $ \lambda_{1} = \SI{0,40}{\mu m}; \lambda_{2} = \SI{0,45}{\mu m} $. 
		\end{mcq}
	}
	
	\loigiai
	{		\textbf{Đáp án: A.}
		
		Để không xảy ra hiện tượng phát quang thì $ \lambda \geq \lambda_{0} $. Với $ \lambda_{0} $ cho bởi:
		$$
		\lambda_{0} = \dfrac{c}{f_{0}} = \SI{0,5}{\mu m}.
		$$	
	}
	\item \mkstar{1}
	\cauhoi
	{Muốn một chất phát quang ra ánh sáng khả kiến có bước sóng $\lambda$ lúc được chiếu sáng thì
		\begin{mcq} 
			\item phải kích thích bằng ánh sáng có bước sóng $\lambda$.
			\item phải kích thích bằng ánh sáng có bước sóng nhỏ hơn $\lambda$.
			\item phải kích thích bằng ánh sáng có bước sóng lớn hơn $\lambda$.
			\item phải kích thích bằng tia hồng ngoại.
			
		\end{mcq}
	}
	
	\loigiai
	{		\textbf{Đáp án: B.}
		
		
	}
		\item \mkstar{1}
	\cauhoi
	{Chọn câu trả lời sai khi nói về sự phát quang?
		
		\begin{mcq}
			\item Sự huỳnh quang của chất khí, chất lỏng và sự lân quang của các chất rắn gọi là sự phát quang.
			
			\item Đèn huỳnh quang là việc áp dụng sự phát quang của các chất rắn.
			
			\item Sự phát quang còn được gọi là sự phát sáng lạnh.
			
			\item Khi chất khí được kích thích bởi ánh sáng có tần số $f$, sẽ phát ra ánh sáng có tần số $f'$  với $f'>f$.
		\end{mcq}
	}
	
	\loigiai
	{		\textbf{Đáp án: D.}
		
		
	}
		\item \mkstar{1}
		\cauhoi
	{Phát biểu nào sau đây sai khi nói về hiện tượng huỳnh quang?
		
		\begin{mcq}
			\item Hiện tượng huỳnh quang là hiện tượng phát quang của các chất khí bị chiếu ánh sáng kích thích.
			
			\item Khi tắt ánh sáng kích thích thì hiện tượng huỳnh quang còn kéo dài khoảng cách thời gian trước khi tắt.
			
			\item Năng lượng phôtôn phát ra từ hiện tượng huỳnh quang bao giờ cũng nhỏ hơn năng lượng phôtôn của ánh sáng kích thích.
			
			\item Huỳnh quang còn được gọi là sự phát sáng lạnh.
			
		\end{mcq}
	}
	
	\loigiai
	{		\textbf{Đáp án: B.}
		
		 Khi tắt ánh sáng kích thích thì hiện tượng huỳnh quang tắt ngay.
	}
		\item \mkstar{1}
		\cauhoi
	{Phát biểu nào sau đây sai khi nói về hiện tượng lân quang?
		
		\begin{mcq}
			\item Sự phát sáng của các tinh thể khi bị chiếu sáng thích hợp được gọi là hiện tượng lân quang.
			
			\item Nguyên nhân chính của sự lân quang là do các tinh thể phản xạ ánh sáng chiếu vào nó.
			\item Ánh sáng lân quang có thể tồn tại một thời gian sau khi tắt ánh sáng kích thích.
			
			\item  Hiện tượng lân quang là hiện tượng phát quang của chất rắn.
			
		\end{mcq}
	}
	
	\loigiai
	{		\textbf{Đáp án: B.}
		
		Nguyên nhân chính của sự lân quang là do các tinh thể hấp thụ ánh sáng chiếu vào nó và phát quang.
	}
		\item \mkstar{1}
		\cauhoi
	{Theo thuyết lượng tử ánh sáng, để phát ánh sáng huỳnh quang, mỗi nguyên tử hydroay phân tử của chất phát quang hấp thụ hoàn toàn một phôtôn của ánh sáng kích thích có năng lượng  để chuyển sang trạng thái kích thích, sau đó
		
		\begin{mcq}
			\item giải phóng một electron tự do có năng lượng lớn hơn $\varepsilon$ do có bổ sung năng lượng.
			
			\item giải phóng một electron tự do có năng lượng nhỏ hơn $\varepsilon$ do có mất mát năng lượng.
			
			\item phát ra một phôtôn khác có năng lượng lớn hơn $\varepsilon$ do có bổ sung năng lượng.
			\item phát ra một phôtôn khác có năng lượng nhỏ hơn $\varepsilon$  do có mất mát năng lượng.
			
		\end{mcq}
	}
	
	\loigiai
	{		\textbf{Đáp án: C.}
		
		
	}
		\item \mkstar{1}
	\cauhoi
	{Khi chiếu chùm tia tử ngoại vào một ống nghiệm đựng dung dịch fluorexêin thì thấy dung dịch này phát ra ánh sáng màu lục. Đó là hiện tượng
		
		\begin{mcq}(2)
			\item phản xạ ánh sáng. 	
			\item quang – phát quang.	
			
			\item hóa – phát quang. 	
			\item tán sắc ánh sáng.
			
		\end{mcq}
	}
	
	\loigiai
	{		\textbf{Đáp án: B.}
		
		
	}
		\item \mkstar{1}
		\cauhoi
	{Sự phát sáng của nguồn sáng nào dưới đây gọi là sự phát quang?
		
		\begin{mcq}(2)
			\item Ngọn nến.
			\item Đèn pin.
			\item Con đom đóm.
			\item Ngôi sao băng.
			
		\end{mcq}
	}
	
	\loigiai
	{		\textbf{Đáp án: B.}
		
		
	}
		\item \mkstar{1}
	\cauhoi
	{Ánh sáng lân quang
		
		\begin{mcq}
			\item là ánh sáng được phát ra bởi chất rắn, chất lỏng lẫn chất khí.
			
			\item hầu như tắt ngay sau khi tắt ánh sáng kích thích.
			
			\item có thể tồn tại một thời gian sau khi tắt ánh sáng kích thích.
			
			\item có bước sóng nhỏ hơn bước sóng ánh sáng kích thích.
			
		\end{mcq}
	}
	
	\loigiai
	{		\textbf{Đáp án: C.}
		
		
	}
		\item \mkstar{1}
	\cauhoi
	{Một chất phát quang có khả năng phát ra ánh sáng màu vàng khi được kích thích phát sáng. Hỏi khi chiếu vào chất đó ánh sáng đơn sắc nào dưới đây thì chất đó sẽ có thể phát quang?
		
		\begin{mcq}(4)
			\item Vàng.
			\item Lục.
			\item Đỏ.
			\item Da cam.
		\end{mcq}
	}
	
	\loigiai
	{		\textbf{Đáp án: B.}
		
		
	}
\end{enumerate}


