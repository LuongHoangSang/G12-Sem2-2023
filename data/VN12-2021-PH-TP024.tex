\whiteBGstarBegin
\setcounter{section}{0}
\section{Lý thuyết: Tán sắc ánh sáng}
\begin{enumerate}[label=\bfseries Câu \arabic*:]

%========================================
    \item \mkstar{1} [2]
    
	\cauhoi
	{Trong chân không, một ánh sáng đơn sắc có tần số $\xsi{4.10^{14}}{Hz}$. Biết chiết suất của ánh sáng này trong nước là $4/3$. Tần số của ánh sáng này trong nước là
		\begin{mcq}(4)
			\item \xsi{5,3.10^{14}}{Hz}. 
			\item \xsi{3,0.10^{14}}{Hz}. 
			\item \xsi{4,0.10^{14}}{Hz}. 
			\item \xsi{3,4.10^{14}}{Hz}. 
		\end{mcq}
	}
	
	\loigiai
	{		\textbf{Đáp án: C.}

Khi truyền từ môi trường này sang môi trường khác, tần số ánh sáng là không đổi.		
		
	}

%========================================
    \item \mkstar{1} [2]
    
	\cauhoi
	{Giới hạn quang điện của những kim loại kiềm như Ca, Na, K, Cs nằm trong vùng nào
		\begin{mcq}(1)
			\item Vùng tử ngoại và ánh sáng nhìn thấy. 
			\item Vùng tử ngoại. 
			\item Vùng hồng ngoại. 
			\item Ánh sáng nhìn thấy. 
		\end{mcq}
	}
	
	\loigiai
	{		\textbf{Đáp án: D.}
		
Giới hạn quang điện của những kim loại kiềm như Ca, Na, K, Cs nằm trong vùng ánh sáng nhìn thấy.
		
	}

%========================================
    \item \mkstar{1} [9]
    
	\cauhoi
	{Bước sóng của một trong các bức xạ màu lục có trị số là
		\begin{mcq}(4)
			\item \SI{55}{nm}. 
			\item \SI{0,55}{mm}. 
			\item \SI{0,55}{\mu m}. 
			\item \SI{0,55}{nm}. 
		\end{mcq}
	}
	
	\loigiai
	{		\textbf{Đáp án: C.}
		
Bước sóng của một trong các bức xạ màu lục có trị số là $\SI{0,55}{\mu m}$.
		
	}

%========================================
    \item \mkstar{1} [7]
    
	\cauhoi
	{Hiện tượng tán sắc ánh sáng thực chất là hiện tượng
		\begin{mcq}(1)
			\item tạo thành chùm ánh sáng trắng từ sự hoà trộn của các chùm ánh sáng đơn sắc.
			\item tạo thành chùm ánh sáng đơn sắc từ sự phân tích chùm ánh sáng trắng.
			\item đổi màu của các tia sáng.
			\item chùm sáng trắng bị mất đi một số màu.
		\end{mcq}
	}
	
	\loigiai
	{		\textbf{Đáp án: B.}
		
Hiện tượng tán sắc ánh sáng thực chất là hiện tượng tạo thành chùm ánh sáng đơn sắc từ sự phân tích chùm ánh sáng trắng.
		
	}

%========================================
    \item \mkstar{1} [7]
    
	\cauhoi
	{Cầu vồng là kết quả của hiện tượng
		\begin{mcq}(2)
			\item tán sắc ánh sáng. 
			\item nhiễu xạ ánh sáng. 
			\item khúc xạ ánh sáng. 
			\item giao thoa ánh sáng. 
		\end{mcq}
	}
	
	\loigiai
	{		\textbf{Đáp án: A.}
		
Cầu vồng là kết quả của hiện tượng tán sắc ánh sáng.
		
	}

%========================================
    \item \mkstar{1} [10]
    
	\cauhoi
	{Khi một chùm ánh sáng song song, hẹp truyền qua một lăng kính thì bị phân tách thành các chùm sáng đơn sắc khác nhau. Đây là hiện tượng
		\begin{mcq}(2)
			\item giao thoa ánh sáng. 
			\item tán sắc ánh sáng. 
			\item nhiễu xạ ánh sáng. 
			\item phản xạ ánh sáng. 
		\end{mcq}
	}
	
	\loigiai
	{		\textbf{Đáp án: B.}
		
Khi một chùm ánh sáng song song, hẹp truyền qua một lăng kính thì bị phân tách thành các chùm sáng đơn sắc khác nhau. Đây là hiện tượng tán sắc ánh sáng.
		
	}

%=======================	
	\item \mkstar{1} [1]
	
	\cauhoi
	{Chiết suất của một môi trường trong suốt đối với các ánh sáng đơn sắc đỏ, lục, tím lần lượt là $n_{d}, n_{l}, n_{t}$. Quan hệ nào sau đây là \textbf{đúng}?
		\begin{mcq}(2)
			\item $n_{l} > n_{d} > n_{t}$. 
			\item $n_{d} < n_{l} < n_{t}$. 
			\item $n_{d} > n_{l} > n_{t}$. 
			\item $n_{l} < n_{d} < n_{t}$. 
		\end{mcq}
	}
	
	\loigiai
	{		\textbf{Đáp án: B.}
		
Ánh sáng màu tím có chiết suất lớn nhất, màu đỏ nhỏ nhất. Nên mối quan hệ đúng là
$$
n_{d} < n_{l} < n_{t}.
$$
		
	}
	
%========================================
    \item \mkstar{1} [4]
    
	\cauhoi
	{Phát biểu nào sau đây là \textbf{sai}?
		\begin{mcq}(1)
			\item Trong chân không, mỗi ánh sáng đơn sắc có một bước sóng xác định. 
			\item Trong chân không, các ánh sáng đơn sắc khác nhau truyền với cùng một tốc độ. 
			\item Trong chân không, bước sóng của ánh sáng đỏ nhỏ hơn bước sóng của ánh sáng tím. 
			\item Trong ánh sáng trắng có vô số ánh sáng đơn sắc. 
		\end{mcq}
	}
	
	\loigiai
	{		\textbf{Đáp án: C.}
		
Trong chân không, bước sóng của ánh sáng đỏ nhỏ hơn bước sóng của ánh sáng tím là sai.
		
	}

%========================================
    \item \mkstar{1} [4]
    
	\cauhoi
	{Gọi $n_{d}, n_{t}$ và $n_{v}$ lần lượt là chiết suất của một môi trường trong suốt đối với các ánh sáng đơn sắc đỏ, tím và vàng. Sắp xếp nào sau đây là đúng?
		\begin{mcq}(2)
			\item $n_{d} < n_{v} < n_{t}$. 
			\item $n_{v} > n_{d} > n_{t}$. 
			\item $n_{d} > n_{t} > n_{v}$. 
			\item $n_{t} > n_{d} > n_{v}$. 
		\end{mcq}
	}
	
	\loigiai
	{		\textbf{Đáp án: A.}
		
Ánh sáng màu đỏ có chiết suất nhỏ nhất và ánh sáng màu tìm có chiết suất lớn nhất nên sắp xếp đúng là
$$
n_{d} < n_{v} < n_{t}.
$$
		
	}
	
%========================================
	\item \mkstar{1} [2]

	\cauhoi
	{Phát biểu nào sau đây là \textbf{sai} khi nói về ánh sáng đơn sắc?
		\begin{mcq}(1)
			\item Ánh sáng đơn sắc không bị tán sắc khi đi qua lăng kính. 
			\item Trong chân không bước sóng của ánh sáng đỏ nhỏ hơn bước sóng của ánh sáng tím. 
			\item Mỗi ánh sáng đơn sắc có tần số sóng xác định. 
			\item Trong chân không, các ánh sáng đơn sắc khác nhau truyền với cùng một tốc độ. 
		\end{mcq}
	}
	
	\loigiai
	{		\textbf{Đáp án: B.}
		
Trong chân không bước sóng của ánh sáng đỏ nhỏ hơn bước sóng của ánh sáng tím là sai.
		
	}
	
%========================================
    \item \mkstar{2} [2]
    
	\cauhoi
	{Chiếu xiên góc lần lượt bốn ánh sáng đơn sắc màu cam, màu lam, màu vàng, màu chàm từ không khí vào nước với cùng một góc tới. So với phương của tia tới, tia khúc xạ bị lệch ít nhất là tia màu
		\begin{mcq}(4)
			\item chàm. 
			\item cam. 
			\item vàng. 
			\item lam. 
		\end{mcq}
	}
	
	\loigiai
	{		\textbf{Đáp án: B.}
		
Tia bị lệch ít nhất là tia có chiết suất nhỏ nhất. Vậy nên tia màu cam bị lệch ít nhất.
		
	}
	
%========================================
    \item \mkstar{2} [3]
    
	\cauhoi
	{Chiếu xiên một chùm sáng hẹp gồm bốn ánh sáng đơn sắc đỏ, vàng, lam, tím từ trong nước ra không khí. Khi tia màu lam nằm là là mặt phân cách giữa hai môi trường thì
		\begin{mcq}(1)
			\item hai ánh sáng đỏ và tím bị phản xạ toàn phần. 
			\item so với phương tia tới, tia khúc xạ đỏ bị lệch ít hơn tia khúc xạ vàng.
			\item tia khúc xạ chỉ là ánh sáng tím, còn tia sáng đỏ và vàng bị phản xạ toàn phần. 
			\item so với phương tia tới, tia khúc xạ vàng bị lệch ít hơn tia khúc xạ đỏ. 
		\end{mcq}
	}
	
	\loigiai
	{		\textbf{Đáp án: B.}
		
Khi tia màu lam nằm là là mặt phân cách giữa hai môi trường thì tia màu đỏ và vàng bị khúc xạ, còn tia màu tím bị phản xạ toàn phần.\\
Do chiết suất của ánh sáng màu đỏ nhỏ hơn chiết suất của ánh sáng màu vàng nên so với phương tia tới, tia khúc xạ đỏ bị lệch ít hơn tia khúc xạ vàng.
		
	}
	
%========================================
    \item \mkstar{2} [7]
    
	\cauhoi
	{Gọi $n_{c}, n_{l}, n_{L}, n_{v}$ lần lượt là chiết suất của thủy tinh đối với các tia chàm, lam, lục và vàng. Sắp xếp theo thứ tự nào dưới đây là đúng
		\begin{mcq}(2)
			\item $n_{c} > n_{L} > n_{l} > n_{v}$. 
			\item $n_{c} < n_{L} < n_{l} < n_{v}$. 
			\item $n_{c} < n_{l} < n_{L} < n_{v}$. 
			\item $n_{c} > n_{l} > n_{L} > n_{v}$. 
		\end{mcq}
	}
	
	\loigiai
	{		\textbf{Đáp án: D.}
		
Trong các màu theo sắp xếp đỏ, cam, vàng, lục, lam, chàm, tím thì màu tím có chiết suất lớn nhất và màu đỏ nhỏ nhất. Vậy nên sắp xếp đúng phải là
$$
n_{c} > n_{l} > n_{L} > n_{v}.
$$
		
	}

\end{enumerate}

\loigiai
{
	\begin{center}
		\textbf{BẢNG ĐÁP ÁN}
	\end{center}
	\begin{center}
		\begin{tabular}{|m{2.8em}|m{2.8em}|m{2.8em}|m{2.8em}|m{2.8em}|m{2.8em}|m{2.8em}|m{2.8em}|m{2.8em}|m{2.8em}|}
			\hline
			01.C  & 02.D  & 03.C  & 04.B  & 05.A  & 06.B  & 07.B & 08.C & 09.A & 10.B \\
			\hline
			11.B  & 12.B  & 13.D  &       &       &       &      &      &      &      \\
			\hline
			
		\end{tabular}
	\end{center}
}

\section{Dạng bài: Tán sắc ánh sáng}
\begin{enumerate}[label=\bfseries Câu \arabic*:]

%========================================
    \item \mkstar{2} [10]
    
	\cauhoi
	{Một lăng kính thủy tinh có góc chiết quang $A = 4 ^\circ$, đặt trong không khí. Chiết suất của lăng kính đối với ánh sáng đỏ và tím lần lượt là 1,624 và 1,685. Chiếu một chùm tia sáng song song, hẹp gồm hai bức xạ đỏ và tím vào mặt bên của lăng kính theo phương vuông góc với mặt này. Góc tạo bởi tia đỏ và tia tím sau khi ló ra khỏi mặt bên kia của lăng kính xấp xỉ bằng
		\begin{mcq}(4)
			\item $4,258^\circ$. 
			\item $0,336^\circ$. 
			\item $0,244^\circ$. 
			\item $0,061^\circ$. 
		\end{mcq}
	}
	
	\loigiai
	{		\textbf{Đáp án: C.}
		
Góc lệch giữa tia màu đỏ và màu tím khi truyền qua lăng kính cho bởi:
$$
\Delta D = (n_{t} - n_{d})A = (1,685 - 1,624) \cdot 4^\circ = 0,244^\circ.
$$
		
	}

%========================================
    \item \mkstar{3} [3]
    
	\cauhoi
	{Từ không khí chiếu một chùm sáng hẹp (coi như một tia sáng) gồm hai bức xạ đơn sắc màu cam và màu tím tới mặt nước với góc tới $52^\circ$  thì xảy ra hiện tượng phản xạ và khúc xạ biết tia khúc xạ màu cam vuông góc với tia phản xạ, góc giữa tia khúc xạ màu tím và màu cam là $1,9^\circ$. Chiết suất của nước đối với ánh sáng màu tím là
		\begin{mcq}(4)
			\item $1,333$. 
			\item $1,349$. 
			\item $1,337$. 
			\item $1,306$. 
		\end{mcq}
	}
	
	\loigiai
	{		\textbf{Đáp án: C.}
		
Từ định luật khúc xạ ánh sáng ta có:
$$
i' = i = 52^\circ.
$$
Đối với tia màu cam, do tia phản xạ vuông góc với tia khúc xạ nên:
$$
i' + r_{c} = 90^\circ \rightarrow r_{c} = 38^\circ.
$$
Tia màu tím có chiết suất lớn hơn tia màu cam nên khi bị khúc xạ sẽ lệch gần pháp tuyến hơn. Ta có:
$$
r_{t} = r_{c} - 1,9^\circ = 36,1^\circ.
$$
Chiết suất của ánh sáng màu tím cho bởi:
$$
n_{t} = \dfrac{\sin{52^\circ}}{\sin{36,1^\circ}} = 1,337.
$$
		
	}

\end{enumerate}

\loigiai
{
	\begin{center}
		\textbf{BẢNG ĐÁP ÁN}
	\end{center}
	\begin{center}
		\begin{tabular}{|m{2.8em}|m{2.8em}|m{2.8em}|m{2.8em}|m{2.8em}|m{2.8em}|m{2.8em}|m{2.8em}|m{2.8em}|m{2.8em}|}
			\hline
			01.C  & 02.C  &   &   &   &   &  &  &  &  \\
			\hline
			
		\end{tabular}
	\end{center}
}


\whiteBGstarEnd
