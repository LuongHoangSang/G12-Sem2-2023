% --- chapter
\newcommand{\chapter}[2][]{
	\newcommand{\chapname}{#2}
	\begin{flushleft}
		\begin{minipage}[t]{\linewidth}
			\includegraphics[height=1cm]{hdht-logo.png}
			\hspace{0pt}	
			\sffamily\bfseries\large Bài  27. Tia hồng ngoại và tia tử ngoại
			\begin{flushleft}
				\huge\bfseries #1
			\end{flushleft}
		\end{minipage}
	\end{flushleft}
	\vspace{1cm}
	\normalfont\normalsize
}
%-----------------------------------------------------
\chapter[Tia hồng ngoại]{Tia hồng ngoại}

\subsection {Bản chất}
\begin{itemize}
	\item Tia hồng ngoại là những bức xạ không nhìn thấy được, có bước sóng lớn hơn bước sóng của ánh sáng đỏ ($> \text{0,76}\ \mu \text{m}$). 
	\item Thu được cùng với các tia sáng thông thường.
	\item Có cùng bản chất với ánh sáng.
\end{itemize}

\subsection{Tính chất chung}
\begin{itemize}
	\item Tuân theo các định luật: truyền thẳng, phản xạ, khúc xạ.
	\item Gây được hiện tượng nhiễu xạ, giao thoa như ánh sáng thông thường. 
\end{itemize}

\subsection{Nguồn phát}
\begin{itemize}
	\item Mọi vật có nhiệt độ cao hơn 0 K ($-273^\circ \text{C}$) đều phát ra tia hồng ngoại.
	\item Vật có nhiệt độ cao hơn môi trường xung quanh thì phát bức xạ hồng ngoại ra môi trường.
	\item Nguồn phát tia hồng ngoại thông dụng: bóng đèn dây tóc, bếp ga, bếp than, điôt hồng ngoại,...
\end{itemize}

\subsection{Tính chất và công dụng}

\subsubsection{Tính chất}

\begin{itemize}
	\item  Tác dụng nhiệt là tính chất nổi bật nhất. Vật hấp thụ tia hồng ngoại sẽ nóng lên.
	\item  Có khả năng gây ra một số phản ứng hóa học, có thể tác dụng lên phim ảnh.
	\item  Có thể biến điệu như sóng điện từ cao tần.
	\item  Gây ra hiện tượng quang điện trong với một số chất bán dẫn.
\end{itemize}

\subsubsection{Công dụng}

\begin{itemize}
	\item Sấy khô, sưởi ấm, đun nấu.
	\item Chụp ảnh ban đêm, chụp ảnh của nhiều thiên thể, chụp ảnh trái đất từ vệ tinh.
	\item Sử dụng trong các bộ điều khiển từ xa (điều khiển ti vi, điều hòa, ...).
	\item Quân sự: ống nhòm hồng ngoại dùng để quan sát và lái xe ban đêm, camera hồng ngoại chụp ảnh và quay phim ban đêm, tên lửa tự động tìm mục tiêu dựa vào tia hồng ngoại do mục tiêu phát ra.
\end{itemize}
