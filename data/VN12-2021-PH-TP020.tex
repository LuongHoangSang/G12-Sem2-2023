
\setcounter{section}{0}

\section{Lý thuyết: Cấu tạo và nguyên tắc của mạch dao động}
\begin{enumerate}[label=\bfseries Câu \arabic*:]
%------------------------------------------------------------------------------------------
	\item \mkstar{1} [1] %Cau 1
	
	\cauhoi
	{Một mạch dao động lí tưởng gồm tụ điện có điện dung $C$ và cuộn cảm thuần có độ tự cảm $L$, đang thực hiện dao động điện từ tự do. Biết dòng điện qua mạch có dạng $i = I_0 \cos \left( \omega t \right)$. Điện tích tụ điện có giá trị cực đại bằng
		\begin{mcq}(4)
			\item $I_0 \sqrt{\omega}$. 
			\item $\dfrac{I_0}{\sqrt{\omega}}$.
			\item $\dfrac{I_0}{\omega}$. 
			\item $\omega I_o$. 
		\end{mcq}
	}
	
	\loigiai
	{		\textbf{Đáp án: C.}
		
		Mối quan hệ giữa cường độ dòng điện cực đại trong mạch $I_0$ và điện tích cực đại trên bản tụ $Q_0$ là $I_0 = \omega Q_0$. Vậy nên điện tích cực đại trên bản tụ là 
		$$Q_0 = \dfrac{I_0}{\omega}.$$
		
	}

%--------------------------------------------------------------------------------------------------------------
%--------------------------------------------------------------------------------------------------------------
	\item \mkstar{1} [2] %Cau 2
	
	\cauhoi
	{Trong máy thu thanh vô tuyến, bộ phận dùng để biến đổi trực tiếp dao động điện thành dao động âm có cùng tần số là
		\begin{mcq}(4)
			\item mạch tách sóng. 
			\item mạch chọn sóng. 
			\item micrô. 
			\item loa. 
		\end{mcq}
	}
	
	\loigiai
	{		\textbf{Đáp án: D.}
		
		Trong máy thu thanh vô tuyến, bộ phận dùng để biến đổi trực tiếp dao động điện thành dao động âm có cùng tần số là loa.
		
	}

%--------------------------------------------------------------------------------------------------------------
%--------------------------------------------------------------------------------------------------------------
	\item \mkstar{1} [2] %Cau 3
	
	\cauhoi
	{Mạch dao động điện từ lí tưởng, cuộn cảm có độ tự cảm L và tụ điện có điện dung C. Chu kì dao động là
		\begin{mcq}(4)
			\item $T = \dfrac{2\pi}{\sqrt{LC}}$. 
			\item $T = \sqrt{LC}$. 
			\item $T = 2\pi \sqrt{LC}$. 
			\item $T = \dfrac{1}{2\pi \sqrt{LC}}$. 
		\end{mcq}
	}
	
	\loigiai
	{		\textbf{Đáp án: C.}
		
		Công thức  tính chu kì dao động của vật là $T = 2\pi \sqrt{LC}$.
		
	}

%--------------------------------------------------------------------------------------------------------------
%--------------------------------------------------------------------------------------------------------------
	\item \mkstar{1} [4] %Cau 4
	
	\cauhoi
	{Trong mạch dao động điện từ lí tưởng đang hoạt động, điện tích trên một bản tụ điện biến thiên điều hòa
		\begin{mcq}(1)
			\item cùng pha với cường độ dòng điện trong đoạn mạch. 
			\item lệch pha $\text{0,25} \pi$ so với cường độ dòng điện trong đoạn mạch. 
			\item ngược pha với cường độ dòng điện trong đoạn mạch. 
			\item lệch pha $\text{0,5} \pi$ so với cường độ dòng điện trong đoạn mạch. 
		\end{mcq}
	}
	
	\loigiai
	{		\textbf{Đáp án: D.}
		
		Trong mạch dao động đang hoạt động, cường độ dòng điện luôn sớm pha $\pi /2$ so với điện tích trên một bản tụ.
		
	}

%--------------------------------------------------------------------------------------------------------------
%--------------------------------------------------------------------------------------------------------------
	\item \mkstar{1} [12] %Cau 5
	
	\cauhoi
	{Trong mạch dao động  $LC$ lí tưởng $L$ và $C$ thay đổi được, muốn giảm tần số dao động riêng của mạch thì có thể
		\begin{mcq}(2)
			\item tăng $C$, giữ nguyên $L$. 
			\item giảm $C$ một nửa, tăng $L$ gấp 2 lần. 
			\item giảm $C$ và giảm $L$. 
			\item giảm $C$ và giữ nguyên $L$. 
		\end{mcq}
	}
	
	\loigiai
	{		\textbf{Đáp án: A.}
		
		Công thức tính tần số dao động riêng là
		$$
		f = \dfrac{1}{2\pi \sqrt{LC}}.
		$$ \\
		Vậy nên, để giảm tần số dao động riêng $f$, ta tăng $C$ và giữ nguyên $L$.
	}

%--------------------------------------------------------------------------------------------------------------
%--------------------------------------------------------------------------------------------------------------
	\item \mkstar{1} [7] %Cau 6
	
	\cauhoi
	{Một mạch dao động LC lí tưởng đang có dao động điện từ tự do với tần số góc $\omega$. Gọi $q_0$ là điện tích cực đại của một bản tụ điện thì cường độ dòng điện cực đại trong mạch là
		\begin{mcq}(4)
			\item $\dfrac{q_0}{\omega^2}$. 
			\item $q_0 \omega^2$. 
			\item $q_0 \omega$. 
			\item $\dfrac{q_0}{\omega}$. 
		\end{mcq}
	}
	
	\loigiai
	{		\textbf{Đáp án: C.}
		
		Mối liên hệ giữa điện tích cực đại $q_0$ và cường độ dòng điện cực đại $I_0$ là
		$$
		I_0 = \omega q_0.
		$$
		
	}	

%--------------------------------------------------------------------------------------------------------------
%--------------------------------------------------------------------------------------------------------------
	\item \mkstar{1} [7] %Cau 7
	
	\cauhoi
	{Trong mạch dao động điện từ tự do $LC$, so với cường độ dòng điện trong mạch thì điện tích trên một bản tụ luôn
		\begin{mcq}(2)
			\item trễ pha hơn một góc $\pi/2$. 
			\item sớm pha hơn một góc $\pi/2$. 
			\item cùng pha. 
			\item sớm pha hơn một góc $\pi/4$. 
		\end{mcq}
	}
	
	\loigiai
	{		\textbf{Đáp án: A.}
		
		Trong mạch dao động điện từ tự do $LC$, so với cường độ dòng điện trong mạch thì điện tích trên một bản tụ luôn trễ pha hơn một góc $\pi/2$.
		
	}

%--------------------------------------------------------------------------------------------------------------
%--------------------------------------------------------------------------------------------------------------
	\item \mkstar{1} [7]
	
	\cauhoi
	{Một mạch dao động gồm cuộn cảm thuần có độ tự cảm L và tụ điện có điện dung C. Chu kì dao động riêng của mạch là
		\begin{mcq}(4)
			\item $\dfrac{\sqrt{LC}}{2\pi}$. 
			\item $2\pi \sqrt{LC}$. 
			\item $\dfrac{1}{2\pi \sqrt{LC}}$. 
			\item $\dfrac{2\pi}{\sqrt{LC}}$. 
		\end{mcq}
	}
	
	\loigiai
	{		\textbf{Đáp án: B.}
		
		Chu kì dao động riêng của mạch dao động là $2\pi \sqrt{LC}$.
		
	}	

%--------------------------------------------------------------------------------------------------------------
%--------------------------------------------------------------------------------------------------------------
	\item \mkstar{1} [12]
	
	\cauhoi
	{Điện tích của một bản tụ trong mạch dao động LC đang thực hiện dao động điện từ tự do là $q = 4\cdot10^{-7} \cos \left( 4000t \right)$ (C). Điện tích cực đại có độ lớn
		\begin{mcq}(4)
			\item $\xsi{2\sqrt{2}\cdot10^{3}}{C}$. 
			\item $\xsi{22\cdot10^{-7}}{C}$. 
			\item $\xsi{4\cdot10^{-7}}{C}$. 
			\item $\xsi{4\cdot10^{3}}{C}$. 
		\end{mcq}
	}
	
	\loigiai
	{		\textbf{Đáp án: C.}
		
		Từ biểu thức của cường độ dòng điện $q = 4\cdot10^{-7} \cos \left( 4000t \right)$ (C) ta suy ra điện tích cực đại trên bản tụ điện là $\xsi{4\cdot10^{-7}}{C}$.
		
	}

%--------------------------------------------------------------------------------------------------------------
%--------------------------------------------------------------------------------------------------------------
	\item \mkstar{2} [3]
	
	\cauhoi
	{Gọi $I_0$ và $Q_0$ là giá trị cực đại của điện tích trên bản tụ và cường độ dòng điện cực đại trong mạch dao động điện từ LC. Chu kì dao động điện từ tự do trong mạch có biểu thức
		\begin{mcq}(4)
			\item $T = \dfrac{2\pi}{I_0 Q_0}$. 
			\item $T = \dfrac{2\pi Q_0}{I_0}$. 
			\item $T = 2\pi\sqrt{Q_0 I_0}$. 
			\item $T = 2\pi Q_0 I_0$. 
		\end{mcq}
	}
	
	\loigiai
	{		\textbf{Đáp án: B.}
		
		Ta có chu kì dao động tự do của mạch cho bởi $T = \dfrac{2\pi}{\omega}$. \\
		Lại có $\omega = \dfrac{I_0}{Q_0}$. \\
		Từ đó:
		$$T = \dfrac{2\pi}{\omega} = \dfrac{2\pi}{\dfrac{I_0}{Q_0}} = \dfrac{2\pi Q_0}{I_0}.$$
		
	}

%--------------------------------------------------------------------------------------------------------------
%--------------------------------------------------------------------------------------------------------------
	\item \mkstar{2} [12]
	
	\cauhoi
	{Điện tích của một bản tụ trong mạch dao động LC lí tưởng là $q = 6\cdot10^{-6} \cos \left( 4000t + \dfrac{\pi}{2} \right)$ (C). Tại thời điểm $t = \xsi{10^{-4}}{s}$, điện tích trên bản tụ có độ lớn xấp xỉ bằng 
		\begin{mcq}(4)
			\item $\SI{2,30e-6}{C}$. 
			\item $\SI{5,90e-6}{C}$. 
			\item $\SI{1,15e-6}{C}$. 
			\item $\SI{4,60e-6}{C}$. 
		\end{mcq}
	}
	
	\loigiai
	{		\textbf{Đáp án: A.}
		
		Ta thay thời điểm $t = \SI{e-4}{s}$ vào biểu thức điện tích, ta được:
		$$
		q = \num{6e-6} \cos \left( 4000t + \dfrac{\pi}{2} \right) = \num{6e-6}\cos \left( 4000\cdot10^{-4} + \dfrac{\pi}{2} \right) = -\SI{2,30 e-6}{C}
		$$
		
	}

%--------------------------------------------------------------------------------------------------------------
%--------------------------------------------------------------------------------------------------------------
	\item \mkstar{2} [9]
	
	\cauhoi
	{Một mạch dao động LC có điện trở thuần không đáng kể, tụ điện có điện dung $\SI{5}{\mu F}$. Dao động điện từ tự do của mạch LC với hiệu điện thế cực đại ở hai đầu tụ điện bằng $\SI{6}{V}$. Khi hiệu điện thế ở hai đầu tụ điện là $\SI{4}{V}$ thì năng lượng điện từ trong mạch bằng
		\begin{mcq}(4)
			\item $\xsi{9\cdot10^{-5}}{J}$. 
			\item $\xsi{5\cdot10^{-5}}{J}$. 
			\item $\xsi{4\cdot10^{-5}}{J}$. 
			\item $\xsi{10^{-5}}{J}$. 
		\end{mcq}
	}
	
	\loigiai
	{		\textbf{Đáp án: A.}
		
		Năng lượng điện từ trong mạch ở thời điểm bất kì cũng chính là năng lượng cực đại của điện trường bên trong tụ điện:
		$$
		W = W_C = \dfrac{1}{2}CU^{2} = \dfrac{1}{2} \cdot 5\cdot10^{-6} \cdot 6^{2} = \xsi{9\cdot10^{-5}}{J}.
		$$
	}

%------------------------------------------------------------------------------------------
	\item \mkstar{2} [7]
	
	\cauhoi
	{Mạch dao động LC lí tưởng gồm cuộn cảm có độ tự cảm $L = \SI{2}{mH}$ và tụ điện có điện dung $C = \SI{2}{pF}$. Tần số dao động của mạch gần bằng
		\begin{mcq}(4)
			\item $\SI{1}{MHz}$. 
			\item $\SI{2,5}{MHz}$. 
			\item $\SI{1}{kHz}$. 
			\item $\SI{2,5}{kHz}$. 
		\end{mcq}
	}
	
	\loigiai
	{		\textbf{Đáp án: B.}
		
		Tần số dao động riêng của mạch cho bởi
		$$
		f = \dfrac{1}{2\pi \sqrt{LC}} = \dfrac{1}{2\pi \sqrt{2\cdot 10^{-3} \cdot 2\cdot 10^{-12}}} = \SI{2,5}{MHz}.
		$$
		
	}


%--------------------------------------------------------------------------------------------------------------
%--------------------------------------------------------------------------------------------------------------
	\item \mkstar{2} [4]
	
	\cauhoi
	{Mạch dao động điện từ lí tưởng gồm cuộn dây thuần cảm có độ tự cảm L và tụ điện có điện dung C. Khi tăng điện dung của tụ điện lên 9 lần thì chu kì dao động riêng của mạch
		\begin{mcq}(4)
			\item tăng lên 9 lần. 
			\item tăng lên 3 lần. 
			\item giảm đi 9 lần. 
			\item giảm 3 lần. 
		\end{mcq}
	}
	
	\loigiai
	{		\textbf{Đáp án: B.}
		
		Ban đầu, chu kì dao động riêng của mạch cho bởi công thức $T = 2\pi \sqrt{LC}$. \\
		Lúc sau, chu kì dao động riêng của mạch cho bởi công thức $T' = 2\pi \sqrt{LC'}$. \\
		Lập tỉ lệ hai biểu thức trên ta được $\dfrac{T'}{T} = \sqrt{\dfrac{C'}{C}}$. \\
		Thay $C' = 9C$ vào biểu thức trên ta được $T' = 3T$.
		
	}

%--------------------------------------------------------------------------------------------------------------
%--------------------------------------------------------------------------------------------------------------
	\item \mkstar{2} [10]
	
	\cauhoi
	{Coi dao động điện từ của một mạch dao động LC là dao động tự do. Biết độ tự cảm của cuộn dây là $\xsi{2\cdot10^{-2}}{H}$, điện dung tụ điện là $\xsi{2\cdot10^{-10}}{F}$. Chu kì dao động điện từ tự do trong mạch này là
		\begin{mcq}(4)
			\item $\xsi{4\pi . 10^{-6}}{s}$. 
			\item $\xsi{2\pi . 10^{-6}}{s}$. 
			\item $\xsi{4\pi}{s}$. 
			\item $\xsi{2\pi}{s}$. 
		\end{mcq}
	}
	
	\loigiai
	{		\textbf{Đáp án: A.}
		
		Chu kì dao động riêng của mạch cho bởi:
		$$T = 2\pi \sqrt{LC} = 2\pi \sqrt{2\cdot10^{-2} \cdot 2\cdot10^{-10}} = \xsi{4\pi . 10^{-6}}{s}$$.
		
	}

%--------------------------------------------------------------------------------------------------------------
%--------------------------------------------------------------------------------------------------------------
	\item \mkstar{2} [10]
	
	\cauhoi
	{Mạch dao động điện từ LC lí tưởng gồm cuộn thuần cảm có độ tự cảm $\xsi{1}{mH}$ và tụ điện có điện dung $\xsi{0,1}{\mu F}$. Dao động điện từ riêng của mạch có tần số góc là
		\begin{mcq}(4)
			\item $\xsi{2\cdot10^{5}}{rad/s}$. 
			\item $\xsi{10^{5}}{rad/s}$. 
			\item $\xsi{3\cdot10^{5}}{rad/s}$. 
			\item $\xsi{4\cdot10^{5}}{rad/s}$. 
		\end{mcq}
	}
	
	\loigiai
	{		\textbf{Đáp án: B.}
		
		Tần số góc dao động riêng của mạch cho bởi:
		$$\omega = \dfrac{1}{\sqrt{LC}} = \dfrac{1}{\sqrt{1\cdot10^{-3}\cdot \text{0,1}\cdot10^{-6}}} = \xsi{10^{5}}{rad/s}.$$
		
	}

%--------------------------------------------------------------------------------------------------------------
%--------------------------------------------------------------------------------------------------------------
	\item \mkstar{2} [1] %Cau17
	
	\cauhoi
	{Một mạch dao động lí tưởng gồm tụ điện có điện dung $C = 2 \mu F$ và cuộn cảm thuần có độ tự cảm $L$, đang thực hiện dao động điện từ tự do. Biết dòng điện qua mạch có dạng $i = 2 \cos \left( 5000t \right)(mA,s)$. Giá trị của L bằng
		\begin{mcq}(4)
			\item $\SI{0,05}{H}$. 
			\item $\SI{0,01}{H}$. 
			\item $\SI{0,02}{H}$. 
			\item $\SI{0,04}{H}$. 
		\end{mcq}
	}
	
	\loigiai
	{		\textbf{Đáp án: C.}
		
		Từ phương trình dòng điện qua mạch $i = 2 \cos \left( 5000t \right)$ (mA,s) ta rút ra tần số dao động riêng của mạch $\omega = \SI{5000}{rad/s}$. \\
		Lại có $\omega = \dfrac{1}{\sqrt{LC}}$ nên suy ra 
		$$L = \dfrac{1}{C \omega^2} = \dfrac{1}{2\cdot10^{-6} \cdot 5000^2} = \SI{0,02}{H}.$$ 
		
	}

%--------------------------------------------------------------------------------------------------------------
%--------------------------------------------------------------------------------------------------------------
	\item \mkstar{2} [2] %Cau18
	
	\cauhoi
	{Điện tích của một bản tụ trong mạch dao động điện từ có phương trình là $q = Q_0 \cos \left( 4\pi \cdot 10^{4}t \right)$. Trong đó $t$ tính theo giây. Tần số dao động của mạch là
		\begin{mcq}(4)
			\item $ \SI{1e4}{Hz} $.
			\item $ \SI{20e14}{Hz} $.
			\item $ \SI{2e4}{Hz} $.
			\item $ \SI{2e4}{kHz} $.
		\end{mcq}
	}
	
	\loigiai
	{		\textbf{Đáp án: C.}
	
		Từ phương trình $q = Q_0 \cos \left( 4\pi \cdot 10^{4}t \right)$ ta rút ra tần số góc của mạch là $\omega = \xsi{4 \pi e4}{rad/s}.$ \\
		Tần số dao động của mạch cho bởi
		$$f = \dfrac{\omega}{2\pi} = \dfrac{4\pi \cdot 10^{4}}{2\pi} = \SI{2 e4}{Hz}.$$
		
	}

%--------------------------------------------------------------------------------------------------------------
%--------------------------------------------------------------------------------------------------------------
	\item \mkstar{3} [10] %Cau19
	
	\cauhoi
	{Một mạch dao động lí tưởng gồm cuộn cảm thuần có độ tự cảm $\xsi{4}{\mu H}$ và một tụ điện có điện dung biến đổi từ $\xsi{10}{pF}$ đến $\xsi{640}{pF}$. Lấy $\pi^2 = 10$. Chu kì dao động riêng của mạch có giá trị
		\begin{mcq}(2)
			\item từ $\xsi{4,2\cdot10^{-8}}{s}$ đến $\xsi{2,4\cdot10^{-7}}{s}$. 
			\item từ $\xsi{2,24\cdot10^{-8}}{s}$ đến $\xsi{3\cdot10^{-7}}{s}$. 
			\item từ $\xsi{2\cdot10^{-8}}{s}$ đến $\xsi{3,6\cdot10^{-7}}{s}$. 
			\item từ $\xsi{4\cdot10^{-8}}{s}$ đến $\xsi{3,2\cdot10^{-7}}{s}$. 
		\end{mcq}
	}
	
	\loigiai
	{		\textbf{Đáp án: D.}
		
		Chu kì dao động riêng trong mạch cho bởi: $T = 2\pi \sqrt{LC}$. \\
		Khi $C = \xsi{10}{pF}$, ta có:
		$$
			T = 2\pi \sqrt{LC} = 2\pi \sqrt{4\cdot10^{-6} \cdot 10\cdot10^{-12}} = \xsi{4\cdot10^{-8}}{s}.
		$$ \\
		Khi $C = \xsi{640}{pF}$, ta có;
		$$
			T = 2\pi \sqrt{LC} = 2\pi \sqrt{4\cdot10^{-6} \cdot 640\cdot10^{-12}} = \xsi{3,2\cdot10^{-7}}{s}.
		$$ \\
		Vậy chu kì dao động riêng của mạch biến thiên từ $\xsi{4\cdot10^{-8}}{s}$ đến $\SI{3,2 e-7}{s}$.
	}
	
%--------------------------------------------------------------------------------------------------------------
%--------------------------------------------------------------------------------------------------------------
	\item \mkstar{3} [1] %Cau20
	
	\cauhoi
	{Một mạch dao động lí tưởng gồm tụ điện có điện dung $C$ và cuộn cảm thuần có độ tự cảm $L$, đang thực hiện dao động điện từ tự do với tần số $f$. Nếu tăng điện dung của tụ điện lên 16 lần thì tần số dao động của mạch giảm lượng $\SI{24}{MHz}$. Giá trị của $f$ bằng
		\begin{mcq}(4)
			\item $\SI{48}{MHz}$. 
			\item $\SI{32}{MHz}$. 
			\item $\SI{40}{MHz}$. 
			\item $\SI{36}{MHz}$. 
		\end{mcq}
	}
	
	\loigiai
	{		\textbf{Đáp án: B.}
		
		Ban đầu, tần số dao động riêng của mạch cho bởi 
		$$f = \dfrac{1}{2\pi \sqrt{LC}}.$$
		Lúc sau, khi tăng điện dung của tụ điện lên 16 lần, tần số dao động riêng của mạch là
		$$f' = \dfrac{1}{2\pi \sqrt{LC'}}.$$
		Lập tỉ lệ giữa hai biểu thức trên, ta được:
		$$\dfrac{f'}{f} = \dfrac{C}{C'}.$$
		Thay $\xsi{f' = f - 24}{MHz}$ và $C' = 16C$, ta được:
		$$\dfrac{f - 24}{f} = \dfrac{C}{16C}.$$
		Sử dụng chức năng Solve trên máy tính cầm tay, ta được:
		$$f = \SI{32}{MHz}.$$
		
	}

\end{enumerate}

\loigiai
{
	\begin{center}
		\textbf{BẢNG ĐÁP ÁN}
	\end{center}
	\begin{center}
		\begin{tabular}{|m{2.8em}|m{2.8em}|m{2.8em}|m{2.8em}|m{2.8em}|m{2.8em}|m{2.8em}|m{2.8em}|m{2.8em}|m{2.8em}|}
			\hline
			01.C  & 02.D  & 03.C  & 04.D  & 05.A  & 06.C  & 07.A &  08.B & 09.C & 10.B \\
			\hline
			11.A  & 12.A  & 13.B  & 14.B  & 15.A  & 16.B  & 17.C &  18.C & 19.D & 20.B \\ 
			\hline
			
		\end{tabular}
	\end{center}
}
\section{Dạng bài: Bài toán tương tự dao động cơ}
\begin{enumerate}[label=\bfseries Câu \arabic*:]

%--------------------------------------------------------------------------------------------------------------%--------------------------------------------------------------------------------------------------------------	
	\item \mkstar{1} [4]
	
	\cauhoi
	{Cường độ dòng điện tức thời trong mạch dao động LC có dạng $i = \text{0,5} \cos \left( 3000t \right)$. Tần số dao động riêng của mạch là  
		\begin{mcq}(4)
			\item $\SI{3000}{rad/s}$. 
			\item $\SI{477,46}{Hz}$. 
			\item $\xsi{6000\pi}{Hz}$. 
			\item $\SI{3000}{Hz}$. 
		\end{mcq}
	}
	
	\loigiai
	{		\textbf{Đáp án: B.}
		
Tần số dao động riêng trong mạch cho bởi biểu thức:
$$
f=\dfrac{\omega}{2 \pi}=\dfrac{3000}{2 \pi}= \SI{477,76}{Hz}.
$$
		
	}
	
%--------------------------------------------------------------------------------------------------------------
%--------------------------------------------------------------------------------------------------------------
	\item \mkstar{1} [12]
	
	\cauhoi
	{Một mạch dao động LC lí tưởng có $L = \SI{20}{mH}$ và $C = \SI{200}{pF}$. Chu kì riêng của dao động điện từ trong mạch xấp xỉ bằng
		\begin{mcq}(4)
			\item $\xsi{1,3\cdot10^{-5}}{s}$. 
			\item $\xsi{1,9\cdot10^{-4}}{s}$. 
			\item $\xsi{12,5\cdot10^{-3}}{s}$. 
			\item $\xsi{3,9\cdot10^{-4}}{s}$. 
		\end{mcq}
	}
	
	\loigiai
	{		\textbf{Đáp án: A.}
		
	Chu kì dao động riêng của mạch cho bởi 
	$$ T = 2\pi \sqrt{LC} = 2\pi \sqrt{20\cdot10^{-3} \cdot 200\cdot10^{-12}} = \xsi{1,3\cdot10^{-5}}{s} $$.
		
	}
	
%--------------------------------------------------------------------------------------------------------------
%--------------------------------------------------------------------------------------------------------------
	\item \mkstar{3} [10]
	
	\cauhoi
	{Một mạch dao động điện từ lí tưởng đang có dao động điện từ tự do. Biết điện tích cực đại trên một bản tụ điện là $\xsi{4\sqrt{2}}{\mu C}$ và cường độ dòng điện cực đại trong mạch là $\xsi{0,5\sqrt{2}}{A}$. Thời gian ngắn nhất để điện tích trên một bản tụ giảm từ giá trị cực đại đến nửa giá trị cực đại là
		\begin{mcq}(4)
			\item $\xsi{\dfrac{4\pi}{3}}{\mu s}$. 
			\item $\xsi{\dfrac{16\pi}{3}}{\mu s}$. 
			\item $\xsi{\dfrac{2\pi}{3}}{\mu s}$. 
			\item $\xsi{\dfrac{8\pi}{3}}{\mu s}$. 
		\end{mcq}
	}
	
	\loigiai
	{		\textbf{Đáp án: D.}
		
		Tần số riêng của mạch dao động là 
		$$
		\omega = \dfrac{I_0}{q_0} = \dfrac{\text{0,5} \sqrt{2}}{4 \sqrt{2} \cdot10^{-6}} = \SI{125 e3}{rad/s}.
		$$ \\
		Từ đường tròn pha, ta thấy để điện tích trên một bản tụ giảm từ giá trị cực đại đến nửa giá trị cực đại thì vector quay quay được một góc nhỏ nhất là $\Delta \varphi = \pi /3.$ \\
		Vậy thời gian ngắn nhất để điện tích trên một bản tụ giảm từ giá trị cực đại đến nửa giá trị cực đại là
		$$
		\Delta t = \dfrac{\Delta \varphi}{\omega} = \dfrac{\pi / 3}{125\cdot10^{3}} = \xsi{\dfrac{8\pi}{3}}{\mu s}.
		$$
	}



%--------------------------------------------------------------------------------------------------------------
%--------------------------------------------------------------------------------------------------------------
	\item \mkstar{3} [12]
	
	\cauhoi
	{Một mạch dao động LC có $C = \SI{2}{nF}$ đang thực hiện dao động điện từ tự do. Tại thời điểm $t_1$, cường độ dòng điện trong mạch có độ lớn $\SI{8}{mA}$, tại thời điểm $t_2 = t_1 +T/4$, hiệu điện thế giữa hai bản tụ có độ lớn $\SI{6}{V}$. Giá trị của L là 
		\begin{mcq}(4)
			\item $\SI{2,250}{H}$. 
			\item $\SI{1,125}{H}$. 
			\item $\SI{2,250}{mH}$. 
			\item $\SI{1,125}{mH}$. 
		\end{mcq}
	}
	
	\loigiai
	{		\textbf{Đáp án: D.}
		
Ta có thời điểm $t_2$ trễ hơn $T/4$ so với thời điểm $t_1$ nên đây là hai thời điểm vuông pha. \\
Suy ra $i_2$ vuông pha với $i_1$. \\
Mà $u_2$ thì vuông pha với $i_2$. \\
Nên $i_1$ và $u_2$ hoặc là cùng pha, hoặc là ngược pha với nhau. \\
Nếu chỉ xét độ lớn, ta có:
	$$
	\dfrac{|i_1|}{I_0} = \dfrac{|u_2|}{U_0}.
	$$ \\
Lại có $U_0 = \dfrac{Q_0}{C} = \dfrac{I_0}{\omega C}$. Thay vào biểu thức trên, ta được:
	$$
	\dfrac{|i_1|}{I_0} = \dfrac{|u_2|}{\dfrac{I_0}{\omega C}}.
	$$
Suy ra 

$$|i_1| = |u_2|\omega C \Rightarrow |i_1| = |u_2|\dfrac{1}{\sqrt{LC}} C \Rightarrow 8\cdot10^{-3} = 6 \cdot \dfrac{1}{\sqrt{L \cdot 2\cdot10^{-9}}} \cdot 2\cdot10^{-9}$$

Từ đó suy ra $L = \SI{1,125}{mH}.$
		
	}
	
%--------------------------------------------------------------------------------------------------------------
%--------------------------------------------------------------------------------------------------------------

\item \mkstar{3} [10]
	
	\cauhoi
	{Trong mạch dao động LC lí tưởng đang có dao động điện từ tự do. Thời gian ngắn nhất để năng lượng điện trường giảm từ giá trị cực đại xuống còn một nửa giá trị cực đại là $\xsi{1,5\cdot10^{-4}}{s}$. Thời gian ngắn nhất để điện tích trên tụ giảm từ giá trị cực đại xuống còn một nửa giá trị đó là
		\begin{mcq}(4)
			\item $\xsi{2\cdot10^{-4}}{s}$. 
			\item $\xsi{6\cdot10^{-4}}{s}$. 
			\item $\xsi{12\cdot10^{-4}}{s}$. 
			\item $\xsi{3\cdot10^{-4}}{s}$. 
		\end{mcq}
	}
	
	\loigiai
	{		\textbf{Đáp án: A.}
		
	Thời gian ngắn nhất để năng lượng điện trường giảm từ giá trị cực đại xuống còn một nửa giá trị cực đại là $T/4$. \\
Ta có $T/8 = \SI{1,5 e-4}{s} \Rightarrow T = \SI{1,2 e-3}{s}.$ \\ 
Từ đường tròn pha, ta xác định được thời gian ngắn nhất để điện tích trên tụ giảm từ giá trị cực đại xuống còn một nửa giá trị đó là
    $$
    \Delta t = \dfrac{T}{6} = \xsi{2\cdot10^{-4}}{s}.
    $$	 
	}

%--------------------------------------------------------------------------------------------------------------
%--------------------------------------------------------------------------------------------------------------	
	\item \mkstar{3} [10]
	
	\cauhoi
	{Hai mạch dao động điện từ lí tưởng đang có dao động điện từ tự do. Điện tích của tụ điện trong mạch thứ nhất và mạch thứ hai lần lượt là $q_1$ và $q_2$ với $4{q_1}^{2}+{q_2}^{2} = \text{1,3}\cdot10^{-17}$, q tính bằng C. Ở thời điểm $t$ điện tích của tụ điện và cường độ dòng điện trong mạch dao động thứ nhất lần lượt là $\xsi{10^{-9}}{C}$ và $\SI{6}{mA}$, cường độ dòng điện trong mạch thứ hai có độ lớn bằng
		\begin{mcq}(4)
			\item $\SI{4}{mA}$. 
			\item $\SI{10}{mA}$. 
			\item $\SI{8}{mA}$. 
			\item $\SI{6}{mA}$. 
		\end{mcq}
	}
	
	\loigiai
	{		\textbf{Đáp án: C.}
		
		Thay $q_1 = \xsi{10^{-9}}{C}$ vào $4{q_1}^{2}+{q_2}^{2} = \text{1,3}\cdot10^{-17}$ ta tìm được $q_2 = \SI{3}{nC}$. \\
Lấy đạo hàm hai vế phương trình $4{q_1}^{2}+{q_2}^{2} = \text{1,3}\cdot10^{-17}$ ta được:
    $$
    8{q_1}{i_1} + 2{q_2}{i_2} = 0
    $$
Thay $q_1, i_1, q_2$ vào phương trình trên ta tìm được $i_2 = \SI{-8}{mA}.$
		
	}
	
	
	
	
%--------------------------------------------------------------------------------------------------------------
%--------------------------------------------------------------------------------------------------------------
	\item \mkstar{3} [10]
	
	\cauhoi
	{Một mạch dao động ở máy vào của một máy thu thanh gồm cuộn thuần cảm có độ tự cảm $\SI{3}{\mu H}$ và tụ điện có điện dung biến thiên trong khoảng $\SI{10}{pF}$ và $\SI{500}{pF}$. Biết rằng, muốn thu tần số riêng của mạch dao động phải bằng tần số của mạch cần thu (để có cộng hưởng). Trong không khí, tốc độ truyền sóng điện từ là $\xsi{3\cdot10^{8}}{m/s}$, máy thu này có thể thu được sóng điện từ trong khoảng
		\begin{mcq}(2)
			\item từ $\SI{100}{m}$ đến $\SI{730}{m}$. 
			\item từ $\SI{10,32}{m}$ đến $\SI{73}{m}$. 
			\item từ $\SI{1,24}{m}$ đến $\SI{73}{m}$. 
			\item từ $\SI{10}{m}$ đến $\SI{730}{m}$. 
		\end{mcq}
	}
	
	\loigiai
	{		\textbf{Đáp án: B.}
		
	Sóng điện từ thu được từ mạch dao động cho bởi biểu thức:
    $$
    \lambda = 2\pi c \sqrt{LC}.
    $$ 
Sóng điện từ có bước sóng dài nhất thu được khi $C = C_{max} =\SI{500}{pF}$. Khi đó:
    $$
    \lambda_{max} = 2\pi c \sqrt{LC_{max}} = 2\pi \cdot 3\cdot10^{8} \cdot \sqrt{3\cdot10^{-6} \cdot 500\cdot10^{-12}} = \SI{73}{m}.
    $$
Sóng điện từ có bước sóng ngắn nhất thu được khi $C = C_{min} =\SI{10}{pF}$. Khi đó:
    $$
    \lambda_{min} = 2\pi c \sqrt{LC_{min}} = 2\pi \cdot 3\cdot10^{8} \cdot \sqrt{3\cdot10^{-6} \cdot 10\cdot10^{-12}} = \SI{10,32}{m}.
    $$
		
	}

%--------------------------------------------------------------------------------------------------------------
%--------------------------------------------------------------------------------------------------------------	
	\item \mkstar{3} [9]
	
	\cauhoi
	{Một mạch dao động LC lí tưởng với $L = \SI{2,4}{mH}$ và $C = \SI{1,5}{mF}$. Gọi $I_0$ là cường độ dòng điện cực đại trong mạch. Khoảng thời gian ngắn nhất giữa hai lần liên tiếp mà $i = I_o /3$ là
		\begin{mcq}(4)
			\item $\SI{4,76}{ms}$. 
			\item $\SI{4,67}{ms}$. 
			\item $\SI{0,29}{ms}$. 
			\item $\SI{4,54}{ms}$. 
		\end{mcq}
	}
	
	\loigiai
	{		\textbf{Đáp án: B.}

Từ đường tròn pha, ta xác định được khoảng thời gian ngắn nhất giữa hai lần liên tiếp $i=I_{0} / 3$ là
$$
\Delta t=\dfrac{2}{\omega} \arccos \left(\dfrac{i}{I_{0}}\right)=\dfrac{2}{\omega} \arccos \dfrac{1}{3}.
$$
Lại có,
$$
\omega=\dfrac{1}{\sqrt{L C}}=\dfrac{1}{\sqrt{\text{2,4}\cdot10^{-3} \cdot \text{1,5} \cdot 10^{-3}}} \approx \SI{527}{rad/s}.
$$
Khi đó,
$$
\Delta t=\dfrac{2}{527} \arccos \dfrac{1}{3}= \xsi{4,67\cdot10^{-3}}{s}.
$$

		
	}

%--------------------------------------------------------------------------------------------------------------%--------------------------------------------------------------------------------------------------------------	
	\item \mkstar{3} [9]
	
	\cauhoi
	{Một mạch dao động lí tưởng gồm cuộn cảm thuần có độ tự cảm $\SI{4}{\mu H}$ và một tụ điện có điện dung biến đổi từ $\SI{10}{pF}$ đến $\SI{360}{pF}$. Lấy $\pi^2 = 10$. Chu kì dao động riêng của mạch có giá trị
		\begin{mcq}(2)
			\item $\xsi{4\cdot10^{-8}}{s}$ đến $\xsi{3,2\cdot10^{-7}}{s}$. 
			\item $\xsi{2\cdot10^{-8}}{s}$ đến $\xsi{3,6\cdot10^{-7}}{s}$. 
			\item $\xsi{4\cdot10^{-8}}{s}$ đến $\xsi{2,4\cdot10^{-7}}{s}$. 
			\item $\xsi{2\cdot10^{-8}}{s}$ đến $\xsi{3\cdot10^{-7}}{s}$. 
		\end{mcq}
	}
	
	\loigiai
	{		\textbf{Đáp án: C.}
		
Chu kì dao động riêng của mạch cho bời:
$$
T=2 \pi \sqrt{L C}
$$
Khi $\SI{10}{pF}$, ta có:
$$
T=2 \pi \sqrt{4\cdot10^{-6} \cdot 10\cdot10^{-12}}= \xsi{4\cdot10^{-8}}{s}.
$$
Khi $\SI{360}{pF}$, ta có:
$$
T=2 \pi \sqrt{4\cdot10^{-6} \cdot 360\cdot10^{-12}}= \SI{2,4 e-7}{s}.
$$
Vậy chu kì dao động riêng của mạch biến thiên từ $\SI{4 e-8}{s}$ đến $\SI{2,4 e-7}{s}$.
		
	}

%--------------------------------------------------------------------------------------------------------------%--------------------------------------------------------------------------------------------------------------	
	\item \mkstar{3} [1]
	
	\cauhoi
	{Một mạch dao động lí tưởng gồm tụ điện có điện dung $C = \SI{2}{\mu F}$ và cuộn cảm thuần có độ tự cảm L, đang thực hiện dao động điện từ tự do tại thời điểm hiệu điện thế giữa hai bản tụ điện là $\SI{5}{V}$ thì điện tích trên một bản tụ điện bằng
		\begin{mcq}(4)
			\item $\SI{5}{\mu C}$. 
			\item $\SI{2}{\mu C}$. 
			\item $\SI{4}{\mu C}$. 
			\item $\SI{10}{\mu C}$. 
		\end{mcq}
	}
	
	\loigiai
	{		\textbf{Đáp án: D.}
		
Trong một mạch dao động đang hoạt động, mối quan hệ giữa điện thế $u$ và điện tích $q$ ở một thời điểm bất kì cho bởi
$$
q=C u=2\cdot10^{-6} \cdot 5=\SI{10}{\mu C}.
$$
		
	}

%--------------------------------------------------------------------------------------------------------------%--------------------------------------------------------------------------------------------------------------	
	\item \mkstar{3} [1]
	
	\cauhoi
	{Một mạch dao động lí tưởng gồm tụ điện có điện dung $C = \SI{4}{\mu F}$ và cuộn cảm thuần có độ tự $L = \SI{1}{H}$, đang thực hiện dao động điện từ tự do với hiệu điện thế cực đại giữa hai bản tụ điện là $\SI{6}{V}$ thì dòng điện qua cuộn cảm có giá trị cực đại là
		\begin{mcq}(4)
			\item $\xsi{24 \sqrt{2}}{mA}$. 
			\item $\SI{12}{mA}$. 
			\item $\xsi{12 \sqrt{2}}{mA}$. 
			\item $\SI{24}{mA}$. 
		\end{mcq}
	}
	
	\loigiai
	{		\textbf{Đáp án: B.}
		
Trong một mạch dao động, mối quan hệ giữa dòng điện cực đại $I_{0}$ và hiệu điện thế cực đại $U_{0}$ giữa hai bàn tụ là
$$
I_{0}=\sqrt{\dfrac{C}{L}} U_{0}=\sqrt{\dfrac{4\cdot10^{-6}}{1}} \cdot 6=  \SI{12}{mA}.
$$

		
	}


%--------------------------------------------------------------------------------------------------------------%--------------------------------------------------------------------------------------------------------------	
	\item \mkstar{3} [3]
	
	\cauhoi
	{Trong mạch dao động LC lí tưởng đang có dao động điện từ tự do, với hiệu điện thế cực đại giữa hai bản tụ là $U_0$ và cường độ dòng điện cực đại trong mạch là $I_0$. Tại thời điểm $t$, hiệu điện thế giữa hai bản tụ điện là $u$ và cường độ dòng điện trong mạch là $i$. Hệ thức liên hệ giữa $u$ và $i$ là
		\begin{mcq}(2)
			\item $i^{2} = LC ({U_0}^{2} - u^{2})$. 
			\item $i^{2} = \dfrac{L ({U_0}^{2} - u^{2})}{C}$. 
			\item $i^{2} = \sqrt{LC} ({U_0}^{2} - u^{2})$. 
			\item $i^{2} = \dfrac{C({U_0}^2 - u^{2})}{L}$. 
		\end{mcq}
	}
	
	\loigiai
	{		\textbf{Đáp án: D.}
		
Điện áp hai đầu tụ điện $u$ và cường độ dòng điện $i$ ở một thời điểm $t$ là những đại lượng vuông pha nhau. Vậy nên ta có hệ thức độc lập:
$$
\dfrac{i^{2}}{I_{0}^{2}}+\dfrac{u^{2}}{U_{0}^{2}}=1 \rightarrow i^{2}=\dfrac{I_{0}^{2}}{U_{o}^{2}}\left(U_{0}^{2}-u^{2}\right)=\dfrac{C}{L}\left(U_{0}^{2}-u^{2}\right).
$$
	}

%--------------------------------------------------------------------------------------------------------------%--------------------------------------------------------------------------------------------------------------	
	\item \mkstar{3} [4]
	
	\cauhoi
	{Một mạch dao động điện từ lí tưởng đang có dao động điện từ tự do. Tại thời điểm $t = 0$, điện tích trên một bản tụ điện có giá trị cực đại. Sau khoảng thời gian ngắn nhất $\Delta t$ thì điện tích trên bản tụ này bằng một nửa giá trị cực đại. Chu kì dao động riêng của mạch dao động này là
		\begin{mcq}(4)
			\item $4 \Delta t$. 
			\item $6 \Delta t$. 
			\item $3 \Delta t$. 
			\item $12 \Delta t$. 
		\end{mcq}
	}
	
	\loigiai
	{		\textbf{Đáp án: B.}
		
Từ đường tròn pha, ta có khoảng thời gian ngắn nhất để điện tích trên bản tụ điện giảm từ giá trị cực đại xuống còn một nửa giá trị cực đại là
$$
\Delta t=\dfrac{T}{6} \rightarrow T=6 \Delta t.
$$
		
	}

%--------------------------------------------------------------------------------------------------------------%--------------------------------------------------------------------------------------------------------------	
	\item \mkstar{3} [10]
	
	\cauhoi
	{Trong mạch dao động LC có dao động điện từ tự do (dao động riêng) với tần số góc $\xsi{10^{4}}{rad/s}$. Điện tích cực đại trên tụ điện là $\xsi{10^{-9}}{C}$. Cường độ dòng điện trong mạch cực đại bằng
		\begin{mcq}(4)
			\item $\xsi{2\cdot10^{-5}}{A}$. 
			\item $\xsi{10^{-5}}{A}$. 
			\item $\xsi{10^{-4}}{A}$. 
			\item $\xsi{2\cdot10^{-4}}{A}$. 
		\end{mcq}
	}
	
	\loigiai
	{		\textbf{Đáp án: B.}
		
Cường độ dòng điện cực đại trong mạch cho bởi
$$
I_{0}=\omega Q_{0}=10^{4} \cdot 10^{-9}= \xsi{10^{-5}}{A}.
$$
		
	}

%--------------------------------------------------------------------------------------------------------------%--------------------------------------------------------------------------------------------------------------	
	\item \mkstar{3} [10]
	
	\cauhoi
	{Mạch dao động gồm tụ điện có điện dung $\SI{4500}{pF}$ và cuộn dây thuần cảm có độ tự cảm $\SI{5}{\mu H}$. Hiệu điện thế cực đại ở hai đầu tụ điện là $\SI{2}{V}$. Cường độ dòng điện trong mạch cực đại bằng
		\begin{mcq}(4)
			\item $\SI{0,03}{A}$. 
			\item $\SI{0,06}{A}$. 
			\item $\xsi{6\cdot10^{-4}}{A}$. 
			\item $\xsi{3\cdot10^{-4}}{A}$. 
		\end{mcq}
	}
	
	\loigiai
	{		\textbf{Đáp án: B.}
		
Cường độ dòng điện cực đại trong mạch cho bởi biểu thức;
$$
I_{0}=\sqrt{\dfrac{C}{L}} U_{0}=\sqrt{\dfrac{4500\cdot10^{-12}}{5\cdot10^{-6}}} \cdot 2=\SI{0,06}{A}
$$
		
	}

%--------------------------------------------------------------------------------------------------------------%--------------------------------------------------------------------------------------------------------------	
	\item \mkstar{3} [10]
	
	\cauhoi
	{Cường độ dòng điện tức thời trong mạch dao động LC lí tưởng là $i = \text{0,08} \sin \left( 2000t \right)$ (A). Cuộn dây có độ tự cảm là $L = \SI{50}{mH}$. Hiệu điện thế giữa hai bản tụ tại thời điểm cường độ dòng điện tức thời trong mạch bằng cường độ dòng điện hiệu dụng là
		\begin{mcq}(4)
			\item $\SI{32}{V}$. 
			\item $\xsi{4\sqrt{2}}{V}$. 
			\item $\SI{8}{V}$. 
			\item $\xsi{2\sqrt{2}}{V}$. 
		\end{mcq}
	}
	
	\loigiai
	{		\textbf{Đáp án: B.}
		
Điện dung của tụ điện cho bởi biều thức
$$
C=\dfrac{1}{\omega^{2} L}=\dfrac{1}{2000^{2} \cdot 50\cdot10^{-3}}= \SI{5}{\mu C}.
$$
Điện áp cực đại giữa hai đầu bản tụ cho bởi:
$$
U_{0}=\sqrt{\dfrac{L}{C}} I_{0}=\sqrt{\dfrac{50\cdot10^{-3}}{5\cdot10^{-6}}} \cdot \text{0,08}= \SI{8}{V}.
$$
Thay $i=\dfrac{I_{0} \sqrt{2}}{2}$ vào hệ thức độc lập giữa $i$ và $u$ ta được:
$$
\left(\dfrac{i}{I_{0}}\right)^{2}+\left(\dfrac{u}{U_{0}}\right)^{2}=1 \rightarrow\left(\dfrac{\sqrt{2}}{2}\right)^{2}+\left(\dfrac{u}{8}\right)^{2}=1 \rightarrow|u|= \xsi{4\sqrt{2}}{V}.
$$
		
	}
			
\end{enumerate}

\loigiai
{
	\begin{center}
		\textbf{BẢNG ĐÁP ÁN}
	\end{center}
	\begin{center}
		\begin{tabular}{|m{2.8em}|m{2.8em}|m{2.8em}|m{2.8em}|m{2.8em}|m{2.8em}|m{2.8em}|m{2.8em}|m{2.8em}|m{2.8em}|}
			\hline
			01.B  & 02.A  & 03.D  & 04.D  & 05.A  & 06.C  & 07.B & 08.B & 09.C & 10.D \\
			\hline
			11.B  & 12.D  & 13.B  & 14.B  & 15.B  & 16.B  &      &      &      &      \\
			\hline
			
		\end{tabular}
	\end{center}
}


\section{Dạng bài: Bài toán ghép tụ - tụ xoay}
\begin{enumerate}[label=\bfseries Câu \arabic*:]

%--------------------------------------------------------------------------------------------------------------
	\item \mkstar{3} [10]
	
	\cauhoi
	{Cho một mạch dao động điện từ gồm một cuộn dây thuần cảm và hai tụ điện có điện dung $C_1 = \SI{2}{nF}$ và $C_2 = \SI{6}{nF}$ mắc song song với nhau. Mạch có tần số là $\SI{4000}{Hz}$. Nếu  tháo dời khỏi mạch tụ điện thứ hai thì mạch còn lại dao động với tần số
		\begin{mcq}(4)
			\item $\SI{2000}{Hz}$. 
			\item $\SI{4000}{Hz}$. 
			\item $\SI{8000}{Hz}$. 
			\item $\SI{16000}{Hz}$. 
		\end{mcq}
	}
	
	\loigiai
	{		\textbf{Đáp án: C.}
		
Điện dung của tụ điện mắc song song cho bởi
$$
C=C_{1}+C_{2}=2+6= \SI{8}{nF}.
$$
Chu kì của mạch khi mắc song song hai tụ là
$$
T=\dfrac{1}{f}=\dfrac{1}{4000}= \SI{2,5 e-4}{s}.
$$
Khi tháo dời khỏi mạch tụ điện $C_{2}$ mạch chỉ còn tụ điện $C_{1}$. Chu kì của mạch chỉ chứa tụ $C_{1}$ và mạch chứa tụ $C$ cho bởi
$$
\begin{aligned}
T_{1} &=2 \pi \sqrt{L C_{1}}, \\
T &=2 \pi \sqrt{L C}.
\end{aligned}
$$
Lập tỉ lệ hai biểu thức trên, ta được:
$$
\dfrac{T_{1}}{T}=\sqrt{\dfrac{C_{1}}{C}} \rightarrow \dfrac{T_{1}}{\text{2,5} \cdot 10^{-4}}=\sqrt{\dfrac{2}{8}} \rightarrow T_{1}= \SI{1,25 e4}{s}.
$$
Tần số dao động của mạch chỉ chứa $C_{1}$ là
$$
f_{1}=\dfrac{1}{T_{1}}= \SI{8000}{Hz}.
$$	
	}


%----------------------------------------------------------------------------------------	
	\item \mkstar{3} [10]
	
	\cauhoi
	{Một mạch dao động gồm một cuộn dây có độ tự cảm $L = \SI{1,5}{mH}$ và một tụ xoay có điện dung biến thiên từ $C_1 = \SI{50}{pF}$ đến $C_2 = \SI{450}{pF}$ khi một trong hai bản tụ xoay từ $0^\circ$ đến $180^\circ$. Biết điện dung của tụ phụ thuộc vào góc xoay theo hàm bậc nhất. Để mạch thu được sóng điện từ có bước sóng $\SI{1200}{m}$ cần xoay bản động của tụ điện một góc bằng bao nhiêu kể từ vị trí mà tụ điện có điện dung cực đại? Cho $\pi^2 = 10$.
		\begin{mcq}(4)
			\item $99^\circ$. 
			\item $81^\circ$. 
			\item $121^\circ$. 
			\item $108^\circ$. 
		\end{mcq}
	}

	\loigiai
	{		\textbf{Đáp án: A.}
		
Bước sóng của một mạch thu sóng cho bởi
$$
\lambda=2 \pi c \sqrt{L C}
$$
Khi mạch thu được bước sóng $1200 \mathrm{~m}$, ta có:
$$
\lambda=2 \pi c \sqrt{L C} \rightarrow 1200=2 \pi \cdot 3\cdot10^{8} \sqrt{\text{1,5}\cdot10^{-3} \cdot C} \rightarrow \SI{270}{pF}.
$$
Công thức tụ xoay cho bời:
$$
\dfrac{C-C_{1}}{C_{2}-C_{1}}=\dfrac{\alpha-\alpha_{1}}{\alpha_{2}-\alpha_{1}} \rightarrow \dfrac{270-50}{450-50}=\dfrac{\alpha-0^{\circ}}{180^{\circ}-0^{\circ}} \rightarrow \alpha=99^{\circ}.
$$
		
	}


%--------------------------------------------------------------------------------------------------------------
%--------------------------------------------------------------------------------------------------------------	
	\item \mkstar{3} [9] 
	
	\cauhoi
	{Một mạch dao động LC lí tưởng gồm cuộn cảm thuần có độ tự cảm không đổi, tụ điện có điện dung C thay đổi. Khi $C = C_1$ thì tần số dao động riêng của mạch là $\SI{7,5}{MHz}$ và khi $C = C_2$ thì tần số dao động riêng của mạch là $\SI{10}{MHz}$. Nếu $C = C_1 + C_2$ thì tần số dao động riêng của mạch là
		\begin{mcq}(4)
			\item $\SI{6,0}{MHz}$. 
			\item $\SI{17,5}{MHz}$. 
			\item $\SI{12,5}{MHz}$. 
			\item $\SI{2,5}{MHz}$. 
		\end{mcq}
	}
	
	\loigiai
	{		\textbf{Đáp án: A.}
		
Tần số dao động riêng của mạch cho bời
$$
f=\dfrac{1}{2 \pi \sqrt{L C}} \rightarrow C \sim \dfrac{1}{f^{2}}.
$$
Khi đó biểu thức $C=C_{1}+C_{2}$ trở thành
$$
\dfrac{1}{f^{2}}=\dfrac{1}{f_{1}^{2}}+\dfrac{1}{f_{2}^{2}} \rightarrow \dfrac{1}{f^{2}}=\dfrac{1}{\left(\text{7,5}\cdot10^{6}\right)^{2}}+\dfrac{1}{\left(10\cdot10^{6}\right)^{2}} \rightarrow f = \SI{6}{MHz}.
$$
	
	}

\end{enumerate}

\loigiai
{
	\begin{center}
		\textbf{BẢNG ĐÁP ÁN}
	\end{center}
	\begin{center}
		\begin{tabular}{|m{2.8em}|m{2.8em}|m{2.8em}|m{2.8em}|m{2.8em}|m{2.8em}|m{2.8em}|m{2.8em}|m{2.8em}|m{2.8em}|}
			\hline
			01.C  & 02.A  & 03.A  &   &   &   &  &  &  &  \\
			\hline
			
		\end{tabular}
	\end{center}
}

