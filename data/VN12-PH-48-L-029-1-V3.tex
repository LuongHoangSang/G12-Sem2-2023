
\chapter[Hiện tượng phóng xạ hạt nhân \\ Các loại tia phóng xạ]{Hiện tượng phóng xạ hạt nhân\\ Các loại tia phóng xạ}
\section{Lý thuyết}
\subsection{Định nghĩa hiện tượng phóng xạ}
Phóng xạ là quá trình phân rã tự phát của một hạt nhân không bền vững (tự nhiên hay nhân tạo). Quá trình phân rã này kèm theo sự tạo ra các hạt và có thể kèm theo sự phát ra các bức xạ điện từ. 

Hạt nhân tự phân rã gọi là hạt nhân mẹ, hạt nhân được tạo thành sau phân rã gọi là hạt nhân con.
\subsection{Các loại tia phóng xạ}
\subsubsection{Tia alpha $\alpha$}					
\begin{enumerate}[label=\alph*)]
	\item Bản chất
	
	Tia $\alpha$ là dòng hạt nhân $^4_2\text{He}$, mang điện tích dương.
	\item Tính chất
	\begin{itemize}
		\item Vận tốc tia $\alpha$ lớn, khoảng $\SI{2e7}{\meter/\second}$;
		\item Tia $\alpha$ lệch về bản tụ âm khi bay vào giữa hai bản của tụ điện;
		\item Tia $\alpha$ ion hóa chất khí mạnh (đi được $\SI{8}{\centi\meter}$ trong không khí);
		\item Khả năng đâm xuyên của tia $\alpha$ yếu (không xuyên qua được tấm bìa dày cỡ $\SI{1}{\milli\meter}$).
	\end{itemize}
\end{enumerate}
\subsubsection{Tia beta $\beta$}
\begin{enumerate}[label=\alph*)]
	\item Bản chất
	Tia $\beta$ gồm hai loại:
	\begin{itemize}
		\item Tia $\beta^-$ hay $^{\ 0}_{-1}e$: là loại tia phổ biến, có bản chất là chùm electron mang điện tích $-e$.
		\item Tia $\beta^+$ hay $^{\ 0}_{+1}e$: hiếm hơn $\beta^-$, bản chất là chùm hạt có khối lượng như electron nhưng mang điện tích $+e$, gọi là các pozitron.
	\end{itemize}
	
	\item Tính chất
	
	\begin{itemize}
		\item Vận tốc của $\beta$ rất lớn, có thể đạt xấp xỉ vận tốc ánh sáng;
		\item Tia $\beta^-$ lệch về bản tụ dương, tia $\beta^+$ lệch về bản tụ âm khi bay vào giữa hai bản của tụ điện;
		\item Tia $\beta$ ion hóa chất khí mạnh, nhưng yếu hơn so với tia $\alpha$ (đi được khoảng vài mét trong không khí);
		\item Khả năng đâm xuyên của tia $\beta$ mạnh hơn tia $\alpha$ (có thể xuyên qua được lá nhôm dày cỡ milimet).			
	\end{itemize}
\end{enumerate}
\subsubsection{Tia gama $\gamma$}
\begin{enumerate}[label=\alph*)]
	\item Bản chất
	
	Tia $\gamma$ có bản chất là sóng điện từ có bước sóng rất ngắn cũng là hạt photon có năng lượng cao.
	
	Trong phân rã $\alpha$ và $\beta$, hạt nhân con có thể ở trong trạng thái kích thích và phóng xạ tia $\gamma$ để trở về trạng thái cơ bản.
	
	\item Tính chất
	\begin{itemize}
		\item Vận tốc của $\gamma$ bằng vận tốc ánh sáng (vì bản chất tia gamma là chùm photon);
		\item Tia $\gamma$ không bị lệch khi bay vào giữa hai bản của tụ điện;
		\item Khả năng đâm xuyên của tia $\gamma$ lớn hơn nhiều so với tia $\alpha$ và tia $\beta$ (khoảng vài mét trong bê tông và vài centimet trong chì).
	\end{itemize}
	
\end{enumerate}		
\section{Bài tập tự luyện}
\begin{enumerate}[label=\bfseries Câu \arabic*:]
	\item \mkstar{1} [1]
	\cauhoi
	{Phóng xạ hạt nhân là phản ứng ...
		\begin{mcq}(2)
			\item nhiệt hạch.
			\item hạt nhân thu năng lượng.
			\item phân hạch.
			\item hạt nhân tỏa năng lượng.
		\end{mcq}
	}
	
	\loigiai
	{		\textbf{Đáp án: D.}
		
		Phóng xạ hạt nhân là phản ứng hạt nhân tỏa năng lượng.
		
	}
	
	\item \mkstar{1} [1]
	\cauhoi
	{Phản ứng hạt nhân nào sau đây là quá trình phóng xạ?
		\begin{mcq}(2)
			\item $\ce{^210_84 Po} \longrightarrow \ce{^4_2 He} + \ce{^206_82 Pb}$.
			\item $\ce{^1_0 n} + \ce{^235_92 U} \longrightarrow \ce{^139_54 Xe} + \ce{^95_38 Sr} + 2 \ce{^1_0 n}$.
			\item $\ce{^7_3 Li} + \ce{^2_1 H} \longrightarrow 2 \ce{^4_2 He} + \ce{^1_0 n}$.
			\item $\ce{^4_2 He} + \ce{^27_13 Al} \longrightarrow \ce{^1_0 n} + \ce{^30_15 P}$.
		\end{mcq}
	}
	
	\loigiai
	{		\textbf{Đáp án: A.}
		
		Phản ứng
		$$\ce{^210_84 Po} \longrightarrow \ce{^4_2 He} + \ce{^206_82 Pb}$$
		là quá trình phóng xạ.
		
		Các phản ứng còn lại là phản ứng nhiệt hạch, phân hạch hoặc phản ứng hạt nhân thông thường.
		
	}
	\item \mkstar{1} [3]
	\cauhoi
	{Chọn phát biểu đúng về tia phóng xạ $\alpha$.
		\begin{mcq}
			\item Không bị lệch khi đi qua điện trường và từ trường.
			\item Có vận tốc bằng vận tốc ánh sáng trong chân không.
			\item Là dòng các hạt nhân $\ce{^4_2 He}$.
			\item Có khả năng đâm xuyên mạnh hơn tia phóng xạ $\gamma$.
		\end{mcq}
	}
	
	\loigiai
	{		\textbf{Đáp án: C.}
		
		Tia phóng xạ $\alpha$ là dòng các hạt nhân $\ce{^4_2 He}$, có bị lệch trong điện trường và từ trường, khả năng đâm xuyên yếu hơn và vận tốc nhỏ hơn nhiều so với tia $\gamma$.
		
	}
	
	\item \mkstar{1} [3]
	\cauhoi
	{Tia nào sau đây \textbf{không} phải là tia phóng xạ?
		\begin{mcq}(4)
			\item Tia $\alpha$.
			\item Tia $\beta^+$.
			\item Tia $\gamma$.
			\item Tia X.
		\end{mcq}
	}
	
	\loigiai
	{		\textbf{Đáp án: D.}
		
		Tia X bản chất là sóng điện từ, không phải tia phóng xạ.
		
	}
	\item \mkstar{1} [12]
	\cauhoi
	{Cặp tia nào dưới đây có cùng bản chất là sóng điện từ?
		\begin{mcq}(2)
			\item Tia $\beta ^+$ và tia $\alpha$.
			\item Tia hồng ngoại và tia tử ngoại.
			\item Tia $\alpha$ và tia tử ngoại.
			\item Tia $\beta ^+$ và tia $\beta ^-$.
		\end{mcq}
	}
	
	\loigiai
	{		\textbf{Đáp án: B.}
		
		Tia hồng ngoại và tia tử ngoại đều có bản chất là sóng điện từ.
		
	}
	\item \mkstar{1} [5]
	\cauhoi
	{Hạt nhân $\ce{^A_Z X}$ phóng xạ $\alpha$ tạo ra hạt nhân $\ce{Y}$. Phương trình phản ứng có dạng:
		\begin{mcq}(2)
			\item $\ce{^A_Z X} \longrightarrow \alpha + \ce{^{A-4}_{Z-2} Y}$.
			\item $\ce{^A_Z X} \longrightarrow \alpha + \ce{^{A-2}_{Z-4} Y}$.
			\item $\ce{^A_Z X} \longrightarrow \alpha + \ce{^{A-2}_{Z-2} Y}$.
			\item $\ce{^A_Z X} \longrightarrow \alpha + \ce{^{A-4}_{Z-4} Y}$.
		\end{mcq}
	}
	
	\loigiai
	{		\textbf{Đáp án: A.}
		
		Áp dụng định luật bảo toàn số khối và bảo toàn điện tích thì phản ứng đúng là
		$$\ce{^A_Z X} \longrightarrow \alpha + \ce{^{A-4}_{Z-2} Y}$$
		
	}
	\item \mkstar{1} [5]
	\cauhoi
	{Hạt nhân $\ce{^A_Z X}$ phóng xạ $\beta^-$ tạo ra hạt nhân $\ce{Y}$. Phương trình phản ứng có dạng:
		\begin{mcq}(2)
			\item $\ce{^A_Z X} \longrightarrow \beta^- + \ce{^{A}_{Z-1} Y}$.
			\item $\ce{^A_Z X} \longrightarrow \beta^- + \ce{^{A-1}_{Z} Y}$.
			\item $\ce{^A_Z X} \longrightarrow \beta^- + \ce{^{A+1}_{Z} Y}$.
			\item $\ce{^A_Z X} \longrightarrow \beta^- + \ce{^{A}_{Z+1} Y}$.
		\end{mcq}
	}
	
	\loigiai
	{		\textbf{Đáp án: D.}
		
		Áp dụng định luật bảo toàn số khối và bảo toàn điện tích thì phản ứng đúng là
		$$\ce{^A_Z X} \longrightarrow \beta^- + \ce{^{A}_{Z+1} Y}$$
		
	}
	\item \mkstar{2} [1]
	\cauhoi
	{Đồng vị $\ce{^30_15 P}$ biến thành hạt nhân $\ce{^30_14 Si}$ sau khi phóng xạ tia
		\begin{mcq}(4)
			\item $\beta^-$. 
			\item $\gamma$. 
			\item $\beta^+$. 
			\item $\alpha$. 
		\end{mcq}
	}
	
	\loigiai
	{		\textbf{Đáp án: C.}
		
		Phương trình phản ứng:
		$$\ce{^30_15 P} \longrightarrow \ce{^30_14 Si} + \ce{^A_Z X}$$
		
		Áp dụng định luật bảo toàn số khối và bảo toàn điện tích, tìm được X là hạt $\beta^+$ ($\ce{^0_1 e}$).
		
	}
	
	
	\item \mkstar{2} [13]
	\cauhoi
	{Hạt nhân $\ce{^226_88 Ra}$ phóng xạ ra 3 hạt $\alpha$ và một hạt $\beta^-$ trong chuỗi phóng xạ liên tiếp. Khi đó hạt nhân tạo thành là
		\begin{mcq}(4)
			\item $\ce{^214_82 X}$.
			\item $\ce{^212_83 X}$.
			\item $\ce{^214_83 X}$.
			\item $\ce{^212_82 X}$.
		\end{mcq}
	}
	
	\loigiai
	{		\textbf{Đáp án: C.}
		
		Phương trình phản ứng:
		$$\ce{^226_88 Ra}\ \ldots \longrightarrow \ldots\ 3 \ce{^4_2 He} + \ce{^0_{-1} e} + \ce{X}$$
		
		Áp dụng định luật bảo toàn số khối và bảo toàn điện tích, tìm được $X$ là $\ce{^214_83 X}$.
		
	}
	\item \mkstar{2} [5]
	\cauhoi
	{Khi một hạt nhân nguyên tử phóng xạ lần lượt một tia $\alpha$ rồi một tia $\beta^+$ thì hạt nhân nguyên tử sẽ biến đổi như thế nào?
		\begin{mcq}
			\item Số khối giảm 4, số neutron giảm 1.
			\item Số neutron giảm 3, số proton giảm 1.
			\item Số proton giảm 1, số neutron tăng 3.
			\item Số khối giảm 4, số proton tăng 1.
		\end{mcq}
	}
	
	\loigiai
	{		\textbf{Đáp án: A.}
		
		Phương trình phản ứng:
		$$\ce{^A_Z X}\ \ldots \longrightarrow \ldots\ \ce{^4_2 He} + \ce{^0_1 e} + Y$$
		
		Áp dụng định luật bảo toàn số khối và bảo toàn điện tích, tìm được Y là $\ce{^{A-4}_{Z-3} Y}$.
		
		Vậy số khối giảm 4, số proton giảm 3, dẫn đến số neutron giảm 1.
		
	}
	
\end{enumerate}

