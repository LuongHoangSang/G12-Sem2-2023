%\setcounter{chapter}{1}
\chapter{Luyện tập: Các loại quang phổ - Tia hồng ngoại - Tia tử ngoại - Tia X}
\section{Các loại quang phổ}

\begin{enumerate}
	\item \textbf{(ĐH - CĐ 2010):} Quang phổ vạch phát xạ
	\begin{mcq}
		\item của các nguyên tố khác nhau, ở cùng một nhiệt độ thì như nhau về độ sáng tỉ đối của các vạch.
		\item là một hệ thống những vạch sáng (vạch màu) riêng lẻ, ngăn cách nhau bởi những khoảng tối.
		\item do các chất rắn, chất lỏng hoặc chất khí có áp suất lớn phát ra khi bị nung nóng.
		\item là một dải có màu từ đỏ đến tím nối liền nhau một cách liên tục.
	\end{mcq}
	
	
	\item \textbf{(ĐH - 2009):} Phát biểu nào sau đây là đúng?
	\begin{mcq}
		\item Chất khí hay hơi ở áp suất thấp được kích thích bằng nhiệt hay bằng điện cho quang phổ liên tục.
		\item Chất khí hay hơi được kích thích bằng nhiệt hay bằng điện luôn cho quang phổ vạch.
		\item Quang phổ liên tục của nguyên tố nào thì đặc trưng cho nguyên tố ấy.
		\item Quang phổ vạch của nguyên tố nào thì đặc trưng cho nguyên tố ấy.
	\end{mcq}
	
	\item   Ống chuẩn trực trong máy quang phổ có tác dụng
	\begin{mcq}
		\item tạo ra chùm tia sáng song song.
		\item tập trung ánh sáng chiếu vào lăng kính.
		\item tăng cường độ sáng.
		\item tán sắc ánh sáng.
	\end{mcq}
	
	\item \textbf{(CĐ-2009):} Khi nói về quang phổ, phát biểu nào sau đây là đúng?
	\begin{mcq}
		\item Các chất rắn bị nung nóng thì phát ra quang phổ vạch.
		\item Mỗi nguyên tố hóa học có một quang phổ vạch đặc trưng của nguyên tố ấy.
		\item Các chất khí ở áp suất lớn bị nung nóng thì phát ra quang phổ vạch.
		\item Quang phổ liên tục của nguyên tố nào thì đặc trưng cho nguyên tố đó.
	\end{mcq}
	
	\item \textbf{(ĐH - 2008): }Phát biểu nào sau đây là đúng khi nói về quang phổ?
	\begin{mcq}
		\item Quang phổ liên tục của nguồn sáng nào thì phụ thuộc thành phần cấu tạo của nguồn sáng ấy.
		\item Mỗi nguyên tố hóa học ở trạng thái khí hay hơi nóng sáng dưới áp suất thấp cho một quang phổ vạch riêng, đặc trưng cho nguyên tố đó.
		\item Để thu được quang phổ hấp thụ thì nhiệt độ của đám khí hay hơi hấp thụ phải cao hơn nhiệt độ của nguồn sáng phát ra quang phổ liên tục.
		\item Quang phổ hấp thụ là quang phổ của ánh sáng do một vật rắn phát ra khi vật đó được nung nóng.
	\end{mcq}
	
	\item Phát biểu nào sau đây là \textbf{không đúng}?
	\begin{mcq}
		\item Trong máy quang phổ, ống chuẩn trực có tác dụng tạo ra chùm tia sáng song song.
		\item Trong máy quang phổ, buồng ảnh nằm ở phía sau lăng kính.
		\item Trong máy quang phổ, lăng kính có tác dụng phân tích chùm ánh sáng phức tạp song song thành các chùm sáng đơn sắc song song.
		\item Trong máy quang phổ, quang phổ của một chùm sáng thu được trong buồng ảnh luôn máy là một dải sáng có màu cầu vồng.
	\end{mcq}
	
	\item Hiện tượng quang học nào sau đây sử dụng trong máy phân tích quang phổ?
	\begin{mcq}
		\item Hiện tượng khúc xạ ánh sáng.
		\item Hiện tượng phản xạ ánh sáng.
		\item Hiện tượng giao thoa ánh sáng. 
		\item Hiện tượng tán sắc ánh sáng.
	\end{mcq}
	
	\item Máy quang phổ là dụng cụ dùng để
	
	\begin{mcq}
		\item đo bước sóng các vạch quang phổ.
		\item tiến hành các phép phân tích quang phổ.
		\item quan sát và chụp quang phổ của các vật.
		\item phân tích một chùm ánh sáng phức tạp thành những thành phần đơn sắc.
	\end{mcq}
	
	\item Những chất nào sau đây phát ra quang phổ liên tục?
	\begin{mcq}
		\item Chất khí ở nhiệt độ cao. 
		\item Chất rắn ở nhiệt độ thường.
		\item Hơi kim loại ở nhiệt độ cao. 
		\item Chất khí có áp suất lớn, ở nhiệt độ cao.
	\end{mcq}
	
	\item Đặc điểm quan trọng của quang phổ liên tục là
	\begin{mcq}
		\item chỉ phụ thuộc vào thành phần cấu tạo và nhiệt độ của nguồn sáng.
		\item chỉ phụ thuộc vào thành phần cấu tạo của nguồn sáng và không phụ thuộc vào nhiệt độ của nguồn sáng.
		\item không phụ thuộc vào thành phần cấu tạo của nguồn sáng và chỉ phụ thuộc vào nhiệt độ của nguồn sáng.
		\item không phụ thuộc vào thành phần cấu tạo của nguồn sáng và không phụ thuộc vào nhiệt độ của nguồn sáng.
	\end{mcq}
	
	\item Quang phổ của nguồn sáng nào sau đây không phải là quang phổ liên tục?
	
	\begin{mcq}
		\item Sợi dây tóc nóng sáng trong bóng đèn. 
		\item Một đèn LED đỏ đang nóng sáng.
		\item Mặt trời.
		\item Miếng sắt nung nóng.
	\end{mcq}
	
	\item Để nhận biết sự có mặt của nguyên tố hoá học trong một mẫu vật, ta phải nghiên cứu loại quang phổ nào của mẫu đó?
	\begin{mcq}
		\item Quang phổ vạch phát xạ.
		\item Quang phổ liên tục.
		\item Quang phổ hấp thụ.
		\item Cả ba loại quang phổ trên.
	\end{mcq}
	
	\item Quang phổ vạch phát xạ được phát ra do
	\begin{mcq}
		\item các chất khi hay hơi ở áp suất thấp khi bị kích thích phát sáng.
		\item chiếu ánh sáng trắng qua chất khi hay hơi bị nung nóng.
		\item các chất rắn, lỏng hoặc khí khi bị nung nóng.
		\item các chất rắn, lỏng hoặc khí có tỉ khối lớn khi bị nung nóng.
	\end{mcq}
	
	\item \textbf{Tìm phát biểu sai.}
	
	Hai nguyên tố khác nhau có đặc điểm quang phổ vạch phát xạ khác nhau về 
	\begin{mcq}
		\item số lượng các vạch quang phổ.
		\item bề rộng các vạch quang phổ.
		\item độ sáng tỉ đối giữa các vạch quang phổ.
		\item màu sắc các vạch và vị trí các vạch màu.
	\end{mcq}
	
	\item Phát biểu nào sau đây là \textbf{không đúng}?
	\begin{mcq}
		\item Quang phổ vạch phát xạ của các nguyên tố khác nhau thì khác nhau về số lượng vạch màu, màu sắc vạch, vị trí và độ sáng tỉ đối của các vạch quang phổ.
		\item Mỗi nguyên tố hoá học ở trạng thái khí hay hơi ở áp suất thấp được kích thích phát sáng có một quang phổ vạch phát xạ đặc trưng.
		\item Quang phổ vạch phát xạ là những dải màu biến đổi liên tục nằm trên một nền tối.
		\item Quang phổ vạch phát xạ là một hệ thống các vạch sáng màu nằm riêng rẽ trên một nền tối.
	\end{mcq}
	
	\item Chọn câu đúng khi nói về quang phổ liên tục?
	\begin{mcq}
		\item Quang phổ liên tục của một vật phụ thuộc vào bản chất của vật nóng sáng.
		\item Quang phổ liên tục phụ thuộc vào nhiệt độ của vật nóng sáng.
		\item Quang phổ liên tục không phụ thuộc vào nhiệt độ và bản chất của vật nóng sáng.
		\item Quang phổ liên tục phụ thuộc cả nhiệt độ và bản chất của vật nóng sáng.
	\end{mcq}
	
	\item Nguồn sáng phát ra quang phổ vạch phát xạ là
	\begin{mcq}
		\item mặt trời.
		\item khối sắt nóng chảy.
		\item bóng đèn nê-on của bút thử điện.
		\item ngọn lửa đèn cồn trên có rắc vài hạt muối.
	\end{mcq}
	
\end{enumerate}

\section{Các loại tia: tia hồng ngoại, tia tử ngoại, tia X...}
\begin{enumerate}
	\item  Khi nói về tia hồng ngoại, phát biểu nào dưới đây là \textbf{sai}?
	\begin{mcq}
		\item Tia hồng ngoại cũng có thể biến điệu được như sóng điện từ cao tần.
		\item Tia hồng ngoại có khả năng gây ra một số phản ứng hóa học.
		\item Tia hồng ngoại có tần số lớn hơn tần số của ánh sáng đỏ.
		\item Tác dụng nổi bật nhất của tia hồng ngoại là tác dụng nhiệt.
	\end{mcq}
	
	
	\item \textbf{(CĐ 2008): }Ánh sáng đơn sắc có tần số $5\cdot 10^{14}\ \text{Hz}$ truyền trong chân không với bước sóng $600\ \text{nm}$. Chiết suất tuyệt đối của một môi trường trong suốt ứng với ánh sáng này là $\text{1,52}$. Tần số của ánh sáng trên khi truyền trong môi trường trong suốt này
	\begin{mcq}
		\item nhỏ hơn $5\cdot 10^{14}\ \text{Hz}$ còn bước sóng bằng $600\ \text{nm}$.
		\item lớn hơn $5\cdot 10^{14}\ \text{Hz}$ còn bước sóng nhỏ hơn $600\ \text{nm}$.
		\item vẫn bằng $5\cdot 10^{14}\ \text{Hz}$ còn bước sóng nhỏ hơn $600\ \text{nm}$.
		\item vẫn bằng $5\cdot 10^{14}\ \text{Hz}$ còn bước sóng lớn hơn $600\ \text{nm}$.
	\end{mcq}
	
	\item Bức xạ (hay tia) hồng ngoại là bức xạ
	\begin{mcq} 
		\item đơn sắc, có màu hồng.
		\item đơn sắc, không màu ở ngoài đầu đỏ của quang phổ.
		\item có bước sóng nhỏ dưới $\text{0,4}\ \mu\text{m}$.
		\item có bước sóng từ $\text{0,76}\ \mu\text{m}$ tới cỡ milimét.
	\end{mcq}
	
	\item  Công dụng phổ biến nhất của tia hồng ngoại là
	\begin{mcq}
		\item Sấy khô, sưởi ấm.
		\item Chiếu sáng.
		\item Chụp ảnh ban đêm. 
		\item Chữa bệnh.
	\end{mcq}
	
	\item  Bức xạ tử ngoại là bức xạ điện từ
	\begin{mcq}
		\item có màu tím sẫm.
		\item có tần số thấp hơn so với ánh sáng khả kiến.
		\item có bước sóng lớn hơn so với bức xạ hồng ngoại.
		\item có bước sóng nhỏ hơn so với ánh sáng khả kiến.
	\end{mcq}
	
	\item \textbf{(CĐ 2008):} Tia hồng ngoại là những bức xạ có
	\begin{mcq}
		\item bản chất là sóng điện từ.
		\item khả năng ion hoá mạnh không khí.
		\item khả năng đâm xuyên mạnh, có thể xuyên qua lớp chì dày cỡ cm.
		\item bước sóng nhỏ hơn bước sóng của ánh sáng đỏ.
	\end{mcq}
	
	\item  Phát biểu nào sau đây là \textbf{không đúng}?
	\begin{mcq}
		\item Vật có nhiệt độ trên $3000^\circ \text{C}$ phát ra tia tử ngoại rất mạnh.
		\item Tia tử ngoại không bị thuỷ tinh hấp thụ.
		\item Tia tử ngoại là sóng điện từ có bước sóng nhỏ hơn bước sóng của ánh sáng đỏ.
		\item Tia tử ngoại có tác dụng nhiệt.
	\end{mcq}
	
	\item \textbf{(CĐ 2008):} Khi nói về tia tử ngoại, phát biểu nào dưới đây là \textbf{sai}?
	\begin{mcq}
		\item Tia tử ngoại có tác dụng mạnh lên kính ảnh.
		\item Tia tử ngoại có bản chất là sóng điện từ.
		\item Tia tử ngoại có bước sóng lớn hơn bước sóng của ánh sáng tím.
		\item Tia tử ngoại bị thuỷ tinh hấp thụ mạnh và làm ion hoá không khí.
	\end{mcq}
	
	\item   Tìm phát biểu \textbf{sai} về tia hồng ngoại.
	\begin{mcq}
		\item Tia hồng ngoại có bản chất là sóng điện từ.
		\item Tia hồng ngoại kích thích thị giác làm cho ta nhìn thấy màu hồng.
		\item vật nung nóng ở nhiệt độ thấp chỉ phát ra tia hồng ngoại. Nhiệt độ của vật trên $500^\circ \text{C}$ mới bắt đầu phát ra ánh sáng khả kiến.
		\item Tia hồng ngoại nằm ngoài vùng ánh sáng khả kiến, bước sóng của tia hồng ngoại dài hơn bước sóng của ánh sáng đỏ.
	\end{mcq}
	
	\item  Chọn câu đúng khi nói về tia X?
	\begin{mcq}
		\item Tia X là sóng điện từ có bước sóng nhỏ hơn bước sóng của tia tử ngoại.
		\item Tia X do các vật bị nung nóng ở nhiệt độ cao phát ra.
		\item Tia X có thể được phát ra từ các đèn điện.
		\item Tia X có thể xuyên qua tất cả mọi vật.
	\end{mcq}
	\item  Tia Rơn-ghen hay tia X là sóng điện từ có bước sóng
	\begin{mcq}
		\item lớn hơn tia hồng ngoại.
		\item nhỏ hơn tia tử ngoại.
		\item nhỏ quá không đo được.
		\item vài na-no-mét đến vài mi-li-mét.
	\end{mcq}
	
	\item  Chọn câu \textbf{không đúng}?
	\begin{mcq}
		\item Tia X có khả năng xuyên qua một lá nhôm mỏng.
		\item Tia X có tác dụng mạnh lên kính ảnh.
		\item Tia X là bức xạ có thể trông thấy được vì nó làm cho một số chất phát quang.
		\item Tia X là bức xạ có hại đối với sức khỏe con người.
	\end{mcq}
	
	\item  Tia X được ứng dụng nhiều nhất, là nhờ có
	\begin{mcq}
		\item khả năng xuyên qua vải, gỗ, các cơ mềm.
		\item tác dụng làm đen phim ảnh.
		\item tác dụng làm phát quang nhiều chất.
		\item tác dụng hủy diệt tế bào.
	\end{mcq}
	
	\item  \textbf{Chọn phát biểu sai.}
	
	Tia X
	\begin{mcq}
		\item có bản chất là sóng điện từ.
		\item có năng lượng lớn vì bước sóng lớn.
		\item không bị lệch phương trong điện trường và từ trường.
		\item có bước sóng ngắn hơn bước sóng của tia tử ngoại.
	\end{mcq}
\end{enumerate}
\section{Bài toán về tia X}
\begin{enumerate}
	\item  Cho một ống phát tia X có $U_\text{AK}=30\ \text{kV}$. Bỏ qua động năng ban đầu. Cho hằng số điện tích nguyên tố $e=\text{1,6}\cdot 10^{-19}\ \text{C}$ và khối lượng của electron $m_\text{e}=\text{9,1}\cdot 10^{-31}\ \text{kg}$. Tốc độ lớn nhất của electron ngay trước khi đập vào anốt là
	\begin{mcq}(2)
		\item $\text{1,3}\cdot 10^{7}\ \text{m/s}$.
		\item $\text{1,3}\cdot 10^{6}\ \text{m/s}$.
		\item $\text{1,03}\cdot 10^{7}\ \text{m/s}$.
		\item $\text{3,1}\cdot 10^{8}\ \text{m/s}$.
	\end{mcq}
	
	\item \textbf{[Trích đề thi ĐH năm 2007] } Hiệu điện thế giữa anốt và catốt của một ống Rơnghen là $U_\text{AK}=\text{18,75}\ \text{kV}$. Cho $h=\text{6,625}\cdot 10^{-34}\ \text{Js}$, độ lớn điện tích của lectron $e=\text{1,6}\cdot 10^{-19}\ \text{C}$ và $c=3\cdot 10^8\ \text{m/s}$. Bỏ qua động năng ban đầu của êlectrôn. Bước sóng nhỏ nhất của tia Rơnghen do ống phát ra là
	\begin{mcq}(2)
		\item $\text{0,4625}\cdot 10^{-9}\ \text{m}$.
		\item $\text{0,6625}\cdot 10^{-9}\ \text{m}$.
		\item $\text{0,5625}\cdot 10^{-9}\ \text{m}$.
		\item $\text{0,6625}\cdot 10^{-10}\ \text{m}$.
	\end{mcq}
	
	\item \textbf{[Trích đề thi ĐH năm 2010]} Chùm tia X phát ra từ một ống tia X (ống Cu-lít-giơ) có tần số lớn nhất là  $\text{6,4}\ \cdot 10^{18}\ \text{Hz}$. Bỏ qua động năng các êlectron khi bứt ra khỏi catôt. Hiệu điện thế giữa anôt và catôt của ống tia X là
	\begin{mcq} (2)
		\item $U_\text{AK}=\text{13,25}\ \text{kV}$.
		\item $U_\text{AK}=\text{2,65}\ \text{kV}$.
		\item $U_\text{AK}=\text{26,50}\ \text{kV}$.
		\item $U_\text{AK}=\text{5,30}\ \text{kV}$.
	\end{mcq}
\end{enumerate}

\begin{center}
	\textbf{ĐÁP ÁN}
	
\end{center}

\textbf{1. Các loại quang phổ}

\begin{longtable}[\textwidth]{|p{0.1\textwidth}|p{0.1\textwidth}|p{0.1\textwidth}|p{0.1\textwidth}|p{0.1\textwidth}|p{0.1\textwidth}|p{0.1\textwidth}|p{0.1\textwidth}|}
	% --- first head
	\hline%\hspace{2 pt}
	\multicolumn{1}{|c}{\textbf{Câu 1}} & \multicolumn{1}{|c|}{\textbf{Câu 2}} & \multicolumn{1}{c|}{\textbf{Câu 3}} &
	\multicolumn{1}{c|}{\textbf{Câu 4}} &
	\multicolumn{1}{c|}{\textbf{Câu 5}} &
	\multicolumn{1}{c|}{\textbf{Câu 6}} &
	\multicolumn{1}{c|}{\textbf{Câu 7}} &
	\multicolumn{1}{c|}{\textbf{Câu 8}}\\
	\hline
	B. &D. &A. &B. &B. &D. &D. &D.\\
	\hline
	
	\multicolumn{1}{|c|}{\textbf{Câu 9}} & \multicolumn{1}{c|}{\textbf{Câu 10}} & \multicolumn{1}{c|}{\textbf{Câu 11}} &
	\multicolumn{1}{c|}{\textbf{Câu 12}} &
	\multicolumn{1}{c|}{\textbf{Câu 13}} &
	\multicolumn{1}{c|}{\textbf{Câu 14}} &
	\multicolumn{1}{c|}{\textbf{Câu 15}} &
	\multicolumn{1}{c|}{\textbf{Câu 16}} \\
	\hline
	D. &C. &B. &A. &A. &B. &C. &B.\\
	\hline	
	
	\multicolumn{1}{|c|}{\textbf{Câu 17}} & \multicolumn{1}{c|}{\textbf{}} & \multicolumn{1}{c|}{\textbf{}} &
	\multicolumn{1}{c|}{\textbf{}} &
	\multicolumn{1}{c|}{\textbf{}} &
	\multicolumn{1}{c|}{\textbf{}} &
	\multicolumn{1}{c|}{\textbf{}} &
	\multicolumn{1}{c|}{} \\
	\hline
	D. & & & & & & &\\
	\hline	
	
	
\end{longtable}


\textbf{2. Các loại tia: tia hồng ngoại, tia tử ngoại, tia X...}


\begin{longtable}[\textwidth]{|p{0.1\textwidth}|p{0.1\textwidth}|p{0.1\textwidth}|p{0.1\textwidth}|p{0.1\textwidth}|p{0.1\textwidth}|p{0.1\textwidth}|p{0.1\textwidth}|}
	% --- first head
	\hline%\hspace{2 pt}
	\multicolumn{1}{|c}{\textbf{Câu 1}} & \multicolumn{1}{|c|}{\textbf{Câu 2}} & \multicolumn{1}{c|}{\textbf{Câu 3}} &
	\multicolumn{1}{c|}{\textbf{Câu 4}} &
	\multicolumn{1}{c|}{\textbf{Câu 5}} &
	\multicolumn{1}{c|}{\textbf{Câu 6}} &
	\multicolumn{1}{c|}{\textbf{Câu 7}} &
	\multicolumn{1}{c|}{\textbf{Câu 8}}\\
	\hline
	C. &C. &D. &A. &D. &A. &B. &C.\\
	\hline
	
	\multicolumn{1}{|c|}{\textbf{Câu 9}} & \multicolumn{1}{c|}{\textbf{Câu 10}} & \multicolumn{1}{c|}{\textbf{Câu 11}} &
	\multicolumn{1}{c|}{\textbf{Câu 12}} &
	\multicolumn{1}{c|}{\textbf{Câu 13}} &
	\multicolumn{1}{c|}{\textbf{Câu 14}} &
	\multicolumn{1}{c|}{\textbf{}} &
	\multicolumn{1}{c|}{} \\
	\hline
	B. &A. &B. &C. &A. &B. & &\\
	\hline		
\end{longtable}


\textbf{3. Bài toán về tia X}
\begin{longtable}[\textwidth]{|p{0.1\textwidth}|p{0.1\textwidth}|p{0.1\textwidth}|p{0.1\textwidth}|p{0.1\textwidth}|p{0.1\textwidth}|p{0.1\textwidth}|p{0.1\textwidth}|}
	% --- first head
	\hline%\hspace{2 pt}
	\multicolumn{1}{|c}{\textbf{Câu 1}} & \multicolumn{1}{|c|}{\textbf{Câu 2}} & \multicolumn{1}{c|}{\textbf{Câu 3}} &
	\multicolumn{1}{c|}{\textbf{}} &
	\multicolumn{1}{c|}{\textbf{}} &
	\multicolumn{1}{c|}{\textbf{}} &
	\multicolumn{1}{c|}{\textbf{}} &
	\multicolumn{1}{c|}{\textbf{}}\\
	\hline
	C. &D. &C. & & & & & \\
	\hline
\end{longtable}	



