
\chapter[Giao thoa ánh sáng trắng]{Giao thoa ánh sáng trắng}

\section{Lý thuyết}

\begin{itemize}
	
	\item Ánh sáng trắng là tập hợp  nhiều ánh sáng đơn sắc khác nhau có bước sóng biến thiên liên tục từ  $\lambda_{\text{đ}} =\text{0,38}\ \mu \text{m}$  đến $\lambda_{\text{t}} =\text{0,76}\ \mu \text{m}$.
	
	\item Mỗi ánh sáng đơn sắc cho một hệ thống vân giao thoa riêng không chồng khít lên nhau. Tại trung tâm tất cả các ánh sáng đơn sắc đều cho vân sáng bậc 0 nên vân trung tâm là vân màu trắng.
	
	\item Các vân sáng bậc 1, 2, 3,...n của các ánh sáng đơn sắc không còn chồng khít lên nhau nữa nên chúng tạo thành các vạch sáng viền màu sắc tím bên trong và đỏ bên ngoài.
	\item Độ rộng quang phổ bậc $k$ là khoảng cách từ vân sáng đỏ bậc $k$ đến vân sáng tím bậc $k$ (cùng một phía đối với vân trung tâm)
	\begin{equation}
		\Delta x_{k} = x_{\text{đ}(k)}-x_{\text{t}(k)}=k\dfrac{D}{a}(\lambda_\text{đ}-\lambda_\text{t})
	\end{equation}
\end{itemize}

\section{Ví dụ minh họa}

\ppgiai{
	Để tìm số bức xạ cho vân sáng (hoặc vân tối) tại một điểm nhất định trên màn ta làm như sau:
	\begin{description}
		\item[Bước 1] Xác định dữ kiện đề bài cung cấp: 
		\begin{itemize}
			\item Vân sáng: 
			\begin{equation*}
				x_{\text{M}}= k\dfrac{\lambda D}{a} \Rightarrow \lambda = \dfrac{a x_{\text{M}}}{kD}.
			\end{equation*}
			\item Vân tối: 
			\begin{equation*}
				x_{\text{M}}= (m+\text{0,5})\dfrac{\lambda D}{a} \Rightarrow \lambda = \dfrac{a x_{\text{M}}}{(m+\text{0,5})D}.
			\end{equation*}
		\end{itemize}
		\item [Bước 2]Dựa vào điều kiện  $\text{0,38}\ \mu \text{m} \leq  \lambda \leq  \text{0,76}\ \mu \text{m}$  để tìm $k$. Mỗi giá trị $k$
		ứng với một bước sóng cho vân sáng (hoặc vân tối) tại vị trí đó. 
	\end{description}
}

\viduii{1}
{
	Trong thí nghiệm lâng về giao thoa ánh sáng, khoảng cách giữa hai khe 0,3mm, khoảng cách từ mặt phẳng chứa hai khe đến màn quan sát 2 m. Hai khe được chiếu bằng ánh sáng trắng. Khoảng cách từ vân sáng bậc 1 màu đỏ (bước sóng 0,76 $\mu$m) đến vân sáng bậc 1 màu tím (bước sóng 0,4 $\mu$ m) cùng phía so với vân trung tâm là
	\begin{mcq}(4)
		\item 1,8 mm.			
		\item 2,7 mm.			
		\item 1,5 mm.			
		\item 2,4 mm.
\end{mcq}}
{\begin{center}
		\textbf{Hướng dẫn giải}
	\end{center}
	
	
	Độ rộng quang phổ ánh sáng trắng
	\begin{equation*}
		\Delta x_{k} = x_{\text{đ}_(k)}-x_{\text{t}_(k)}=k\dfrac{D}{a}(\lambda_\text{đ}-\lambda_\text{t})=\text{2,4} \cdot 10^{-8}\ \text{m}.
	\end{equation*}
	
	\begin{center}
		\textbf{Câu hỏi tương tự}
	\end{center}
	
	Trong thí nghiệm lâng về giao thoa ánh sáng, khoảng cách giữa hai khe 0,3mm, khoảng cách từ mặt phẳng chứa hai khe đến màn quan sát 2 m. Hai khe được chiếu bằng ánh sáng trắng. Khoảng cách từ vân sáng bậc 1 màu đỏ (bước sóng 0,76 $\mu$m) đến vân sáng bậc 3 màu tím (bước sóng 0,4 $\mu$ m) cùng phía so với vân trung tâm là
	\begin{mcq}(4)
		\item 1,8 mm.			
		\item 2,9 mm.			
		\item 1,5 mm.			
		\item 2,4 mm.
	\end{mcq}
	
	\textbf{Đáp án:} B.
}

\viduii{2}
{Thực hiện giao thoa ánh sáng với thiết bị của Y-âng, khoảng cách giữa hai khe $a =$ 2 mm, từ hai khe đến màn $D =$ 2 m. Người ta chiếu sáng hai khe bằng ánh sáng trắng ($\text{0,4}\ \mu \text{m} \leq  \lambda \leq  \text{0,75}\ \mu \text{m}$). Quan sát điểm A trên màn ảnh, cách vân sáng trung tâm 3,3 mm. Hỏi tại A bức xạ cho vân tối có bước sóng ngắn nhất bằng bao nhiêu?
	\begin{mcq}(4)
		\item 0,440 $\mu$m.		
		\item 0,508 $\mu$m.			
		\item 0,400 $\mu$m.			
		\item 0,490 $\mu$m.
	\end{mcq}
}
{\begin{center}
		\textbf{Hướng dẫn giải}
	\end{center}
	
	\begin{itemize}
		\item Tại A bức xạ có vân tối nên 
		\begin{equation*}
			x_{\text{A}} =(m+\text{0,5})\dfrac{Ơ\lambda D}{a}.
		\end{equation*}
		\item Bước sóng 
		\begin{equation*}
			\lambda=\dfrac{a x_{\text{A}}}{(m+\text{0,5})D} = dfrac{\text{3,3}}{m+\text{0,5}}
		\end{equation*}
		\item Thay $\lambda$ vào điều kiện $\text{0,4}\ \mu \text{m} \leq  \lambda \leq  \text{0,75}\ \mu \text{m}$.
		\begin{equation*}
			\text{0,4} \leq  \dfrac{\text{3,3}}{m+\text{0,5}} \leq  \text{0,75} \Rightarrow \text{0,9}\leq  m \leq  \text{7,75}.
		\end{equation*}
		
		\item Suy ra $m=4; 5; 6; 7$.
		\item Bước sóng ngắn nhất
		\begin{equation*}
			\lambda_{\text{min}}=\dfrac{\text{3,3}}{7+\text{0,5}}=\text{0,44}\ \mu \text{m}.
		\end{equation*}
	\end{itemize}
	
	\begin{center}
		\textbf{Câu hỏi tương tự}
	\end{center}
	
	Thực hiện giao thoa ánh sáng với thiết bị của Y-âng, khoảng cách giữa hai khe $a =$ 2 mm, từ hai khe đến màn $D =$ 2 m. Người ta chiếu sáng hai khe bằng ánh sáng trắng ($\text{0,4}\ \mu \text{m} \leq  \lambda \leq  \text{0,75}\ \mu \text{m}$). Quan sát điểm A trên màn ảnh, cách vân sáng trung tâm 3,3 mm. Hỏi tại A bức xạ cho vân sáng có bước sóng ngắn nhất bằng bao nhiêu?
	\begin{mcq}(4)
		\item 0,440 $\mu$m.		
		\item 0,508 $\mu$m.			
		\item 0,413 $\mu$m.			
		\item 0,490 $\mu$m.
	\end{mcq}
	
	\textbf{Đáp án:} C.
}
\section{Bài tập tự luyện}
\begin{enumerate}[label=\bfseries Câu \arabic*:]
	
	%========================================
	\item \mkstar{2} [4]
	\cauhoi
	{Khi thực hiện thí nghiệm Y-âng về giao thoa với ánh sáng trắng, ta quan sát thấy
		\begin{mcq}(1)
			\item một dải màu liên tục từ đỏ tới tím. 
			\item vân sáng trắng ở chính giữa, hai bên có những dải màu, với tím ở trong và đỏ ở ngoài. 
			\item vân sáng trắng ở chính giữa, hai bên có những dải màu, với đỏ ở trong và tím ở ngoài. 
			\item các dải sáng trắng xen kẻ với những vạch tối.. 
		\end{mcq}
	}
	
	\loigiai
	{		\textbf{Đáp án: B.}
		
		Khi thực hiện thí nghiệm Y-âng về giao thoa với ánh sáng trắn, ta quan sát thấy vân sáng trắng ở chính giữa, hai bên có những dải màu, với tím ở trong và đỏ ở ngoài.
	}
	
	%========================================
	\item \mkstar{3} [4]
	\cauhoi
	{Trong thí nghiệm Y-âng về giao thoa ánh sáng, hai khe được chiếu bằng ánh sáng trắng biến thiên liên tục từ $\SI{0,38}{\mu m}$ đến $\SI{0,75}{\mu m}$. Khoảng cách giữa hai khe là $\SI{0,2}{mm}$. Khoảng cách từ hai khe tới màn là $\SI{2}{m}$. Khoảng cách giữa vân sáng bậc 3 màu đỏ và vân sáng bậc 3 màu tím ở cùng một bên so với vân trung tâm là
		\begin{mcq}(4)
			\item $\SI{11,1}{mm}$. 
			\item $\SI{11,4}{mm}$. 
			\item $\SI{7,4}{mm}$. 
			\item $\SI{2,5}{mm}$. 
		\end{mcq}
	}
	
	\loigiai
	{		\textbf{Đáp án: A.}
		
		Khoảng cách giữa vân sáng bậc 3 màu đỏ và vân sáng bậc 3 màu tím là
		$$
		\Delta d = 3(i_{d} - i_{t}) = 3\dfrac{(\lambda_{d} - \lambda_{t})D}{a} = \SI{11,1}{mm}.
		$$
	}
	
	%========================================
	\item \mkstar{3} [7]
	\cauhoi
	{Trong thí nghiệm giao thoa với ánh sáng trắng có bước sóng $\lambda$ từ $\SI{0,38}{\mu m}$ đến $\SI{0,76}{\mu m}$. Tại vị trí vân sáng bậc 4 của ánh sáng đỏ có bước sóng $\lambda_{\text{đ}} =\SI{0,75}{\mu m}$, số bức xạ khác cũng cho vân sáng tại đó là
		\begin{mcq}(4)
			\item 5. 
			\item 4. 
			\item 3. 
			\item 2. 
		\end{mcq}
	}
	
	\loigiai
	{		\textbf{Đáp án: C.}
		
		Ta có $4\lambda_{d} = k\lambda \rightarrow \lambda = \dfrac{3}{k}$. \\
		Lại có $\SI{0,38}{\mu m} \leq \lambda \leq \SI{0,76}{\mu m} \rightarrow 3,95 \leq k \leq 7,89$.  \\
		Vậy có ba ánh sáng khác cũng có vân sáng tại vị trí trên.
	}
	
	%========================================
	\item \mkstar{4} [3]
	\cauhoi
	{Trong thí nghiệm Y-âng về giao thoa ánh sáng, nguồn S phát ra ánh sáng đơn sắc: $\lambda_{1} = \SI{0,4}{\mu m}$ (màu tím); $\lambda_{2} = \SI{0,5}{\mu m}$ (màu lục) và $\lambda_{3} = \SI{0,7}{\mu m}$ (màu đỏ). Số vân sáng màu tím và màu đỏ quan sát được trên màn, nằm giữa hai vân sáng liên tiếp giống màu vân trung tâm là
		\begin{mcq}(2)
			\item 34 vân tím, 19 vân đỏ. 
			\item 24 vân tím, 12 vân đỏ. 
			\item 35 vân tím, 20 vân đỏ. 
			\item 27 vân tím, 18 vân đỏ. 
		\end{mcq}
	}
	
	\loigiai
	{		\textbf{Đáp án: B.}
		
		Bước sóng vân trùng là
		$$
		\lambda_{\equiv} = \mathrm{LCM}(\lambda_{1}; \lambda_{2}; \lambda_{3}) = \SI{14}{\mu m}.
		$$
		Ta có $\dfrac{\lambda_{\equiv}}{\lambda_{1}} = \num{35}$. Suy ra trong khoảng giữa hai vân sáng cùng màu vân trung tâm có 34 vị trí vân màu tím. \\
		Ta có $\dfrac{\lambda_{\equiv}}{\lambda_{3}} = \num{20}$. Suy ra trong khoảng giữa hai vân sáng cùng màu vân trung tâm có 19 vị trí vân màu đỏ. \\
		Ta có khoảng vân trùng (tím, đỏ) (tím, lục) và (lục, đỏ) lần lượt là: \\
		$\left\{
		\begin{aligned}
			& \lambda_{13} = \mathrm{LCM} (\lambda_{1}; \lambda_{3}) =\SI{2,8}{\mu m} \\
			& \lambda_{1,2} = \mathrm{LCM} (\lambda_{1}; \lambda_{2}) = \SI{2,0}{\mu m} \\
			& \lambda_{2,3} = \mathrm{LCM} (\lambda_{2}; \lambda_{3}) = \SI{3,5}{\mu m}. \\
		\end{aligned} 
		\right.$ \\
		Ta có $\dfrac{\lambda_{\equiv}}{\lambda_{13}} = \num{5}$. Suy ra trong khoảng giữa hai vân sáng cùng màu vân trung tâm có 4 vị trí vân màu tím - đỏ. \\
		Ta có $\dfrac{\lambda_{\equiv}}{\lambda_{12}} = \num{7}$. Suy ra trong khoảng giữa hai vân sáng cùng màu vân trung tâm có 6 vị trí vân màu tím - lục. \\
		Suy ra số vân màu tím quan sát được là $34 - 4 - 6 = \num{24}$ vân sáng.
		Ta có $\dfrac{\lambda_{\equiv}}{\lambda_{23}} = \num{4}$. Suy ra trong khoảng giữa hai vân sáng cùng màu vân trung tâm có 3 vị trí vân màu lục - đỏ. \\
		Suy ra số vân màu tím quan sát được là $19 - 4 - 3 = \num{12}$ vân sáng.
	}
	
	%========================================
	\item \mkstar{4} [2]
	\cauhoi
	{Trong thí nghiệm Y-âng về giao thoa ánh sáng trắng (có bước sóng thay đối từ $\SI{380}{nm}$ đến $\SI{760}{nm}$), hai khe được chiếu sáng đồng thời bởi hai bức xạ đơn sắc có bước sóng lần lượt là $\lambda_{1}$ và $\lambda_{2}\left(\lambda_{2}>\lambda_{1}\right)$, khoảng cách giữa hai khe là $\SI{0,5}{mm}$, khoảng cách từ hai khe đến màn quan sát là $\SI{2}{m}$. Thấy vân sáng bậc 3 của bức xạ $\lambda_{1}$ trùng với vân sáng bậc $k$ của bức xạ $\lambda_{2}$ và cách vân trung tâm $\SI{6}{mm}$. Giá tri $k$ và $\lambda_{2}$ là
		\begin{mcq}(2)
			\item $k = 2$ và $\lambda_{2} = \SI{0,75}{\mu m}$. 
			\item $k = 2$ và $\lambda_{2} = \SI{4,2}{\mu m}$. 
			\item $k = 1$ và $\lambda_{2} = \SI{1,2}{\mu m}$. 
			\item $k = 1$ và $\lambda_{2} = \SI{4,8}{\mu m}$. 
		\end{mcq}
	}
	
	\loigiai
	{		\textbf{Đáp án: A.}
		
		Ta có:
		
		$$
		x_{\equiv} = 3 \dfrac{\lambda_{1} D}{a} \rightarrow \lambda_{1} = \SI{0,5}{\mu m}.
		$$
		Ta có:
		
		$$
		3 \lambda_{1} = k \lambda_{2} \rightarrow \lambda_{2} = \dfrac{1500}{k} \; (\si{nm}).
		$$
		Lại có:
		$$
		380 \leq \lambda_{2} \leq 760 \rightarrow 1,97 \leq k \leq 3,95.
		$$
		Vậy $k = 2$ và $\lambda_{2} = \SI{0,75}{\mu m}$.
	}   
	
	%========================================
	\item \mkstar{4} [1]
	\cauhoi
	{Trong thí nghiệm Y-âng về giao thoa ánh sáng, hai khe được chiếu bằng ánh sáng trắng có bước sóng thay đổi liên tục từ $\SI{380}{nm}$ đến $\SI{760}{nm}$. Trên màn, tại vị trí vân sáng bậc 5 của bức xạ lục có bước sóng $\SI{540}{nm}$, số bức xạ khác cũng cho vân sáng tại đó là
		\begin{mcq}(4)
			\item 3. 
			\item 6. 
			\item 5. 
			\item 4. 
		\end{mcq}
	}
	
	\loigiai
	{		\textbf{Đáp án: A.}
		
		Ta có:
		$$
		k_{5}\lambda_{l} = k \lambda \rightarrow \lambda = \dfrac{2700}{k} .
		$$
		Mặt khác:
		$$
		380 \leq \lambda \leq 760 \rightarrow 3,5 \leq \lambda \leq 7,1.
		$$
		Vậy có tổng cộng 3 bức xạ khác cũng cho vân sáng tại đó.
	}
	
	
\end{enumerate}

