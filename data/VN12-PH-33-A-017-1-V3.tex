
\chapter[Giao thoa ánh sáng đơn sắc]{Giao thoa ánh sáng đơn sắc}

\section{Lý thuyết}

\subsection{Vị trí vân sáng, vân tối - khoảng vân}


\begin{itemize}
	\item Hiệu đường đi của hai sóng kết hợp đến 1 điểm trên màn: 
	\begin{equation}
		d_2-d_1=\dfrac{ax}{D}.
	\end{equation}
	\item Khoảng vân: 
	\begin{equation}
		i=\dfrac{\lambda D}{a}.
	\end{equation}
	\item Vân sáng: 
	\begin{equation}
		d_2-d_1=\dfrac{ax}{D}=k\lambda \Leftrightarrow x=k\dfrac{\lambda D}{a}.
	\end{equation}
	
	Vân sáng trung tâm: $d_2 -d_1=0 \lambda \Leftrightarrow x= 0i$.
	
	Vân sáng bậc 1: $d_2 -d_1=\pm  \lambda \Leftrightarrow x= \pm i$.
	
	Vân sáng bậc 2: $d_2 -d_1=\pm  2\lambda \Leftrightarrow x= \pm 2i$.
	
	...
	
	Vân sáng bậc k: $d_2 -d_1=\pm  k\lambda \Leftrightarrow x= \pm ki$.
	
	\item Vân tối: 
	\begin{equation}
		d_2-d_1=\dfrac{ax}{D}=\left(k'-\dfrac{1}{2}\right) \lambda \Leftrightarrow x=\left(k'-\dfrac{1}{2}\right) i.
	\end{equation}
	
	Vân tối thứ 1: $d_2 -d_1=\pm (1-\text{0,5})  \lambda \Leftrightarrow x= \pm \text{0,5} i$.
	
	Vân tối thứ 2: $d_2 -d_1=\pm (2-\text{0,5})  \lambda \Leftrightarrow x= \pm \text{1,5} i$.
	
	...
	
	Vân tối thứ k': $d_2-d_1= \pm \left(k'-\dfrac{1}{2}\right) \lambda \Leftrightarrow  x=\pm\left(k'-\dfrac{1}{2}\right) i$.
\end{itemize}

\subsection{Xác định số vân trên trường giao thoa}


Vị trí vân sáng:
\begin{equation}
	x=k\dfrac{\lambda D}{a},\ k=\pm 1; \pm 2,...
\end{equation}

Vị trí vân tối: 
\begin{equation}
	x=\left(k'+\dfrac{1}{2}\right) \dfrac{\lambda D}{a},\ k'=0; \pm 1; \pm 2,... 
\end{equation}
trong đó:
\begin{itemize}
	\item $a$ là khoảng cách giữa hai nguồn kết hợp,
	\item $D$ là khoảng cách từ hai nguồn đến màn,
	\item $\lambda$ là bước sóng ánh sáng.
\end{itemize} 

\section{Mục tiêu bài học - Ví dụ minh họa}

\begin{dang}{Vị trí vân sáng, vân tối - khoảng vân.}
	\ppgiai{
		\begin{description}
			\item [Bước 1] Xác định số khoảng vân khi biết được bậc của vân sáng hoặc thứ của vân tối. 
			\item [Bước 2] Sử dụng công thức tính khoảng vân để tìm các đại lượng để bài yêu cầu.
		\end{description}
	}
	
	\viduii{2}{Trong thí nghiệm Y-âng về giao thoa ánh sáng, hai khe được chiếu bằng ánh sáng đơn sắc có bước sóng $\lambda$. Nếu tại điểm M trên màn quan sát có vân tối thứ tư (tính vân sáng trung tâm) thì hiệu đường đi của ánh sáng từ hai khe $\text{S}_1, \text{S}_2$ đến M có độ lớn bằng
		\begin{mcq}(4)
			\item 3,5 $\lambda$.
			\item 3 $\lambda$.
			\item 2,5 $\lambda$.
			\item 2 $\lambda$.
		\end{mcq}
	}
	{	\begin{center}
			\textbf{Hướng dẫn giải}
		\end{center}
		
		Nếu tại điểm M trên màn quan sát có vân tối thứ tư (tính vân sáng trung tâm) thì hiệu đường đi:
		
		\begin{equation*}
			d_2 - d_1 = (4-\text{0,5})\lambda = \text{3,5} \lambda.
		\end{equation*}
		
		
		\begin{center}
			\textbf{Câu hỏi tương tự}
		\end{center}
		
		Trong thí nghiệm Y-âng về giao thoa ánh sáng, hai khe được chiếu bằng ánh sáng đơn sắc có bước sóng $ \lambda $. Nếu tại điểm $ M $ trên màn quan sát có vân tối thứ ba (tính từ vân trung tâm) thì hiệu đường đi của ánh sáng từ hai khe đến $ M $ có độ lớn bằng
		\begin{mcq}(4)
			\item $ 3\lambda $.
			\item $ 2\lambda $.
			\item $ \num{1,5}\lambda $.
			\item $ \num{2,5}\lambda $.
		\end{mcq}
		
		
		\textbf{Đáp án:} D.
	}
	
	\viduii{2}
	{
		Trong thí nghiệm Y-âng về giao thoa ánh sáng, các khe hẹp được chiếu sáng bởi ánh sáng đơn sắc. Khoảng vân trên màn là $\text{1,2}\ \text{mm}$. Trong khoảng giữa hai điểm M và N trên màn ở cùng một phía so với vân sáng trung tâm, cách vân trung tâm lần lượt $2 \text{mm}$ và $\text{4,5}\ \text{mm}$, quan sát được
		\begin{mcq}(2)
			\item 2 vân sáng và 2 vân tối.		
			\item 3 vân sáng và 2 vân tối.
			\item 2 vân sáng và 3 vân tối.		
			\item 2 vân sáng và 1 vân tối.
		\end{mcq}
	}
	{
		\begin{center}
			\textbf{Hướng dẫn giải}
		\end{center}
		
		\begin{itemize}
			
			\item Vì hai điểm M và N trên màn ở cùng một phía so với vân sáng trung tâm nên có thể chọn $x_{\text{M}}= 2\ \text{mm}$ và $x_{\text{N}}= \text{4,5} \text{mm}$.
			
			\item Số vân sáng:
			\begin{equation*}
				x_{\text{M}} \leq  ki \leq x_{\text{N}} \Rightarrow 2 \leq \text{1,2} k \leq \text{4,5} \Rightarrow \text{1,67}  \leq k \leq \text{3,75} \Rightarrow k = 2, 3. 
			\end{equation*}
			\item Số vân sáng:
			\begin{equation*}
				x_{\text{M}} \leq  (m+\text{0,5}) \leq x_{\text{N}} \Rightarrow 2 \leq \text{1,2} (m+\text{0,5}) \leq \text{4,5} \Rightarrow \text{1,17}  \leq m \leq \text{3,25} \Rightarrow k = 2, 3. 
			\end{equation*}
			
			Suy ra có thể quan sát được 2 vân sáng và 2 vân tối.
		\end{itemize}		
		
		\begin{center}
			\textbf{Câu hỏi tương tự}
		\end{center}
		
		Trong thí nghiệm Y-âng về giao thoa ánh sáng, các khe hẹp được chiếu sáng bởi ánh sáng đơn sắc. Khoảng vân trên màn là $ \SI{1,4}{mm} $. Trong khoảng giữa $ M $ và $ N $ trên màn ở cùng một phía so với vân sáng trung tâm lần lượt $ \SI{2}{mm} $ và $ \SI{4,5}{mm} $, quan sát được	
		\begin{mcq}(2)
			\item 2 vân sáng và 2 vân tối.		
			\item 3 vân sáng và 2 vân tối.
			\item 2 vân sáng và 3 vân tối.		
			\item 2 vân sáng và 1 vân tối.
		\end{mcq}	
		\textbf{Đáp án: A.}
	}
\end{dang}

\begin{dang}{Xác định số vân trên trường giao thoa.}
	\ppgiai{
		Với $L$ là bề rộng trường giao thoa. 
		
		\begin{description}
			
			\item [Bước 1] Tìm khoảng vân $i$. 
			\item [Bước 2] Tính tỉ số $y=\dfrac{L}{2i}$.
			\item [Bước 3] Xác định số vân sáng hoặc số vân tối theo quy tắc:
			
			Số vân sáng $=2[y] +1$, với $[y]$ có nghĩa là lấy phần nguyên của $y$.
			
			Số vân tối $=2\{y\}$, trong đó $\{y\}$ có nghĩa là làm tròn số đến số nguyên gần nhất của $y$.	
		\end{description}
	}
	
	\viduii{2}
	{Trong một thí nghiệm về Giao thoa ánh sáng bằng khe I âng với ánh sáng đơn sắc $\lambda  = \text{0,7}\ \mu \text{m}$, khoảng cách giữa 2 khe $S_1, S_2$ là $a = \text{0,35}\ \text{mm}$ , khoảng cách từ 2 khe đến màn quan sát là $D = 1\ \text{m}$, bề rộng của vùng có giao thoa là $\text{13,5}\ \text{mm}$. Số vân sáng, vân tối quan sát được trên màn là
		\begin{mcq}(2)
			\item 7 vân sáng, 6 vân tối.        
			\item 6 vân sáng, 7 vân tối.
			\item 6 vân sáng, 6 vân tối.          
			\item 7 vân sáng, 7 vân tối.
		\end{mcq}
	}
	{
		\begin{center}
			\textbf{Hướng dẫn giải}
		\end{center}
		
		
		\begin{itemize}
			\item Khoảng vân: 
			\begin{equation*}
				i=\dfrac{\lambda D}{a}=\dfrac{\text {0,7} \cdot 10^{-6} \cdot 1}{\text{0,35} \cdot 10^{-3}}= 2 \cdot 10^{-3}\ \text{m}.
			\end{equation*}
			\item Xét: $\dfrac{\text{13,5}\cdot 10^{-3}}{2} : (2 \cdot 10^{-3}) =\text{3,375}$.
			\item Số vân sáng: $2 \cdot 3 + 1 = 7$.
			\item Số vân tối: $2 \cdot 3 =6$. 
			
		\end{itemize}
		
		\begin{center}
			\textbf{Câu hỏi tương tự}
		\end{center}
		
		Trong một thí nghiệm về giao thoa ánh sáng bằng khe Young với ánh sáng đơn sắc $ \SI{0,74}{\mu m} $, khoảng cách giữa hai khe $ S_{1} S_{2} $ là $ a = \SI{2}{mm} $, khoảng cách từ hai khe đến màn quan sát là $ D = \SI{1}{m} $. Bề rộng của vùng giao thoa là $ \SI{8}{mm} $. Số vân sáng, vân tối quan sát được trên màn là
		\begin{mcq}(2)
			\item 11 vân sáng, 10 vân tối.        
			\item 22 vân sáng, 20 vân tối.
			\item 21 vân sáng, 22 vân tối.          
			\item 22 vân sáng, 22 vân tối.
		\end{mcq}
		\textbf{Đáp án: C.}
	}
\end{dang}

\section{Bài tập tự luyện}
\begin{enumerate}[label=\bfseries Câu \arabic*:]
	
	%========================================
	\item \mkstar{2} [13]
	\cauhoi
	{Trong thí nghiệm Y-âng về giao thoa ánh sáng. Khi tiến hành trong không khí, người ta đo được khoảng vân là $i = \SI{2}{mm}$. Tiến hành thí nghiệm trong nước, nước có chiết suất tuyệt đối là $n = \dfrac{4}{3}$ thì khoảng vân đo được trong nước là
		\begin{mcq}(4)
			\item $\SI{2,5}{mm}$. 
			\item $\SI{2}{mm}$. 
			\item $\SI{1,5}{mm}$. 
			\item $\SI{1,25}{mm}$. 
		\end{mcq}
	}
	
	\loigiai
	{		\textbf{Đáp án: C.}
		
		Gọi $i'$ là khoảng vân đo được trong môi trường nước. Ta có
		$$
		i' = \dfrac{i}{n} = \SI{1,5}{mm}.
		$$
	}
	
	%========================================
	\item \mkstar{3} [9]
	\cauhoi
	{Trong thí nghiệm Y-âng về giao thoa ánh sáng đơn sắc, khoảng cách giữa hai khe là $\SI{1}{mm}$, khoảng cách từ mặt phẳng chứa hai khe đến màn là $\SI{2}{m}$. Trong hệ vân trên màn, vân sáng bậc 3 cách vân trung tâm là $\SI{2,4}{mm}$. Bước sóng của ánh sáng đơn sắc dùng trong thí nghiệm là 
		\begin{mcq}(4)
			\item $\SI{0,5}{\mu m}$. 
			\item $\SI{0,7}{\mu m}$. 
			\item $\SI{0,4}{\mu m}$. 
			\item $\SI{0,6}{\mu m}$. 
		\end{mcq}
	}
	
	\loigiai
	{		\textbf{Đáp án: C.}
		
		Vân sáng bậc 3 cách vân trung tâm $3i$ nên
		$$
		3i = \SI{2,4}{mm} \rightarrow i = \SI{0,8}{mm}.
		$$
		Ta có
		$$
		i = \dfrac{\lambda D}{a} \rightarrow \lambda = \SI{0,4}{\mu m}.
		$$
	}
	
	%========================================
	\item \mkstar{3} [9]
	\cauhoi
	{Thực hiện giao thoa ánh sáng theo khe Y-âng với $a = \SI{2}{mm}$, $D= \SI{1}{m}$, nguồn S phát ra ánh sáng đơn sắc có bước sóng $\lambda = \SI{0,5}{\mu m}$. Khoảng cách từ vân sáng bậc 5 đến vân tối bậc 7 ở hai bên vân sáng trung tâm là
		\begin{mcq}(4)
			\item $\SI{2,875}{mm}$. 
			\item $\SI{11,5}{mm}$. 
			\item $\SI{2,6}{mm}$. 
			\item $\SI{12,5}{mm}$. 
		\end{mcq}
	}
	
	\loigiai
	{		\textbf{Đáp án: A.}
		
		Khoảng cách giữa vân sáng bậc 5 và vân tối bậc 7 ở hai phía so với vân trung tâm là
		$$
		5i + 6,5i = 11,5i = 11,5 \dfrac{\lambda D}{a} = \SI{2,875}{mm}.
		$$
	}
	
	%========================================
	\item \mkstar{3} [9]
	\cauhoi
	{Trong thí nghiệm Y-âng về giao thoa ánh sáng đơn sắc, người ta đo được khoảng cách từ vân sáng bậc 2 đến vân sáng bậc 5 cùng một phía so với vân trung tâm là $\SI{3}{mm}$. Số vân sáng quan sát được trên MN đối xứng với nhau qua vân sáng trung tâm có bề rộng là $\SI{13}{mm}$ là
		\begin{mcq}(4)
			\item 9 vân. 
			\item 13 vân. 
			\item 15 vân. 
			\item 11 vân. 
		\end{mcq}
	}
	
	\loigiai
	{		\textbf{Đáp án: B.}
		
		Khoảng cách từ vân sáng bậc 2 đến vân sáng bậc 5 là $3i$ nên
		$$
		3i = \SI{3}{mm} \rightarrow i = \SI{1}{mm}.
		$$
		Ta có:
		$$
		\dfrac{L}{2i} = \num{6,5}.
		$$
		Số vân sáng trên đoạn MN là 13 vân.
	}
	
	%========================================
	\item \mkstar{1} [23]
	\cauhoi
	{ Trong thí nghiệm giao thoa khe Y-âng, khoảng cách giữa hai vân sáng cạnh nhau là
		
		\begin{mcq}(4)
			\item $\dfrac{\lambda}{aD}$. 
			\item $\dfrac{ax}{D}$. 
			\item $\dfrac{\lambda a}{D}$. 
			\item $\dfrac{\lambda D}{a}$. 
		\end{mcq}
	}
	
	\loigiai
	{		\textbf{Đáp án: D.}
		
		Khoảng cách giữa hai vân sáng cạnh nhau là khoảng vân $i = \dfrac{\lambda D}{a}$.		
	}
	
	%========================================
	\item \mkstar{1} [13]
	\cauhoi
	{Trong một thí nghiệm Y-âng về giao thoa ánh sáng, bước sóng ánh sáng đơn sắc là $\lambda$ (m). Khoảng cách giữa hai khe hẹp là $a$ (m. Khoảng cách từ mặt phẳng phân cách chứa hai khe đến màn là $D$ (m). Vị trí vân tối có tọa độ $x_{k}$ là
		\begin{mcq}(2)
			\item $x_{k} = (2k+1)\dfrac{\lambda D}{a}$. 
			\item $x_{k} = k \dfrac{\lambda D}{a}$. 
			\item $x_{k} = (2k+1) \dfrac{\lambda D}{2a}$. 
			\item $x_{k} = k \dfrac{\lambda D}{2a}$. 
		\end{mcq}
	}
	
	\loigiai
	{		\textbf{Đáp án: C.}
		
		Vị trí vân tối là 
		$$
		x_{k} = (2k+1) \dfrac{\lambda D}{2a}.
		$$
	}
	
	%========================================
	\item \mkstar{2} [7]
	\cauhoi
	{Trong thí nghiệm Y-âng về giao thoa ánh sáng, hai khe được chiếu bằng ánh sáng đơn sắc có bước sóng $\SI{0,6}{\mu m}$, khoảng cách giữa hai khe hẹp là $\SI{1}{mm}$, khoảng cách từ mặt phẳng chứa hai khe đến màn quan sát là $\SI{2,5}{m}$. Khoảng vân giao thoa trên màn là
		
		\begin{mcq}(4)
			\item $\SI{1,5}{mm}$.
			\item $\SI{1,0}{mm}$. 	
			\item $\SI{2,0}{mm}$.  	
			\item $\SI{0,5}{mm}$. 
		\end{mcq}
	}
	
	\loigiai
	{		\textbf{Đáp án: A.}
		
		Khoảng vân trên màn cho bởi
		$$
		i = \dfrac{\lambda D}{a} = \SI{1,5}{mm}.
		$$
	}
	
	
	%========================================
	\item \mkstar{3} [5]
	\cauhoi
	{Trong thí nghiệm giao thoa khe Y-âng, nguồn sóng có bước sóng là $\SI{380}{nm}$.  Khoảng cách giữa hai khe hẹp là $\SI{2}{mm}$. Khoảng cách giữa hai khe đến màn là $\SI{2}{m}$. Khoảng vân là
		\begin{mcq}(4)
			\item $\SI{3}{mm}$. 
			\item $\SI{0,38}{mm}$. 
			\item $\SI{0,62}{mm}$. 
			\item $\SI{0,54}{mm}$. 
		\end{mcq}
	}
	
	\loigiai
	{		\textbf{Đáp án: B.}
		
		Khoảng vân của hệ cho bởi
		$$
		i = \dfrac{\lambda D}{a} = \SI{0,38}{mm}.
		$$
	}
	
	%========================================
	\item \mkstar{3} [5]
	\cauhoi
	{Khi thực hiện giao thoa với ánh sáng đơn sắc, nếu hai khe Y-âng cách nhau $\SI{1,2}{mm}$ thì khoảng vân là $i = \SI{1,21}{mm}$. Nếu khoảng cách giữa hai khe giảm đi $\SI{0,2}{mm}$ thì khoảng vân sẽ 
		\begin{mcq}(2)
			\item giảm đi $\SI{0,24}{mm}$. 
			\item giảm đi $\SI{0,11}{mm}$. 
			\item tăng thêm $\SI{0,24}{mm}$. 
			\item tăng thêm $\SI{0,11}{mm}$. 
		\end{mcq}
	}
	
	\loigiai
	{		\textbf{Đáp án: C.}
		
		Ta có:
		$$
		\dfrac{i'}{i} = \dfrac{a}{a'} = \dfrac{a}{a- \Delta a} \rightarrow i' = \SI{1,452}{mm}.
		$$
		Như vậy khoảng vân sẽ tăng thêm $\SI{0,24}{mm}$.
	}
	
	%========================================
	\item \mkstar{3} [13]
	\cauhoi
	{Trong thí nghiệm Y-âng về giao thoa ánh sáng có $\lambda = \SI{0,75}{\mu m}$, $a = \SI{0,6}{mm}$ và $D = \SI{1,2}{m}$. Tại điểm M trên màn quan sát cách vân trung tâm $\SI{5,25}{mm}$ có vân
		\begin{mcq}(2)
			\item sáng bậc 3. 
			\item tối thứ 3. 
			\item tối thứ 4. 
			\item sáng bậc 4. 
		\end{mcq}
	}
	
	\loigiai
	{		\textbf{Đáp án: C.}
		
		Khoảng vân $i$ là
		$$
		i = \dfrac{\lambda D}{a} = \SI{1,5}{mm}. 
		$$
		Ta có:
		$$
		\dfrac{x}{i} = 3,5.
		$$
		Vậy tại $x = \SI{5,25}{mm}$ là vân tối thứ 4.
	}
	
	%========================================
	\item \mkstar{3} [13]
	\cauhoi
	{Trong thí nghiệm Y-âng về giao thoa ánh sáng có $a = \SI{1}{mm}$ và $D = \SI{1}{m}$. Trên màn quan sát ta thấy vân sáng bậc 5 cách vân sáng trung tâm là $\SI{3,0}{mm}$. Bước sóng ánh sáng dùng trong thí nghiệm là
		\begin{mcq}(4)
			\item $\lambda = \SI{0,70}{\mu m}$. 
			\item $\lambda = \SI{0,50}{\mu m}$. 
			\item $\lambda = \SI{0,60}{\mu m}$. 
			\item $\lambda = \SI{0,53}{\mu m}$. 
		\end{mcq}
	}
	
	\loigiai
	{		\textbf{Đáp án: C.}
		
		Khoảng cách từ vân sáng bậc 5 đến vân sáng trung tâm là
		$$
		\Delta d = 5i \rightarrow  i = \SI{0,6}{mm}.
		$$
		Khoảng vân cho bởi
		$$
		i = \dfrac{\lambda D}{a} \rightarrow \lambda = \SI{0,60}{\mu m}.
		$$
	}
	
	%========================================
	\item \mkstar{2} [13]
	\cauhoi
	{Trong thí nghiệm Y-âng về giao thoa ánh sáng với khoảng vân là $i$, khoảng cách từ vân sáng bậc 3 đến vân sáng bậc 7 ở cùng một phía so với vân sáng trung tâm là 
		\begin{mcq}(4)
			\item $3i$. 
			\item $4i$. 
			\item $7i$. 
			\item $5i$. 
		\end{mcq}
	}
	
	\loigiai
	{		\textbf{Đáp án: B.}
		
		Khoảng cách giữa vân sáng bậc 3 và vân sáng bậc 7 cho bởi
		$$
		\Delta d = 7i - 3i = 4i.
		$$
	}
	
	%========================================
	\item \mkstar{2} [13]
	\cauhoi
	{Trong thí nghiệm Y-âng về giao thoa ánh sáng có $\lambda = \SI{0,5}{\mu m}$, $a = \SI{1,2}{mm}$ và $D = \SI{2,5}{m}$. Vân tối thứ ba cách vân trung tâm một khoảng bằng
		\begin{mcq}(4)
			\item $\SI{2,1}{mm}$. 
			\item $\SI{3,1}{mm}$. 
			\item $\SI{2,6}{mm}$. 
			\item $\SI{3,6}{mm}$. 
		\end{mcq}
	}
	
	\loigiai
	{		\textbf{Đáp án: C.}
		
		Vân tối thứ ba cách vân trung tâm một đoạn là
		$$
		x_{t_{3}} = 2,5i = 2,5 \dfrac{\lambda D}{a} = \SI{2,6}{mm}.
		$$
	}
	
	%========================================
	\item \mkstar{2} [13]
	\cauhoi
	{Trong thí nghiệm Y-âng về giao thoa ánh sáng có $\lambda = \SI{0,5}{\mu m}$, $a = \SI{1}{mm}$ và $D = \SI{2}{m}$. Khoảng vân $i$ bằng
		\begin{mcq}(4)
			\item $\SI{2,5}{mm}$. 
			\item $\SI{5,0}{mm}$. 
			\item $\SI{2,0}{mm}$. 
			\item $\SI{1,0}{mm}$. 
		\end{mcq}
	}
	
	\loigiai
	{		\textbf{Đáp án: D.}
		
		Khoảng vân $i$ là
		$$
		i = \dfrac{\lambda D}{a} = \SI{1,0}{mm}.
		$$
	}
	
	%========================================
	\item \mkstar{2} [13]
	\cauhoi
	{Trong thí nghiệm về giao thoa ánh sáng có $\lambda = \SI{0,7}{\mu m}$, $a = \SI{0,35}{mm}$ và $D = \SI{1}{m}$. Bề rộng của trường giao thoa là $L = \SI{18,2}{mm}$. Số vân sáng quan sát được là
		\begin{mcq}(4)
			\item $11$. 
			\item $8$. 
			\item $10$. 
			\item $9$. 
		\end{mcq}
	}
	
	\loigiai
	{		\textbf{Đáp án: D.}
		
		Khoảng vân là
		$$
		i = \dfrac{\lambda D}{a} = \SI{2}{mm}.
		$$
		Ta có
		$$
		\dfrac{L}{2i} = \num{4,55}.
		$$
		Số vân quan sát được là 9 vân.
	}
	
	%========================================
	\item \mkstar{3} [13]
	\cauhoi
	{Đoạn MN trên màn quan sát trong thí nghiệm Y-âng về giao thoa ánh sáng. Khi dùng ánh sáng có bước sóng $\lambda$ thì khoảng cách giữa hai vân sáng liên tiếp là $\SI{1,2}{mm}$. Trên MN có 16 vân sáng quan sát được (tại M, N đều là vân sáng). Đoạn MN có giá trị là
		\begin{mcq}(4)
			\item $\SI{10}{mm}$. 
			\item $\SI{12}{mm}$. 
			\item $\SI{16}{mm}$. 
			\item $\SI{18}{mm}$. 
		\end{mcq}
	}
	
	\loigiai
	{		\textbf{Đáp án: D.}
		
		Khoảng cách giữa hai vân sáng liên tiếp cũng là khoảng vân nên $i = \SI{1,2}{mm}$.
		Tại M và N là các vân sáng và trên MN có tổng cộng 16 vân sáng quan sát được thì
		$$
		MN = 15i = \SI{18}{mm}.
		$$
	}
	%========================================
	\item \mkstar{2} [12]
	\cauhoi
	{Thực hiện thí nghiệm Y-âng về giao thoa ánh sáng với khoảng cách giữa hai khe sáng là $\SI{2}{mm}$, khoảng cách từ mặt phẳng chứa hai khe đến màn quan sát là $\SI{2}{m}$, ánh sáng đơn sắc có bước sóng $\SI{0,7}{\mu m}$. Khoảng cách giữa vân sáng và vân tối liền kề là 
		
		\begin{mcq}(4)
			\item $\SI{0,0875}{mm}$. 
			\item $\SI{0,3500}{mm}$.
			\item $\SI{0,7000}{mm}$.
			\item $\SI{0,1750}{mm}$.
		\end{mcq} 
	}
	
	\loigiai
	{		\textbf{Đáp án: B.}
		
		Khoảng cách giữa vân sáng và vân tối liền kề là $\num{0,5} i = \num{0,5} \dfrac{\lambda D}{a} = \SI{0,3500}{mm}$.		
	}
	
	%========================================
	\item \mkstar{2} [12]
	\cauhoi
	{Thực hiện thí nghiệm Y-âng với ánh sáng đơn sắc có bước sóng $\lambda$. Biết khoảng cách giữa hai khe là $a$, khoảng cách từ mặt phẳng chứa hai khe đến màn quan sát là $D$ thì vân sáng bậc 3 cách vân trung tâm 
		\begin{mcq}(4)
			\item $4 \dfrac{\lambda D}{a}$. 
			\item $3 \dfrac{\lambda a}{D}$. 
			\item $3 \dfrac{\lambda D}{a}$. 
			\item $4 \dfrac{\lambda a}{D}$. 
		\end{mcq}
	}
	
	\loigiai
	{		\textbf{Đáp án: C.}
		
		Vân sáng bậc 3 cách vân trung tâm một khoảng là $3 \dfrac{\lambda D}{a}$.	
	}
	
	%========================================
	\item \mkstar{2 } [12]
	\cauhoi
	{Khoảng cách giữa 2 vân sáng liền kề là $\SI{0,5}{mm}$ thì khoảng vân có giá trị 
		\begin{mcq}(4)
			\item $\SI{1,5}{mm}$.
			\item $\SI{1,0}{mm}$. 
			\item $\SI{0,5}{mm}$.
			\item $\SI{0,25}{mm}$.
		\end{mcq}
	}
	
	\loigiai
	{		\textbf{Đáp án: C.}
		
		Khoảng cách giữa 2 vân sáng liền kề là $\SI{0,5}{mm}$ thì khoảng vân cũng có giá trị $\SI{0,5}{mm}$.	
	}
	
	%========================================
	\item \mkstar{3} [10]
	\cauhoi
	{Trong thí nghiệm Y-âng về giao thoa với ánh sáng đơn sắc, khoảng cách giữa hai khe là $\SI{1}{mm}$, khoảng cách từ mặt phẳng chứa hai khe đến màn quan sát là $\SI{2}{m}$ và khoảng vân là $\SI{0,8}{mm}$. Cho $c = 3 \cdot 10^{8} \; \si{m/s}$. Tần số ánh sáng đơn sắc dùng trong thí nghiệm là
		
		\begin{mcq}(4)
			\item $\text{5,5} \cdot 10^{14} \si{Hz}$. 
			\item $\text{4,5} \cdot 10^{14} \si{Hz}$. 
			\item $\text{7,5} \cdot 10^{14} \si{Hz}$.
			\item $\text{6,5} \cdot 10^{14} \si{Hz}$. 
		\end{mcq}
	}
	
	\loigiai
	{		\textbf{Đáp án: C.}
		
		Khoảng vân cho bởi
		$$
		i = \dfrac{\lambda D}{a}  \rightarrow \lambda = \SI{0,4}{\mu m}
		$$
		Tần số ánh sáng cho bởi
		$$
		f = \dfrac{c}{\lambda} = \SI{7,5 e14}{Hz}.
		$$
	}
	%========================================
	\item \mkstar{2} [10]
	\cauhoi
	{Trong thí nghiệm Y-âng về giao thoa ánh sáng, nguồn sáng phát ra ánh sáng đơn sắc có bước sóng $\SI{500}{nm}$. Khoảng cách giữa hai khe là $\SI{1}{mm}$, khoảng cách từ mặt phẳng chứa hai khe đến màn quan sát là $\SI{1}{m}$. Trên màn, khoảng cách giữa hai vân sáng liên tiếp bằng
		\begin{mcq}(4)
			\item $\SI{0,50}{mm}$.
			\item $\SI{1,0}{mm}$.
			\item $\SI{1,5}{mm}$. 
			\item $\SI{0,75}{mm}$. 
		\end{mcq}
	}
	
	\loigiai
	{		\textbf{Đáp án: A.}
		
		Khoảng cách giữa hai vân sáng là
		$$
		i = \dfrac{\lambda D}{a} = \SI{0,5}{mm}.
		$$
	}
	
	%========================================
	\item \mkstar{2} [4]
	\cauhoi
	{Trong thí nghiệm Y-âng về giao thoa ánh sáng đơn sắc có bước sóng $\lambda = \SI{0,6}{\mu m}$. Khoảng cách giữa hai khe là $\SI{1}{mm}$, khoảng cách từ hai khe đến màn là $\SI{2}{m}$. Vân sáng thứ tư cách vân trung tâm một khoảng là
		\begin{mcq}(4)
			\item $\SI{4,2}{mm}$. 
			\item $\SI{3,6}{mm}$. 
			\item $\SI{4,8}{mm}$. 
			\item $\SI{6,0}{mm}$. 
		\end{mcq}
	}
	
	\loigiai
	{		\textbf{Đáp án: C.}
		
		Vân sáng thứ tư cách vân trung tâm một khoảng là
		$$
		x_{4} = 4 \dfrac{\lambda D}{a} = \SI{4,8}{mm}.
		$$
	}
	
	%========================================
	\item \mkstar{3} [4]
	\cauhoi
	{Trong thí nghiệm Y-âng về giao thoa ánh sáng, hai khe cách nhau $\SI{3}{mm}$ được chiếu bằng ánh sáng đơn sắc có bước sóng $\SI{0,6}{\mu m}$. Các vân giao thoa được hứng trên màn cách hai khe $\SI{2}{m}$. Tại điểm M cách vân trung tâm $\SI{1,2}{mm}$ có
		\begin{mcq}(2)
			\item vân sáng bậc 3. 
			\item vân tối thứ 3. 
			\item vân sáng bậc 5. 
			\item vân sáng bậc 4. 
		\end{mcq}
	}
	
	\loigiai
	{		\textbf{Đáp án: A.}
		
		Khoảng vân là $i = \dfrac{\lambda D}{a} = \SI{0,4}{mm}$.
		Tại điểm M, ta có:
		$$
		\dfrac{x_{M}}{i} = 3.
		$$
		Vậy tại M là vân sáng bậc 3.
	}
	
	%========================================
	\item \mkstar{3} [2]
	\cauhoi
	{Trong thí nghiệm Y-âng về giao thoa ánh sáng đơn sắc, ta thấy tại điểm M trên màn có vân sáng bậc 5. Dịch chuyển màn quan sát ra xa thêm $\SI{20}{cm}$ thì tại M có vân tối thứ 5 tính từ trung tâm. Trước lúc dịch chuyển, khoảng cách từ màn đến hai khe bằng
		\begin{mcq}(4)
			\item $\SI{2}{m}$. 
			\item $\SI{1,8}{m}$. 
			\item $\SI{1,6}{m}$. 
			\item $\SI{2,2}{m}$. 
		\end{mcq}
	}
	
	\loigiai
	{		\textbf{Đáp án: B.}
		
		Ban đầu, vị trí của điểm M cho bởi
		$$
		x_{M} = 5 \dfrac{\lambda D}{a}.
		$$
		Lúc sau, vị trí điểm M cho bởi
		$$
		x_{M} = \num{4,5} \dfrac{\lambda (D + \Delta D)}{a}.
		$$
		Từ hai phương trình trên ta được:
		$$
		5D = \num{4,5}(D + \Delta D) \rightarrow D = \SI{1,8}{m}.
		$$
	}
	
	
	%========================================
	\item \mkstar{3} [3]
	\cauhoi
	{Trong thí nghiệm Y-âng về giao thoa ánh sáng, hai khe được chiếu bằng ánh sáng đơn sắc có bước sóng $\lambda$. Khoảng cách giữa hai khe sáng là $a$, khoảng cách từ mặt phẳng chứa hai khe đến màn quan sát là $\SI{1}{m}$. Trên màn quan sát, hai vân sáng bậc 3 nằm ở hai điểm M và N. Dịch màn quan sát một đoạn $\SI{50}{cm}$ theo hướng ra xa 2 khe thì số vân sáng trên đoạn MN giảm so với lúc đầu là 
		\begin{mcq}(4)
			\item 2 vân. 
			\item 5 vân. 
			\item 7 vân. 
			\item 4 vân. 
		\end{mcq}
	}
	
	\loigiai
	{		\textbf{Đáp án: A.}
		
		Vì tại M và N là vị trí vân sáng bậc 3 ở hai bên vân trung tâm nên $\text{MN} = 6i$.
		Ta có:
		$$
		\dfrac{i'}{i} = \dfrac{D'}{D} = \dfrac{D + \Delta D}{D} = 1,5 \rightarrow i' = 1,5i.
		$$
		Ta có:
		$$
		\dfrac{\text{MN}}{2i'} = \dfrac{6i}{3i} = 2 \rightarrow MN = 2i'.
		$$
		Vậy số vân sáng trên đoạn MN lúc sau là 5 vân, giảm 2 vân so với lúc đầu là 7 vân.
	}
	
	%=========================================
	\item \mkstar{3} [1]
	\cauhoi
	{Trong thí nghiệm Y-âng về giao thoa ánh sáng, hai khe được chiếu bằng ánh sáng đơn sắc. Khoảng cách giữa hai khe là $\SI{1,2}{mm}$. Khoảng cách từ mặt phẳng chứa hai khe đến màn quan sát là $\SI{1,5}{m}$. Trên màn quan sát, xét đoạn gồm 6 vân sáng liên tiếp cạnh nhau thì hai vân sáng ngoài cùng cách nhau $\SI{3}{mm}$. Bước sóng của ánh sáng đơn sắc này bằng
		\begin{mcq}(4)
			\item $\SI{600}{nm}$. 
			\item $\SI{400}{nm}$. 
			\item $\SI{500}{nm}$. 
			\item $\SI{480}{nm}$. 
		\end{mcq}
	}
	
	\loigiai
	{		\textbf{Đáp án: D.}
		
		Độ rộng của 6 vân sáng liên tiếp là $5i$, suy ra $i = \SI{0,6}{mm}$.
		Khoảng vân cho bởi:
		$$
		i = \dfrac{\lambda D}{a} \rightarrow \lambda = \SI{480}{nm}.
		$$
	}
	
	%========================================
	\item \mkstar{2} [13]
	\cauhoi
	{Trong thí nghiệm Y-âng về giao thoa ánh sáng có $a = \SI{1}{mm}$ và $D = \SI{1}{m}$. Trên màn quan sát thấy vân sáng bậc 5 cách vân sáng trung tâm $\SI{3,0}{mm}$. Bước sóng ánh sáng đơn sắc dùng trong thí nghiệm là
		\begin{mcq}(4)
			\item $\lambda = \SI{0,70}{\mu m}$. 
			\item $\lambda = \SI{0,50}{\mu m}$. 
			\item $\lambda = \SI{0,60}{\mu m}$. 
			\item $\lambda = \SI{0,53}{\mu m}$. 
		\end{mcq}
	}
	
	\loigiai
	{		\textbf{Đáp án: C.}
		
		Vị trí của vân sáng bậc 5 cho bởi:
		$$
		x_{5} = 5 \dfrac{\lambda D}{a} \rightarrow \lambda = \SI{0,6}{\mu m}.
		$$
	}
	
	%========================================
	\item \mkstar{3} [1]
	\cauhoi
	{Trong thí nghiệm Y-âng về giao thoa ánh sáng, hai khe được chiếu bằng ánh sáng đơn sắc có bước sóng $\SI{0,48}{\mu m}$. Khoảng cách giữa hai khe là $\SI{0.9}{mm}$, khoảng cách từ mặt phẳng chứa hai khe đến màn quan sát là $\SI{1,5}{m}$. Trên màn quan sát, so với vân sáng trung tâm, vân tối thứ ba cách vân sáng trung tâm
		\begin{mcq}(4)
			\item $\SI{2,8}{mm}$. 
			\item $\SI{2,4}{mm}$. 
			\item $\SI{2}{mm}$. 
			\item $\SI{1,6}{mm}$. 
		\end{mcq}
	}
	
	\loigiai
	{		\textbf{Đáp án: C.}
		
		Khoảng vân cho bởi
		$$
		i = \dfrac{\lambda D}{a} = \SI{0,8}{mm}.
		$$
		Khoảng cách từ vân trung tâm tới vân tối thứ ba là $2,5i = \SI{2}{mm}$.
	}
	
	%========================================
	\item \mkstar{4} [32]
	\cauhoi
	{Trong thí nghiệm Y-âng về giao thoa ánh sáng đơn sắc. Khoảng cách giữa hai khe là $\SI{1}{mm}$ và khoảng cách từ hai khe tới màn quan sát là $\SI{2}{m}$. Trong khoảng bề rộng $\SI{24,5}{mm}$ trên màn quan sát có 25 vân tối. Biết một đầu bề rộng là vân tối, đầu còn lại là vân sáng. Bước sóng của ánh sáng đó là
		\begin{mcq}(4)
			\item $\SI{0,48}{\mu m}$. 
			\item $\SI{0,40}{\mu m}$. 
			\item $\SI{0,50}{\mu m}$. 
			\item $\SI{0,52}{\mu m}$. 
		\end{mcq}
	}
	
	\loigiai
	{		\textbf{Đáp án: C.}
		
		Một khoảng bề rộng có một đầu là vân sáng, đầu còn lại là vân tối với 25 vân tối thì bề rộng này có kích thước là $24,5i$. Ta có:
		$$
		\num{24,5}i = \SI{24,5}{mm} \rightarrow i = \SI{1}{mm}.
		$$
		Ta có:
		$$
		i = \dfrac{\lambda D}{a} \rightarrow \lambda = \SI{0,50}{\mu m}.
		$$
	}
	
	%========================================
	\item \mkstar{4} [4]
	\cauhoi
	{Trong thí nghiệm Y-âng về giao thoa ánh có bước sóng $\lambda$, khoảng cách giữa hai khe là $\SI{1}{mm}$. Ban đầu, tại M cách vị trí trung tâm $\SI{5,25}{mm}$ người ta quan sát được vân sáng bậc 5. Giữ cố định màn chứa hai khe, di chuyển màn quan sát ra xa và dọc theo đường vuông góc với mặt phẳng chứa hai khe một đoạn $\SI{0,75}{m}$ thì ta thấy điểm M chuyển thành vân tối lần thứ hai. Bước sóng $\lambda$ có giá trị là
		\begin{mcq}(4)
			\item $\SI{0,60}{\mu m}$. 
			\item $\SI{0,50}{\mu m}$. 
			\item $\SI{0,70}{\mu m}$. 
			\item $\SI{0,64}{\mu m}$. 
		\end{mcq}
	}
	
	\loigiai
	{		\textbf{Đáp án: A.}
		
		Vì khoảng cách từ hai khe đến màn tăng lên nên vị trí điểm M sẽ bị dời về vị trí các vân tối gần vân trung tâm hơn. Vì trong quá trình này, vị trí điểm M bị chuyển thành vân tối hai lần nên nó ở vị trí vân tối thứ 4.
		
		Tại vị trí điểm M lúc đầu và lúc sau, ta có:
		$$
		x_{M} = 5 \dfrac{\lambda D}{a} = 3,5 \dfrac{\lambda (D + \Delta D)}{a} \rightarrow D = \SI{1,75}{m}.
		$$
		Tại vị trí điểm M lúc đầu, ta có:
		$$
		x_{M} = 5 \dfrac{\lambda D}{a} \rightarrow \lambda = \SI{0,60}{\mu m}.
		$$
	}
	\item \mkstar{2} 
	\cauhoi
	{Giao thoa khe Young với ánh sáng đơn sắc có bước sóng $\lambda = \SI{0,5}{\mu m}; a = \SI{0,5}{mm}; D = \SI{2}{m}$. Tại M cách vân trung tâm $\SI{7}{mm}$ và tại điểm N cách vân trung tâm $\SI{10}{mm}$ thì
		
		\begin{mcq}
			\item M, N đều là vân sáng.
			\item M là vân tối, N là vân sáng.	
			
			\item M, N đều là vân tối.	
			\item M là vân sáng, N là vân tối.
			
		\end{mcq}
	}
	
	\loigiai
	{		\textbf{Đáp án: B.}
		
		Ta có: 
		
		$$ i = \SI{2}{mm} \Rightarrow \begin{cases}
			x_\text{M} = \text{3,5}i.\\
			x_\text{N} = 5i.
		\end{cases}
		$$
		
		Suy ra M là vân tối, N là vân sáng.
	
	}
		\item \mkstar{2} 
	\cauhoi
	{Trong thí nghiệm giao thoa ánh sáng bằng khe Young. Cho biết S$_1$S$_2= a = \SI{1}{mm}$, khoảng cách giữa hai khe S$_1$S$_2$ đến màn (E) là $\SI{2}{m}$, bước sóng ánh sáng dùng trong thí nghiệm là  $\lambda = \SI{0,5}{\micro\meter}$. Để M trên màn (E) là một vân sáng thì $x_\text M$ có thể nhận giá trị nào trong các giá trị sau đây?
		
		\begin{mcq}(4)
			\item $x_\text M = \SI{2,25}{mm}.$ 
			\item $x_\text M = \SI{4}{mm}.$
			\item $x_\text M = \SI{3,5}{mm}.$
			\item $x_\text M = \SI{4,5}{mm}.$
		\end{mcq}
	}
	
	\loigiai
	{		\textbf{Đáp án: B.}
		
		Khoảng vân $i = \SI{1}{mm}.$
		
		Để M trên màn là một vân sáng $\Rightarrow x_\text{M} = ki$ với $k$ là số nguyên $\Rightarrow x_\text M = \SI{4}{mm}$ thỏa mãn. 
	}
		\item \mkstar{3} 
	\cauhoi
	{Trong thí nghiệm Young về giao thoa ánh sáng, người ta đo được khoảng vân là $\text{1,12} \cdot 10^3\ \mu \text m$. Xét 2 điểm M và N cùng một phía so với vân chính giữa, với $\text{OM} = \text{0,56}\cdot 10^4\ \mu \text m$ và $\text{ON} = \text{1,288}\cdot 10^4\ \mu \text m$, giữa M và N có bao nhiêu vân tối?
		
		\begin{mcq}(4)
			\item 5.
			\item 6.
			\item 7.
			\item 8.
		\end{mcq}
	}
	
	\loigiai
	{		\textbf{Đáp án: B.}
		
		Giải điều kiện: 
		
		$$\text{OM} < (k+\text{0,5}) < \text{ON} \Leftrightarrow \text{4,5} \leq k \leq 11 \Rightarrow k = 5,6,7,8,9,10.$$
		
		Suy ra có 6 vân tối nằm giữa M và N (nằm giữa nên ta không tính ở 2 đầu mút). 
	}
		\item \mkstar{3} 
	\cauhoi
	{Trong thí nghiệm giao thoa khe Young, khoảng cách giữa hai khe F$_1$F$_2$ là $a = \SI{2}{mm}$; khoảng cách từ hai khe F$_1$F$_2$ đến màn là $D = \SI{1,5}{m}$, dùng ánh sáng đơn sắc có bước sóng  $\lambda = \SI{0,6}{\mu m}$. Xét trên khoảng MN, với MO $= \SI{5}{mm}$, ON $= \SI{10}{mm}$, (O là vị trí vân sáng trung tâm), MN nằm hai phía vân sáng trung tâm. Số vân sáng trong đoạn MN là
		
		\begin{mcq}(4)
			\item 31.
			\item 32.
			\item 33.
			\item 34.
		\end{mcq}
	}
	
	\loigiai
	{		\textbf{Đáp án: D.}
		
		Khoảng vân 
		
		$$ i  = \dfrac{\lambda D}{a} = \SI{0,45}{mm}.$$
		
		Số vân sáng là số giá trị $k$ nguyên thỏa mãn
		
		$$ - 5 \leq ki \leq 10 \Leftrightarrow - \text{11,1} \leq k \leq \text{22,2}.$$
		
		Có 34 vân sáng.
	}
		\item \mkstar{3} 
	\cauhoi
	{Trong thí nghiệm về giao thoa ánh sáng của Young, chùm sáng đơn sắc có bước sóng  $\lambda= \SI{0,5}{\mu m}$, khoảng cách giữa 2 khe là $\SI{1,2}{mm}$, khoảng cách từ 2 khe đến màn ảnh là $\SI{3}{m}$. Hai điểm M, N nằm cùng phía với vân sáng trung tâm, cách vân trung tâm các khoảng $\SI{4}{mm}$ và $\SI{18}{mm}$. Giữa M và N có bao nhiêu vân sáng?
		
		\begin{mcq}(4)
			\item 11 vân.
			\item 7 vân.
			\item 8 vân.
			\item 9 vân.
		\end{mcq}
	}
	
	\loigiai
	{		\textbf{Đáp án: A.}
		
		
		Khoảng vân $i = \dfrac{\lambda D}{a} = \SI{1,25}{mm.}$
		
		Số vân sáng là số giá trị $k$ nguyên thỏa mãn:
		
		$$8 <ki<18 \Leftrightarrow \text{3,2} <k <\text{14,4}$$
		
		Có 11 vân sáng.
		
	}
	
\end{enumerate}






